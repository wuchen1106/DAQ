\subparagraph{Examples of MPET Perlscripts for run control}\label{RC_mhttpd_perlrc}
\par


 \label{RC_mhttpd_perlrc_idx_script_perlmidas}
\hypertarget{RC_mhttpd_perlrc_idx_script_perlmidas}{}


Part of the run control perlscripts for MPET experiment at TRIUMF (written by Vladimir Rykov) are reproduced below. The script \hyperlink{RC_mhttpd_perlrc_RC_mhttpd_perlrc_script}{perlrc.pl} calls a script called \hyperlink{RC_mhttpd_perlrc_RC_mhttpd_perlmidas_script}{perlmidas.pl} to access the ODB.

\hyperlink{RC_mhttpd_perlrc_RC_mhttpd_perlmidas_script}{perlmidas.pl} may be of interest to users who wish to interact with the ODB through scripts.\hypertarget{RC_mhttpd_perlrc_RC_mhttpd_perlmidas_script}{}\subsubsection{perlmidas.pl}\label{RC_mhttpd_perlrc_RC_mhttpd_perlmidas_script}

\begin{DoxyCode}
# common subroutines
use strict;
use warnings;
##############################################################
sub MIDAS_env
# set up proper MIDAS environment...
##############################################################
{
    our ($MIDAS_HOSTNAME,$MIDAS_EXPERIMENT,$ODB_SUCCESS,$DEBUG);
    our ($CMDFLAG_HOST, $CMDFLAG_EXPT);

    $ODB_SUCCESS=0;

    $MIDAS_HOSTNAME = $ENV{"MIDAS_SERVER_HOST"};
    if (defined($MIDAS_HOSTNAME) &&   $MIDAS_HOSTNAME ne "")
    {
        $CMDFLAG_HOST = "-h $MIDAS_HOSTNAME";
    }
    else
    {
        $MIDAS_HOSTNAME = "";
        $CMDFLAG_HOST = "";
    }

    $MIDAS_EXPERIMENT = $ENV{"MIDAS_EXPT_NAME"};
    if (defined($MIDAS_EXPERIMENT) &&   $MIDAS_EXPERIMENT ne "")
    {
        $CMDFLAG_EXPT = "-e ${MIDAS_EXPERIMENT}";
    }
    else
    {
        $MIDAS_EXPERIMENT = "";
        $CMDFLAG_EXPT = "";
    }

}


##############################################################
sub MIDAS_sendmsg
##############################################################
{
# send a message to odb message logger
    my ($name, $message) =  @_;

    our ($MIDAS_HOSTNAME,$MIDAS_EXPERIMENT,$ODB_SUCCESS,$DEBUG);
    our ($CMDFLAG_HOST, $CMDFLAG_EXPT);
    our ($COMMAND, $ANSWER);

    my $status;
    my $host="";
    my $dquote='"';
    my $squote="'";
    my $command="${dquote}msg ${name} ${squote}${message}${squote}${dquote}";
    print "name=$name, message=$message\n";
    print "command is: $command \n";

    $COMMAND ="`odb ${CMDFLAG_EXPT} ${CMDFLAG_HOST} -c ${command}`";
    $ANSWER=`odb ${CMDFLAG_EXPT} ${CMDFLAG_HOST} -c ${command}`;
    $status=$?;
    chomp $ANSWER;  # strip trailing linefeed
    if($DEBUG)
    {
        print "command: $COMMAND\n";
        print " answer: $ANSWER\n";
    }

    if($status != $ODB_SUCCESS) 
    { # this status value is NOT the midas status code
        print "send_message:  Failure status returned from odb msg (status=$statu
      s)\n";
    }
    return;
}

sub strip
{
# removes / from end of string, // becomes /
    my $string=shift;
    $string=~ (s!//!/!g);
    $string=~s!/$!!;
    print "strip: now \"$string\"\n";
    return ($string);
}

sub MIDAS_varset
##############################################################
{
# set a value of an odb key
    my ($key, $value) =  @_;

    our ($MIDAS_HOSTNAME,$MIDAS_EXPERIMENT,$ODB_SUCCESS,$DEBUG);
    our ($CMDFLAG_HOST, $CMDFLAG_EXPT);
    our ($COMMAND, $ANSWER);

    my $status;
    my $host="";
    my $dquote='"';
    my $squote="'";
    my $command="${dquote}set ${squote}${key}${squote} ${squote}${value}${squote}
      ${dquote}";
    print "key=$key, new value=${value}\n";
    print "command is: $command \n";

    $COMMAND ="`odb ${CMDFLAG_EXPT} ${CMDFLAG_HOST} -c command`";
    $ANSWER=`odb ${CMDFLAG_EXPT} ${CMDFLAG_HOST} -c $command `;
    $status=$?;
    chomp $ANSWER;  # strip trailing linefeed
    if($DEBUG)
    {
        print "command: $COMMAND\n";
        print " answer: $ANSWER\n";
    }

    if($status != $ODB_SUCCESS) 
    { # this status value is NOT the midas status code
        print "send_message:  Failure status returned from odb msg (status=$statu
      s)\n";
    }
    return;
}

sub MIDAS_varget
##############################################################
{
# set a value of an odb key
    my ($key) =  @_;

    our ($MIDAS_HOSTNAME,$MIDAS_EXPERIMENT,$ODB_SUCCESS,$DEBUG);
    our ($CMDFLAG_HOST, $CMDFLAG_EXPT);
    our ($COMMAND, $ANSWER);

    my $status;
    my $host="";
    my $dquote='"';
    my $squote="'";
    my $command="${dquote}ls -v ${squote}${key}${squote}${dquote}";
    print "key=$key\n";
    print "command is: $command \n";
    
    $COMMAND ="`odb ${CMDFLAG_EXPT} ${CMDFLAG_HOST} -c command`";
    $ANSWER=`odb ${CMDFLAG_EXPT} ${CMDFLAG_HOST} -c $command `;  
    $status=$?;
    chomp $ANSWER;  # strip trailing linefeed
    if($DEBUG)
    {
        print "command: $COMMAND\n";
        print " answer: $ANSWER\n";
    }

    if($status != 0) 
    { # this status value is NOT the midas status code
        print "send_varset  Failure status returned from odb msg (status=$status)
      \n";
    }
    return $ANSWER;
}

sub MIDAS_dirlist
##############################################################
{
# return a directory list of directory given by odb key
    my ($key) =  @_;

    our ($MIDAS_HOSTNAME,$MIDAS_EXPERIMENT,$ODB_SUCCESS,$DEBUG);
    our ($CMDFLAG_HOST, $CMDFLAG_EXPT);
    our ($COMMAND, $ANSWER);

    my $status;
    my $host="";
    my $dquote='"';
    my $squote="'";
    my $command="${dquote}ls ${squote}${key}${squote}${dquote}";
    print "key=$key\n";
    print "command is: $command \n";
    
    $COMMAND ="`odb ${CMDFLAG_EXPT} ${CMDFLAG_HOST} -c command`";
    $ANSWER=`odb ${CMDFLAG_EXPT} ${CMDFLAG_HOST} -c $command `;  
    $status=$?;
    chomp $ANSWER;  # strip trailing linefeed
    if($DEBUG)
    {
        print "command: $COMMAND\n";
        print " answer: $ANSWER\n";
    }

    if($status != 0) 
    { # this status value is NOT the midas status code
        print "send_varset  Failure status returned from odb msg (status=$status)
      \n";
    }
    return $ANSWER;
}

sub MIDAS_startrun
##############################################################
{
# start MIDAS run
    my ($key) =  @_;

    our ($MIDAS_HOSTNAME,$MIDAS_EXPERIMENT,$ODB_SUCCESS,$DEBUG);
    our ($CMDFLAG_HOST, $CMDFLAG_EXPT);
    our ($COMMAND, $ANSWER);

    our ($SCANLOG_FH);

    my $status;
    my $host="";
    my $dquote='"';
    my $squote="'";
    my $command="${dquote}start now${dquote}";
    print "command is: $command \n";

    #sleep(10);

    $COMMAND ="`odb ${CMDFLAG_EXPT} ${CMDFLAG_HOST} -c ${command}`";
    $ANSWER=`odb ${CMDFLAG_EXPT} ${CMDFLAG_HOST} -c ${command}`;
    $status=$?;
    chomp $ANSWER;  # strip trailing linefeed
    if($DEBUG)
    {
        print "command: $COMMAND\n";
        print " answer: $ANSWER\n";

        #print $SCANLOG_FH "status: $status\n";
        #print $SCANLOG_FH "command: $COMMAND\n";
        #print $SCANLOG_FH " answer: $ANSWER\n";

    }

    if($status != 0)
    { # this status value is NOT the midas status code
        print "startrun:  Failure status returned from odb msg (status=$status)\n
      ";
        print $SCANLOG_FH " answer: $ANSWER\n";

    }
    return $ANSWER;
}   
1;
\end{DoxyCode}


\par


\par
\hypertarget{RC_mhttpd_perlrc_RC_mhttpd_perlrc_script}{}\subsubsection{perlrc.pl}\label{RC_mhttpd_perlrc_RC_mhttpd_perlrc_script}

\begin{DoxyCode}
 #!/usr/bin/perl

################################################################
#
#  PerlRC
#
#  MIDAS piggyback perl script that is exectuted upon completion
#  of a run. It checks its parameters, modifies the MIDAS variables
#  as required, and starts a new run. This way it can run through
#  different DAQ settings. Implemented scans:
#  1) Scan1D - scans a set of variables from beginning values
#     to ending values. All valiables are changed simultaneously.
#  2) Scan2D - scans 2 sets of variables.
#  3) SettingsSwitch - switches between different settings sets
#     typically to be used to switch between ion species.
#
#  V. Ryjkov
#  June 2008
#
################################################################

require "/home/mpet/vr/perl/PerlRC/perlmidas.pl";

our $DEBUG = true;
our $PERLSCAN_PREF = "/PerlRC";
our $PERLSCAN_CONTROLVARS = $PERLSCAN_PREF . "/ControlVariables";
our $PERLSCAN_START = $PERLSCAN_PREF . "/RunControl/RCActive";
our $PERLSCAN_NRUNS = $PERLSCAN_PREF . "/RunControl/RCTotalRuns";
our $PERLSCAN_CURRUN = $PERLSCAN_PREF . "/RunControl/RCCurrentRun";
our $SCANLOG_PATH = "/data/mpet/PerlRC.log";
our $SCANLOG_FH;
our $MIDAS_RUNNO = "/Runinfo/Run number";
my  $PERLSCAN_SCANTYPE = $PERLSCAN_PREF . "/RunControl/RCType";

MIDAS_env();
# MIDAS_sendmsg("test","run stop");
my $ScanStart  =MIDAS_varget($PERLSCAN_START);
my $ScanType   =MIDAS_varget($PERLSCAN_SCANTYPE);
my $NRuns      =MIDAS_varget($PERLSCAN_NRUNS);
my $CurrentRun =MIDAS_varget($PERLSCAN_CURRUN);
my $retval;
my $MIDASrunno;

open(SCANLOG,">>${SCANLOG_PATH}");
$SCANLOG_FH=\*SCANLOG;

if(scalar(@ARGV)==1 && $ARGV[0] =~ /start/) {
    MIDAS_varset($PERLSCAN_START,'y');
    $ScanStart = "y";
}
if(scalar(@ARGV)==1 && $ARGV[0] =~ /stop/) {
    MIDAS_varset($PERLSCAN_START,'n');
    $ScanStart = "n";
}
if( $ScanStart eq "y") {
    if( $CurrentRun == 0) {
        print $SCANLOG_FH "=== NEW PerlRC scan. Scan type is \"${ScanType}\" ===\
      n";
        print $SCANLOG_FH "===    Number of runs in this scan is ${NRuns}    ===\
      n";
    }
    if( $CurrentRun == $NRuns) {
        print $SCANLOG_FH "=== Finished PerlRC scan ===\n";
        print $SCANLOG_FH "============================\n";
    }
    if( ++$CurrentRun <= $NRuns ) {
        $MIDASrunno=MIDAS_varget($MIDAS_RUNNO);
        $MIDASrunno++;
        print $SCANLOG_FH "<Run #${MIDASrunno}> ";
        MIDAS_varset($PERLSCAN_CURRUN,$CurrentRun);
        for ($ScanType) {
            if    (/Scan1D/)   {$retval=Scan1D(); }     # do something
            elsif (/Scan2D/)   {$retval=Scan2D(); }     # do something else
            elsif (/TuneSwitch/) {$retval=TuneSwitch(); } # do something else
        }
        if($retval != 0) {
            MIDAS_varset($PERLSCAN_CURRUN,0);
            MIDAS_varset($PERLSCAN_START,"n");
            print $SCANLOG_FH "!!!#### Aborting scan! ####!!!\n";
        }
        else {
            sleep(1);
            #print $SCANLOG_FH "pausing 10 sec...\n";
            MIDAS_startrun();
            #print $SCANLOG_FH "start the run\n";
        }
    }
    else {
        MIDAS_varset($PERLSCAN_CURRUN,0);
        MIDAS_varset($PERLSCAN_START,"n");
    }
}
else {
    if(scalar(@ARGV)==2 && $ARGV[0] =~ /tune/) {
        SwitchToTune($ARGV[1]);
    }
}
close(SCANLOG);

sub Scan1D
{

    ............


}    


sub SetControlVar
{
    our $SCANLOG_FH;
    our $PERLSCAN_CONTROLVARS;
    my ($varname,$varvalue)=@_;
    my $retval;
    my $varpath;

    #print $SCANLOG_FH "variablename: $varname \n";

    $varpath=MIDAS_varget($PERLSCAN_CONTROLVARS . "/" . $varname);
    if($varpath =~ /^key (.*) not found/) {
        print $SCANLOG_FH "! Control variable ${varname}(${1}) is not listed in $
      {PERLSCAN_CONTROLVARS}\n";
        return -4;
    }

    .............
    
    
    val=MIDAS_varset($varpath,$varvalue);
        if($retval =~ /^key not found/) {return -5;}
    }
    return 0;
}

sub SwitchToTune
{
    our $SCANLOG_FH;
    our $PERLSCAN_CONTROLVARS;
    our $PERLSCAN_PREF;
    my $PERLSCAN_TUNEDIR = $PERLSCAN_PREF . "/Tunes";
    my ($tunename)=@_;
    my $retval;
    my $varpath;
    my $varval;
    my $cvarname;

    $retval = MIDAS_dirlist($PERLSCAN_TUNEDIR . "/" . $tunename);
    if($retval =~ /^key not found/){
        print $SCANLOG_FH "! Could not locate tune ${tunename} in the tune direct
      ory ${PERLSCAN_TUNEDIR}\n";
        return -7;
    }
    my @TuneVars=split(/\n/,$retval);
    foreach (@TuneVars) {
        if (/^(.+\S)\s{2,}.*/) {
            $cvarname = $1;
            $varval = MIDAS_varget($PERLSCAN_TUNEDIR . "/" . $tunename . "/" .$cv
      arname);
            $retval = SetControlVar($cvarname, $varval);
            if($retval < 0) {return $retval;}
        }
        else {
            print $SCANLOG_FH "! Cannot decipher tune variable list, offending li
      ne: $_\n";
            return -8;
        }
        sleep(1);
    }
    return 0;
}

sub Scan2D
{
   .................
}


sub TuneSwitch
{   
    our  ($PERLSCAN_PREF, $PERLSCAN_START);
    our $SCANLOG_FH;
    my $PERLSCAN_TUNESWITCHDIR = "/RunControl/TuneSwitch";
    my $PERLSCAN_TUNESLIST = $PERLSCAN_PREF . $PERLSCAN_TUNESWITCHDIR .  "/TunesL
      ist";
    my $PERLSCAN_TUNEIDX = $PERLSCAN_PREF . $PERLSCAN_TUNESWITCHDIR .  "/CurrentT
      uneIndex";
    my $PERLSCAN_TUNENAME = $PERLSCAN_PREF . $PERLSCAN_TUNESWITCHDIR .  "/Current
      TuneName";

    my $TunesList = MIDAS_varget($PERLSCAN_TUNESLIST);
    my $TuneIdx = MIDAS_varget($PERLSCAN_TUNEIDX);
    my $TuneName = MIDAS_varget($PERLSCAN_TUNENAME);
    
    my @tunes = split(/\s*;\s*/,$TunesList);
    print "tunes length= ",scalar(@tunes),"\n";
    if( ++$TuneIdx > scalar(@tunes) ) {
            $TuneIdx=1;
    }
    MIDAS_varset($PERLSCAN_TUNEIDX,$TuneIdx);

    $retval=SwitchToTune($tunes[$TuneIdx-1]);
    if($retval < 0) {return $retval;}
    MIDAS_varset($PERLSCAN_TUNENAME,$tunes[$TuneIdx-1]);
    print $SCANLOG_FH "Tune is \"",$tunes[$TuneIdx-1],"\"\n";
    return 0;
}
\end{DoxyCode}


 \par
 \label{index_end}
\hypertarget{index_end}{}
 