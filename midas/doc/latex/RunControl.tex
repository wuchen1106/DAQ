\par
  \par
\hypertarget{RunControl_RC_Intro}{}\subsection{Introduction}\label{RunControl_RC_Intro}
This section describes {\bfseries Run} {\bfseries Control} and {\bfseries Monitoring} of the experiment.

A {\bfseries \char`\"{}run\char`\"{}} starts when the MIDAS system receivs a {\bfseries  Start Transition }, and continues until it receives a {\bfseries  Stop transition }. In the simplest case, this is when the commands {\bfseries \char`\"{}Start\char`\"{}} and {\bfseries \char`\"{}Stop\char`\"{}} are issued.

{\bfseries  \char`\"{}Run Control\char`\"{} } describes how that run may be controlled, including
\begin{DoxyItemize}
\item where the data is to be stored (e.g. tape),
\item on what conditions the run should be allowed to start (e.g. hardware is ready, beam is on)
\item when it should stop (e.g. beam goes off, hardware failure).
\end{DoxyItemize}

{\bfseries  \char`\"{}Monitoring\char`\"{} } involves
\begin{DoxyItemize}
\item informing the user on the progress of the run,
\item displaying statistics or history plots,
\item sending information or alarm messages to warn the user of any problems.
\end{DoxyItemize}

\par
 Fundamental to Run Control and Monitoring is the \hyperlink{F_MainElements_F_Online_Database_overview}{Online Database (ODB)} which contains all of the information for an experiment. Therefore a run control program requires access only to information in the ODB.

\par
 Programming exactly what data is read during the run falls under the category \hyperlink{FrontendOperation}{SECTION 6: Frontend Operation} , and analyzing the data during or after the run is described under \hyperlink{DataAnalysis}{SECTION 7: Data Analysis} . \par



\begin{DoxyItemize}
\item \hyperlink{RC_Run_States_and_Transitions}{Run States and Transitions}
\item \hyperlink{RC_run_control}{Run Control Programs}
\begin{DoxyItemize}
\item \hyperlink{RC_odbedit_utility}{odbedit}
\item \hyperlink{RC_mhttpd}{mhttpd}
\end{DoxyItemize}
\item \hyperlink{RC_Monitor}{Monitoring the Experiment}
\item \hyperlink{RC_customize_ODB}{Customizing the Experiment}
\item \hyperlink{RC_Hot_Link}{Event Notification (Hot-\/Link)}
\end{DoxyItemize}

\par
\par


\label{index_end}
\hypertarget{index_end}{}
  \par
 \subsection{Run States and Transitions}\label{RC_Run_States_and_Transitions}
\par
 

\par
 \label{RC_Run_States_and_Transitions_idx_run_states}
\hypertarget{RC_Run_States_and_Transitions_idx_run_states}{}
 \label{RC_Run_States_and_Transitions_idx_transition}
\hypertarget{RC_Run_States_and_Transitions_idx_transition}{}
 Three {\bfseries Run States} define the states of the MIDAS DAQ system: \par
 \par
\begin{center} {\bfseries {\itshape Stopped\/}, {\itshape Paused\/}, and {\itshape Running\/}.} \end{center}  \par
 In order to change from one state to another, MIDAS provides four basic {\bfseries transition} {\bfseries functions} : \par
 \par
\begin{center}{\bfseries {\itshape TR\_\-START\/}, {\itshape TR\_\-PAUSE\/}, {\itshape TR\_\-RESUME\/}, {\itshape TR\_\-STOP\/} }\end{center} 

\begin{center} Transitions  \end{center} 

\label{RC_Run_States_and_Transitions_idx_run_state_codes}
\hypertarget{RC_Run_States_and_Transitions_idx_run_state_codes}{}
 \label{RC_Run_States_and_Transitions_idx_run_transition_codes}
\hypertarget{RC_Run_States_and_Transitions_idx_run_transition_codes}{}
 \label{RC_Run_States_and_Transitions_state_transition}
\hypertarget{RC_Run_States_and_Transitions_state_transition}{}
 \hypertarget{RC_Run_States_and_Transitions_RC_state_transition}{}\subsubsection{MIDAS State and Transition Codes}\label{RC_Run_States_and_Transitions_RC_state_transition}
MIDAS provides three {\bfseries  State Codes } (defined in \hyperlink{midas_8h}{midas.h}) for the three run states (Stopped, Paused and Running). The current State of an experiment is available under the \hyperlink{RC_Run_States_and_Transitions_RC_ODB_RunInfo_Tree}{ODB /RunInfo Tree}.

\begin{TabularC}{2}
\hline
\multirow{1}{\linewidth}{{\bfseries Transition} \par
  }&\multirow{1}{\linewidth}{{\bfseries Value}\par
   }\\\cline{1-2}
STATE\_\-STOPPED&1 \\\cline{1-2}
STATE\_\-PAUSED&2 \\\cline{1-2}
STATE\_\-RUNNING&3 \\\cline{1-2}
\end{TabularC}


and four {\bfseries  Transition Codes } for the four Transitions:


\begin{DoxyItemize}
\item TR\_\-START
\item TR\_\-STOP
\item TR\_\-PAUSE
\item TR\_\-RESUME
\end{DoxyItemize}

While a transition is in progress, these Transition Codes will be found under the \hyperlink{RC_Run_States_and_Transitions_RC_ODB_RunInfo_Tree}{ODB /RunInfo Tree} for an experiment.



 \label{RC_Run_States_and_Transitions_idx_run_transition_priority}
\hypertarget{RC_Run_States_and_Transitions_idx_run_transition_priority}{}
 \hypertarget{RC_Run_States_and_Transitions_RC_Transition_priority}{}\subsubsection{Run Transition Priority}\label{RC_Run_States_and_Transitions_RC_Transition_priority}
During these transition periods, any MIDAS client registered to receive notification of such a transition will be able to perform dedicated tasks in either synchronized or asynchronized mode, within the overall run control of the experiment.

In order to provide more flexibility to the transition sequence of all the MIDAS clients connected to a given experiment, each transition function has a {\bfseries  transition sequence number } attached to it. This transition sequence is used to establish within a given transition the order of the invocation of the MIDAS clients (from the lowest sequence number to the highest). By this means, {\bfseries  MIDAS provides the user with full control of the sequencing of any MIDAS client. }

Click on the links to see an example of how the transition priority operates at \hyperlink{RC_odbedit_examples_RC_transition_start}{start of run} and at \hyperlink{RC_odbedit_examples_RC_transition_stop}{end of run}.

\par


\par
 \label{RC_Run_States_and_Transitions_idx_transition_register}
\hypertarget{RC_Run_States_and_Transitions_idx_transition_register}{}
 \hypertarget{RC_Run_States_and_Transitions_RC_Register_for_run_transition}{}\paragraph{Registering for a run transition}\label{RC_Run_States_and_Transitions_RC_Register_for_run_transition}
Any MIDAS client can register to receive notification for a run transition. This notification is done by registering to the system for a given transition ( \hyperlink{group__cmfunctionc_ga00950930acadf846be75239b6d0e80dc}{cm\_\-register\_\-transition()} ) by specifying the transition type and the sequencing number (1 to 1000). \par
 The following example show registering to the TR\_\-START transition with a sequencing number of 450 and to a TR\_\-STOP transition with a sequencing number of 650.


\begin{DoxyCode}
  INT main()
  {
    ...
  cm_register_transition(TR_START, tr_prestart,450);
  cm_register_transition(TR_STOP, tr_poststop,650);
    ...
  }

  // callback
  INT tr_prestart(INT run_number, char *error)
  {
        // code to perform actions prior to frontend starting 

        return (status);
  }

  // callback
  INT tr_poststop(INT run_number, char *error)
  {
        // code to perform actions after frontend has stopped 

        return (status);
  }
\end{DoxyCode}



\begin{DoxyItemize}
\item On a TR\_\-START transition, tr\_\-prestart will run {\bfseries after} the logger and event builder, but {\bfseries before} the Frontend (see \hyperlink{RC_Run_States_and_Transitions_RC_Default_Seq_Numbers}{Default Transition Sequence Numbers}). 
\item On a TR\_\-STOP transition, tr\_\-poststop will run {\bfseries after} the frontend, but {\bfseries before} the logger and event builder. 
\end{DoxyItemize}

\par


\par
 \hypertarget{RC_Run_States_and_Transitions_RC_Multiple_Registration}{}\subparagraph{Multiple Registration to a Transition}\label{RC_Run_States_and_Transitions_RC_Multiple_Registration}
Multiple registration to a given transition can be requested. This last option allows (for example) {\bfseries  two callback functions to be invoked before and after a given transition }, such as the start of the logger.


\begin{DoxyCode}
my_application.c
  // Callback 
  INT before_logger(INT run_number, char *error)
  {
    printf("Initialize ... before the logger gets the Start Transition");
    ...
    return CM_SUCCESS;
  }

  // Callback 
  INT after_logger(INT run_number, char *error)



  {
    printf("Log initial info to file... after logger gets the Start Transition");
      
    ...
    return CM_SUCCESS;
  }

  INT main()
  {
    ...
    cm_register_transition(TR_START, before_logger, 100);
    cm_register_transition(TR_START, after_logger, 300);
    ...
  }
\end{DoxyCode}


\par


\par
 \label{RC_Run_States_and_Transitions_idx_run_transition_sequence}
\hypertarget{RC_Run_States_and_Transitions_idx_run_transition_sequence}{}
 \hypertarget{RC_Run_States_and_Transitions_RC_Default_Seq_Numbers}{}\paragraph{Default Transition Sequence Numbers}\label{RC_Run_States_and_Transitions_RC_Default_Seq_Numbers}
By default the following sequence numbers are used:

\par
 \begin{TabularC}{5}
\hline
\multirow{2}{\linewidth}{{\bfseries Client} \par
  }&\multirow{1}{\linewidth}{{\bfseries Default Transition Sequence Number}\par
   }\\\cline{1-2}
TR\_\-START\par
  &TR\_\-PAUSE\par
  &TR\_\-RESUME\par
  &TR\_\-STOP\par
   \\\cline{1-4}
Frontend\par
  &500\par
  &500\par
  &500\par
  &500\par
   \\\cline{1-5}
Analyzer\par
  &500 &500 &500 &500  \\\cline{1-5}
Logger\par
  &200\par
  &500 &500 &800\par
   \\\cline{1-5}
EventBuilder \par
  &300\par
  &500 &500 &700\par
   \\\cline{1-5}
\end{TabularC}
\par
 \par


\par
 \hypertarget{RC_Run_States_and_Transitions_RC_Review_Seq_Num}{}\paragraph{Review the Client Transition Sequence Numbers in the ODB}\label{RC_Run_States_and_Transitions_RC_Review_Seq_Num}
\label{RC_Run_States_and_Transitions_RC_odb_system_tree}
\hypertarget{RC_Run_States_and_Transitions_RC_odb_system_tree}{}
 \label{RC_Run_States_and_Transitions_idx_ODB_tree_System}
\hypertarget{RC_Run_States_and_Transitions_idx_ODB_tree_System}{}
 The {\bfseries  ODB /System tree } contains information specific to each \char`\"{}MIDAS client\char`\"{} currently connected to the experiment. The sequence number appears in the ODB under /System/Clients/ 
\begin{DoxyCode}
[local:midas:S]Clients>ls -lr
Key name                        Type    #Val  Size  Last Opn Mode Value
---------------------------------------------------------------------------
Clients                         DIR
    1832                        DIR     <------------ Frontend 1
        Name                    STRING  1     32    21h  0   R    ebfe01
        Host                    STRING  1     256   21h  0   R    pierre2
        Hardware type           INT     1     4     21h  0   R    42
        Server Port             INT     1     4     21h  0   R    2582
        Transition START        INT     1     4     21h  0   R    500
        Transition STOP         INT     1     4     21h  0   R    500
        Transition PAUSE        INT     1     4     21h  0   R    500
        Transition RESUME       INT     1     4     21h  0   R    500
        RPC                     DIR
            17000               BOOL    1     4     21h  0   R    y
    3872                        DIR     <------------ Frontend 2 
        Name                    STRING  1     32    21h  0   R    ebfe02
        Host                    STRING  1     256   21h  0   R    pierre2
        Hardware type           INT     1     4     21h  0   R    42
        Server Port             INT     1     4     21h  0   R    2585
        Transition START        INT     1     4     21h  0   R    500
        Transition STOP         INT     1     4     21h  0   R    500
        Transition PAUSE        INT     1     4     21h  0   R    500
        Transition RESUME       INT     1     4     21h  0   R    500
        RPC                     DIR
            17000               BOOL    1     4     21h  0   R    y
    2220                        DIR     <------------ ODBedit doesn't need transi
      tion
        Name                    STRING  1     32    42s  0   R    ODBEdit
        Host                    STRING  1     256   42s  0   R    pierre2
        Hardware type           INT     1     4     42s  0   R    42
        Server Port             INT     1     4     42s  0   R    3429
    568                         DIR     <------------ Event Builder
        Name                    STRING  1     32    26s  0   R    Ebuilder
        Host                    STRING  1     256   26s  0   R    pierre2
        Hardware type           INT     1     4     26s  0   R    42
        Server Port             INT     1     4     26s  0   R    3432
        Transition START        INT     1     4     26s  0   R    300
        Transition STOP         INT     1     4     26s  0   R    700
    2848                        DIR     <------------ Logger 
        Name                    STRING  1     32    5s   0   R    Logger
        Host                    STRING  1     256   5s   0   R    pierre2
        Hardware type           INT     1     4     5s   0   R    42
        Server Port             INT     1     4     5s   0   R    3436
        Transition START        INT     1     4     5s   0   R    200
        Transition STOP         INT     1     4     5s   0   R    800
        Transition PAUSE        INT     1     4     5s   0   R    500
        Transition RESUME       INT     1     4     5s   0   R    500
        RPC                     DIR
            14000               BOOL    1     4     5s   0   R    y
\end{DoxyCode}


\par


\par
 \hypertarget{RC_Run_States_and_Transitions_RC_Change_Seq_Num}{}\paragraph{Change the Client Sequence Number}\label{RC_Run_States_and_Transitions_RC_Change_Seq_Num}
The {\itshape /System/Clients/\/}... tree reflects the system at a given time. If a permanent change of a client sequence number is required, the MIDAS system call \hyperlink{group__cmfunctionc_gac4b2b97d9cd12320b6f910f32e131aef}{cm\_\-set\_\-transition\_\-sequence()} can be used.

\par


\par
 \hypertarget{RC_Run_States_and_Transitions_RC_Deferred_Transition}{}\subsubsection{Deferred Transitions}\label{RC_Run_States_and_Transitions_RC_Deferred_Transition}
Any transition may be deferred until some condition is satisfied. This is usually set up in a frontend (see \hyperlink{FE_event_notification_FE_Deferred_Transition}{setup Deferred Transition} for further information).

\par
 

 \par
 \label{RC_Run_States_and_Transitions_idx_ODB_tree_RunInfo}
\hypertarget{RC_Run_States_and_Transitions_idx_ODB_tree_RunInfo}{}
 \hypertarget{RC_Run_States_and_Transitions_RC_ODB_RunInfo}{}\subsubsection{Run Information}\label{RC_Run_States_and_Transitions_RC_ODB_RunInfo}
Basic information about the state of the current run is available in the \hyperlink{RC_Run_States_and_Transitions_RC_ODB_RunInfo_Tree}{ODB /RunInfo Tree} . This information is displayed on the \hyperlink{RC_mhttpd_utility_RC_mhttpd_minimal_status_page}{Main Status page of mhttpd} or can viewed with \hyperlink{RC_odbedit_utility}{odbedit} as shown \hyperlink{RC_Run_States_and_Transitions_RC_ODB_RunInfo_Tree}{below} .\hypertarget{RC_Run_States_and_Transitions_RC_ODB_RunInfo_Tree}{}\paragraph{ODB /RunInfo Tree}\label{RC_Run_States_and_Transitions_RC_ODB_RunInfo_Tree}
This branch of the ODB contains system information related to the run information. Several time fields are available for run time statistics. 
\begin{DoxyCode}
odb -e expt -h host
[host:expt:Running]/>ls -r -l /runinfo
Key name                      Type    #Val  Size  Last Opn Mode Value
------------------------------------------------------------------------
Runinfo                        DIR
    State                      INT     1     4     2h   0   RWD  3
    Online Mode                INT     1     4     2h   0   RWD  1
    Run number                 INT     1     4     2h   0   RWD  8521
    Transition in progress     INT     1     4     2h   0   RWD  0
    Requested transition       INT     1     4     2h   0   RWD  0
    Start time                 STRING  1     32    2h   0   RWD  Thu Mar 23 10:03
      :44 2000
    Start time binary          DWORD   1     4     2h   0   RWD  953834624
    Stop time                  STRING  1     32    2h   0   RWD  Thu Mar 23 10:03
      :33 2000
    Stop time binary           DWORD   1     4     2h   0   RWD  0
\end{DoxyCode}


\begin{table}[h]\begin{TabularC}{2}
\hline
Keys in the ODB tree /RunInfo   \\\cline{1-1}
ODB Key  &Explanation  \\\cline{1-2}
State  &Specifies the state of the current run (see \hyperlink{RC_Run_States_and_Transitions_RC_state_transition}{MIDAS State and Transition Codes}) . The possible states are
\begin{DoxyItemize}
\item 1: STOPPED ( STATE\_\-STOPPED )
\item 2: PAUSED ( STATE\_\-PAUSED )
\item 3: RUNNING ( STATE\_\-RUNNING )
\end{DoxyItemize}

\\\cline{1-2}
Online Mode  &Specifies the expected acquisition mode. This parameter allows the user to detect if the data are coming from a \char`\"{}real-\/time\char`\"{} hardware source or from a data save-\/set. Note that for analysis replay using \char`\"{}analyzer\char`\"{} this flag will be switched off.   \\\cline{1-2}
Run number  &Specifies the current run number. This number is automatically incremented by a successful run start procedure.   \\\cline{1-2}
Transition in progress  &Specifies the current internal state of the system. This parameter is used for multiple source of \char`\"{}run start\char`\"{} synchronization.   \\\cline{1-2}
Requested transition  &Specifies the current internal state of the \hyperlink{RC_Run_States_and_Transitions_RC_Deferred_Transition}{Deferred Transitions} state of the system.   \\\cline{1-2}
Start Time  &Specifies in an ASCII format the time at which the last run has been started.   \\\cline{1-2}
Start Time binary  &Specifies in a binary format at the time at which the last run has been started This field is useful for time interval computation.   \\\cline{1-2}
Stop Time  &Specifies in an ASCII format the time at which the last run has been stopped.   \\\cline{1-2}
Stop Time binary  &Specifies in a binary format the time at which the last run has been stopped. This field is useful for time interval computation.   \\\cline{1-2}
\end{TabularC}
\centering
\caption{Above: meaning of keys in the /RunInfo ODB tree }
\end{table}


\label{index_end}
\hypertarget{index_end}{}
  \par
 \subsection{Run Control Programs}\label{RC_run_control}
\par
  \par


Users must have some way of controlling the experiment (i.e. starting and stopping a run, changing the run \hyperlink{structparameters}{parameters} etc.) and of determining the progress of the run (e.g. whether the run started successfully, whether data is being taken and saved, and whether any error conditions have occurred). \par
 Two options for {\bfseries  Run Control } are provided in the MIDAS package. They are \hyperlink{RC_odbedit_utility}{odbedit}, a program with a simple command line interface, and \hyperlink{RC_mhttpd_utility}{mhttpd}, the web-\/based run control program. \par
 {\bfseries mhttpd} is the usual choice for experimenters, since it provides a graphical interface, and is used both for run control and monitoring. It has many features not available in odbedit, such as history display, electronic logbook, custom pages and alias-\/links. However, it has limited functionality as an ODB editor. \par
 {\bfseries odbedit} is very useful for debugging, is often quicker and simpler to use, and is a fully functional odb editor (hence its name). It has limited monitoring capability. \par
 Other utilities for monitoring are also provided, such as \hyperlink{RC_Monitor_RC_mstat_utility}{mstat}, a simple monitoring task, and \hyperlink{RC_Monitor_RC_mdump_utility}{mdump} which can dump the raw data. \par


Various keys in the ODB can be customized for Run Control. This involves creating and/or editing keys using odbedit or mhttpd. Where the customization is only relevent to mhttpd, it will be described in the appropriate mhttpd section.

The odbedit and mhttpd utilities will be described next.


\begin{DoxyItemize}
\item \hyperlink{RC_odbedit}{odbedit: The ODB Editor and Run Control utility}
\item \hyperlink{RC_mhttpd}{mhttpd: the MIDAS Web-\/based Run Control utility} \par
 \par

\end{DoxyItemize}

\par
  \label{index_end}
\hypertarget{index_end}{}
 \subsubsection{odbedit: The ODB Editor and Run Control utility}\label{RC_odbedit}
\par
 

\par


The \hyperlink{RC_odbedit_utility}{odbedit utility} is both an \hyperlink{F_MainElements_F_Online_Database_overview}{Online Database (ODB)} Editor and a {\bfseries Run Control} program. \par
 This is one of the main applications to interact with the different components of the MIDAS system. It is a simpler alternative to the web-\/based run control program \hyperlink{RC_mhttpd_utility}{mhttpd}. There are many occasions where {\bfseries odbedit} is more convenient and faster to use than mhttpd. {\bfseries odbedit} is used to create the initial ODB, and is a powerful editing tool. It is often used for debugging and troubleshooting. \par



\begin{DoxyItemize}
\item \hyperlink{RC_odbedit_utility}{odbedit -\/ ODB Editor and run control utility}
\item \hyperlink{RC_odbedit_utility_RC_odbedit_help}{odbedit command list}
\item \hyperlink{RC_odbedit_examples}{Using odbedit}
\end{DoxyItemize}

\par
 

\par
 \label{index_end}
\hypertarget{index_end}{}
 \paragraph{odbedit -\/ ODB Editor and run control utility}\label{RC_odbedit_utility}
\label{RC_odbedit_utility_idx_odbedit-utility}
\hypertarget{RC_odbedit_utility_idx_odbedit-utility}{}
 \par
 

\par


\label{RC_odbedit_utility_idx_edit_ODB_using-odbedit}
\hypertarget{RC_odbedit_utility_idx_edit_ODB_using-odbedit}{}
 {\bfseries odbedit} is primarily an \hyperlink{F_MainElements_F_Online_Database_overview}{Online Database (ODB)} Editor. It also acts as a run control and has limited run monitoring features. It is an alternative to the web-\/based run control program \hyperlink{RC_mhttpd_utility}{The mhttpd daemon}.


\begin{DoxyItemize}
\item {\bfseries  Arguments }
\begin{DoxyItemize}
\item \mbox{[}-\/h hostname \mbox{]} :Specifies host to connect to. See \hyperlink{F_Utilities_List_F_utilities_params}{hostname} for details.
\item \mbox{[}-\/e exptname \mbox{]} :Specifies the experiment to connect to. See \hyperlink{F_Utilities_List_F_utilities_params}{experiment} for details.
\item \mbox{[}-\/c command \mbox{]} :Perform a single command
\item \mbox{[}-\/c @commandFile \mbox{]} :Perform commands in sequence found in the commandFile. Can be used to perform operations in script files. See \hyperlink{RC_odbedit_examples_RC_odbedit_extcommand}{examples}.
\item \mbox{[}-\/s size \mbox{]} : size in bytes (for \hyperlink{RC_odbedit_examples_RC_odbedit_create_ODB}{ODB creation}). Specify the size of the ODB file to be created when no shared file is present in the experiment directory (default 128KB).
\item \mbox{[}-\/d ODB Subtree\mbox{]} :Specify the initial entry ODB path to go to.
\item \mbox{[}-\/g\mbox{]} debug
\item \mbox{[}-\/C \mbox{]} connect to corrupted ODB
\end{DoxyItemize}
\end{DoxyItemize}


\begin{DoxyItemize}
\item {\bfseries  Usage } ODBedit has a simple command line interface with command line editing similar to the UNIX tcsh shell. The following edit keys are implemented:
\begin{DoxyItemize}
\item \mbox{[}Backspace\mbox{]} Erase the character left from cursor
\item \mbox{[}Delete/Ctrl-\/D\mbox{]} Erase the character under cursor
\item \mbox{[}Ctrl-\/W/Ctrl-\/U\mbox{]} Erase the current line
\item \mbox{[}Ctrl-\/K\mbox{]} Erase the line from cursor to end
\item \mbox{[}Left arrow/Ctrl-\/B\mbox{]} Move cursor left
\item \mbox{[}Right arrow/Ctrl-\/F\mbox{]} Move cursor right
\item \mbox{[}Home/Ctrl-\/A\mbox{]} Move cursor to beginning of line
\item \mbox{[}End/Ctrl-\/E\mbox{]} Move cursor to end of line
\item \mbox{[}Up arrow/Ctrl-\/P\mbox{]} Recall previous command
\item \mbox{[}Down arrow/Ctrl-\/N\mbox{]} Recall next command
\item \mbox{[}Ctrl-\/F\mbox{]} Find most recent command which starts with current line
\item \mbox{[}Tab/Ctrl-\/I\mbox{]} Complete directory. The command {\bfseries ls} /Sy $<$tab$>$ yields to {\bfseries ls} /System.
\end{DoxyItemize}
\end{DoxyItemize}


\begin{DoxyItemize}
\item {\bfseries  Remarks }
\begin{DoxyItemize}
\item ODBedit treats the hierarchical online database very much like a file system. Most commands are similar to UNIX file commands like ls, cd, chmod, ln etc. The help command displays a short description of all commands.
\end{DoxyItemize}
\end{DoxyItemize}

The odbedit commands and mode of operation are described fully in the following sections.

\label{RC_odbedit_utility_idx_odbedit-utility_command_list}
\hypertarget{RC_odbedit_utility_idx_odbedit-utility_command_list}{}
 \hypertarget{RC_odbedit_utility_RC_odbedit_help}{}\subparagraph{odbedit command list}\label{RC_odbedit_utility_RC_odbedit_help}
Running \hyperlink{RC_odbedit_utility}{odbedit} and issuing the command \char`\"{}help\char`\"{} displays the list of commands: 
\begin{DoxyCode}
$ odbedit

[local:pol:S]/>help
Database commands ([] are options, <> are placeholders):

alarm                   - reset all alarms
cd <dir>                - change current directory
chat                    - enter chat mode
chmod <mode> <key>      - change access mode of a key
                          1=read | 2=write | 4=delete
cleanup [client] [-f]   - delete hanging clients [force]
copy <src> <dest>       - copy a subtree to a new location
create <type> <key>     - create a key of a certain type
create <type> <key>[n]  - create an array of size [n]
del/rm [-l] [-f] <key>  - delete a key and its subkeys
  -l                      follow links
  -f                      force deletion without asking
exec <key>/<cmd>        - execute shell command (stored in key) on server
export <key> <filename> - export key into ASCII file
find <pattern>          - find a key with wildcard pattern
help/? [command]        - print this help [for a specific command]
hi [analyzer] [id]      - tell analyzer to clear histos
import <filename> [key] - import ASCII file into string key
ln <source> <linkname>  - create a link to <source> key
load <file>             - load database from .ODB file at current position
ls/dir [-lhvrp] [<pat>] - show database entries which match pattern
  -l                      detailed info
  -h                      hex format
  -v                      only value
  -r                      show database entries recursively
  -p                      pause between screens
make [analyzer name]    - create experim.h
mem [-v]                - show memeory usage [verbose]
mkdir <subdir>          - make new <subdir>
move <key> [top/bottom/[n]] - move key to position in keylist
msg [type] [user] <msg> - compose user message
old [n]                 - display old n messages
passwd                  - change MIDAS password
pause                   - pause current run
pwd                     - show current directory
resume                  - resume current run
rename <old> <new>      - rename key
rewind [channel]        - rewind tapes in logger
save [-c -s -x -cs] <file>  - save database at current position
                          in ASCII format
  -c                      as a C structure
  -s                      as a #define'd string
  -x                      as a XML file
set <key> <value>       - set the value of a key
set <key>[i] <value>    - set the value of index i
set <key>[*] <value>    - set the value of all indices of a key
set <key>[i..j] <value> - set the value of all indices i..j
scl [-w]                - show all active clients [with watchdog info]
shutdown <client>/all   - shutdown individual or all clients
sor                     - show open records in current subtree
start [number][now][-v] - start a run [with a specific number],
                          [now] w/o asking parameters, [-v] debug output
stop [-v]               - stop current run, [-v] debug output
trunc <key> <index>     - truncate key to [index] values
ver                     - show MIDAS library version
webpasswd               - change WWW password for mhttpd
wait <key>              - wait for key to get modified
quit/exit               - exit
\end{DoxyCode}


\par
 

\label{index_end}
\hypertarget{index_end}{}
 \paragraph{Using odbedit}\label{RC_odbedit_examples}
\label{RC_odbedit_examples_idx_odbedit-utility_examples}
\hypertarget{RC_odbedit_examples_idx_odbedit-utility_examples}{}
 \par
 


\begin{DoxyItemize}
\item \hyperlink{RC_odbedit_examples_RC_odbedit_prompt}{Setting odbedit's prompt} 
\begin{DoxyItemize}
\item \hyperlink{RC_odbedit_examples_RC_odbedit_prompt_examples}{Examples of changing the odbedit prompt:} 
\end{DoxyItemize}
\item \hyperlink{RC_odbedit_examples_RC_odbedit_create_ODB}{ODB Creation} 
\item \hyperlink{RC_odbedit_examples_RC_odbedit_keynames}{ODB Key names: UPPER/lower case, spaces in key names} 
\item \hyperlink{RC_odbedit_examples_RC_odbedit_corrupted}{Corrupted ODB} 
\item \hyperlink{RC_odbedit_examples_RC_odbedit_extcommand}{Using the external command (the -\/c argument)} 
\begin{DoxyItemize}
\item \hyperlink{RC_odbedit_examples_RC_odbedit_script_examples}{Examples of scripts sending odbedit commands} 
\begin{DoxyItemize}
\item \hyperlink{RC_odbedit_examples_RC_example_script_1}{Shell script run at end-\/of-\/run} 
\item \hyperlink{RC_odbedit_examples_RC_example_script_2}{Shell script run at beginning of run} 
\end{DoxyItemize}
\end{DoxyItemize}
\item \hyperlink{RC_odbedit_examples_RC_odbedit_cmd_examples}{Examples using odbedit commands} 
\begin{DoxyItemize}
\item \hyperlink{RC_odbedit_examples_RC_odbedit_cd}{cd -\/ change current directory} 
\item \hyperlink{RC_odbedit_examples_RC_odbedit_chat}{chat -\/ enter chat mode} 
\item \hyperlink{RC_odbedit_examples_RC_odbedit_chmod}{chmod -\/ change access mode} 
\item \hyperlink{RC_odbedit_examples_RC_odbedit_cr}{create -\/ create a key of a certain type} 
\item \hyperlink{RC_odbedit_examples_RC_odbedit_export}{export -\/ export ASCII file} 
\item \hyperlink{RC_odbedit_utility_RC_odbedit_help}{help -\/ list of commands} 
\item \hyperlink{RC_odbedit_examples_RC_odbedit_import}{import -\/ import ASCII file} 
\item \hyperlink{RC_odbedit_examples_RC_odbedit_ln}{ln -\/ create a link} 
\item \hyperlink{RC_odbedit_examples_RC_odbedit_load}{load -\/ load database from a saved file} 
\item \hyperlink{RC_odbedit_examples_RC_odbedit_ls}{ls -\/ list the database entries} 
\item \hyperlink{RC_odbedit_examples_RC_odbedit_make}{make -\/ create experim.h} 
\item \hyperlink{RC_odbedit_examples_RC_odbedit_mkdir}{mkdir -\/ make new subdirectory} 
\item \hyperlink{RC_odbedit_examples_RC_odbedit_move}{move -\/ move a key to a new position} 
\item \hyperlink{RC_odbedit_examples_RC_odbedit_msg}{msg -\/ send a user message} 
\item \hyperlink{RC_odbedit_examples_RC_odbedit_old}{old -\/ display old messages} 
\item \hyperlink{RC_odbedit_examples_RC_odbedit_passwd}{passwd -\/ change/set up the MIDAS password} 
\item \hyperlink{RC_odbedit_examples_RC_odbedit_pwd}{pwd -\/ show current directory} 
\item \hyperlink{RC_odbedit_examples_RC_odbedit_rename}{rename -\/ rename a key} 
\item \hyperlink{RC_odbedit_examples_RC_odbedit_rm}{rm/del -\/ delete a key and its subkeys} 
\item \hyperlink{RC_odbedit_examples_RC_odbedit_set}{set -\/ set the value of a key} 
\item \hyperlink{RC_odbedit_examples_RC_odbedit_sor}{sor -\/ show open records} 
\item \hyperlink{RC_odbedit_examples_RC_odbedit_save}{save -\/ save database at current position} 
\item \hyperlink{RC_odbedit_examples_RC_odbedit_scl}{scl -\/ show active clients} 
\item \hyperlink{RC_odbedit_examples_RC_odbedit_sh}{sh -\/ shutdown a client} 
\item \hyperlink{RC_odbedit_examples_RC_odbedit_start}{start -\/ start a run} 
\item \hyperlink{RC_odbedit_examples_RC_odbedit_stop}{stop -\/ stop a run} 
\item \hyperlink{RC_odbedit_examples_RC_odbedit_trunc}{trunc -\/ truncate a key} 
\item \hyperlink{RC_odbedit_examples_RC_odbedit_webpasswd}{webpasswd -\/ change/set up the web password for mhttpd} 
\end{DoxyItemize}
\end{DoxyItemize}



\hypertarget{RC_odbedit_examples_RC_odbedit_prompt}{}\subparagraph{Setting odbedit's prompt}\label{RC_odbedit_examples_RC_odbedit_prompt}
When \char`\"{}odbedit\char`\"{} is entered on the command line, it returns a prompt, e.g. 
\begin{DoxyCode}
odbedit
[local:midas:S]/>
\end{DoxyCode}
 The format of the {\bfseries prompt} (in the above example {\bfseries  \mbox{[}local:midas:S\mbox{]}/$>$ } ) is controlled in the ODB by the key {\bfseries  /System/Prompt }. The default value is shown below: 
\begin{DoxyCode}
odbedit
[local:midas:S]/>cd /System/
[local:midas:S]/System>ls
Clients                         
Tmp                             
Client Notify                   0
Prompt                          [%h:%e:%s]%p>
\end{DoxyCode}
 \par
 where the meanings of the Prompt symbols are shown below:

\begin{table}[h]\begin{TabularC}{2}
\hline
{\bfseries Symbol}  &{\bfseries Substitute}   \\\cline{1-2}
\%{\bfseries h} &Host name  \\\cline{1-2}
\%{\bfseries e}  &Experiment name  \\\cline{1-2}
\%{\bfseries s}  &Run state symbols (U,S,P,R)  \\\cline{1-2}
\%{\bfseries S}  &Run state long form (Unknown,Stopped,Pause,Running)  \\\cline{1-2}
\%{\bfseries p}  &Current ODB Path  \\\cline{1-2}
\%{\bfseries t}  &Current time  \\\cline{1-2}
\end{TabularC}
\centering
\caption{Above: Meaning of Prompt symbols }
\end{table}
\hypertarget{RC_odbedit_examples_RC_odbedit_prompt_examples}{}\subparagraph{Examples of changing the odbedit prompt:}\label{RC_odbedit_examples_RC_odbedit_prompt_examples}

\begin{DoxyEnumerate}
\item Set the prompt to show the {\bfseries  long-\/form of the run state } 
\begin{DoxyCode}
  [local:midas:S]/System>set Prompt "[%h:%e:%S]%p>"
  [local:midas:Stopped]/System>
\end{DoxyCode}
 
\item Set the prompt to the phrase {\bfseries my\_\-prompt} 
\begin{DoxyCode}
    [local:midas:Stopped]/System>set Prompt my_prompt>
    my_prompt>
\end{DoxyCode}



\item Set the prompt to {\bfseries  name the fields } (i.e. Host, Expt, State, Path) 
\begin{DoxyCode}
    my_prompt>set Prompt [Host:%h-Expt:%e:State:%s]Path:%p>
    [Host:local-Expt:midas-State:S]Path:/System>
\end{DoxyCode}



\item Set the prompt to the {\bfseries current} {\bfseries time} 
\begin{DoxyCode}
    [Host:local-Expt:midas-State:S]Path:/System>set Prompt "%t>"
    13:29:08>
\end{DoxyCode}






\label{RC_odbedit_examples_idx_ODB_create}
\hypertarget{RC_odbedit_examples_idx_ODB_create}{}
 
\end{DoxyEnumerate}\hypertarget{RC_odbedit_examples_RC_odbedit_create_ODB}{}\subparagraph{ODB Creation}\label{RC_odbedit_examples_RC_odbedit_create_ODB}
After installation of MIDAS, before any other tasks are started, the ODB is created for the first time by starting the \hyperlink{RC_odbedit_utility}{odbedit utility}. This automatically creates all the shared-\/memory files needed for the experiment. By default, these files will be created in the area indicated in the \hyperlink{Q_Linux_Q_Linux_Exptab}{exptab file} for your experiment. \par
 If MIDAS\_\-EXPT\_\-NAME is defined, this experiment will be used, unless superceded by the -\/e option (see \hyperlink{RC_odbedit_utility}{odbedit}). \par
 
\begin{DoxyCode}
[mpet@titan01 ~/online]$ ls .*.SHM
ls: No match.
[mpet@titan01 ~/online] odbedit
[local:mpet:Stopped]/>quit
[mpet@titan01 ~/online]$ ls .*.SHM
.ALARM.SHM  .ELOG.SHM  .HISTORY.SHM  .MSG.SHM  .ODB.SHM  .SYSMSG.SHM  .SYSTEM.SHM
      
\end{DoxyCode}


The default size of the ODB is 128KB. A different size can be specified by using the -\/s option (see \hyperlink{RC_odbedit_utility}{odbedit utility}) e.g. 
\begin{DoxyCode}
odbedit -s 204000
\end{DoxyCode}


The other shared memory files created are the system buffer .SYSTEM.SHM, the \hyperlink{F_Messaging}{system messaging} buffer .SYSMSG.SHM, the message buffer .MSG.SHM, \begin{Desc}
\item[\hyperlink{todo__todo000016}{Todo}]( MSG.SHM what for? )\end{Desc}
the \hyperlink{F_History_logging}{history} buffer .HISTORY.SHM, the \hyperlink{F_Elog}{Elog} buffer .ELOG.SHM, the alarm buffer .ALARM.SHM . \par


{\bfseries Note:} \par
to change the size of the event buffer(s) (e.g. SYSTEM buffer) see \hyperlink{FE_event_buffer_size}{Increase the Event Buffer Size(s)} . \par
  Running odbedit for the first time creates the trees /Runinfo, /Experiment, /System in the ODB.  Each application will then add its own set of \hyperlink{structparameters}{parameters} to the database depending on its requirements. For example, starting the \hyperlink{F_Logging_F_mlogger_utility}{MIDAS logger} will cause the tree /Logger to be created.



 \par
 \label{RC_odbedit_examples_idx_ODB_key_names}
\hypertarget{RC_odbedit_examples_idx_ODB_key_names}{}
 \hypertarget{RC_odbedit_examples_RC_odbedit_keynames}{}\subparagraph{ODB Key names: UPPER/lower case, spaces in key names}\label{RC_odbedit_examples_RC_odbedit_keynames}
{\bfseries ODB Key names are case-\/independent,} 
\begin{DoxyCode}
[mpet@titan01 ~/online] odbedit
[local:mpet:Stopped]/>ls
PerlRC
[local:mpet:Stopped]/>ls perlrc
ControlVariables
RunControl
Tunes
[local:mpet:Stopped]/>ls PERLRC
ControlVariables
RunControl
Tunes
\end{DoxyCode}
 {\bfseries Key names containing spaces must be enclosed in quotes} 
\begin{DoxyCode}
[local:mpet:Stopped]/>ls "/Equipment/TITAN_ACQ/ppg cycle/"
transition_HV
stdpulse_START
begin_scan
stdpulse_3
\end{DoxyCode}
 If the quotes are omitted 
\begin{DoxyCode}
[local:mpet:Stopped]/>ls /Equipment/TITAN_ACQ/ppg cycle
key /Equipment/TITAN_ACQ/ppg not found
\end{DoxyCode}
 Using {\bfseries TAB completion,} one could write 
\begin{DoxyCode}
[local:mpet:Stopped]/>ls /Equipment/TITAN_ACQ/ppg  
\end{DoxyCode}
 then pressing the TAB key would replace the line above with that below 
\begin{DoxyCode}
[local:mpet:Stopped]/>ls "/Equipment/TITAN_ACQ/ppg cycle/ 
\end{DoxyCode}
 then press ENTER key to see the list 
\begin{DoxyCode}
transition_HV
stdpulse_START
begin_scan
stdpulse_3
\end{DoxyCode}


\label{RC_odbedit_examples_idx_ODB_corrupted}
\hypertarget{RC_odbedit_examples_idx_ODB_corrupted}{}
 

\hypertarget{RC_odbedit_examples_RC_odbedit_corrupted}{}\subparagraph{Corrupted ODB}\label{RC_odbedit_examples_RC_odbedit_corrupted}
If the \hyperlink{F_MainElements_F_Online_Database_overview}{Online Database} becomes corrupted, \hyperlink{RC_odbedit_utility}{odbedit} may no longer work, and other clients will also fail to open the database. In this case, the old ODB should be deleted and a new one created. The contents of the ODB can be reloaded from a \hyperlink{RC_odbedit_examples_RC_odbedit_save}{saved file}. Since the ODB may become corrupted, it is advisable to \hyperlink{F_Logging_Data_F_Logger_ODB_Dump}{save a copy automatically} at the end of each run. \par
 To delete the corrupted ODB, delete the $\ast$.SHM files created in the area indicated in the \hyperlink{Q_Linux_Q_Linux_Exptab}{exptab file} for your experiment. 
\begin{DoxyCode}
[mpet@titan01 ~/online]$ ls .*.SHM
.ALARM.SHM  .ELOG.SHM  .HISTORY.SHM  .MSG.SHM  .ODB.SHM  .SYSMSG.SHM  .SYSTEM.SHM
      
[mpet@titan01 ~/online]$ rm .*.SHM
\end{DoxyCode}
 Create new $\ast$.SHM files by running odbedit (see \hyperlink{RC_odbedit_examples_RC_odbedit_create_ODB}{ODB Creation}), then load a \hyperlink{RC_odbedit_examples_RC_odbedit_save}{saved file} containing the latest copy of the odb contents. 
\begin{DoxyCode}
[mpet@titan01 ~/online] odbedit
[local:mpet:Stopped]/>load mpet.odb
\end{DoxyCode}


\par
 

 \par


\label{RC_odbedit_examples_idx_script_odbedit}
\hypertarget{RC_odbedit_examples_idx_script_odbedit}{}
 \label{RC_odbedit_examples_idx_odbedit_scripts}
\hypertarget{RC_odbedit_examples_idx_odbedit_scripts}{}
\hypertarget{RC_odbedit_examples_RC_odbedit_extcommand}{}\subparagraph{Using the external command (the  -\/c argument)}\label{RC_odbedit_examples_RC_odbedit_extcommand}
\hyperlink{RC_odbedit_utility}{odbedit -\/ ODB Editor and run control utility} -\/c argument In the simplest case, a single odbedit command can be entered on the command line, 
\begin{DoxyCode}
[pol@isdaq01 src]$ odbedit -c start
Starting run #401
Run #401 started
[pol@isdaq01 src]$ odbedit -c stop
Run #401 stopped
\end{DoxyCode}
 or a value can be set (note the use of the \hyperlink{RC_odbedit_utility}{odbedit} {\bfseries -\/d} argument) 
\begin{DoxyCode}
[pol@isdaq01 pol]$ odbedit -d /test -c "set testval 3"
[pol@isdaq01 pol]$ odb
[local:pol:S]/>ls test
testval                         3

[pol@isdaq01 pol]$ odbedit -d /test -c "ls testval"
testval 
\end{DoxyCode}


Note that the syntax to create an ODB STRING array using the \char`\"{}-\/c\char`\"{} command is 
\begin{DoxyCode}
odbedit -c "create STRING Test[1][40]"
odbedit -c "create STRING Test[8][40]"
\end{DoxyCode}


A filename containing a number of odbedit commands can also be entered, using the \hyperlink{RC_odbedit_utility}{odbedit} -\/c @commandfile argument e.g. 
\begin{DoxyCode}
[pol@isdaq01 pol]$ odbedit -d /test <b> -c @testfile.com </b>
testval                         4
Starting run #403
Run #403 started
Run #403 paused
Run #403 stopped
[pol@isdaq01 pol]$ 
\end{DoxyCode}
 where the file \char`\"{}testfile\char`\"{} contains odbedit commands, such as 
\begin{DoxyCode}
set testval 4
ls testval
start
stop
\end{DoxyCode}
 This external command feature of odbedit allows for sophisticated scripts to be created that can manipulate the odb. \par
Such scripts can for example
\begin{DoxyItemize}
\item check ODB \hyperlink{structparameters}{parameters} prior to beginning of run
\item send run \hyperlink{structparameters}{parameters} to the electronic logbook
\item act as a run controller, starting and stopping a series of runs with varying \hyperlink{structparameters}{parameters}
\end{DoxyItemize}

Some examples are shown below.\hypertarget{RC_odbedit_examples_RC_odbedit_script_examples}{}\subparagraph{Examples of scripts sending odbedit commands}\label{RC_odbedit_examples_RC_odbedit_script_examples}

\begin{DoxyItemize}
\item \hyperlink{RC_odbedit_examples_RC_example_script_1}{Shell script run at end-\/of-\/run}
\item \hyperlink{RC_odbedit_examples_RC_example_script_2}{Shell script run at beginning of run}
\end{DoxyItemize}

See also \hyperlink{RC_mhttpd_defining_script_buttons_RC_odb_script_ex2_perlscript}{MPET perlscripts to perform run control} .

\label{RC_odbedit_examples_idx_script_end-of-run}
\hypertarget{RC_odbedit_examples_idx_script_end-of-run}{}
 \hypertarget{RC_odbedit_examples_RC_example_script_1}{}\subparagraph{Shell script run at end-\/of-\/run}\label{RC_odbedit_examples_RC_example_script_1}
This script runs at the end of run, and reads some \hyperlink{structparameters}{parameters} from the odb and sends them to the elog by using the \hyperlink{F_Elog_F_melog_utility}{melog -\/ submits an entry to the Elog}. To make the script run at end of run, the name of the script is entered in the \char`\"{}Execute on stop run\char`\"{} key in the \hyperlink{RC_customize_ODB_RC_ODB_Programs_Tree}{The ODB /Programs tree} .


\begin{DoxyCode}
#!/bin/tcsh

# This script is started at the end of each run. It takes some parameters
# from the odb and creates an entry in the elog 
#
# Match to at_start_run.csh script
# Check for input files
if ($#argv == 1) then
  if (-e $1) then
    echo "Processing from file"
    set cmd = `echo 'load '$1`
    odb -e $MIDAS_EXPT_NAME -c "$cmd"
  endif
endif
echo "exp:   $MIDAS_EXPT_NAME"

# This is the file where the elog entry is saved temporarily
set fin = "/home/$MIDAS_EXPT_NAME/tmp/info_for_elog.txt"
if (-e $fin) then
  rm -f $fin
endif
touch $fin

# set port for mhttpd
set port='8080'

# Start collecting information from ODB first
set Run_number = `odb -e $MIDAS_EXPT_NAME -c 'ls "/Runinfo/Run number"'`
set number = `echo $Run_number | awk '{print $3}'`
set sample = `odb -e $MIDAS_EXPT_NAME -c 'ls "/Experiment/Edit on Start/sample"'`
      
set Sample = `echo $sample | awk '{print $2}'`
set temperature = `odb -e $MIDAS_EXPT_NAME -c 'ls "/Experiment/Edit on Start/temp
      erature"'`
set T = `echo $temperature | awk '{print $2}'`
set field = `odb -e $MIDAS_EXPT_NAME -c 'ls "/Experiment/Edit on Start/field"'`
set H = `echo $field | awk '{print $2}'`
set RF = '??'
set experimenter = `odb -e $MIDAS_EXPT_NAME -c 'ls "/Experiment/Edit on Start/exp
      erimenter"'`
set author = `echo $experimenter | awk '{print $2}'`
set run_title = `odb -e $MIDAS_EXPT_NAME -c 'ls "/Experiment/Edit on Start/run_ti
      tle"'`
set title = `echo $run_title | awk -F'run_title' '{print $2}'`
set experiment_number = `odb -e $MIDAS_EXPT_NAME -c 'ls "/Experiment/Edit on Star
      t/experiment number"'`
set exp_num = `echo $experiment_number | awk '{print $3}'`
set Experiment_name = `odb -e $MIDAS_EXPT_NAME -c 'ls "/Equipment/FIFO_acq/sis mc
      s/Input/Experiment name"'`
set type = `echo $Experiment_name | awk '{print $3}'`
set type_dir = `echo 'ls -r /PPG/PPG'$type`

# Now create the temporary file to be sent to the elog
echo "Run # $number" >> $fin
odb -e $MIDAS_EXPT_NAME -c 'ls "/Runinfo/Start time"' >> $fin
odb -e $MIDAS_EXPT_NAME -c 'ls "/Runinfo/Stop time"' >> $fin
echo "$Sample at T = $T K, H = $H T and RF = $RF mW">> $fin
echo "Run Title   : $title" >> $fin
echo "Experimenter: $author" >> $fin
echo "Experiment #: $exp_num" >> $fin
echo "-------------------------------------------------------------" >>$fin
odb -e $MIDAS_EXPT_NAME -c "$type_dir" >> $fin
echo "-------------------------------------------------------------" >>$fin

if ("x$Sample" == "x") then 
   set Sample = 'none'
endif

if ("x$author" == "x") then 
   set author = 'Auto'
endif

# Send information to the elog

echo "about to send elog (expt $MIDAS_EXPT_NAME, port $port)"
melog -h isdaq01 -p $port -l $MIDAS_EXPT_NAME -a author=$author -a Type="Automati
      c Elog" -a System="Elog" -a Subject="$Sample"  -m $fin
cat $fin

# done
\end{DoxyCode}
 \label{RC_odbedit_examples_idx_script_start-of-run}
\hypertarget{RC_odbedit_examples_idx_script_start-of-run}{}
 

\hypertarget{RC_odbedit_examples_RC_example_script_2}{}\subparagraph{Shell script run at beginning of run}\label{RC_odbedit_examples_RC_example_script_2}
The following example is part of a shell script run at the beginning of run for the TRIUMF BNMR experiment to check the status of various slow controls required for logging during the run. To make the script run at beginning of run, the name of the script is entered in the \char`\"{}Execute on start run\char`\"{} key in the \hyperlink{RC_customize_ODB_RC_ODB_Programs_Tree}{The ODB /Programs tree} .


\begin{DoxyCode}
#!/bin/csh
#

# Add an input parameter
#  0 default
#  1 (for redo camp slow controls from custom page) to stop statistics being zero
      ed
#  2 (redo epics slow controls from custom page)
#
#
# NOTE: msg [type] [user] <msg> - compose user message
#
#  odb msg 1 -> error msg  (in black/red)
#  odb msg 2 -> info msg   (in black/white)

#echo "argv: $argv ; number of args: $#argv"
set param = 0
if  ($#argv > 0) then
    set param = $argv[1];
   endif
#echo "param: $param ;  $MIDAS_EXPT_NAME"

set my_path = "/home/$MIDAS_EXPT_NAME/online/$MIDAS_EXPT_NAME/bin"
#echo "my_path:$my_path"

odb -e $MIDAS_EXPT_NAME -c "msg 'at_start_run'  '(at start) starting with param= 
      $param' "

# clear elog (camp log) alarm flag
odb -e $MIDAS_EXPT_NAME -c 'set "/equipment/fifo_acq/client flags/elog alarm" 0'
# clear epicslog alarm flag
odb -e $MIDAS_EXPT_NAME -c 'set "/equipment/fifo_acq/client flags/epicslog alarm"
       0'

if ($param == 1) then
# redo CAMP only
   odb -e $MIDAS_EXPT_NAME -c "msg 'at_start_run'  '(at start) sets camp ok to 4'
      "
   odb -e $MIDAS_EXPT_NAME -c "set '/equipment/camp/settings/camp ok' 4" # camp_o
      k = 4 indicates redoing CAMP only
else if  ($param == 2) then
# redo EPICS only
   odb -e $MIDAS_EXPT_NAME -c "msg 'at_start_run'  '(at start) sets epics ok to4'
      "
   odb -e $MIDAS_EXPT_NAME -c "set '/equipment/epicslog/settings/epics ok' 4" #ep
      ics_ok =4 indicates redoing EPICS only
else
# run start - redo EPICS and CAMP
   odb -e $MIDAS_EXPT_NAME  -c "msg 'at_start_run'  '(at start) sets epics ok and
       camp ok to 2' "
   odb -e $MIDAS_EXPT_NAME -c "set '/equipment/camp/settings/camp ok' 2"      #ca
      mp_ok=2 indicates redoing CAMP
   odb -e $MIDAS_EXPT_NAME -c "set '/equipment/epicslog/settings/epics ok' 2" #ep
      ics_ok=2 indicates redoing EPICS
endif

if ($param == 0 || $param == 2) then
#
#  Check epics logged variables
#
   odb -e $MIDAS_EXPT_NAME -c "msg 'at_start_run' 'calling check_epics.csh' "
   $my_path/check_epics.csh
    set stat = $status
    if ($stat != 0) then
       echo "error return from check_epics.csh"
       odb -e $MIDAS_EXPT_NAME -c "msg '1' 'at_start_run' 'error return from chec
      k_epics.csh' "
       odb -e $MIDAS_EXPT_NAME -c "set '/equipment/epicslog/settings/epics ok' 0
" # EPICS failure
    else
       odb -e $MIDAS_EXPT_NAME -c "msg 'at_start_run' 'after check_epics.csh (suc
      cess)' "
       odb -e $MIDAS_EXPT_NAME -c "set '/equipment/epicslog/settings/epics ok' 1"
       # EPICS success
    endif

else
   odb -e $MIDAS_EXPT_NAME -c "msg 'at_start_run' '(at start) NOT calling check_e
      pics.csh ($param)' "
endif

if ($param == 2) then  # checks epics only ; no camp
  exit
endif

 ..............
    etc.
\end{DoxyCode}




\hypertarget{RC_odbedit_examples_RC_odbedit_cmd_examples}{}\subparagraph{Examples using odbedit commands}\label{RC_odbedit_examples_RC_odbedit_cmd_examples}
Here are some examples of the most commonly used \hyperlink{RC_odbedit_utility}{odbedit} commands:\hypertarget{RC_odbedit_examples_RC_odbedit_pwd}{}\subparagraph{pwd -\/ show current directory}\label{RC_odbedit_examples_RC_odbedit_pwd}

\footnotesize  One of the \hyperlink{RC_odbedit_utility}{odbedit} \hyperlink{RC_odbedit_utility_RC_odbedit_help}{commands} 
\normalsize \par
\par
 
\begin{DoxyCode}
$ odbedit
[local:mpet:Stopped]/>
[local:mpet:Stopped]/>pwd
/
\end{DoxyCode}




\hypertarget{RC_odbedit_examples_RC_odbedit_cd}{}\subparagraph{cd -\/ change current directory}\label{RC_odbedit_examples_RC_odbedit_cd}

\footnotesize  One of the \hyperlink{RC_odbedit_utility}{odbedit} \hyperlink{RC_odbedit_utility_RC_odbedit_help}{commands} 
\normalsize \par
\par



\begin{DoxyCode}
cd <dir>                - change current directory
\end{DoxyCode}
 For example, 
\begin{DoxyCode}
[local:mpet:Stopped]/>cd system
[local:mpet:Stopped]/>pwd
/System
\end{DoxyCode}




\hypertarget{RC_odbedit_examples_RC_odbedit_chat}{}\subparagraph{chat  -\/ enter chat mode}\label{RC_odbedit_examples_RC_odbedit_chat}
This mode is used to communicate with another person also running odbedit on the same experiment. It is useful where a telephone connection is not available. e.g. \par
 \begin{table}[h]\begin{TabularC}{2}
\hline
Anna's console &Fred's console \\\cline{1-2}

\begin{DoxyCode}
[anna@isdaq01 ~]$ odb -e bnmr -h dasdevpc
[dasdevpc:bnmr:S]/>chat
Your name> anna
Exit chat mode with empty line.
> hi
12:46:12 [anna] hi
12:46:51 [ODBEdit2] Program ODBEdit on host dasdevpc started
12:47:21 [fred] hi
> hi
12:47:34 [anna] hi fred
12:48:12 [fred] please cycle crate power now
> done - all lights green
12:48:21 [anna] done - all lights green
> bye
12:50:27 [anna] bye
>
[dasdevpc:bnmr:S]/>
\end{DoxyCode}
  &
\begin{DoxyCode}
odbedit -e bnmr
[fred@dasdevpc ~]$ odb -e bnmr
[local:bnmr:S]/>chat
Your name> fred
Exit chat mode with empty line.
> hi
12:47:21 [fred] hi
12:47:34 [anna] hi fred
> please cycle crate power now
12:48:12 [fred] please cycle crate power now
12:48:21 [anna] done - all lights green
12:50:27 [anna] bye
>
[local:bnmr:S]/>  
\end{DoxyCode}
   \\\cline{1-2}
\end{TabularC}
\centering
\caption{Two users communicate using {\bfseries chat} mode}
\end{table}


The chat conversation can also be heard over the speakers if \hyperlink{F_Messaging_F_mspeaker_utility}{m\mbox{[}lx\mbox{]}speaker -\/ audible messaging} is running.

\label{RC_odbedit_examples_idx_access-control_ODB_keys}
\hypertarget{RC_odbedit_examples_idx_access-control_ODB_keys}{}
 

\hypertarget{RC_odbedit_examples_RC_odbedit_chmod}{}\subparagraph{chmod -\/ change access mode}\label{RC_odbedit_examples_RC_odbedit_chmod}
The access mode can be changed with the chmod command between


\begin{DoxyItemize}
\item 1 read R
\item 2 write W
\item 4 delete D
\end{DoxyItemize}

By default the access mode of ODB keys are RWD i.e. the permission is to Read, Write or Delete them. To avoid them being inadvertently changed or deleted the mode can be set to read-\/only, e.g 
\begin{DoxyCode}
[local:t2kgas:S]GasMain.gif>chmod 1 background
Are you sure to change the mode of key
  /Custom/Images/GasMain.gif/background
and all its subkeys
to mode [R]? (y/[n]) y
[local:t2kgas:S]GasMain.gif>ls -lt
Key name                        Type    #Val  Size  Last Opn Mode Value
---------------------------------------------------------------------------
refresh time (s)                DWORD   1     4     13h  0   RWD  10
background                      STRING  1     256   13h  0   R    /home/suz/onlin
      e/t2kgas/images/GasMain.gif
labels                          DIR
fills                           DIR
\end{DoxyCode}
 After setting the mode to read-\/only, the key cannot be written to: 
\begin{DoxyCode}
[local:t2kgas:S]GasMain.gif>set background dddd
Write access not allowed
\end{DoxyCode}
 or deleted 
\begin{DoxyCode}
[local:t2kgas:S]GasMain.gif>rm background
Are you sure to delete the key
"/Custom/Images/GasMain.gif/background"
(y/[n]) y
deletion of key not allowed
[local:t2kgas:S]GasMain.gif> 
\end{DoxyCode}


To restore the key to mode RWD, 
\begin{DoxyCode}
[local:t2kgas:S]GasMain.gif>chmod 7 background
Are you sure to change the mode of key
  /Custom/Images/GasMain.gif/background
and all its subkeys
to mode [RWD]? (y/[n]) y
[local:t2kgas:S]GasMain.gif>ls -lt
Key name                        Type    #Val  Size  Last Opn Mode Value
---------------------------------------------------------------------------
refresh time (s)                DWORD   1     4     13h  0   RWD  10
background                      STRING  1     256   13h  0   RWD  /home/suz/onlin
      e/t2kgas/images/GasMain.gif
labels                          DIR
fills                           DIR
[local:t2kgas:S]GasMain.gif>
\end{DoxyCode}




\hypertarget{RC_odbedit_examples_RC_odbedit_ls}{}\subparagraph{ls -\/ list the database entries}\label{RC_odbedit_examples_RC_odbedit_ls}

\begin{DoxyCode}
ls/dir [-lhvrp] [<pat>] - show database entries which match pattern
  -l                      detailed info
  -h                      hex format
  -v                      only value
  -r                      show database entries recursively
  -p                      pause between screens
\end{DoxyCode}
 \par
List the keys (\char`\"{}dir\char`\"{} is an alternative to \char`\"{}ls\char`\"{}). 
\begin{DoxyCode}
[local:mpet:Stopped]/System>ls
Clients
Tmp
Client Notify                   0
Prompt                          [%h:%e:%S]%p>
[local:mpet:Stopped]/System>     
\end{DoxyCode}
 \par
The \char`\"{}-\/l\char`\"{} option gives detailed information, such as the key type and size 
\begin{DoxyCode}
[local:mpet:Stopped]/>ls /experiment
Name                            mpet
Buffer sizes
Variables
Edit on start
[local:mpet:Stopped]/>ls -lt  /experiment
Key name                        Type    #Val  Size  Last Opn Mode Value
---------------------------------------------------------------------------
Name                            STRING  1     32    3m   0   RWD  mpet
Buffer sizes                    DIR
Variables                       DIR
Edit on start                   DIR
\end{DoxyCode}
 \par
The recursive option \char`\"{}-\/r\char`\"{} shows the database entries recursively 
\begin{DoxyCode}
[local:pol:S]/>ls -r ppg
PPG
    PPGcommon
        Experiment name -> /Equipment/FIFO_acq/sis mcs/Input/Experiment name
                                1h
        CFG path -> /Equipment/FIFO_acq/sis mcs/Input/CFG path
                                /home/pol/online/pol/ppcobj
        PPG path -> /Equipment/FIFO_acq/sis mcs/Input/PPG path
                                /home/pol/online/ppg-templates
        Time slice (ms) -> /Equipment/FIFO_acq/sis mcs/Input/Time slice (ms)
                                1e-04
        Minimal delay (ms) -> /Equipment/FIFO_acq/sis mcs/Input/Minimal delay (ms
      )
                                0.0005
        DAQ service time (ms) -> /Equipment/FIFO_acq/sis mcs/Input/DAQ service ti
      me (ms)
                                3000
\end{DoxyCode}
 \par
The values can be displayed in hexadecimal using the \char`\"{}-\/h\char`\"{} option 
\begin{DoxyCode}
[local:pol:S]/>ls  "/Equipment/FIFO_acq/sis mcs/Input/e00 rf frequency (Hz)"
e00 rf frequency (Hz)           22064585
[local:pol:S]/>ls  -h "/Equipment/FIFO_acq/sis mcs/Input/e00 rf frequency (Hz)"
e00 rf frequency (Hz)           0x150ADC9
\end{DoxyCode}
 \par
or the value only displayed with the \char`\"{}-\/v\char`\"{} option 
\begin{DoxyCode}
[local:pol:S]/test>ls
val                             5
[local:pol:S]/test>ls -v
5
\end{DoxyCode}




\hypertarget{RC_odbedit_examples_RC_odbedit_mkdir}{}\subparagraph{mkdir -\/ make new subdirectory}\label{RC_odbedit_examples_RC_odbedit_mkdir}

\begin{DoxyCode}
mkdir <subdir>          - make new <subdir>
\end{DoxyCode}
 \par
This example shows how to make a new subdirectory \char`\"{}/custom\char`\"{}. 
\begin{DoxyCode}
[mpet@titan01 ~/online] odbedit
[local:mpet:Stopped]/>ls
System
Programs
Experiment
Runinfo
Alarms

[local:mpet:Stopped]/>mkdir custom
[local:mpet:Stopped]/>ls
System
Programs
Experiment
Runinfo
Alarms
Custom
\end{DoxyCode}


More than one level of subdirectory can be made with one command 
\begin{DoxyCode}
[local:pol:S]/>mkdir /Equipment/test/settings
[local:pol:S]/>ls /Equipment
test
[local:pol:S]/>ls /Equipment/test
[local:pol:S]/>
settings
\end{DoxyCode}




\hypertarget{RC_odbedit_examples_RC_odbedit_msg}{}\subparagraph{msg -\/ send a user message}\label{RC_odbedit_examples_RC_odbedit_msg}
This command can be used to send a message to any client that is receiving MIDAS messages, including the MIDAS message logger, e.g 
\begin{DoxyCode}
[local:cg:S]/>
16:30:48 [fred] hi
[local:cg:S]/>msg george "is there a problem?"
[george,USER] is there a problem?
16:30:48 [fred] power supply has failed
\end{DoxyCode}


The messages have gone into the MIDAS log (see \hyperlink{RC_odbedit_examples_RC_odbedit_old}{below}).



\hypertarget{RC_odbedit_examples_RC_odbedit_old}{}\subparagraph{old -\/ display old messages}\label{RC_odbedit_examples_RC_odbedit_old}
This command displays the last N MIDAS messages, e.g. 
\begin{DoxyCode}
[local:customgas:S]/>old 9
Fri May 21 13:36:27 2010 [ODBEdit,INFO] Program ODBEdit on host dasdevpc2 started
      
Fri May 21 13:36:40 2010 [ODBEdit,INFO] Program ODBEdit on host dasdevpc2 stopped
      
Mon May 31 15:56:28 2010 [ODBEdit,INFO] Program ODBEdit on host dasdevpc2 started
      
Wed Jun  9 20:49:42 2010 [mhttpd] Program mhttpd on host dasdevpc2 stopped
Wed Jun  9 20:49:42 2010 [mhttpd] Program mhttpd on host dasdevpc2 started
Wed Jun 16 16:27:51 2010 [fred] hi
Wed Jun 16 16:29:15 2010 [fred] hi
Wed Jun 16 16:30:10 2010 [george] is there a problem?
Wed Jun 16 16:30:48 2010 [fred] power supply has failed
\end{DoxyCode}




\hypertarget{RC_odbedit_examples_RC_odbedit_passwd}{}\subparagraph{passwd -\/ change/set up the MIDAS password}\label{RC_odbedit_examples_RC_odbedit_passwd}
Example is shown \hyperlink{RC_customize_ODB_RC_Setup_Security}{here}.



\hypertarget{RC_odbedit_examples_RC_odbedit_webpasswd}{}\subparagraph{webpasswd -\/ change/set up the web password for mhttpd}\label{RC_odbedit_examples_RC_odbedit_webpasswd}
Example is shown \hyperlink{RC_customize_ODB_RC_Setup_Web_Security}{here}.



\hypertarget{RC_odbedit_examples_RC_odbedit_move}{}\subparagraph{move -\/ move a key to a new position}\label{RC_odbedit_examples_RC_odbedit_move}

\begin{DoxyCode}
move <key> [top/bottom/[n]] - move key to position in keylist
\end{DoxyCode}


The \char`\"{}move\char`\"{} command provides a means of re-\/ordering the keys. 
\begin{DoxyCode}
[local:mpet:Stopped]/>ls
System
Programs
Experiment
Runinfo
Alarms
Custom
\end{DoxyCode}


The key \char`\"{}custom\char`\"{} can be moved to the top (or bottom) of the list, e.g. 
\begin{DoxyCode}
[local:mpet:Stopped]/>move custom top
[local:mpet:Stopped]/>ls
Custom
System
Programs
Experiment
Runinfo
Alarms
\end{DoxyCode}


or to any position, e.g. 
\begin{DoxyCode}
[local:mpet:Stopped]/>move custom 1
[local:mpet:Stopped]/>ls
System
Custom
Programs
Experiment
Runinfo
Alarms
\end{DoxyCode}




\hypertarget{RC_odbedit_examples_RC_odbedit_rename}{}\subparagraph{rename -\/ rename a key}\label{RC_odbedit_examples_RC_odbedit_rename}

\begin{DoxyCode}
rename <old> <new>      - rename key
\end{DoxyCode}
 \par
 
\begin{DoxyCode}
[local:pol:S]/>ls "my string"
my string                       this is a test string
[local:pol:S]/>rename "my string" "your string"
[local:pol:S]/>ls
your string                       this is a test string
\end{DoxyCode}




\hypertarget{RC_odbedit_examples_RC_odbedit_copy}{}\subparagraph{copy -\/ copy a subtree}\label{RC_odbedit_examples_RC_odbedit_copy}

\begin{DoxyCode}
copy <src> <dest>       - copy a subtree to a new location
\code
To make a copy of a subtree:
\code
[local:pol:S]/>ls test
testval                         4
[local:pol:S]/>copy test test1
[local:pol:S]/>ls
test
test1
[local:pol:S]/>ls test1
testval       
\end{DoxyCode}




\hypertarget{RC_odbedit_examples_RC_odbedit_import}{}\subparagraph{import -\/  import ASCII file}\label{RC_odbedit_examples_RC_odbedit_import}

\begin{DoxyCode}
import  <filename> [key]    - import ASCII file into string key 
\end{DoxyCode}


e.g. import an \hyperlink{RC_mhttpd_Internal}{Internal custom page} into a key, 
\begin{DoxyCode}
Tue> odbedit
[local:midas:Stopped]>cd custom
[local:midas:Stopped]/Custom>import mcustom.html   <-- import an html file
  Key name: Test&  
\end{DoxyCode}




\hypertarget{RC_odbedit_examples_RC_odbedit_export}{}\subparagraph{export -\/  export ASCII file}\label{RC_odbedit_examples_RC_odbedit_export}

\begin{DoxyCode}
export  <filename> [key]    - import ASCII file into string key 
\end{DoxyCode}


e.g. export an \hyperlink{RC_mhttpd_Internal}{Internal custom page} into a key,


\begin{DoxyCode}
  [local:midas:Stopped]/>cd Custom/
  [local:midas:Stopped]/Custom>export test&
  File name: mcustom.html
  [local:midas:Stopped]/Custom>
\end{DoxyCode}




\hypertarget{RC_odbedit_examples_RC_odbedit_ln}{}\subparagraph{ln -\/ create a link}\label{RC_odbedit_examples_RC_odbedit_ln}

\begin{DoxyCode}
ln <source> <linkname>  - create a link to <source> key
\end{DoxyCode}


The \hyperlink{RC_customize_ODB_RC_Edit_On_Start}{Edit on start} area often contains links to ODB \hyperlink{structparameters}{parameters} e.g. 
\begin{DoxyCode}
[local:mpet:Stopped]/Experiment>cd "/Experiment/Edit on start/
[local:mpet:Stopped]Edit on start>ls
num ppg cycles                  /Equipment/TITAN_acq/ppg cycle/begin_scan/loop co
      unt -> 50
Pedestals run                   n
Write Data                      /Logger/Write data -> y
Capture delay (ms)              /Equipment/TITAN_acq/ppg cycle/evset_2/time offse
      t (ms) -> 0.0955
PLT pulsedown delay (ms)        /Equipment/TITAN_acq/ppg cycle/pulse_1/time offse
      t (ms) -> 0.0922
Start Frequency in MHz          /Experiment/Variables/StartFreq (MHz) -> 1.458877
      5
End Frequency in MHz            /Experiment/Variables/EndFreq (MHz) -> 1.4588815
Number of frequency steps       /Experiment/Variables/NFreq -> 41
Feedbackfilename                /Feedback/fbfilename -> /home/mpet/online/mpetfee
      dbackfnv1.txt
\end{DoxyCode}


These links were made using the \char`\"{}ln\char`\"{} command, e.g. 
\begin{DoxyCode}
[local:mpet:Stopped]Edit on start>ln  "/Equipment/TITAN_acq/ppg cycle/evset_2/tim
      e offset (ms)" "Capture delay (ms)"
\end{DoxyCode}




\hypertarget{RC_odbedit_examples_RC_odbedit_cr}{}\subparagraph{create -\/ create a key of a certain type}\label{RC_odbedit_examples_RC_odbedit_cr}

\begin{DoxyCode}
create <type> <key>     - create a key of a certain type
create <type> <key>[n]  - create an array of size [n]
\end{DoxyCode}


Keys can be created of the types supported by MIDAS i.e. \par
 INT DWORD BOOL FLOAT DOUBLE STRING


\begin{DoxyCode}
[local:pol:S]/test>create dword my_dword
[local:pol:S]/test>create int my_int
[local:pol:S]/test>create float my_float
[local:pol:S]/test>create double my_double
[local:pol:S]/test>create bool my_bool
[local:pol:S]/>create string "my string"
String length [32]: 64

[local:pol:S]/test>ls
my_dword                        0
my_int                          0
my_float                        0
my_double                       0
my_bool                         n
my_string

[local:pol:S]/test>ls -lt
Key name                        Type    #Val  Size  Last Opn Mode Value
---------------------------------------------------------------------------
my_dword                        DWORD   1     4     >99d 0   RWD  0
my_int                          INT     1     4     >99d 0   RWD  0
my_float                        FLOAT   1     4     >99d 0   RWD  0
my_double                       DOUBLE  1     8     >99d 0   RWD  0
my_bool                         BOOL    1     4     >99d 0   RWD  n
my_string                       STRING  1     62    9s   0   RWD
[local:pol:S]/test>
\end{DoxyCode}


Arrays of all these types can also be created, e.g. 
\begin{DoxyCode}
[local:pol:S]/test>create int fred[5]
[local:pol:S]/test>ls
fred
                                0
                                0
                                0
                                0
                                0
\end{DoxyCode}


The \hyperlink{RC_odbedit_examples_RC_odbedit_set}{set} command is used to assign values to the keys.



\hypertarget{RC_odbedit_examples_RC_odbedit_set}{}\subparagraph{set -\/ set the value of a key}\label{RC_odbedit_examples_RC_odbedit_set}

\begin{DoxyCode}
set <key> <value>       - set the value of a key
set <key>[i] <value>    - set the value of index i
set <key>[*] <value>    - set the value of all indices of a key
set <key>[i..j] <value> - set the value of all indices i..j
\end{DoxyCode}


After keys are \hyperlink{RC_odbedit_examples_RC_odbedit_cr}{created}, they can be assigned values with the {\bfseries set} command: 
\begin{DoxyCode}
[pol@isdaq01 src]$ odb
[local:pol:S]/>create string "my string"
String length [32]: 64
[local:pol:S]/>ls -lt "my string"
Key name                        Type    #Val  Size  Last Opn Mode Value
---------------------------------------------------------------------------
my string                       STRING  1     64    9s   0   RWD
[local:pol:S]/>set "my string" "this is a test string"
[local:pol:S]/>ls "my string"
my string                       this is a test string
[local:pol:S]/>create INT ival
[local:pol:S]/>set ival 8
[local:pol:S]/>ls -lt ival
Key name                        Type    #Val  Size  Last Opn Mode Value
---------------------------------------------------------------------------
ival                            INT     1     4     3s   0   RWD  8
[local:pol:S]/>ls ival
ival                            8
[local:pol:S]/>   
\end{DoxyCode}


Values of {\bfseries arrays} can also be set:


\begin{DoxyCode}
[local:pol:S]/test>set fred[4] 6
[local:pol:S]/test>ls
fred
                                0
                                0
                                0
                                0
                                6
[local:pol:S]/test>set fred 2
[local:pol:S]/test>ls
fred
                                2
                                0
                                0
                                0
                                6
[local:pol:S]/test>set fred[*] 5
[local:pol:S]/test>ls
fred
                                5
                                5
                                5
                                5
                                5
[local:pol:S]/test>set fred[1..3] 6
[local:pol:S]/test>ls
fred
                                5
                                6
                                6
                                6
                                5
[local:pol:S]/test>           
\end{DoxyCode}


The array can easily be expanded (see also \hyperlink{RC_odbedit_examples_RC_odbedit_trunc}{trunc}) :


\begin{DoxyCode}
[local:pol:S]/test>set fred[8] 9
[local:pol:S]/test>ls
fred
                                5
                                6
                                6
                                6
                                5
                                0
                                0
                                0
                                9
\end{DoxyCode}


\label{RC_odbedit_examples_RC_odbedit_set_wp}
\hypertarget{RC_odbedit_examples_RC_odbedit_set_wp}{}
 {\bfseries NOTE} that the \char`\"{}set\char`\"{} command may not work if the ODB parameter is {\bfseries write-\/protected}. See \hyperlink{RC_odbedit_examples_RC_odbedit_chmod}{chmod -\/ change access mode} and \hyperlink{RC_customize_ODB_RC_Lock_when_Running}{Lock when Running}.



\hypertarget{RC_odbedit_examples_RC_odbedit_chmod}{}\subparagraph{chmod -\/ change access mode}\label{RC_odbedit_examples_RC_odbedit_chmod}

\begin{DoxyCode}
chmod <mode> <key>       change access mode of a key
                          1=read | 2=write | 3=RWD | 4=delete
\end{DoxyCode}


By default, a key is created in mode 3 (i.e. RWD read/write/delete). The {\bfseries chmod} command may be used to change the protection of the key. 
\begin{DoxyCode}
[local:bnmr:S]/>create int my_test
[local:bnmr:S]/>set my_test 3
[local:bnmr:S]/>ls -lt my_test
Key name                        Type    #Val  Size  Last Opn Mode Value
---------------------------------------------------------------------------
my_test                         INT     1     4     20s  0   RWD  3
[local:bnmr:S]/>chmod 1 my_test
Are you sure to change the mode of key
  /my_test
and all its subkeys
to mode [R]? (y/[n]) y
[local:bnmr:S]/>ls -lt my_test
Key name                        Type    #Val  Size  Last Opn Mode Value
---------------------------------------------------------------------------
my_test                         INT     1     4     46s  0   R    3
[local:bnmr:S]/>set my_test 6
Write access not allowed
[local:bnmr:S]/> 
\end{DoxyCode}


{\bfseries NOTE:} \par
 \par
Write protection when running can also be performed -\/ see \hyperlink{RC_customize_ODB_RC_Lock_when_Running}{Lock when Running}.



\hypertarget{RC_odbedit_examples_RC_odbedit_trunc}{}\subparagraph{trunc  -\/ truncate a key}\label{RC_odbedit_examples_RC_odbedit_trunc}

\begin{DoxyCode}
trunc <key> <index>     - truncate key to [index] values
\end{DoxyCode}


This command is used to truncate or expand an array. 
\begin{DoxyCode}
[local:pol:S]/>ls fred 
fred
                                5
                                6
                                6
                                6
                                5
                                0
                                0
                                0
                                9
[local:pol:S]/>trunc fred 4
[local:pol:S]/>ls fred
fred
                                5
                                6
                                6
                                6
[local:pol:S]/>trunc fred 9
[local:pol:S]/>ls fred
fred
                                5
                                6
                                6
                                6
                                0
                                0
                                0
                                0
                                0
\end{DoxyCode}


\par
 

\hypertarget{RC_odbedit_examples_RC_odbedit_rm}{}\subparagraph{rm/del -\/ delete a key and its subkeys}\label{RC_odbedit_examples_RC_odbedit_rm}

\begin{DoxyCode}
del/rm [-l] [-f] <key>  - delete a key and its subkeys
  -l                      follow links
  -f                      force deletion without asking
\end{DoxyCode}
 \par



\begin{DoxyCode}
[local:pol:S]/>rm ival
Are you sure to delete the key
"/ival"
(y/[n]) y
 
[local:pol:S]/>rm test/try
Are you sure to delete the key "/test/try"
and all its subkeys? (y/[n]) y
[local:pol:S]/> 
\end{DoxyCode}


If you answer \char`\"{}n\char`\"{} the key will not be deleted.



\hypertarget{RC_odbedit_examples_RC_odbedit_sor}{}\subparagraph{sor -\/ show open records}\label{RC_odbedit_examples_RC_odbedit_sor}
This shows which records are open, i.e. \hyperlink{RC_Hot_Link_RC_Hot_Link_Intro}{hot-\/linked} . 
\begin{DoxyCode}
[local:mpet:Stopped]/>sor
/Runinfo/Requested transition open 1 times by fempet
/Equipment/Trigger/Common open 1 times by fempet
/Equipment/Trigger/Statistics open 1 times by fempet
/Equipment/Trigger/Statistics/Events per sec. open 1 times by Logger
/Equipment/Trigger/Statistics/kBytes per sec. open 1 times by Logger
/Equipment/Trigger/Settings open 1 times by fempet
/Equipment/Scaler/Common open 1 times by fempet
/Equipment/Scaler/Statistics open 1 times by fempet
/Equipment/SlowDac/Variables open 1 times by Logger
/Equipment/SlowDac/Variables/Demand open 1 times by fesdac
/Equipment/SlowDac/Common open 1 times by fesdac
/Equipment/SlowDac/Statistics open 1 times by fesdac
/Equipment/Beamline/Settings/Names open 1 times by scEpics
/Equipment/Beamline/Settings/Update Threshold Measured open 1 times by scEpics
/Equipment/Beamline/Common open 1 times by scEpics
/Equipment/Beamline/Variables open 1 times by Logger
/Equipment/Beamline/Variables/Demand open 1 times by scEpics
/Equipment/Beamline/Statistics open 1 times by scEpics
/Equipment/RF/Variables open 1 times by Logger
/Equipment/TITAN_ACQ/Common open 1 times by fempet
/Equipment/TITAN_ACQ/Statistics open 1 times by fempet
[local:mpet:Stopped]/>    
\end{DoxyCode}




\hypertarget{RC_odbedit_examples_RC_odbedit_save}{}\subparagraph{save -\/ save database at current position}\label{RC_odbedit_examples_RC_odbedit_save}

\begin{DoxyCode}
save [-c -s -x -cs] <file>  - save database at current position
                          in ASCII format
  -c                      as a C structure
  -s                      as a #define'd string
  -x                      as a XML file
\end{DoxyCode}
 Saving the database regularly is essential in case the database becomes corrupted (see \hyperlink{RC_odbedit_examples_RC_odbedit_corrupted}{Corrupted ODB}). To save the complete database into an ASCII file, 
\begin{DoxyCode}
[mpet@titan01 ~/online] odbedit
[local:mpet:Stopped]/>save mpet.odb
\end{DoxyCode}
 \par
 This example shows how to save part of the database in an {\bfseries ASCII} file 
\begin{DoxyCode}
[local:pol:S]>cd "/Equipment/Info ODB/"
[local:pol:S]Info ODB>save info.odb
\end{DoxyCode}
 {\bfseries Contents} of info.odb : 
\begin{DoxyCode}
[/Equipment/Info ODB/Common]
Event ID = WORD : 10
Trigger mask = WORD : 0
Buffer = STRING : [32] 
Type = INT : 1
Source = INT : 0
Format = STRING : [8] FIXED
Enabled = BOOL : y
Read on = INT : 273
Period = INT : 500
Event limit = DOUBLE : 0
Num subevents = DWORD : 0
Log history = INT : 0
Frontend host = STRING : [32] vwisac2
Frontend name = STRING : [32] fePOL
Frontend file name = STRING : [256] febnmr.c

[/Equipment/Info ODB/Variables]
helicity = DWORD : 0
current cycle = DWORD : 2
cancelled cycle = DWORD : 1
current scan = DWORD : 1
Ref P+ thr = DOUBLE : 0
Ref Laser thr = DOUBLE : 10138
Ref Fcup thr = DOUBLE : 0
Current P+ thr = DOUBLE : 0
Current Laser thr = DOUBLE : 10138
Current Fcup thr = DOUBLE : 0
RF state = DWORD : 0
Fluor monitor counts = DWORD : 50010
EpicsDev Set(V) = FLOAT : 0
EpicsDev Read(V) = FLOAT : 0
Campdev set = FLOAT : 0
Campdev read = FLOAT : 0
Pol DAC set = DOUBLE : 0
Pol DAC read = DOUBLE : 0
last failed thr test = DWORD : 0
cycle when last failed thr = DWORD : 0

[/Equipment/Info ODB/Statistics]
Events sent = DOUBLE : 0
Events per sec. = DOUBLE : 0
kBytes per sec. = DOUBLE : 0
\end{DoxyCode}


\begin{TabularC}{2}
\hline
{\bfseries Save as a C structure} 
\begin{DoxyCode}
[local:pol:S]Info ODB>save  -c cfile.c
\end{DoxyCode}
  &{\bfseries Save as a \#defined structure } 
\begin{DoxyCode}
[local:pol:S]Info ODB>save  -s cfile.h
\end{DoxyCode}
  

\\\cline{1-2}
{\bfseries Contents} of cfile.c : &{\bfseries Contents} of cfile.h :   \\\cline{1-2}

\begin{DoxyCode}
typedef struct {
  struct {
    WORD      event_id;
    WORD      trigger_mask;
    char      buffer[32];
    INT       type;
    INT       source;
    char      format[8];
    BOOL      enabled;
    INT       read_on;
    INT       period;
    double    event_limit;
    DWORD     num_subevents;
    INT       log_history;
    char      frontend_host[32];
    char      frontend_name[32];
    char      frontend_file_name[256];
  } common;
  struct {
    DWORD     helicity;
    DWORD     current_cycle;
    DWORD     cancelled_cycle;
    DWORD     current_scan;
    double    ref_p__thr;
    double    ref_laser_thr;
    double    ref_fcup_thr;
    double    current_p__thr;
 double    current_laser_thr;
    double    current_fcup_thr;
    DWORD     rf_state;
    DWORD     fluor_monitor_counts;
    float     epicsdev_set_v_;
    float     epicsdev_read_v_;
    float     campdev_set;
    float     campdev_read;
    double    pol_dac_set;
    double    pol_dac_read;
    DWORD     last_failed_thr_test;
    DWORD     cycle_when_last_failed_thr;
  } variables;
  struct {
    double    events_sent;
    double    events_per_sec_;
    double    kbytes_per_sec_;
  } statistics;
} INFO_ODB;
\end{DoxyCode}
 

&
\begin{DoxyCode}
#define INFO_ODB(_name) char *_name[] = {\
"[Common]",\
"Event ID = WORD : 10",\
"Trigger mask = WORD : 0",\
"Buffer = STRING : [32] ",\
"Type = INT : 1",\
"Source = INT : 0",\
"Format = STRING : [8] FIXED",\
"Enabled = BOOL : y",\
"Read on = INT : 273",\
"Period = INT : 500",\
"Event limit = DOUBLE : 0",\
"Num subevents = DWORD : 0",\
"Log history = INT : 0",\
"Frontend host = STRING : [32] vwisac2",\
"Frontend name = STRING : [32] fePOL",\
"Frontend file name = STRING : [256] febnmr.c",\
"",\
"[Variables]",\
"helicity = DWORD : 0",\
"current cycle = DWORD : 2",\
"cancelled cycle = DWORD : 1",\
"current scan = DWORD : 1",\
"Ref P+ thr = DOUBLE : 0",\
"Ref Laser thr = DOUBLE : 10138",\
"Ref Fcup thr = DOUBLE : 0",\
"Current P+ thr = DOUBLE : 0",\
"Current Laser thr = DOUBLE : 10138",\
"Current Fcup thr = DOUBLE : 0",\
"RF state = DWORD : 0",\
"Fluor monitor counts = DWORD : 50010",\
"EpicsDev Set(V) = FLOAT : 0",\
"EpicsDev Read(V) = FLOAT : 0",\
"Campdev set = FLOAT : 0",\
"Campdev read = FLOAT : 0",\
"Pol DAC set = DOUBLE : 0",\
"Pol DAC read = DOUBLE : 0",\
"last failed thr test = DWORD : 0",\
"cycle when last failed thr = DWORD : 0",\
"",\
"[Statistics]",\
"Events sent = DOUBLE : 0",\
"Events per sec. = DOUBLE : 0",\
"kBytes per sec. = DOUBLE : 0",\
"",\
NULL }
\end{DoxyCode}
  \\\cline{1-2}
\end{TabularC}


{\bfseries  Save as an XML structure } 
\begin{DoxyCode}
[local:pol:S]Info ODB>save  -x xinfo.xml
\end{DoxyCode}
 {\bfseries Contents} of xinfo.xml : 
\begin{DoxyCode}
<?xml version="1.0" encoding="ISO-8859-1"?>
<!-- created by MXML on Wed Sep 23 13:27:05 2009 -->
<odb root="/Equipment/Info ODB" filename="xinfo.xml" xmlns:xsi="http://www.w3.org
      /2001/XMLSchema-instance" xsi:noNamespaceSchemaLocation="/home/pol/packages/midas
      /odb.xsd">
  <dir name="Common">
    <key name="Event ID" type="WORD">10</key>
    <key name="Trigger mask" type="WORD">0</key>
    <key name="Buffer" type="STRING" size="32"></key>
    <key name="Type" type="INT">1</key>
    <key name="Source" type="INT">0</key>
    <key name="Format" type="STRING" size="8">FIXED</key>
    <key name="Enabled" type="BOOL">y</key>
    <key name="Read on" type="INT">273</key>
    <key name="Period" type="INT">500</key>
    <key name="Event limit" type="DOUBLE">0</key>
    <key name="Num subevents" type="DWORD">0</key>
    <key name="Log history" type="INT">0</key>
    <key name="Frontend host" type="STRING" size="32">vwisac2</key>
    <key name="Frontend name" type="STRING" size="32">fePOL</key>
    <key name="Frontend file name" type="STRING" size="256">febnmr.c</key>
  </dir>
  <dir name="Variables">
    <key name="helicity" type="DWORD">0</key>
    <key name="current cycle" type="DWORD">2</key>
    <key name="cancelled cycle" type="DWORD">1</key>
    <key name="current scan" type="DWORD">1</key>
    <key name="Ref P+ thr" type="DOUBLE">0</key>
    <key name="Ref Laser thr" type="DOUBLE">10138</key>
    <key name="Ref Fcup thr" type="DOUBLE">0</key>
    <key name="Current P+ thr" type="DOUBLE">0</key>
    <key name="Current Laser thr" type="DOUBLE">10138</key>
    <key name="Current Fcup thr" type="DOUBLE">0</key>
    <key name="RF state" type="DWORD">0</key>
    <key name="Fluor monitor counts" type="DWORD">50010</key>
    <key name="EpicsDev Set(V)" type="FLOAT">0</key>
    <key name="EpicsDev Read(V)" type="FLOAT">0</key>
    <key name="Campdev set" type="FLOAT">0</key>
    <key name="Campdev read" type="FLOAT">0</key>
    <key name="Pol DAC set" type="DOUBLE">0</key>
    <key name="Pol DAC read" type="DOUBLE">0</key>
    <key name="last failed thr test" type="DWORD">0</key>
    <key name="cycle when last failed thr" type="DWORD">0</key>
  </dir>
  <dir name="Statistics">
    <key name="Events sent" type="DOUBLE">0</key>
    <key name="Events per sec." type="DOUBLE">0</key>
    <key name="kBytes per sec." type="DOUBLE">0</key>
  </dir>
</odb>
\end{DoxyCode}




\hypertarget{RC_odbedit_examples_RC_odbedit_load}{}\subparagraph{load -\/ load database from a saved file}\label{RC_odbedit_examples_RC_odbedit_load}

\begin{DoxyCode}
load <file>             - load database from .ODB file at current position
\end{DoxyCode}
 To load the {\bfseries complete} database from an ASCII file containing a previously \hyperlink{RC_odbedit_examples_RC_odbedit_save}{saved} database: 
\begin{DoxyCode}
[mpet@titan01 ~/online] odbedit
[local:mpet:Stopped]/>load mpet.odb
\end{DoxyCode}
 The entire database need not be loaded. \hyperlink{RC_odbedit_examples_RC_odbedit_save}{Saved ASCII files} can be made of just a part of the database, and these can be reloaded into the database. Since the full path is given in the saved file, the file can be loaded from any position in the database. The saved ASCII file may of course be edited prior to loading, if keynames or values need to be changed. If the keys in the load file do not exist, they will be created. If they do exist, the values from the file will be loaded. 
\begin{DoxyCode}
[mpet@titan01 ~/online] odbedit
[local:mpet:Stopped]/>load awg0.odb
\end{DoxyCode}
 \par


\label{RC_odbedit_examples_idx_experim-dot-h_make}
\hypertarget{RC_odbedit_examples_idx_experim-dot-h_make}{}
 

\hypertarget{RC_odbedit_examples_RC_odbedit_make}{}\subparagraph{make -\/ create experim.h}\label{RC_odbedit_examples_RC_odbedit_make}

\begin{DoxyCode}
make [analyzer name]    - create experim.h
\end{DoxyCode}
 The {\bfseries make} command creates in the current directory {\bfseries \hyperlink{experim_8h}{experim.h}}, a file containing C structures which can be included into frontend and analyzer code to enable easy access to the odb \hyperlink{structparameters}{parameters} (see also RC\_\-experim\_\-dot\_\-h \char`\"{}using experim.h with hot-\/links\char`\"{} \par
 In order to include the \hyperlink{DataAnalysis_DA_analyzer_utility}{analyzer section}, the ODB key {\bfseries /$<$Analyzer$>$/Parameters} has to be present, where $<$Analyzer$>$ is the name of the analyzer. The command used is then \char`\"{}make $<$Analyzer$>$ \char`\"{} \par
 The following example does not have an Analyzer key. 
\begin{DoxyCode}
[pol@isdaq01 pol]$ odbedit
[local:pol:S]/>make
Analyzer "Analyzer" not found in ODB, skipping analyzer parameters
"experim.h" has been written to /home/pol/online/pol
\end{DoxyCode}
 Here is an part of \hyperlink{experim_8h}{experim.h} for an experiment, showing the \char`\"{}Experiment\char`\"{} tree and one of the \char`\"{}Equipment\char`\"{} trees 
\begin{DoxyCode}
/********************************************************************\

  Name:         experim.h
  Created by:   ODBedit program

  Contents:     This file contains C structures for the "Experiment"
                tree in the ODB and the "/Analyzer/Parameters" tree.


                Additionally, it contains the "Settings" subtree for
                all items listed under "/Equipment" as well as their
                event definition.

                It can be used by the frontend and analyzer to work
                with these information.

                All C structures are accompanied with a string represen-
                tation which can be used in the db_create_record function
                to setup an ODB structure which matches the C structure.

  Created on:   Wed Sep 23 13:10:52 2009

\********************************************************************/

#define EXP_EDIT_DEFINED

typedef struct {
  char      run_title[88];
  DWORD     experiment_number;
  char      experimenter[32];
  char      sample[15];
  char      orientation[15];
  char      temperature[15];
  char      field[15];
  char      element[24];
  INT       mass;
  INT       dc_offset_v_;
  double    ion_source__kv_;
  double    laser_wavelength__nm_;
  BOOL      active;
  INT       num_scans;
  char      source_hv_bias[12];
  BOOL      edit_run_number;
} EXP_EDIT;

#define EXP_EDIT_STR(_name) char *_name[] = {\
"[.]",\
"run_title = STRING : [88] test",\
"experiment number = DWORD : 1",\
"experimenter = STRING : [32] Matt ",\
"sample = STRING : [15] test",\
"orientation = STRING : [15] ",\
"temperature = STRING : [15] ",\
"field = STRING : [15] ",\
"Element = STRING : [24] li",\
"Mass = INT : 7",\
"DC offset(V) = INT : 0",\
"Ion source (kV) = DOUBLE : 30",\
"Laser wavelength (nm) = DOUBLE : 123456789",\
"write data = LINK : [35] /Logger/Channels/0/Settings/Active",\
"Number of scans = LINK : [47] /Equipment/FIFO_acq/sis mcs/hardware/num scans",\
"Source HV Bias = STRING : [12] OLIS",\
"Edit run number = BOOL : y",\
"",\
NULL }




#ifndef EXCL_CYCLE_SCALERS

#define CYCLE_SCALERS_COMMON_DEFINED

typedef struct {
  WORD      event_id;
  WORD      trigger_mask;
  char      buffer[32];
  INT       type;
  INT       source;
  char      format[8];
  BOOL      enabled;
  INT       read_on;
  INT       period;
  double    event_limit;
  DWORD     num_subevents;
  INT       log_history;
  char      frontend_host[32];
  char      frontend_name[32];
  char      frontend_file_name[256];
} CYCLE_SCALERS_COMMON;

#define CYCLE_SCALERS_COMMON_STR(_name) char *_name[] = {\
"[.]",\
"Event ID = WORD : 3",\
"Trigger mask = WORD : 1",\
"Buffer = STRING : [32] SYSTEM",\
"Type = INT : 1",\
"Source = INT : 0",\
"Format = STRING : [8] MIDAS",\
"Enabled = BOOL : y",\
"Read on = INT : 257",\
"Period = INT : 100",\
"Event limit = DOUBLE : 0",\
"Num subevents = DWORD : 0",\
"Log history = INT : 0",\
"Frontend host = STRING : [32] vwisac2",\
"Frontend name = STRING : [32] fePOL",\
"Frontend file name = STRING : [256] febnmr.c",\
"",\
NULL }

#define CYCLE_SCALERS_SETTINGS_DEFINED

typedef struct {
  char      names[6][32];
} CYCLE_SCALERS_SETTINGS;

#define CYCLE_SCALERS_SETTINGS_STR(_name) char *_name[] = {\
"[.]",\
"Names = STRING[6] :",\
"[32] Scaler_B%SIS Ref pulse",\
"[32] Scaler_B%Fluor. mon",\
"[32] Scaler_B%P+ beam",\
"[32] Scaler_B%Laser power",\
"[32] Scaler_B%Faraday Cup 15",\
"[32] Scaler_B%Locking Feedback",\
"",\
NULL }

#endif

...................

etc.
\end{DoxyCode}


\label{RC_odbedit_examples_idx_clients_active_odbedit}
\hypertarget{RC_odbedit_examples_idx_clients_active_odbedit}{}
 

\hypertarget{RC_odbedit_examples_RC_odbedit_scl}{}\subparagraph{scl -\/ show active clients}\label{RC_odbedit_examples_RC_odbedit_scl}

\begin{DoxyCode}
scl [-w]                - show all active clients [with watchdog info]
\end{DoxyCode}
 \par
 
\begin{DoxyCode}
[local:mpet:Stopped]/>scl
Name                Host
rucompet            titan01.triumf.ca
Logger              titan01.triumf.ca
scEpics             titan01.triumf.ca
fesdac              lxmpet.triumf.ca
fempet              lxmpet.triumf.ca
mhttpd              titan01.triumf.ca
ODBEdit             titan01.triumf.ca
\end{DoxyCode}




\hypertarget{RC_odbedit_examples_RC_odbedit_sh}{}\subparagraph{sh -\/ shutdown a client}\label{RC_odbedit_examples_RC_odbedit_sh}

\begin{DoxyCode}
shutdown <client>/all   - shutdown individual or all clients
\end{DoxyCode}
 \par
 
\begin{DoxyCode}
[local:mpet:Stopped]/>sh rucompet
[local:mpet:Stopped]/>scl
Name                Host
Logger              titan01.triumf.ca
scEpics             titan01.triumf.ca
fesdac              lxmpet.triumf.ca
fempet              lxmpet.triumf.ca
mhttpd              titan01.triumf.ca
ODBEdit             titan01.triumf.ca
\end{DoxyCode}


\label{RC_odbedit_examples_idx_run_start}
\hypertarget{RC_odbedit_examples_idx_run_start}{}
 

\hypertarget{RC_odbedit_examples_RC_odbedit_start}{}\subparagraph{start -\/ start a run}\label{RC_odbedit_examples_RC_odbedit_start}

\begin{DoxyCode}
start [number][now][-v] - start a run [with a specific number],
                          [now] w/o asking parameters, [-v] debug output
\end{DoxyCode}
 \par


The odbedit {\bfseries start} command is used to start a run.

\par
 \hypertarget{RC_odbedit_examples_RC_EOS_example1}{}\subparagraph{Run start examples}\label{RC_odbedit_examples_RC_EOS_example1}
In the following example, the run number of the new run is supplied. 
\begin{DoxyCode}
[local:Default:S]/Experiment>start 503
\end{DoxyCode}
 \par


In the example below, the run number is not specified. The system will start the next consecutive run. 
\begin{DoxyCode}
[local:Default:S]/Experiment>start
Run number [30004]: 
Are the above parameters correct? ([y]/n/q): y
\end{DoxyCode}
 The user may edit the run number before continuing by typing \char`\"{}n\char`\"{}. Typing \char`\"{}y\char`\"{} will start the run, and typing \char`\"{}q\char`\"{} will abort the run start. \par


In the above example, there are no \hyperlink{RC_customize_ODB_RC_Edit_On_Start}{edit-\/on-\/start paramaters} defined by the user. If any are defined, the command \char`\"{}start\char`\"{} will display the \char`\"{}edit on start\char`\"{} \hyperlink{structparameters}{parameters} e.g.\hypertarget{RC_odbedit_examples_RC_EOS_example2}{}\subparagraph{Run Start example with \char`\"{}Edit on Start\char`\"{} parameters}\label{RC_odbedit_examples_RC_EOS_example2}
Note that when using odbedit, Parameter comments are NOT visible, and the run number IS editable. 
\begin{DoxyCode}
[local:bnmr:S]/>start 
run_title : 2e test
experiment number : 9999
experimenter : gdm
sample : NA
orientation : 
temperature : 285.12K
field : 0G
Number of scans : 0
write data : y
Run number [30004]:
\end{DoxyCode}


\par
 \hypertarget{RC_odbedit_examples_RC_EOS_example3}{}\subparagraph{Run Start Example with \char`\"{}start now\char`\"{}}\label{RC_odbedit_examples_RC_EOS_example3}
\label{RC_odbedit_examples_RC_odbedit_start_now}
\hypertarget{RC_odbedit_examples_RC_odbedit_start_now}{}
 By entering the command {\bfseries \char`\"{}start now\char`\"{}}, all defined \hyperlink{RC_customize_ODB_RC_Edit_On_Start}{Edit-\/on-\/Start parameters} can be skipped. 
\begin{DoxyCode}
[local:bnmr:S]/>start now
Starting run #30129
Run #30129 started
\end{DoxyCode}
\hypertarget{RC_odbedit_examples_RC_EOS_example4}{}\subparagraph{Run Start Example with \char`\"{}-\/v\char`\"{} verbose option}\label{RC_odbedit_examples_RC_EOS_example4}
\label{RC_odbedit_examples_RC_odbedit_start_v}
\hypertarget{RC_odbedit_examples_RC_odbedit_start_v}{}
 Using the {\bfseries \char`\"{}-\/v\char`\"{}} (verbose) option is useful for debugging. It prints messages as each client is started. 
\begin{DoxyCode}
[local:bnmr:S]/>start -v
run_title : test
experiment number : 1165
experimenter : gdm
sample : GaAs
orientation : 100
temperature : 286.01K
field : 0.00G
Number of cycles : 0
write data : y
Run number [30128]:
Are the above parameters correct? ([y]/n/q): y
\end{DoxyCode}
 \label{RC_odbedit_examples_RC_transition_start}
\hypertarget{RC_odbedit_examples_RC_transition_start}{}
 
\begin{DoxyCode}
Starting run #30128
Setting run number 30128 in ODB
---- Transition START started ----

==== Found client "Logger" with sequence number 200
Connecting to client "Logger" on host isdaq01...
Connection established to client "Logger" on host isdaq01
Executing RPC transition client "Logger" on host isdaq01...
RPC transition finished client "Logger" on host isdaq01 with status 1

==== Found client "mheader" with sequence number 200
Connecting to client "mheader" on host isdaq01...
Connection established to client "mheader" on host isdaq01
Executing RPC transition client "mheader" on host isdaq01...
RPC transition finished client "mheader" on host isdaq01 with status 1

==== Found client "rf_config" with sequence number 350
Connecting to client "rf_config" on host isdaq01...
Connection established to client "rf_config" on host isdaq01
Executing RPC transition client "rf_config" on host isdaq01...
RPC transition finished client "rf_config" on host isdaq01 with status 1

==== Found client "rf_config" with sequence number 400
Connecting to client "rf_config" on host isdaq01...
Connection established to client "rf_config" on host isdaq01
Executing RPC transition client "rf_config" on host isdaq01...
RPC transition finished client "rf_config" on host isdaq01 with status 1

==== Found client "Mdarc" with sequence number 450
Connecting to client "Mdarc" on host isdaq01...
Connection established to client "Mdarc" on host isdaq01
Executing RPC transition client "Mdarc" on host isdaq01...
RPC transition finished client "Mdarc" on host isdaq01 with status 1

==== Found client "Epics" with sequence number 500
Connecting to client "Epics" on host isdaq01...
Connection established to client "Epics" on host isdaq01
Executing RPC transition client "Epics" on host isdaq01...
RPC transition finished client "Epics" on host isdaq01 with status 1

==== Found client "mheader" with sequence number 500
Connecting to client "mheader" on host isdaq01...
Connection established to client "mheader" on host isdaq01
Executing RPC transition client "mheader" on host isdaq01...
RPC transition finished client "mheader" on host isdaq01 with status 1

==== Found client "Mdarc" with sequence number 500
Connecting to client "Mdarc" on host isdaq01...
Connection established to client "Mdarc" on host isdaq01
Executing RPC transition client "Mdarc" on host isdaq01...
RPC transition finished client "Mdarc" on host isdaq01 with status 1

==== Found client "feBNMR" with sequence number 500
Connecting to client "feBNMR" on host bnmrhmvw...
Connection established to client "feBNMR" on host bnmrhmvw
Executing RPC transition client "feBNMR" on host bnmrhmvw...
RPC transition finished client "feBNMR" on host bnmrhmvw with status 1
\end{DoxyCode}




 \label{RC_odbedit_examples_idx_run_stop}
\hypertarget{RC_odbedit_examples_idx_run_stop}{}
 \hypertarget{RC_odbedit_examples_RC_odbedit_stop}{}\subparagraph{stop -\/ stop a run}\label{RC_odbedit_examples_RC_odbedit_stop}

\begin{DoxyCode}
stop [-v]               - stop current run, [-v] debug output
\end{DoxyCode}
 \par
 
\begin{DoxyCode}
[local:pol:R]/>stop
Run #399 stopped
[local:pol:S]/>  
\end{DoxyCode}
 \label{RC_odbedit_examples_idx_run_stop_immediately}
\hypertarget{RC_odbedit_examples_idx_run_stop_immediately}{}
 \char`\"{}Stop now\char`\"{} can be used to force a stop if there is a deferred transition. If there is no deferred transition, the \char`\"{}stop now\char`\"{} is the same as \char`\"{}stop\char`\"{}. 
\begin{DoxyCode}
local:bnmr:R]/>stop now
Run #30129 stopped
\end{DoxyCode}
 Using the \char`\"{}-\/v\char`\"{} (verbose) option is useful for debugging. It prints a message as each client is stopped.

\label{RC_odbedit_examples_RC_transition_stop}
\hypertarget{RC_odbedit_examples_RC_transition_stop}{}



\begin{DoxyCode}
[local:bnmr:R]/>stop -v
---- Transition STOP started ----

==== Found client "mheader" with sequence number 200
Connecting to client "mheader" on host isdaq01...
Connection established to client "mheader" on host isdaq01
Executing RPC transition client "mheader" on host isdaq01...
RPC transition finished client "mheader" on host isdaq01 with status 1

==== Found client "Epics" with sequence number 500
Connecting to client "Epics" on host isdaq01...
Connection established to client "Epics" on host isdaq01
Executing RPC transition client "Epics" on host isdaq01...
RPC transition finished client "Epics" on host isdaq01 with status 1

==== Found client "rf_config" with sequence number 500
Connecting to client "rf_config" on host isdaq01...
Connection established to client "rf_config" on host isdaq01
Executing RPC transition client "rf_config" on host isdaq01...
RPC transition finished client "rf_config" on host isdaq01 with status 1

==== Found client "mheader" with sequence number 500
Connecting to client "mheader" on host isdaq01...
Connection established to client "mheader" on host isdaq01
Executing RPC transition client "mheader" on host isdaq01...
RPC transition finished client "mheader" on host isdaq01 with status 1

==== Found client "feBNMR" with sequence number 500
Connecting to client "feBNMR" on host bnmrhmvw...
Connection established to client "feBNMR" on host bnmrhmvw
Executing RPC transition client "feBNMR" on host bnmrhmvw...
RPC transition finished client "feBNMR" on host bnmrhmvw with status 1

==== Found client "mheader" with sequence number 600
Connecting to client "mheader" on host isdaq01...
Connection established to client "mheader" on host isdaq01
Executing RPC transition client "mheader" on host isdaq01...
RPC transition finished client "mheader" on host isdaq01 with status 1

==== Found client "Mdarc" with sequence number 600
Connecting to client "Mdarc" on host isdaq01...
Connection established to client "Mdarc" on host isdaq01
Executing RPC transition client "Mdarc" on host isdaq01...
RPC transition finished client "Mdarc" on host isdaq01 with status 1

==== Found client "feBNMR" with sequence number 750
Connecting to client "feBNMR" on host bnmrhmvw...
Connection established to client "feBNMR" on host bnmrhmvw
Executing RPC transition client "feBNMR" on host bnmrhmvw...
RPC transition finished client "feBNMR" on host bnmrhmvw with status 1

==== Found client "Logger" with sequence number 800
Connecting to client "Logger" on host isdaq01...
Connection established to client "Logger" on host isdaq01
Executing RPC transition client "Logger" on host isdaq01...
RPC transition finished client "Logger" on host isdaq01 with status 1

---- Transition STOP finished ----
Run #30128 stopped
\end{DoxyCode}


\par
 

\label{index_end}
\hypertarget{index_end}{}
 \subsubsection{mhttpd: the MIDAS Web-\/based Run Control utility}\label{RC_mhttpd}
\label{RC_mhttpd_utility_idx_mhttpd-utility}
\hypertarget{RC_mhttpd_utility_idx_mhttpd-utility}{}


\par
 

\label{RC_mhttpd_idx_mhttpd}
\hypertarget{RC_mhttpd_idx_mhttpd}{}
 \label{RC_mhttpd_idx_midas_webserver}
\hypertarget{RC_mhttpd_idx_midas_webserver}{}
 \begin{center} \par
\par
\par
  \end{center} \hypertarget{RC_mhttpd_RC_mhttpd_intro}{}\paragraph{Introduction}\label{RC_mhttpd_RC_mhttpd_intro}
The MIDAS Web Server utility mhttpd provides MIDAS DAQ control through the web using any web browser. Provided the \hyperlink{RC_mhttpd_utility}{daemon application} is running, the user may access any MIDAS experiment running on a given host from a Web browser.

Full monitoring and almost full control of a particular experiment can be achieved through the MIDAS Web server.


\begin{DoxyItemize}
\item \hyperlink{RC_mhttpd_utility}{The mhttpd daemon}
\item \hyperlink{RC_mhttpd_RC_mhttpd_functionality}{Features of mhttpd}
\begin{DoxyItemize}
\item \hyperlink{RC_mhttpd_RC_mhttpd_page_list}{list of mhttpd pages}
\item \hyperlink{RC_mhttpd_RC_mhttpd_tree_list}{list of mhttpd-\/specific odb trees}
\end{DoxyItemize}
\end{DoxyItemize}\hypertarget{RC_mhttpd_RC_mhttpd_functionality}{}\paragraph{Features of mhttpd}\label{RC_mhttpd_RC_mhttpd_functionality}
The basic functionality of mhttpd includes:
\begin{DoxyItemize}
\item {\bfseries Run control} (start/stop/pause).
\item Up-\/to-\/date {\bfseries client status and statistics display} (frontend(s) logger(s) etc.)
\item {\bfseries Graphical history data} display.
\item {\bfseries Electronic LogBook} recording/retrieval messages
\item {\bfseries Alarm} monitoring/control
\item {\bfseries Slow} {\bfseries control} data display.
\item Listing of currently {\bfseries connected} {\bfseries clients} 
\item Basic {\bfseries access to ODB}.
\end{DoxyItemize}

These functions are available on one or more of the various {\bfseries mhttpd} {\bfseries pages} described below : \label{RC_mhttpd_RC_mhttpd_page_list}
\hypertarget{RC_mhttpd_RC_mhttpd_page_list}{}



\begin{DoxyItemize}
\item \hyperlink{RC_mhttpd_Main_Status_page}{Main Status Page} : monitoring, statistics display, access to other pages etc.
\item \hyperlink{RC_mhttpd_Start_page}{Start page} : Run control page
\item \hyperlink{RC_mhttpd_ODB_page}{ODB page} : Online Database manipulation ( an alternative to using odbedit )
\item \hyperlink{RC_mhttpd_Equipment_page}{Equipment page} : Frontend information
\item \hyperlink{RC_mhttpd_sc_page}{Slow Control page} : Slow Control information
\item \hyperlink{RC_mhttpd_Message_page}{Message page} : Message Log
\item \hyperlink{RC_mhttpd_Elog_page}{Elog page} : Electronic Log
\item \hyperlink{RC_mhttpd_Program_page}{Programs page} : Program control (clients)
\item \hyperlink{RC_mhttpd_History_page}{History page} : History display
\item \hyperlink{RC_mhttpd_Alarm_page}{Alarm page} : Alarm system control
\item \hyperlink{RC_mhttpd_MSCB_page}{MSCB page} MIDAS Slow Control Bus
\item \hyperlink{RC_mhttpd_CNAF_page}{CAMAC Access page} : CAMAC access
\item \hyperlink{RC_mhttpd_Alias_page}{mhttpd Alias page} : user defined Alias page(s)
\item \hyperlink{RC_mhttpd_Logger_page}{mhttpd Logger page} : data logger settings information
\item \hyperlink{RC_mhttpd_Config_page}{Config page} : status page update period
\item \hyperlink{RC_mhttpd_Custom_page}{Custom pages} : user defined Web page(s)
\end{DoxyItemize}

\label{RC_mhttpd_RC_mhttpd_tree_list}
\hypertarget{RC_mhttpd_RC_mhttpd_tree_list}{}
 Several directory trees in the ODB are {\bfseries  specific to mhttpd}, or have {\bfseries additional} {\bfseries features} if mhttpd is running. They are described below in the appropriate section.
\begin{DoxyItemize}
\item \hyperlink{RC_mhttpd_Alias_page_RC_odb_alias_tree}{The ODB /Alias Tree}
\item \hyperlink{RC_mhttpd_Activate_RC_odb_custom_tree}{The /Custom ODB tree}
\item \hyperlink{RC_mhttpd_defining_script_buttons_RC_odb_script_tree}{The ODB /Script tree}
\item \hyperlink{F_Elog_F_ODB_Elog_Tree}{The ODB /Elog Tree}
\item \hyperlink{RC_customize_ODB_RC_Edit_On_Start}{Defining Edit-\/on-\/start Parameters}
\end{DoxyItemize}



\par
 \label{index_end}
\hypertarget{index_end}{}
 \paragraph{The mhttpd daemon}\label{RC_mhttpd_utility}
\label{RC_mhttpd_utility_idx_mhttpd-utility}
\hypertarget{RC_mhttpd_utility_idx_mhttpd-utility}{}


\par
 \hypertarget{RC_mhttpd_utility_RC_mhttpd_Usage}{}\subparagraph{Start the mhttpd daemon}\label{RC_mhttpd_utility_RC_mhttpd_Usage}
The mhttpd utility requires the TCP/IP port number as an argument in order to listen to the web-\/based request.


\begin{DoxyItemize}
\item {\bfseries  Arguments }
\end{DoxyItemize}


\begin{DoxyItemize}
\item \mbox{[}-\/h\mbox{]} : connect to midas server (mserver) on given host
\item \mbox{[}-\/e\mbox{]} : experiment to connect to
\item \mbox{[}-\/p port \mbox{]} : port number e.g. 8081 (no default)
\item \mbox{[}-\/v\mbox{]} : display verbose HTTP communication
\item \mbox{[}-\/D\mbox{]} : starts program as a {\bfseries daemon} 
\item \mbox{[}-\/E\mbox{]} : only display ELog system
\item \mbox{[}-\/H\mbox{]} : only display history plots
\item \mbox{[}-\/a\mbox{]} : only allow access for specific host(s). Several \mbox{[}-\/a Hostname\mbox{]} statements might be given
\item \mbox{[}-\/help\mbox{]} : display usage information
\end{DoxyItemize}


\begin{DoxyItemize}
\item {\bfseries  Usage } \par
 mhttpd \mbox{[}-\/h Hostname\mbox{]} \mbox{[}-\/e Experiment\mbox{]} \mbox{[}-\/p port\mbox{]} \mbox{[}-\/v\mbox{]} \mbox{[}-\/D\mbox{]} \mbox{[}-\/c\mbox{]} \mbox{[}-\/a Hostname\mbox{]} \par
e.g. \par
 {\bfseries mhttpd -\/p 8081 -\/D }
\end{DoxyItemize}

\begin{DoxyNote}{Note}

\begin{DoxyItemize}
\item Several copies of mhttpd can run on a single host, as long as they are {\bfseries started on different ports}.
\item If {\bfseries more than one experiment} runs on the same host, a server for each experiment must be started on a {\bfseries different} port, e.g.
\begin{DoxyItemize}
\item mhttpd -\/e midas -\/p 8081 -\/D
\item mhttpd -\/e midgas -\/p 8082 -\/D
\end{DoxyItemize}
\end{DoxyItemize}
\end{DoxyNote}
\par


\par


\label{RC_mhttpd_utility_idx_mhttpd-utility_connect}
\hypertarget{RC_mhttpd_utility_idx_mhttpd-utility_connect}{}
 \hypertarget{RC_mhttpd_utility_RC_mhttpd_connect}{}\subparagraph{Connect to the mhttpd webserver}\label{RC_mhttpd_utility_RC_mhttpd_connect}
To connect to a mhttpd webserver running on port {\itshape $<$nnnn$>$\/} on host {\itshape $<$hostname$>$\/} and experiment {\itshape $<$exptname$>$\/}, {\bfseries enter the URL in your web browser location box} in the form


\begin{DoxyCode}
 http://<hostname>:<nnnn>/?exp=<exptname>
\end{DoxyCode}
 e.g. 
\begin{DoxyCode}
 http://midtis09:8081/?exp=midas
\end{DoxyCode}
 \par


\label{RC_mhttpd_utility_RC_mhttpd_msp_default}
\hypertarget{RC_mhttpd_utility_RC_mhttpd_msp_default}{}


Once the \hyperlink{RC_mhttpd_utility_RC_mhttpd_connect}{connection} to a given experiment is established, the {\bfseries Main Status Page} of the MIDAS webserver is displayed in the web browser window. \par
 An error page will appear instead if the \hyperlink{RC_mhttpd_utility}{mhttpd daemon} has NOT been started on the specified port (or has not been started at all). \par


\label{RC_mhttpd_utility_RC_mhttpd_minimal_status_page}
\hypertarget{RC_mhttpd_utility_RC_mhttpd_minimal_status_page}{}
 \begin{center} mhttpd main status page (no clients are running) \par
\par
\par
   \end{center}  \par


Once clients are started (e.g. frontend, logger etc.) the main status page will look more like \hyperlink{RC_mhttpd_Main_Status_page_RC_mhttpd_msp_customized}{this}.

the image above shows a pre-\/ \hyperlink{NDF_ndf_dec_2009}{Dec 2009} version of mhttpd (see \hyperlink{RC_mhttpd_status_page_redesign}{Redesign of mhttpd Main Status Page}).

\par
 

 \par


\label{RC_mhttpd_utility_idx_mhttpd_proxy-access}
\hypertarget{RC_mhttpd_utility_idx_mhttpd_proxy-access}{}
 \label{RC_mhttpd_utility_idx_access-control_mhttpd-proxy}
\hypertarget{RC_mhttpd_utility_idx_access-control_mhttpd-proxy}{}
 \label{RC_mhttpd_utility_idx_Apache}
\hypertarget{RC_mhttpd_utility_idx_Apache}{}
 \hypertarget{RC_mhttpd_utility_RC_mhttpd_proxy}{}\subparagraph{Proxy Access to mhttpd}\label{RC_mhttpd_utility_RC_mhttpd_proxy}
A major change was made to mhttpd in \hyperlink{NDF_ndf_feb_2008}{Feb 2008} , changing all internal URLs to relative paths. This allows {\bfseries proxy access} to mhttpd via an \href{http://apache.org/}{\tt Apache} server for example, which might be needed to securely access an experiment from outside the lab through a firewall. Apache can also be configured to allow a secure SSL connection to the proxy.

In order to add SSL encryption to mhttpd, the following settings can be placed into an {\bfseries Apache configuration} : \par
 Assuming
\begin{DoxyItemize}
\item the experiment runs on machine {\itshape online1.your.domain\/}, and
\item apache is running on a publically available machine {\itshape www.your.domain\/} \par

\end{DoxyItemize}


\begin{DoxyCode}
Redirect permanent /online1 http://www.your.domain/online1
ProxyPass /online1/  http://online1.your.domain/

<Location "/online1">
  AuthType Basic
  AuthName ...
  AuthUserFile ...
  Require user ...
</Location>
\end{DoxyCode}


If the URL 
\begin{DoxyCode}
http://www.your.domain/online1
\end{DoxyCode}
 is accessed, it will be redirected (after optional authentication) to 
\begin{DoxyCode}
http://online1.your.domain/
\end{DoxyCode}
 \par
 If you click on the mhttpd \hyperlink{RC_mhttpd_History_page}{History page} for example, mhttpd would normally redirect this to 
\begin{DoxyCode}
http://online1.your.domain/HS/
\end{DoxyCode}
 but this is not correct since you want to go through the proxy {\itshape www.your.domain\/}. The new relative redirection inside mhttpd now redirects the history page correctly to 
\begin{DoxyCode}
http://www.your.domain/online1/HS/
\end{DoxyCode}


\label{index_end}
\hypertarget{index_end}{}
 \par
  \paragraph{Main Status Page}\label{RC_mhttpd_Main_Status_page}
\label{RC_mhttpd_status_page_redesign_idx_mhttpd_page_status}
\hypertarget{RC_mhttpd_status_page_redesign_idx_mhttpd_page_status}{}


\par


 \par


\label{RC_mhttpd_Main_Status_page_RC_mhttpd_main_status}
\hypertarget{RC_mhttpd_Main_Status_page_RC_mhttpd_main_status}{}
 \hypertarget{RC_mhttpd_Main_Status_page_RC_mhttpd_msp_customized}{}\subparagraph{Status Page for a Running Experiment}\label{RC_mhttpd_Main_Status_page_RC_mhttpd_msp_customized}
The following image shows the main Status Page of one of the experiments at TRIUMF:

\begin{center} MIDAS Status Page for a running experiment \par
\par
\par
  \end{center}  \par


In this case, the experiment has been customized by
\begin{DoxyItemize}
\item creating and starting some frontend(s), which define various equipment(s)
\item defining \char`\"{}script\char`\"{} and \char`\"{}alias\char`\"{} buttons
\item starting some of the MIDAS utilities (e.g. the MIDAS logger \hyperlink{F_Logging_F_mlogger_utility}{mlogger} and the \hyperlink{F_LogUtil_F_lazylogger_utility}{lazylogger} )
\end{DoxyItemize}

Compared with the \hyperlink{RC_mhttpd_utility_RC_mhttpd_minimal_status_page}{minimal status page} (where no clients are running) it's clear that the the Status Page above shows a lot more information.

The main status page will be discussed line-\/by-\/line in the following section ( \hyperlink{RC_mhttpd_status_page_features}{Features of the Main Status Page} ).


\begin{DoxyItemize}
\item \hyperlink{RC_mhttpd_status_page_features}{Features of the Main Status Page}
\item \hyperlink{RC_mhttpd_status_page_redesign}{Redesign of mhttpd Main Status Page}
\end{DoxyItemize}



\label{index_end}
\hypertarget{index_end}{}
 \subsubsection{Features of the Main Status Page}\label{RC_mhttpd_status_page_features}
 The Status Page is sub-\/divided in several parts:
\begin{DoxyItemize}
\item \hyperlink{RC_mhttpd_status_page_features_RC_mhttpd_status_title}{Experiment/Date/Refresh information}
\item \hyperlink{RC_mhttpd_status_page_features_RC_mhttpd_status_menu_buttons}{Menu buttons}
\begin{DoxyItemize}
\item \hyperlink{RC_mhttpd_status_page_features_RC_mhttpd_status_RC_buttons}{Run Control buttons}
\item \hyperlink{RC_mhttpd_status_page_features_RC_mhttpd_status_Page_buttons}{Page Switch buttons}
\end{DoxyItemize}
\item \hyperlink{RC_mhttpd_status_page_features_RC_mhttpd_status_script_buttons}{Optional Script buttons}
\item \hyperlink{RC_mhttpd_status_page_features_RC_mhttpd_status_Manual_Trigger_buttons}{Manual-\/Trigger Buttons}
\item \hyperlink{RC_mhttpd_status_page_features_RC_mhttpd_status_Alias_buttons}{Alias-\/Buttons}
\item \hyperlink{RC_mhttpd_status_page_features_RC_mhttpd_status_Run_info}{Run status information}
\item \hyperlink{RC_mhttpd_status_page_features_RC_mhttpd_status_Equipment_info}{Equipment information and Event rates}
\item \hyperlink{RC_mhttpd_status_page_features_RC_mhttpd_status_Logger}{Data Logging Information}
\item \hyperlink{RC_mhttpd_status_page_features_RC_mhttpd_status_latest_msg}{Last system message}
\item \hyperlink{RC_mhttpd_status_page_features_RC_mhttpd_status_clients}{Active Client list}
\end{DoxyItemize}

These will be discussed in detail in the following sections. \par
 \label{RC_mhttpd_status_page_features_RC_mhttpd_main_status_new}
\hypertarget{RC_mhttpd_status_page_features_RC_mhttpd_main_status_new}{}
 \begin{center} mhttpd main status page showing Menu Buttons \par
\par
\par
  \end{center}  \par
\hypertarget{RC_mhttpd_status_page_features_RC_mhttpd_status_title}{}\subparagraph{Experiment/Date/Refresh information}\label{RC_mhttpd_status_page_features_RC_mhttpd_status_title}
The top line on the main status page \hyperlink{RC_mhttpd_status_page_features_RC_mhttpd_main_status_new}{above} shows
\begin{DoxyItemize}
\item the Experiment name (\char`\"{}Online\char`\"{})
\item the current date
\item the refresh period (Refr:600) -\/ see note below
\end{DoxyItemize}

It is important to note that the {\bfseries refresh} of the Status Page is not \char`\"{}event driven\char`\"{} but is controlled by a timer whose rate is adjustable through the \hyperlink{RC_mhttpd_status_page_features_RC_mhttpd_Config_button}{Config button}. This means the information at any given time may reflect the experiment state of up to n seconds in the past, where n is the timer setting of the refresh parameter.





\label{RC_mhttpd_status_page_features_idx_mhttpd_buttons_menu}
\hypertarget{RC_mhttpd_status_page_features_idx_mhttpd_buttons_menu}{}
 \hypertarget{RC_mhttpd_status_page_features_RC_mhttpd_status_menu_buttons}{}\subparagraph{Menu buttons}\label{RC_mhttpd_status_page_features_RC_mhttpd_status_menu_buttons}
The top row of buttons on the \hyperlink{RC_mhttpd_status_page_features_RC_mhttpd_main_status_new}{main status page} are the
\begin{DoxyItemize}
\item \hyperlink{RC_mhttpd_status_page_features_RC_mhttpd_status_RC_buttons}{Run Control buttons}
\item \hyperlink{RC_mhttpd_status_page_features_RC_mhttpd_status_Page_buttons}{Page Switch buttons}
\end{DoxyItemize}

\label{RC_mhttpd_status_page_features_idx_mhttpd_buttons_run-control}
\hypertarget{RC_mhttpd_status_page_features_idx_mhttpd_buttons_run-control}{}
 \hypertarget{RC_mhttpd_status_page_features_RC_mhttpd_status_RC_buttons}{}\subparagraph{Run Control buttons}\label{RC_mhttpd_status_page_features_RC_mhttpd_status_RC_buttons}
Depending on the \hyperlink{RC_Run_States_and_Transitions}{run state}, a single or the two first buttons on the \hyperlink{RC_mhttpd_status_page_features_RC_mhttpd_main_status_new}{main status page} will show the possible action that can be taken, i.e.

\label{RC_mhttpd_status_page_features_RC_table_run_state}
\hypertarget{RC_mhttpd_status_page_features_RC_table_run_state}{}


\begin{table}[h]\begin{TabularC}{3}
\hline
Run Control Buttons present\par
(see Note \hyperlink{RC_mhttpd_status_page_features_RC_mhttpd_note1}{below})\par
  &Run State\par
  &Action\par
   \\\cline{1-3}
\begin{TabularC}{1}
\hline
Start\par
   \\\cline{1-1}
\end{TabularC}
&STOPPED\par
  &Start the run\par
   \\\cline{1-3}
\begin{TabularC}{1}
\hline
Stop\par
   \\\cline{1-1}
\end{TabularC}
&\multirow{2}{\linewidth}{RUNNING\par
  }&Stop the run\par
   \\\cline{2-3}
\begin{TabularC}{1}
\hline
Pause\par
   \\\cline{1-1}
\end{TabularC}
&Pause the run\par
   \\\cline{1-2}
\begin{TabularC}{1}
\hline
Resume\par
   \\\cline{1-1}
\end{TabularC}
&\multirow{1}{\linewidth}{PAUSED\par
  }&Resume the run\par
   \\\cline{1-3}
\end{TabularC}
\centering
\caption{Run Control buttons visible depending on Run State }
\end{table}


\label{RC_mhttpd_status_page_features_RC_mhttpd_note1}
\hypertarget{RC_mhttpd_status_page_features_RC_mhttpd_note1}{}
  Since \hyperlink{NDF_ndf_nov_2009}{Nov 2009} , the Run Buttons may be hidden (see \hyperlink{RC_customize_ODB_RC_Experiment_tree_keys}{Hide Run Buttons} ),

\label{RC_mhttpd_status_page_features_idx_mhttpd_buttons_page-switch}
\hypertarget{RC_mhttpd_status_page_features_idx_mhttpd_buttons_page-switch}{}
 \hypertarget{RC_mhttpd_status_page_features_RC_mhttpd_status_Page_buttons}{}\subparagraph{Page Switch buttons}\label{RC_mhttpd_status_page_features_RC_mhttpd_status_Page_buttons}
The {\bfseries Page Switch buttons} on the mhttpd main status page (see \hyperlink{RC_mhttpd_status_page_features_RC_mhttpd_main_status_new}{example above}) change the page to one of the sub-\/pages. The sub-\/pages all provide a button labelled {\bfseries Status}, which returns to the main Status page when clicked. The purpose of each Page Switch button is explained in the following table:

The Page switch buttons can now be \hyperlink{RC_customize_ODB_RC_ODB_Experiment_Tree}{customized} ( since \hyperlink{NDF_ndf_dec_2009}{Dec 2009} ) , so not all the possible Page Switch buttons may be visible on the status page for a particular experiment.

\begin{table}[h]\begin{TabularC}{2}
\hline
Page Switch Button &Explanation 

\\\cline{1-2}
\begin{TabularC}{1}
\hline
 \hyperlink{RC_mhttpd_ODB_page}{ODB}\par
   \\\cline{1-1}
\end{TabularC}
&This button switches to the (see \hyperlink{RC_mhttpd_ODB_page}{ODB page}), which provides access to the Online Data Base. 

\\\cline{1-2}
\begin{TabularC}{1}
\hline
\hyperlink{RC_mhttpd_MSCB_page}{MSCB}\par
   \\\cline{1-1}
\end{TabularC}
&This button switches to the \hyperlink{RC_mhttpd_MSCB_page}{MSCB page} , which gives access to devices in a MIDAS Slow Control Bus system. (Implemented \hyperlink{NDF_ndf_dec_2009}{Dec 2009})



\\\cline{1-2}
\begin{TabularC}{1}
\hline
 \hyperlink{RC_mhttpd_CNAF_page}{CNAF}\par
   \\\cline{1-1}
\end{TabularC}
&In versions since \hyperlink{NDF_ndf_dec_2009}{Dec 2009} the default is that this button has been replaced by the MSCB button. If the CNAF button is needed, it must be added to the list of \hyperlink{RC_customize_ODB_RC_ODB_Experiment_Tree}{menu buttons}. \par
This button switches to the \hyperlink{RC_mhttpd_CNAF_page}{CAMAC Access page} . If one of the equipments is a CAMAC frontend, it is possible to issue CAMAC commands through this button.  \\\cline{1-2}
\begin{TabularC}{1}
\hline
 \hyperlink{RC_mhttpd_Message_page}{Messages}\par
   \\\cline{1-1}
\end{TabularC}
&Clicking this button opens the \hyperlink{RC_mhttpd_Message_page}{message page} and shows the N last entries of the \hyperlink{F_Messaging}{MIDAS system message log}. The last entry is always present in the status page (see \hyperlink{RC_mhttpd_status_page_features_RC_mhttpd_status_latest_msg}{Last system message} ).  \\\cline{1-2}
\begin{TabularC}{1}
\hline
\hyperlink{RC_mhttpd_Elog_page}{ELog}\par
   \\\cline{1-1}
\end{TabularC}
&This button gives access to the Electronic Log book (Elog). The Elog allows the permanent recording (i.e. in a file) of comments, messages, screen captures etc. composed by the users (see \hyperlink{RC_mhttpd_Elog_page}{Elog page}).  \\\cline{1-2}
\begin{TabularC}{1}
\hline
\hyperlink{RC_mhttpd_Alarm_page}{Alarms}\par
   \\\cline{1-1}
\end{TabularC}
&Clicking this button displays the \hyperlink{RC_mhttpd_Alarm_page}{Alarm page} , which shows the current Alarm setting for the entire experiment. The activation of an alarm is done through the ODB under the {\bfseries /Alarms} tree (See \hyperlink{RC_customize_ODB_RC_Alarm_System}{MIDAS Alarm System})  \\\cline{1-2}
\begin{TabularC}{1}
\hline
 \hyperlink{RC_mhttpd_Program_page}{Programs}\par
   \\\cline{1-1}
\end{TabularC}
&This button gives access to the \hyperlink{RC_mhttpd_Program_page}{Programs page}, which displays the status of the current programs (i.e. MIDAS applications/clients) which are or have been running for this experiment. 

\\\cline{1-2}
\begin{TabularC}{1}
\hline
\hyperlink{RC_mhttpd_History_page}{History}\par
   \\\cline{1-1}
\end{TabularC}
&Display History graphs of pre-\/defined variables. The history setting has to be done through ODB under the {\bfseries /History} (see \hyperlink{F_History_logging}{History Logging} , \hyperlink{RC_mhttpd_History_page}{History page}). 

\\\cline{1-2}
\begin{TabularC}{1}
\hline
\label{RC_mhttpd_status_page_features_RC_mhttpd_Config_button}
\hypertarget{RC_mhttpd_status_page_features_RC_mhttpd_Config_button}{}
 \label{RC_mhttpd_status_page_features_RC_mhttpd_refresh}
\hypertarget{RC_mhttpd_status_page_features_RC_mhttpd_refresh}{}
 \hyperlink{RC_mhttpd_Config_page}{Config}\par
   \\\cline{1-1}
\end{TabularC}
&Allows the {\bfseries  page refresh rate } to be changed. See \hyperlink{RC_mhttpd_Config_page}{Config page} .



\\\cline{1-2}
\begin{TabularC}{1}
\hline
Help\par
   \\\cline{1-1}
\end{TabularC}
&Help button will link to the main MIDAS web documentation (i.e. this document). 

\\\cline{1-2}
\end{TabularC}
\centering
\caption{Page Switch Buttons on the Main Status Page}
\end{table}


\par


\par
\hypertarget{RC_mhttpd_status_page_features_RC_mhpptd_optional_buttons}{}\subparagraph{Optional Buttons on the main Status page}\label{RC_mhttpd_status_page_features_RC_mhpptd_optional_buttons}
\begin{center} mhttpd main Status page (part) showing optional buttons \par
  \end{center}  \par


 Since \hyperlink{NDF_ndf_dec_2009}{Dec 2009}  there may be up to three rows of buttons below the Menu buttons
\begin{DoxyItemize}
\item Script (User) buttons
\item Manually triggered event buttons
\item Custom Page and Alias buttons
\end{DoxyItemize}

 Prior to \hyperlink{NDF_ndf_dec_2009}{Dec 2009}  the Custom Page and Alias hyperlinks appeared as {\bfseries links} rather than buttons, as shown \hyperlink{RC_mhttpd_status_page_redesign_RC_mhttpd_old_alias_buttons}{here}.

\par


\par
 \label{RC_mhttpd_status_page_features_idx_mhttpd_buttons_script}
\hypertarget{RC_mhttpd_status_page_features_idx_mhttpd_buttons_script}{}
 \hypertarget{RC_mhttpd_status_page_features_RC_mhttpd_status_script_buttons}{}\subparagraph{Optional Script buttons}\label{RC_mhttpd_status_page_features_RC_mhttpd_status_script_buttons}
Script (or User) buttons that appear on the \hyperlink{RC_mhttpd_status_page_features_RC_mhpptd_optional_buttons}{main status page} are used to execute user-\/defined scripts. These buttons are defined through the optional ODB /script tree.

See
\begin{DoxyItemize}
\item \hyperlink{RC_mhttpd_defining_script_buttons}{Defining script buttons}
\end{DoxyItemize}

for details.

\par


\par
 \label{RC_mhttpd_status_page_features_idx_manual-trigger_button}
\hypertarget{RC_mhttpd_status_page_features_idx_manual-trigger_button}{}
 \hypertarget{RC_mhttpd_status_page_features_RC_mhttpd_status_Manual_Trigger_buttons}{}\subparagraph{Manual-\/Trigger Buttons}\label{RC_mhttpd_status_page_features_RC_mhttpd_status_Manual_Trigger_buttons}
See \hyperlink{FE_eq_event_routines_FE_manual_trigger}{Manual Trigger} .

\par


\par
 \hypertarget{RC_mhttpd_status_page_features_RC_mhttpd_status_Alias_buttons}{}\subparagraph{Alias-\/Buttons}\label{RC_mhttpd_status_page_features_RC_mhttpd_status_Alias_buttons}
User-\/defined {\bfseries Alias-\/buttons} that appear on the \hyperlink{RC_mhttpd_status_page_features_RC_mhpptd_optional_buttons}{main status page} give access to \hyperlink{RC_mhttpd_Alias_page}{Alias pages}.


\begin{DoxyItemize}
\item \hyperlink{RC_mhttpd_Alias_page_RC_mhttpd_alias_define}{How to create Alias-\/Buttons}
\end{DoxyItemize}

\par


\par
\hypertarget{RC_mhttpd_status_page_features_RC_mhttpd_status_Run_info}{}\subparagraph{Run status information}\label{RC_mhttpd_status_page_features_RC_mhttpd_status_Run_info}
\begin{center} mhttpd status page showing Run Status information  \end{center} \par


The run status information on the \hyperlink{RC_mhttpd_Main_Status_page_RC_mhttpd_main_status}{main status page} shows
\begin{DoxyItemize}
\item current run number
\item \hyperlink{RC_mhttpd_status_page_features_RC_table_run_state}{run state}
\item Alarm status
\item \hyperlink{F_Logging_Data_F_Logger_Auto_Restart}{Restart} (automatically restart run)
\item mlogger status
\item run duration
\end{DoxyItemize}

The appearance and contents of this information changes depending on the conditions. The images below demonstrate how the appearance may change when the run is in transition.

\begin{center} mhttpd status page showing Run Status information when the run is stopping  \end{center} \par


\begin{center} mhttpd status page showing Run Status information when the run is starting  \end{center} \par


\par


\par
 \hypertarget{RC_mhttpd_status_page_features_RC_Edit_RP}{}\subparagraph{Comment and Run Description}\label{RC_mhttpd_status_page_features_RC_Edit_RP}
Optionally, the user can define a \char`\"{}comment\char`\"{} and/or a \char`\"{}Run Description\char`\"{} that will appear on the mhttpd main status page. This is done by creating keys {\bfseries \char`\"{}Comment\char`\"{}} and/or {\bfseries \char`\"{}Run Description\char`\"{}} in the \hyperlink{RC_customize_ODB_RC_Run_Parameters}{Run Parameters subdirectory} under /Experiment. The contents of each key will then be displayed on an extra line on the mhttpd main status page. See \hyperlink{RC_customize_ODB_RC_ODB_Experiment_Tree}{The ODB /Experiment tree} for more information.


\begin{DoxyCode}
[local:t2kgas:S]/>ls -lt "/Experiment/Run Parameters/"
Key name                        Type    #Val  Size  Last Opn Mode Value
---------------------------------------------------------------------------
Comment                         STRING  1     32    19h  0   RWD   no beam, test 
      only
Run Description                 STRING  1     32    19h  0   RWD  28.2keV resonan
      t energy 7Li
\end{DoxyCode}


\par
 \begin{center} mhttpd main status page showing \char`\"{}Comment\char`\"{} and \char`\"{}Run Description\char`\"{} fields  \end{center}  \par


\par


\label{RC_mhttpd_status_page_features_idx_mhttpd_page_status_equipment}
\hypertarget{RC_mhttpd_status_page_features_idx_mhttpd_page_status_equipment}{}
 \hypertarget{RC_mhttpd_status_page_features_RC_mhttpd_status_Equipment_info}{}\subparagraph{Equipment information and Event rates}\label{RC_mhttpd_status_page_features_RC_mhttpd_status_Equipment_info}
The mhttpd status page contains a table of \hyperlink{FrontendOperation_FE_sw_equipment}{Equipment} information and event rates. Equipments are usually defined in \hyperlink{FrontendOperation_FE_features}{frontends}. Other MIDAS clients which may define Equipments include slow controls and eventbuilder clients.

\begin{center} mhttpd status page showing Equipment information and Event rate statistics  \end{center} \hypertarget{RC_mhttpd_status_page_features_RC_mhttpd_eq_variables}{}\subparagraph{Monitor the Equipment variables}\label{RC_mhttpd_status_page_features_RC_mhttpd_eq_variables}
The \char`\"{}Equipment\char`\"{} column of this table lists the names of any defined \hyperlink{FrontendOperation_FE_sw_equipment}{Equipments}. These appear in the order in which they are listed in the ODB \hyperlink{FE_ODB_equipment_tree}{/Equipment} tree.

The names of the equipment in this column are hyperlinks to their respective /Equipment/$<$equipment-\/name$>$/Variables sub-\/tree. Clicking on any of the equipment links will show an \hyperlink{RC_mhttpd_Equipment_page}{Equipment page} , allowing a shortcut for the user to access the current values of the equipment. \par


\par
\hypertarget{RC_mhttpd_status_page_features_RC_mhttpd_eq_status}{}\subparagraph{Status display of each Equipment}\label{RC_mhttpd_status_page_features_RC_mhttpd_eq_status}
The \char`\"{}Status\char`\"{} column of the \hyperlink{RC_mhttpd_status_page_features_RC_mhttpd_status_Equipment_info}{mhttpd status page} shows the status of each equipment. It usually shows the name of the client defining that equipment, and the host computer on which that client is running, The background colour of each equipment \char`\"{}Status\char`\"{} box will also change depending on the status of the associated frontend. The usual colours are shown in the following table:

\begin{center} \begin{table}[h]\begin{TabularC}{1}
\hline
Frontend is RUNNING and equipment is ENABLED  \\\cline{1-1}
Frontend is MISSING   \\\cline{1-1}
Frontend is RUNNING but equipment is DISABLED \\\cline{1-1}
\end{TabularC}
\centering
\caption{Default colour coding of Equipment status }
\end{table}
\par
 \end{center} 

When a run is in transition, or when a client takes a long time to respond, the status information may change to give a status report on the client. Optionally, users may program a client to send their own status reports that appear in this area of the mhttpd status page by incorporating calls to the routine {\itshape set\_\-equipment\_\-status\/} (see \hyperlink{FE_sequence_FE_frontend_status}{Reporting Equipment status}). This routine allows the message and the status box background colour to be specified. For example, the last client in the image above (HV\_\-SY2527) gives a Status of \char`\"{}OK\char`\"{} rather than the default client and hostname.

In versions prior to \hyperlink{NDF_ndf_dec_2009}{Dec 2009} , the \char`\"{}Status\char`\"{} column was labelled {\bfseries \char`\"{}FE Node\char`\"{}} and the client status information was not shown (see \hyperlink{RC_mhttpd_status_page_redesign}{Redesign of mhttpd Main Status Page} ).

\par


\par
 \label{RC_mhttpd_status_page_features_idx_mhttpd_page_status_event-rate}
\hypertarget{RC_mhttpd_status_page_features_idx_mhttpd_page_status_event-rate}{}
 \hypertarget{RC_mhttpd_status_page_features_RC_mhttpd_status_Event_Rates}{}\subparagraph{Event Rates}\label{RC_mhttpd_status_page_features_RC_mhttpd_status_Event_Rates}
The event statistics for the current run are also shown on the \hyperlink{RC_mhttpd_status_page_features_RC_mhttpd_status_Equipment_info}{main status page} , in the columns labelled {\bfseries \char`\"{}Events\char`\"{}}, {\bfseries \char`\"{}Events\mbox{[}/s\mbox{]}\char`\"{}} and {\bfseries \char`\"{}Data\mbox{[}MB/s\mbox{]}\char`\"{}}.

\par


\par
 \hypertarget{RC_mhttpd_status_page_features_RC_mhttpd_status_analyzer}{}\subparagraph{Number of events analyzed}\label{RC_mhttpd_status_page_features_RC_mhttpd_status_analyzer}
In versions prior to \hyperlink{NDF_ndf_dec_2009}{Dec 2009} , there is an extra column labelled \char`\"{}analyzer\char`\"{} which shows the number of events analyzed (valid only if the name of the analyzer is \char`\"{}Analyzer\char`\"{}).

\par


\par
\hypertarget{RC_mhttpd_status_page_features_RC_mhttpd_status_Logger}{}\subparagraph{Data Logging Information}\label{RC_mhttpd_status_page_features_RC_mhttpd_status_Logger}
The image below shows the information on the status page if both \hyperlink{F_Logging_F_mlogger_utility}{mlogger} and \hyperlink{F_LogUtil_F_lazylogger_utility}{lazylogger} are running.

\begin{center} logger information on mhttpd main status page \par
\par
\par
  \end{center}  \par


Compare this example with the \hyperlink{RC_mhttpd_utility_RC_mhttpd_minimal_status_page}{minimal} status page where neither of these clients are running.

\par
 In the image above,
\begin{DoxyItemize}
\item one mlogger channel (Channel 0) is active. \par
Multiple logger channels can be active, in which case a line for each channel would be shown. The hyperlink {\bfseries \char`\"{}0\char`\"{}} opens a \hyperlink{RC_mhttpd_Logger_page}{mhttpd Logger page} showing the settings information.
\item one lazylogger channel ({\bfseries Dcache} ) is also active. Multiple lazy applications can be active, in which case multiple lines of Lazy information would be present. Clicking on the hyperlink {\bfseries \char`\"{}Dcache\char`\"{}} opens a \hyperlink{RC_mhttpd_Logger_page}{mhttpd Logger page} showing the \hyperlink{RC_mhttpd_Logger_page_RC_mhttpd_Logger_lazylogger}{lazylogger settings information} .
\end{DoxyItemize}

\par


\par
 \label{RC_mhttpd_status_page_features_idx_message_last}
\hypertarget{RC_mhttpd_status_page_features_idx_message_last}{}
 \hypertarget{RC_mhttpd_status_page_features_RC_mhttpd_status_latest_msg}{}\subparagraph{Last system message}\label{RC_mhttpd_status_page_features_RC_mhttpd_status_latest_msg}
\begin{center} Example of last system message on mhttpd main status page \par
\par
\par
  \end{center}  \par


The last system message to be received at the time of the last display refresh is displayed on the \hyperlink{RC_mhttpd_Main_Status_page_RC_mhttpd_main_status}{main status page} (see \hyperlink{F_Messaging}{Messaging}). More messages can be viewed by pressing the \hyperlink{RC_mhttpd_status_page_features_RC_mhttpd_status_Page_buttons}{Message button}. This opens the \hyperlink{RC_mhttpd_Message_page}{Message page}.

\par


\par


\label{RC_mhttpd_status_page_features_idx_clients_active_mhttpd}
\hypertarget{RC_mhttpd_status_page_features_idx_clients_active_mhttpd}{}
 \hypertarget{RC_mhttpd_status_page_features_RC_mhttpd_status_clients}{}\subparagraph{Active Client list}\label{RC_mhttpd_status_page_features_RC_mhttpd_status_clients}
\begin{center} Example of Active client list on mhttpd main status page \par
\par
\par
  \end{center}  \par


At the bottom of the \hyperlink{RC_mhttpd_Main_Status_page_RC_mhttpd_main_status}{main status page} is a list of the MIDAS clients for this experiment that are currently active. The hostname is also shown. This information is derived from the \hyperlink{RC_Run_States_and_Transitions_RC_odb_system_tree}{ODB /System} tree .

\par
\par


 \par
 \label{index_end}
\hypertarget{index_end}{}
 \paragraph{Defining Script Buttons on the main Status Page}\label{RC_mhttpd_defining_script_buttons}
\par




\label{RC_mhttpd_defining_script_buttons_idx_ODB_tree_Script}
\hypertarget{RC_mhttpd_defining_script_buttons_idx_ODB_tree_Script}{}
 \hypertarget{RC_mhttpd_defining_script_buttons_RC_odb_script_tree}{}\subparagraph{The ODB /Script tree}\label{RC_mhttpd_defining_script_buttons_RC_odb_script_tree}
\begin{DoxyNote}{Note}
The /Script tree is applicable to \hyperlink{RC_mhttpd}{mhttpd}, and ignored by \hyperlink{RC_odbedit}{odbedit}.
\end{DoxyNote}
The optional ODB tree /Script provides the user with a way to execute a script when a button on the mhttpd \hyperlink{RC_mhttpd_Main_Status_page_RC_mhttpd_main_status}{main status page} is clicked, including the {\bfseries capability of passing \hyperlink{structparameters}{parameters} from the ODB to the script}.

\par
 If the user defines a new tree in ODB named /Script , then any key created in this tree will appear as a script-\/button of that name on the default mhttpd main status page. Each sub-\/tree ( /Script/$<$button name$>$/) should contain at least one string key which is the script command to be executed. Further keys will be passed as {\bfseries  arguments } to the script. MIDAS symbolic links are permitted.\hypertarget{RC_mhttpd_defining_script_buttons_RC_odb_script_example1}{}\subparagraph{Example 1: creation of a Script-\/button; parameters passed to the associated script}\label{RC_mhttpd_defining_script_buttons_RC_odb_script_example1}
The {\bfseries  example } below shows the ODB /script/dac subdirectory. The script-\/button {\bfseries \char`\"{}dac\char`\"{}} associated with this subdirectory is shown on the example mhttpd status page below.

The first key in the dac subdirectory is the string key cmd which contains the name and path of the script to be executed (in this case, a perl script). This script is located on the local host computer on which the experiment is running. The subsequent keys are \hyperlink{structparameters}{parameters} input to the script. 
\begin{DoxyCode}
[local:pol:R]/>ls "/script/dac"
cmd                             /home/pol/online/perl/change_mode.pl
include path                    /home/pol/online/perl
experiment name -> /experiment/name
                                pol
select mode                     1h

mode file tag                   none
[local:pol:R]/>  
\end{DoxyCode}


This will cause a script-\/button labelled {\bfseries \char`\"{}DAC\char`\"{}} to appear on the mhttpd main status page : \par
 \begin{center} Script button \char`\"{}DAC\char`\"{} on the mhttpd main status page  \end{center} \par


When the {\bfseries \char`\"{}DAC\char`\"{}} script-\/button is pressed, the script {\bfseries \char`\"{}change\_\-mode.pl\char`\"{}} will be executed with the following key contents as \hyperlink{structparameters}{parameters}, equivalent to the command: 
\begin{DoxyCode}
  "/home/pol/online/perl/change_mode.pl  /home/pol/online/perl pol 1h mode"
\end{DoxyCode}
 \par


The following is part of the code of the script {\bfseries \char`\"{}change\_\-mode.pl\char`\"{}} : 
\begin{DoxyCode}
# input parameters :

our ($inc_dir, $expt, $select_mode, $mode_name ) = @ARGV;
our $len = $#ARGV; # array length
our $name = "change_mode" ; # same as filename
our $outfile = "change_mode.txt"; # path will be added by file open function
our $parameter_msg = "include path , experiment , select_new_mode  mode_name";
our $nparam = 4;  # no. of input parameters
our $beamline = $expt; # beamline is not supplied. Same as $expt for bnm/qr, pol
############################################################################
# local variables:
my ($transition, $run_state, $path, $key, $status);

# Inc_dir needed because when script is invoked by browser it can't find the
# code for require

unless ($inc_dir) { die "$name: No include directory path has been supplied\n";}
$inc_dir =~ s/\/$//;  # remove any trailing slash
require "$inc_dir/odb_access.pl";
require "$inc_dir/do_link.pl";

# init_check.pl checks:
#   one copy of this script running
#   no. of input parameters is correct
#   opens output file:
#
require "$inc_dir/init_check.pl"; 

# Output will be sent to file $outfile (file handle FOUT)
# because this is for use with the browser and STDOUT and STDERR get set to null


print FOUT  "$name starting with parameters:  \n";
print FOUT  "Experiment = $expt, select new mode = $select_mode;  load file mode_
      name=$mode_name \n";

unless ($select_mode)
{
    print FOUT "FAILURE: selected mode  not supplied\n";
        odb_cmd ( "msg","$MERROR","","$name", "FAILURE:  selected mode not suppli
      ed " ) ;
        unless ($status) { print FOUT "$name: Failure return after msg \n"; }
        die  "FAILURE:  selected mode  not supplied \n";

}
unless ($select_mode =~/^[12]/)
{
    print_3 ($name,"FAILURE: invalid selected mode ($select_mode)",$MERROR,1);
}

etc.
\end{DoxyCode}
\hypertarget{RC_mhttpd_defining_script_buttons_RC_odb_script_example2}{}\subparagraph{Example 2: MPET experiment run controller}\label{RC_mhttpd_defining_script_buttons_RC_odb_script_example2}
This example is from the MPET experiment at TRIUMF, which uses a sophisticated run controller. This includes perlscripts actived by script buttons. The experiment can be set to perform a number of consecutive runs without user intervention, changing some condition(s) between each run. The results are written to a log file.

It involves the use of large number of script-\/buttons on the Main Status page to activate the perlscripts (see Figure 1). Clicking on one of these buttons causes a user-\/defined shell-\/script to be run with a particular parameter.

\par
\par
\par
 \begin{center} Figure 1 Main Status page of MPET experiment   \end{center}  \par
\par
\par


This experiment is using an older version of mhttpd (see \hyperlink{RC_mhttpd_status_page_redesign}{Redesign of mhttpd Main Status Page} ).

The script-\/buttons are defined in the ODB /Script tree (see Figure 2). All activate the shell-\/script perlrc.sh with the appropriate parameter. The first two script-\/buttons labelled \char`\"{}Start PerlRC\char`\"{} and \char`\"{}Stop PerlRC\char`\"{} start and stop the run control respectively. These access \hyperlink{structparameters}{parameters} read from the ODB to determine the scan type, the number of runs to be performed, etc. The other buttons \char`\"{}Tune...\char`\"{} are used to set up run \hyperlink{structparameters}{parameters} into particular known states or \char`\"{}Tunes\char`\"{}.

\par
 \par
\par
\par
 \begin{center} Figure 2 /Script ODB tree for the MPET experiment   \end{center}  \par
\par
\par


This script calls the perlscript perlrc.pl, passing through the parameter. (Alternatively, this could have been done by \hyperlink{RC_mhttpd_defining_script_buttons_RC_odb_script_tree}{passing the parameter} directly to the perlscript, eliminating the intermediate shell-\/script).

The following image shows the ODB \hyperlink{structparameters}{parameters} associated with the run control script buttons.

\par
\par
\par
 \begin{center} Run Control ODB \hyperlink{structparameters}{parameters} for the MPET experiment   \end{center}  \par
\par
\par
 
\begin{DoxyItemize}
\item Clicking on ODB...PerlRC...RunControl...Scan2D shows the RunControl Parameters 
\item Clicking on ODB...PerlRC...RunControl...TuneSwitch shows the Tuning Parameters 
\end{DoxyItemize}

\par
 

 \par
\subparagraph{Examples of MPET Perlscripts for run control}\label{RC_mhttpd_perlrc}
\par


 \label{RC_mhttpd_perlrc_idx_script_perlmidas}
\hypertarget{RC_mhttpd_perlrc_idx_script_perlmidas}{}


Part of the run control perlscripts for MPET experiment at TRIUMF (written by Vladimir Rykov) are reproduced below. The script \hyperlink{RC_mhttpd_perlrc_RC_mhttpd_perlrc_script}{perlrc.pl} calls a script called \hyperlink{RC_mhttpd_perlrc_RC_mhttpd_perlmidas_script}{perlmidas.pl} to access the ODB.

\hyperlink{RC_mhttpd_perlrc_RC_mhttpd_perlmidas_script}{perlmidas.pl} may be of interest to users who wish to interact with the ODB through scripts.\hypertarget{RC_mhttpd_perlrc_RC_mhttpd_perlmidas_script}{}\subparagraph{perlmidas.pl}\label{RC_mhttpd_perlrc_RC_mhttpd_perlmidas_script}

\begin{DoxyCode}
# common subroutines
use strict;
use warnings;
##############################################################
sub MIDAS_env
# set up proper MIDAS environment...
##############################################################
{
    our ($MIDAS_HOSTNAME,$MIDAS_EXPERIMENT,$ODB_SUCCESS,$DEBUG);
    our ($CMDFLAG_HOST, $CMDFLAG_EXPT);

    $ODB_SUCCESS=0;

    $MIDAS_HOSTNAME = $ENV{"MIDAS_SERVER_HOST"};
    if (defined($MIDAS_HOSTNAME) &&   $MIDAS_HOSTNAME ne "")
    {
        $CMDFLAG_HOST = "-h $MIDAS_HOSTNAME";
    }
    else
    {
        $MIDAS_HOSTNAME = "";
        $CMDFLAG_HOST = "";
    }

    $MIDAS_EXPERIMENT = $ENV{"MIDAS_EXPT_NAME"};
    if (defined($MIDAS_EXPERIMENT) &&   $MIDAS_EXPERIMENT ne "")
    {
        $CMDFLAG_EXPT = "-e ${MIDAS_EXPERIMENT}";
    }
    else
    {
        $MIDAS_EXPERIMENT = "";
        $CMDFLAG_EXPT = "";
    }

}


##############################################################
sub MIDAS_sendmsg
##############################################################
{
# send a message to odb message logger
    my ($name, $message) =  @_;

    our ($MIDAS_HOSTNAME,$MIDAS_EXPERIMENT,$ODB_SUCCESS,$DEBUG);
    our ($CMDFLAG_HOST, $CMDFLAG_EXPT);
    our ($COMMAND, $ANSWER);

    my $status;
    my $host="";
    my $dquote='"';
    my $squote="'";
    my $command="${dquote}msg ${name} ${squote}${message}${squote}${dquote}";
    print "name=$name, message=$message\n";
    print "command is: $command \n";

    $COMMAND ="`odb ${CMDFLAG_EXPT} ${CMDFLAG_HOST} -c ${command}`";
    $ANSWER=`odb ${CMDFLAG_EXPT} ${CMDFLAG_HOST} -c ${command}`;
    $status=$?;
    chomp $ANSWER;  # strip trailing linefeed
    if($DEBUG)
    {
        print "command: $COMMAND\n";
        print " answer: $ANSWER\n";
    }

    if($status != $ODB_SUCCESS) 
    { # this status value is NOT the midas status code
        print "send_message:  Failure status returned from odb msg (status=$statu
      s)\n";
    }
    return;
}

sub strip
{
# removes / from end of string, // becomes /
    my $string=shift;
    $string=~ (s!//!/!g);
    $string=~s!/$!!;
    print "strip: now \"$string\"\n";
    return ($string);
}

sub MIDAS_varset
##############################################################
{
# set a value of an odb key
    my ($key, $value) =  @_;

    our ($MIDAS_HOSTNAME,$MIDAS_EXPERIMENT,$ODB_SUCCESS,$DEBUG);
    our ($CMDFLAG_HOST, $CMDFLAG_EXPT);
    our ($COMMAND, $ANSWER);

    my $status;
    my $host="";
    my $dquote='"';
    my $squote="'";
    my $command="${dquote}set ${squote}${key}${squote} ${squote}${value}${squote}
      ${dquote}";
    print "key=$key, new value=${value}\n";
    print "command is: $command \n";

    $COMMAND ="`odb ${CMDFLAG_EXPT} ${CMDFLAG_HOST} -c command`";
    $ANSWER=`odb ${CMDFLAG_EXPT} ${CMDFLAG_HOST} -c $command `;
    $status=$?;
    chomp $ANSWER;  # strip trailing linefeed
    if($DEBUG)
    {
        print "command: $COMMAND\n";
        print " answer: $ANSWER\n";
    }

    if($status != $ODB_SUCCESS) 
    { # this status value is NOT the midas status code
        print "send_message:  Failure status returned from odb msg (status=$statu
      s)\n";
    }
    return;
}

sub MIDAS_varget
##############################################################
{
# set a value of an odb key
    my ($key) =  @_;

    our ($MIDAS_HOSTNAME,$MIDAS_EXPERIMENT,$ODB_SUCCESS,$DEBUG);
    our ($CMDFLAG_HOST, $CMDFLAG_EXPT);
    our ($COMMAND, $ANSWER);

    my $status;
    my $host="";
    my $dquote='"';
    my $squote="'";
    my $command="${dquote}ls -v ${squote}${key}${squote}${dquote}";
    print "key=$key\n";
    print "command is: $command \n";
    
    $COMMAND ="`odb ${CMDFLAG_EXPT} ${CMDFLAG_HOST} -c command`";
    $ANSWER=`odb ${CMDFLAG_EXPT} ${CMDFLAG_HOST} -c $command `;  
    $status=$?;
    chomp $ANSWER;  # strip trailing linefeed
    if($DEBUG)
    {
        print "command: $COMMAND\n";
        print " answer: $ANSWER\n";
    }

    if($status != 0) 
    { # this status value is NOT the midas status code
        print "send_varset  Failure status returned from odb msg (status=$status)
      \n";
    }
    return $ANSWER;
}

sub MIDAS_dirlist
##############################################################
{
# return a directory list of directory given by odb key
    my ($key) =  @_;

    our ($MIDAS_HOSTNAME,$MIDAS_EXPERIMENT,$ODB_SUCCESS,$DEBUG);
    our ($CMDFLAG_HOST, $CMDFLAG_EXPT);
    our ($COMMAND, $ANSWER);

    my $status;
    my $host="";
    my $dquote='"';
    my $squote="'";
    my $command="${dquote}ls ${squote}${key}${squote}${dquote}";
    print "key=$key\n";
    print "command is: $command \n";
    
    $COMMAND ="`odb ${CMDFLAG_EXPT} ${CMDFLAG_HOST} -c command`";
    $ANSWER=`odb ${CMDFLAG_EXPT} ${CMDFLAG_HOST} -c $command `;  
    $status=$?;
    chomp $ANSWER;  # strip trailing linefeed
    if($DEBUG)
    {
        print "command: $COMMAND\n";
        print " answer: $ANSWER\n";
    }

    if($status != 0) 
    { # this status value is NOT the midas status code
        print "send_varset  Failure status returned from odb msg (status=$status)
      \n";
    }
    return $ANSWER;
}

sub MIDAS_startrun
##############################################################
{
# start MIDAS run
    my ($key) =  @_;

    our ($MIDAS_HOSTNAME,$MIDAS_EXPERIMENT,$ODB_SUCCESS,$DEBUG);
    our ($CMDFLAG_HOST, $CMDFLAG_EXPT);
    our ($COMMAND, $ANSWER);

    our ($SCANLOG_FH);

    my $status;
    my $host="";
    my $dquote='"';
    my $squote="'";
    my $command="${dquote}start now${dquote}";
    print "command is: $command \n";

    #sleep(10);

    $COMMAND ="`odb ${CMDFLAG_EXPT} ${CMDFLAG_HOST} -c ${command}`";
    $ANSWER=`odb ${CMDFLAG_EXPT} ${CMDFLAG_HOST} -c ${command}`;
    $status=$?;
    chomp $ANSWER;  # strip trailing linefeed
    if($DEBUG)
    {
        print "command: $COMMAND\n";
        print " answer: $ANSWER\n";

        #print $SCANLOG_FH "status: $status\n";
        #print $SCANLOG_FH "command: $COMMAND\n";
        #print $SCANLOG_FH " answer: $ANSWER\n";

    }

    if($status != 0)
    { # this status value is NOT the midas status code
        print "startrun:  Failure status returned from odb msg (status=$status)\n
      ";
        print $SCANLOG_FH " answer: $ANSWER\n";

    }
    return $ANSWER;
}   
1;
\end{DoxyCode}


\par


\par
\hypertarget{RC_mhttpd_perlrc_RC_mhttpd_perlrc_script}{}\subparagraph{perlrc.pl}\label{RC_mhttpd_perlrc_RC_mhttpd_perlrc_script}

\begin{DoxyCode}
 #!/usr/bin/perl

################################################################
#
#  PerlRC
#
#  MIDAS piggyback perl script that is exectuted upon completion
#  of a run. It checks its parameters, modifies the MIDAS variables
#  as required, and starts a new run. This way it can run through
#  different DAQ settings. Implemented scans:
#  1) Scan1D - scans a set of variables from beginning values
#     to ending values. All valiables are changed simultaneously.
#  2) Scan2D - scans 2 sets of variables.
#  3) SettingsSwitch - switches between different settings sets
#     typically to be used to switch between ion species.
#
#  V. Ryjkov
#  June 2008
#
################################################################

require "/home/mpet/vr/perl/PerlRC/perlmidas.pl";

our $DEBUG = true;
our $PERLSCAN_PREF = "/PerlRC";
our $PERLSCAN_CONTROLVARS = $PERLSCAN_PREF . "/ControlVariables";
our $PERLSCAN_START = $PERLSCAN_PREF . "/RunControl/RCActive";
our $PERLSCAN_NRUNS = $PERLSCAN_PREF . "/RunControl/RCTotalRuns";
our $PERLSCAN_CURRUN = $PERLSCAN_PREF . "/RunControl/RCCurrentRun";
our $SCANLOG_PATH = "/data/mpet/PerlRC.log";
our $SCANLOG_FH;
our $MIDAS_RUNNO = "/Runinfo/Run number";
my  $PERLSCAN_SCANTYPE = $PERLSCAN_PREF . "/RunControl/RCType";

MIDAS_env();
# MIDAS_sendmsg("test","run stop");
my $ScanStart  =MIDAS_varget($PERLSCAN_START);
my $ScanType   =MIDAS_varget($PERLSCAN_SCANTYPE);
my $NRuns      =MIDAS_varget($PERLSCAN_NRUNS);
my $CurrentRun =MIDAS_varget($PERLSCAN_CURRUN);
my $retval;
my $MIDASrunno;

open(SCANLOG,">>${SCANLOG_PATH}");
$SCANLOG_FH=\*SCANLOG;

if(scalar(@ARGV)==1 && $ARGV[0] =~ /start/) {
    MIDAS_varset($PERLSCAN_START,'y');
    $ScanStart = "y";
}
if(scalar(@ARGV)==1 && $ARGV[0] =~ /stop/) {
    MIDAS_varset($PERLSCAN_START,'n');
    $ScanStart = "n";
}
if( $ScanStart eq "y") {
    if( $CurrentRun == 0) {
        print $SCANLOG_FH "=== NEW PerlRC scan. Scan type is \"${ScanType}\" ===\
      n";
        print $SCANLOG_FH "===    Number of runs in this scan is ${NRuns}    ===\
      n";
    }
    if( $CurrentRun == $NRuns) {
        print $SCANLOG_FH "=== Finished PerlRC scan ===\n";
        print $SCANLOG_FH "============================\n";
    }
    if( ++$CurrentRun <= $NRuns ) {
        $MIDASrunno=MIDAS_varget($MIDAS_RUNNO);
        $MIDASrunno++;
        print $SCANLOG_FH "<Run #${MIDASrunno}> ";
        MIDAS_varset($PERLSCAN_CURRUN,$CurrentRun);
        for ($ScanType) {
            if    (/Scan1D/)   {$retval=Scan1D(); }     # do something
            elsif (/Scan2D/)   {$retval=Scan2D(); }     # do something else
            elsif (/TuneSwitch/) {$retval=TuneSwitch(); } # do something else
        }
        if($retval != 0) {
            MIDAS_varset($PERLSCAN_CURRUN,0);
            MIDAS_varset($PERLSCAN_START,"n");
            print $SCANLOG_FH "!!!#### Aborting scan! ####!!!\n";
        }
        else {
            sleep(1);
            #print $SCANLOG_FH "pausing 10 sec...\n";
            MIDAS_startrun();
            #print $SCANLOG_FH "start the run\n";
        }
    }
    else {
        MIDAS_varset($PERLSCAN_CURRUN,0);
        MIDAS_varset($PERLSCAN_START,"n");
    }
}
else {
    if(scalar(@ARGV)==2 && $ARGV[0] =~ /tune/) {
        SwitchToTune($ARGV[1]);
    }
}
close(SCANLOG);

sub Scan1D
{

    ............


}    


sub SetControlVar
{
    our $SCANLOG_FH;
    our $PERLSCAN_CONTROLVARS;
    my ($varname,$varvalue)=@_;
    my $retval;
    my $varpath;

    #print $SCANLOG_FH "variablename: $varname \n";

    $varpath=MIDAS_varget($PERLSCAN_CONTROLVARS . "/" . $varname);
    if($varpath =~ /^key (.*) not found/) {
        print $SCANLOG_FH "! Control variable ${varname}(${1}) is not listed in $
      {PERLSCAN_CONTROLVARS}\n";
        return -4;
    }

    .............
    
    
    val=MIDAS_varset($varpath,$varvalue);
        if($retval =~ /^key not found/) {return -5;}
    }
    return 0;
}

sub SwitchToTune
{
    our $SCANLOG_FH;
    our $PERLSCAN_CONTROLVARS;
    our $PERLSCAN_PREF;
    my $PERLSCAN_TUNEDIR = $PERLSCAN_PREF . "/Tunes";
    my ($tunename)=@_;
    my $retval;
    my $varpath;
    my $varval;
    my $cvarname;

    $retval = MIDAS_dirlist($PERLSCAN_TUNEDIR . "/" . $tunename);
    if($retval =~ /^key not found/){
        print $SCANLOG_FH "! Could not locate tune ${tunename} in the tune direct
      ory ${PERLSCAN_TUNEDIR}\n";
        return -7;
    }
    my @TuneVars=split(/\n/,$retval);
    foreach (@TuneVars) {
        if (/^(.+\S)\s{2,}.*/) {
            $cvarname = $1;
            $varval = MIDAS_varget($PERLSCAN_TUNEDIR . "/" . $tunename . "/" .$cv
      arname);
            $retval = SetControlVar($cvarname, $varval);
            if($retval < 0) {return $retval;}
        }
        else {
            print $SCANLOG_FH "! Cannot decipher tune variable list, offending li
      ne: $_\n";
            return -8;
        }
        sleep(1);
    }
    return 0;
}

sub Scan2D
{
   .................
}


sub TuneSwitch
{   
    our  ($PERLSCAN_PREF, $PERLSCAN_START);
    our $SCANLOG_FH;
    my $PERLSCAN_TUNESWITCHDIR = "/RunControl/TuneSwitch";
    my $PERLSCAN_TUNESLIST = $PERLSCAN_PREF . $PERLSCAN_TUNESWITCHDIR .  "/TunesL
      ist";
    my $PERLSCAN_TUNEIDX = $PERLSCAN_PREF . $PERLSCAN_TUNESWITCHDIR .  "/CurrentT
      uneIndex";
    my $PERLSCAN_TUNENAME = $PERLSCAN_PREF . $PERLSCAN_TUNESWITCHDIR .  "/Current
      TuneName";

    my $TunesList = MIDAS_varget($PERLSCAN_TUNESLIST);
    my $TuneIdx = MIDAS_varget($PERLSCAN_TUNEIDX);
    my $TuneName = MIDAS_varget($PERLSCAN_TUNENAME);
    
    my @tunes = split(/\s*;\s*/,$TunesList);
    print "tunes length= ",scalar(@tunes),"\n";
    if( ++$TuneIdx > scalar(@tunes) ) {
            $TuneIdx=1;
    }
    MIDAS_varset($PERLSCAN_TUNEIDX,$TuneIdx);

    $retval=SwitchToTune($tunes[$TuneIdx-1]);
    if($retval < 0) {return $retval;}
    MIDAS_varset($PERLSCAN_TUNENAME,$tunes[$TuneIdx-1]);
    print $SCANLOG_FH "Tune is \"",$tunes[$TuneIdx-1],"\"\n";
    return 0;
}
\end{DoxyCode}


 \par
 \label{index_end}
\hypertarget{index_end}{}
 \subsubsection{Redesign of mhttpd Main Status Page}\label{RC_mhttpd_status_page_redesign}


\label{RC_mhttpd_status_page_redesign_idx_mhttpd_page_status}
\hypertarget{RC_mhttpd_status_page_redesign_idx_mhttpd_page_status}{}


\par
 \label{RC_mhttpd_status_page_redesign_alias_buttons_status_page}
\hypertarget{RC_mhttpd_status_page_redesign_alias_buttons_status_page}{}
  The appearance of the Main Status Page has been changed (in versions after to \hyperlink{NDF_ndf_dec_2009}{Dec 2009}) . 
\begin{DoxyItemize}
\item the {\bfseries \char`\"{}Analyzed\char`\"{}} column has been dropped 
\item the {\bfseries \char`\"{}FE Node\char`\"{}} column is now labelled {\bfseries \char`\"{}Status\char`\"{}} and may show equipment status information 
\item the user-\/defined \hyperlink{RC_mhttpd_Alias_page_RC_mhttpd_alias_buttons}{alias}, \hyperlink{RC_mhttpd_status_page_features_RC_mhttpd_status_script_buttons}{script} and \char`\"{}custom\char`\"{} hyperlinks now appear as {\bfseries buttons} 
\item there are now four different background colors (\hyperlink{NDF_ndf_jan_2010}{Jan 2010}) for the four rows of buttons, i.e. : 
\begin{DoxyItemize}
\item Main menu buttons, 
\item Script buttons, 
\item Manually triggered events, 
\item Alias \& Custom page buttons. 
\end{DoxyItemize}

This change has been made because the original alias hyperlinks were hard to read if they included spaces. It also gives a more homogeneous look to the page. A status page of the new format is shown here. 
\end{DoxyItemize}

\par
 \begin{center} New format Status page showing four rows of buttons \par
\par
\par
  \end{center}  \par
\par
\par


\label{RC_mhttpd_status_page_redesign_RC_mhttpd_old_alias_buttons}
\hypertarget{RC_mhttpd_status_page_redesign_RC_mhttpd_old_alias_buttons}{}
  Prior to \hyperlink{NDF_ndf_dec_2009}{Dec 2009}  the Custom Page and Alias hyperlinks appeared as {\bfseries links} rather than buttons, as shown below: \par
\par
\par
 \begin{center} Older version of mhttpd main Status page showing Custom Page and Alias {\bfseries Links}  \end{center}  \par
\par
\par




\label{index_end}
\hypertarget{index_end}{}
 \par
 \paragraph{Start page}\label{RC_mhttpd_Start_page}
\label{RC_mhttpd_Start_page_idx_mhttpd_page_start}
\hypertarget{RC_mhttpd_Start_page_idx_mhttpd_page_start}{}
 \par




\par


To start a run, the {\bfseries Start} button (\hyperlink{RC_mhttpd_status_page_features_RC_mhttpd_status_menu_buttons}{if present}) is pressed, and the user will be prompted for any defined experiment-\/specific \hyperlink{structparameters}{parameters} (i.e. \hyperlink{RC_customize_ODB_RC_Edit_On_Start}{Edit on start} \hyperlink{structparameters}{parameters}) before starting the run. The minimum set of \hyperlink{structparameters}{parameters} is the {\bfseries  run number }, it will be incremented by one relative to the last value from the status page. The user may edit the run number except \hyperlink{RC_mhttpd_Start_page_RC_Prevent_Edit_RN}{where noted} before continuing.

\par
\par
\par
 \begin{center}  Start run request page. In this case the user defined no run \hyperlink{structparameters}{parameters}. \par
\par
\par
  \end{center}  \par
\par
\par


Once the {\bfseries Start} {\bfseries button} is pressed on the Start request page, the system will attempt to start the run.

 Since \hyperlink{NDF_ndf_jun_2009}{Jun 2009} , the mhttpd function for starting and stopping runs now spawns an external helper program to handle the transition sequencing. This helps with the old problem of looking at a blank screen for a long time if some frontends take a long time to process run transitions. mhttpd now returns immediately with a message {\bfseries Run start/stop requested} until it detects that the helper program is started and sets a message {\bfseries runinfo/transition in progress} (until/unless the run state has changed). Some aspects of this feature are present since rev 4473. 

\par


\par
\hypertarget{RC_mhttpd_Start_page_RC_mhttpd_Edit_On_Start}{}\subparagraph{Run Start with defined /Experiment/Edit on Start tree}\label{RC_mhttpd_Start_page_RC_mhttpd_Edit_On_Start}
If the user has defined \hyperlink{RC_customize_ODB_RC_Edit_On_Start}{Edit on Start parameters}, when a run is started all the \hyperlink{structparameters}{parameters} in the ODB tree {\bfseries  /Experiment/Edit on Start } will be displayed  for possible modification.
\begin{DoxyItemize}
\item Pressing the {\bfseries Ok} button will proceed to start of the run.
\item Pressing the {\bfseries Cancel} button will abort the start procedure, and return to the status page.
\end{DoxyItemize}

\par
\par
\par
 \begin{center} Start run request page. In this case the user has multiple \hyperlink{RC_customize_ODB_RC_Edit_On_Start}{Edit-\/on-\/start parameters} defined. \par
\par
\par
  \end{center}  \par
\par
\par


The title of each field is the ODB key name itself. If the keyname is not self-\/explanatory, more explanation can be supplied by creating \hyperlink{RC_mhttpd_Start_page_RC_Edit_PC}{Edit-\/on-\/start Parameter Comments} .

\par


\par
\hypertarget{RC_mhttpd_Start_page_RC_EOS_web_features}{}\subparagraph{Features available for mhttpd only}\label{RC_mhttpd_Start_page_RC_EOS_web_features}
Several extra {\bfseries Edit on start} features are available when using mhttpd to start the run, which are ignored by odbedit. These are described below:


\begin{DoxyItemize}
\item \hyperlink{RC_mhttpd_Start_page_RC_Edit_PC}{Edit-\/on-\/start Parameter Comments}
\item \hyperlink{RC_mhttpd_status_page_features_RC_Edit_RP}{Comment and Run Description}
\item \hyperlink{RC_mhttpd_Start_page_RC_Prevent_Edit_RN}{Prevent the run number being edited at Run Start}
\end{DoxyItemize}

\par


\par
\hypertarget{RC_mhttpd_Start_page_RC_Edit_PC}{}\subparagraph{Edit-\/on-\/start Parameter Comments}\label{RC_mhttpd_Start_page_RC_Edit_PC}
An optional subdirectory {\bfseries Parameter Comments} can be created under \hyperlink{RC_customize_ODB_RC_ODB_Experiment_Tree}{The ODB /Experiment tree} to display some extra text on the Start page under an {\bfseries Edit on start} parameter. Usually the parameter names are self-\/explanatory, but the parameter name may not contain enough information. In this case, a {\bfseries parameter comment} can be created by the user.

This \char`\"{}parameter comment\char`\"{} option is visible {\bfseries ONLY} under the MIDAS web page ({\bfseries mhttpd} ), the {\bfseries  odbedit start } command will not display this extra information.

The name of the parameter in the {\bfseries Parameter Comments} subdirectory must match that of the {\bfseries Edit on Start} parameter. Comments may contain html tags if desired. \par
 \begin{DoxyNote}{Note}
If the parameter in Edit-\/on-\/start is a link which is named differently from the actual parameter, then the parameter name in {\bfseries Parameter Comments} must match the name of the actual parameter, rather than the link-\/name.
\end{DoxyNote}
This is illustrated below where the {\bfseries Edit on Start} parameter is a link named {\bfseries \char`\"{}number of channels\char`\"{}}, which links to the actual parameter {\bfseries  /sis/nchannels }. 
\begin{DoxyCode}
number of channels              LINK    1     15    22m  0   RWD  /sis/nchannels
\end{DoxyCode}
 The parameter name in {\bfseries Parameter Comments } for this parameter is nchannels STRING 1 64 14m 0 RWD {\itshape maximum 1024\/} and NOT \char`\"{}number of channels\char`\"{}.


\begin{DoxyCode}
[local:midas:S]/Experiment>ls -lr
Key name                        Type   #Val  Size  Last Opn Mode Value
---------------------------------------------------------------------------
Experiment                      DIR
    Name                        STRING  1     32    17s  0   RWD  midas
    Edit on Start               DIR
        Write data              BOOL    1     4     16m  0   RWD  y
        enable                  BOOL    1     4     16m  0   RWD  n
        nchannels               INT     1     4     16m  0   RWD  0
        dwelling time (ns)      INT     1     4     16m  0   RWD  0


    Parameter Comments          DIR
        Write Data              STRING  1     64    44m  0   RWD  Enable logging
        enable                  STRING  1     64    7m   0   RWD  Scaler for expt
       B1 only
        nchannels               STRING  1     64    14m  0   RWD  <i>maximum 1024
      </i>
        dwelling time (ns)      STRING  1     64    8m   0   RWD  <b>Check hardwa
      re now</b>

[local:midas:S]Edit on Start>ls -l

Key name                        Type   #Val  Size  Last Opn Mode Value
---------------------------------------------------------------------------
Write Data                      LINK    1     19    50m  0   RWD  /logger/Write d
      ata
enable                          LINK    1     12    22m  0   RWD  /sis/enable
number of channels              LINK    1     15    22m  0   RWD  /sis/nchannels
dwelling time (ns)              LINK    1     24    12m  0   RWD  /sis/dwelling t
      ime (ns)
\end{DoxyCode}


This results in a start run page as shown below.

\label{RC_mhttpd_Start_page_RC_param_comment_example}
\hypertarget{RC_mhttpd_Start_page_RC_param_comment_example}{}


\par
\par
\par
 \begin{center}  Start run request page. Extra comment on the run condition is displayed below each entry. \par
\par
\par
  \end{center}  \par
\par
\par




\par
\hypertarget{RC_mhttpd_Start_page_RC_Prevent_Edit_RN}{}\subparagraph{Prevent the run number being edited at Run Start}\label{RC_mhttpd_Start_page_RC_Prevent_Edit_RN}
\label{RC_mhttpd_Start_page_RC_Edit_RN}
\hypertarget{RC_mhttpd_Start_page_RC_Edit_RN}{}


By default, the user has the option to edit the run number at begin of run. To prevent this, the user may add an optional key {\bfseries Edit run number} to the Edit on Start subdirectory in the \hyperlink{RC_customize_ODB_RC_ODB_Experiment_Tree}{The ODB /Experiment tree}. If this key is set to \char`\"{}N\char`\"{}, the user will not be able to edit the run number on the mhttpd start page at the begin of run.

This feature is required where the run number is strictly controlled with a custom run number checking system that assigns the run number automatically based on the type of run.

By creating the key /Experiment/Edit on Start/Edit run number as a boolean variable, the ability of editing the run number is enabled or disabled, e.g. 
\begin{DoxyCode}
[local:Default:S]Edit on start>create BOOL "Edit run number"
\end{DoxyCode}


By default, if this key is NOT present the run number IS editable.

\par
 \begin{center} Start run request page showing the run number write-\/protected. \hyperlink{RC_mhttpd_Start_page_RC_Edit_PC}{Edit-\/on-\/start Parameter Comments} are also defined. \par
\par
\par
  \end{center}  \par
\par
\par
 \par


\begin{DoxyNote}{Note}
This feature is ignored by odbedit; regardless of whether the key Edit run number is present, the run number may be edited when starting a run using odbedit.
\end{DoxyNote}
\par
\par
\par




\par


\label{index_end}
\hypertarget{index_end}{}
 \paragraph{ODB page}\label{RC_mhttpd_ODB_page}
\label{RC_mhttpd_ODB_page_idx_mhttpd_page_ODB}
\hypertarget{RC_mhttpd_ODB_page_idx_mhttpd_page_ODB}{}
 \par




\label{RC_mhttpd_ODB_page_idx_edit_ODB_using-mhttpd}
\hypertarget{RC_mhttpd_ODB_page_idx_edit_ODB_using-mhttpd}{}
 \par


The ODB page is displayed by clicking on the \char`\"{}ODB\char`\"{} \hyperlink{RC_mhttpd_status_page_features_RC_mhttpd_status_Page_buttons}{page switch button} (\hyperlink{RC_mhttpd_status_page_features_RC_mhttpd_status_menu_buttons}{if present}) on the \hyperlink{RC_mhttpd_Main_Status_page_RC_mhttpd_main_status}{main status page}.

\par
\par
\par
 \begin{center}  Clicking on the \char`\"{}ODB\char`\"{} button will bring up a page showing the root tree of the ODB  \par
\par
\par
  \end{center}  \par
\par
\par


The ODB page initially displays the ODB {\bfseries root} {\bfseries tree}. Clicking on the hyperlink \char`\"{}/\char`\"{}, then one of the other hyperlinks ( e.g. \char`\"{}Equipment\char`\"{}) will take you to the requested ODB field. Depending on the \hyperlink{RC_customize_ODB_RC_Access_Control}{security}, read/write access to any ODB field can be gained .

The ODB page includes the useful capabilities of {\bfseries editing} the ODB values and {\bfseries creating} or {\bfseries deleting} new keys. But for copying ODB trees, saving and reloading the ODB, re-\/ordering the keys etc., the much more powerful \hyperlink{RC_odbedit_utility}{odbedit} utility must be used. \par
\par


{\bfseries Examples:} 


\begin{DoxyItemize}
\item \hyperlink{RC_mhttpd_ODB_page_RC_mhttpd_ODB_page_example1}{Changing a variable}
\item \hyperlink{RC_mhttpd_ODB_page_RC_mhttpd_ODB_page_example2}{Creating a subdirectory and an array}
\end{DoxyItemize}\hypertarget{RC_mhttpd_ODB_page_RC_mhttpd_ODB_page_example1}{}\subparagraph{Changing a variable}\label{RC_mhttpd_ODB_page_RC_mhttpd_ODB_page_example1}
This example shows how to change the variable \char`\"{}PA\char`\"{} under the /Equipment/PA/Settings/Channels ODB directory. If the ODB is \hyperlink{RC_customize_ODB_RC_Access_Control}{Write protected} (as in the example below) a {\bfseries  popup window will request the web password } before you can change a value.

\par
\par
\par
 \begin{center}  ODB page access  \par
\par
\par
  \end{center}  \par
\par
\par
 \par
\hypertarget{RC_mhttpd_ODB_page_RC_mhttpd_ODB_page_example2}{}\subparagraph{Creating a subdirectory and an array}\label{RC_mhttpd_ODB_page_RC_mhttpd_ODB_page_example2}
The following sequence shows how to
\begin{DoxyItemize}
\item create the subdirectory {\bfseries \char`\"{}Settings\char`\"{}} in the ODB tree /Equipment/Detector,
\item create an array called {\bfseries \char`\"{}Names\char`\"{}} in the Settings subdirectory,
\item fill the first element of {\bfseries \char`\"{}Names\char`\"{}} with the string {\bfseries \char`\"{}LeftFront\char`\"{}} 
\end{DoxyItemize}

\par
\par
\par
 \begin{center}  Creating subdirectory \char`\"{}Settings\char`\"{} and the array \char`\"{}Names\char`\"{} \par
\par
\par
  \end{center}  \par
\par
\par




\par
 \label{index_end}
\hypertarget{index_end}{}
 \paragraph{Equipment page}\label{RC_mhttpd_Equipment_page}
\label{RC_mhttpd_Equipment_page_idx_mhttpd_page_equipment}
\hypertarget{RC_mhttpd_Equipment_page_idx_mhttpd_page_equipment}{}
 \par




\par
\par
 \label{RC_mhttpd_Equipment_page_idx_Equipment_data_display}
\hypertarget{RC_mhttpd_Equipment_page_idx_Equipment_data_display}{}
 \hypertarget{RC_mhttpd_Equipment_page_RC_mhttpd_Equipment_var}{}\subparagraph{How to Display the Equipment Page}\label{RC_mhttpd_Equipment_page_RC_mhttpd_Equipment_var}
The {\bfseries Equipment} {\bfseries Page} is displayed by clicking on one of the \hyperlink{RC_mhttpd_Equipment_page_RC_mhttpd_Equipment_Hyperlink}{Equipment Hyperlinks} on the \hyperlink{RC_mhttpd_Main_Status_page_RC_mhttpd_main_status}{main status page}. \par
 This provides a short-\/cut so the user may see the contents of the defined banks for that equipment {\bfseries providing that the data from the Equipment is being sent to the ODB} because either


\begin{DoxyItemize}
\item the \hyperlink{FE_table_FE_tbl_ReadOn}{RO\_\-ODB} flag or
\item the \hyperlink{FE_table_FE_tbl_History}{History value}
\end{DoxyItemize}

in the corresponding \hyperlink{FrontendOperation_FE_Equipment_list}{Equipment List} is non-\/zero.

In this case, the /Equipment/$<$equipment-\/name$>$/Variables/ subdirectory ( where {\itshape  \char`\"{}$<$equipment-\/name$>$\char`\"{} \/} is replaced with the name of the defined equipment) is filled by the associated frontend (see \hyperlink{FrontendOperation}{SECTION 6: Frontend Operation} and \hyperlink{FE_ODB_equipment_tree}{the ODB /Equipment tree}). The data is written into array(s) with the same name(s) as the bankname(s).



 The data of Equipments that do not have either of these flags set cannot be viewed in this way. Instead it can be viewed with \hyperlink{RC_Monitor_RC_mdump_utility}{mdump} or an \hyperlink{DataAnalysis_DA_analyzer_utility}{analyzer}.

 \par
 \hypertarget{RC_mhttpd_Equipment_page_RC_mhttpd_Equipment_Naming}{}\subparagraph{Naming the Equipment data}\label{RC_mhttpd_Equipment_page_RC_mhttpd_Equipment_Naming}
The Equipment data is \hyperlink{RC_mhttpd_Equipment_page_RC_mhttpd_Equipment_var}{displayed on the Equipment page} using either the {\bfseries default} {\bfseries names} or {\bfseries names assigned by the user}, if a {\bfseries \char`\"{}Names\char`\"{}} array has been created. The following examples illustrate both cases.\hypertarget{RC_mhttpd_Equipment_page_RC_mhttpd_Equipment_Examples}{}\subparagraph{Examples of named Equipment data}\label{RC_mhttpd_Equipment_page_RC_mhttpd_Equipment_Examples}

\begin{DoxyItemize}
\item MIDAS format:
\begin{DoxyItemize}
\item \hyperlink{RC_mhttpd_Equipment_page_RC_mhttpd_Equipment_example1}{Data from an Equipment with one bank using the default Names}
\item \hyperlink{RC_mhttpd_Equipment_page_RC_mhttpd_Equipment_example2}{Named data from an Equipment with one bank}
\item \hyperlink{RC_mhttpd_Equipment_page_RC_mhttpd_Equipment_example3}{Named data from an Equipment with two banks of the same length}
\item \hyperlink{RC_mhttpd_Equipment_page_RC_mhttpd_Equipment_example4}{Event containing two Named Banks of different sizes}
\item \hyperlink{RC_mhttpd_Equipment_page_RC_mhttpd_Equipment_example5}{Names array Grouped for a large number of elements in a bank}
\end{DoxyItemize}
\item FIXED format:
\begin{DoxyItemize}
\item \hyperlink{RC_mhttpd_Equipment_page_RC_mhttpd_Equipment_example6}{Fixed-\/Format event with named Variables}
\item \hyperlink{RC_mhttpd_Equipment_page_RC_mhttpd_Equipment_example7}{Fixed-\/Format event with defined Names array} \par

\end{DoxyItemize}
\end{DoxyItemize}\hypertarget{RC_mhttpd_Equipment_page_RC_mhttpd_Equipment_example1}{}\subparagraph{Data from an Equipment with one bank using the default Names}\label{RC_mhttpd_Equipment_page_RC_mhttpd_Equipment_example1}
\label{RC_mhttpd_Equipment_page_RC_mhttpd_Equipment_Hyperlink}
\hypertarget{RC_mhttpd_Equipment_page_RC_mhttpd_Equipment_Hyperlink}{}
 The main Status page from an experiment with two Equipments defined ({\bfseries \char`\"{}TpcGasPlc\char`\"{}} and {\bfseries \char`\"{}Detector\char`\"{}}) is shown below. Clicking the {\bfseries Equipment Hyperlink} \char`\"{}Detector\char`\"{} (circled in green) will show the data from /Equipment/Detector/Variables. This illustration shows how the mhttpd display combines the names of the variables in a \hyperlink{FE_bank_construction_FE_MIDAS_event_construction}{MIDAS format} event with the variables read out in the equipment's data bank. \par
 The \char`\"{}Detector\char`\"{} equipment has one bank only (SCLR) which is an array of 10 data words. The \char`\"{}Names\char`\"{} column shows the default name, which is derived from the {\bfseries bankname} and the array element number , i.e. \char`\"{}SCLR\mbox{[}0\mbox{]}...SCLR\mbox{[}9\mbox{]}\char`\"{}.

\par
\par
\par
 \begin{center}  Clicking on the \char`\"{}Detector\char`\"{} Equipment Hyperlink shows SCLR bank contents with default Names \par
\par
\par
  \end{center}  \par
\par
\par


The corresponding ODB data is shown below using \hyperlink{RC_odbedit_utility}{odbedit}.


\begin{DoxyCode}
[local:t2kgas:S]Variables>ls /Equipment/Detector/Variables -lt
Key name                        Type    #Val  Size  Last Opn Mode Value
---------------------------------------------------------------------------
SCLR                            DWORD   10    4     55m  0   RWD
                                        [0]             3453
                                        [1]             2701
                                        [2]             896
                                        [3]             4351
                                        [4]             2051
                                        [5]             1467
                                        [6]             1952
                                        [7]             4931
                                        [8]             783
                                        [9]             902
\end{DoxyCode}


\par


\par
\hypertarget{RC_mhttpd_Equipment_page_RC_mhttpd_Equipment_example2}{}\subparagraph{Named data from an Equipment with one bank}\label{RC_mhttpd_Equipment_page_RC_mhttpd_Equipment_example2}
It is often helpful to define an {\bfseries  individual name for each element of the databank}. This is done by creating an array called {\bfseries \char`\"{}Names\char`\"{}} in the /Equipment/$<$equipment-\/name$>$/Settings/ subdirectory, with the same number of elements as the databank. This array is then filled by the user with a suitable name corresponding to each element in the databank. This procedure is shown \hyperlink{RC_mhttpd_ODB_page_RC_mhttpd_ODB_page_example2}{here using the mhttpd ODB page commands} , or with \hyperlink{RC_odbedit_utility}{odbedit} below:


\begin{DoxyCode}
[local:t2kgas:S]>cd /Equipment/Detector
[local:t2kgas:S]>mkdir Settings
[local:t2kgas:S]Detector>cd Settings
[local:t2kgas:S]Settings>create string Names[10]
String length [32]:
[local:t2kgas:S]Settings>set Names[0] LeftFront
[local:t2kgas:S]Settings>set Names[1] LeftCentre
[local:t2kgas:S]Settings>ls
Names
                                LeftFront
                                LeftCentre








[local:t2kgas:S]Settings>
\end{DoxyCode}
 \par
 Now when the hyperlink \char`\"{}Detector\char`\"{} is clicked, the elements of the SCLR bank will be named using data from the {\bfseries Names} array as follows: \par


\par
\par
\par
 \begin{center}  Clicking on \char`\"{}Detector\char`\"{} Equipment Hyperlink shows SCLR bank contents with defined Names \par
\par
\par
  \end{center}  \par
\par
\par


The complete {\bfseries Names} array is shown below using \hyperlink{RC_odbedit_utility}{odbedit} :


\begin{DoxyCode}
[local:t2kgas:S]settings>ls -lt
Key name                        Type    #Val  Size  Last Opn Mode Value
---------------------------------------------------------------------------
 Names                           STRING  10    32    22m  0   RWD
                                        [0]             LeftFront
                                        [1]             LeftCentre
                                        [2]             LeftBack
                                        [3]             RightFront
                                        [4]             RightCentre
                                        [5]             RightBack
                                        [6]             MidFront
                                        [7]             MidCentre
                                        [8]             MidBack
                                        [9]             Dump
\end{DoxyCode}


\par


\par
\hypertarget{RC_mhttpd_Equipment_page_RC_mhttpd_Equipment_example3}{}\subparagraph{Named data from an Equipment with two banks of the same length}\label{RC_mhttpd_Equipment_page_RC_mhttpd_Equipment_example3}
Sometimes the data consists of more than one bank that corresponds to the same list of names, i.e. both banks have the same length. In that case, the {\bfseries Names} array will refer to both, e.g. \par
\par
 \begin{center} Clicking on \char`\"{}Detector\char`\"{} Equipment Hyperlink shows SCLR and TDCT bank contents with the same defined Names \par
\par
\par
  \end{center}  \par
\par
\par


The \char`\"{}Detector\char`\"{} equipment now has two banks (SCLR and TDCT) which are both arrays of 10 data words:


\begin{DoxyCode}
[local:t2kgas:S]Variables>ls /Equipment/Detector/Variables -lt
Key name                        Type    #Val  Size  Last Opn Mode Value
---------------------------------------------------------------------------
SCLR                            DWORD   10    4     55m  0   RWD
                                        [0]             3453
                                        [1]             2701
                                        [2]             896
                                        [3]             4351
                                        [4]             2051
                                        [5]             1467
                                        [6]             1952
                                        [7]             4931
                                        [8]             783
                                        [9]             902  
TDCT                            FLOAT   10    4     3m   0   RWD
                                        [0]             503
                                        [1]             679
                                        [2]             321
                                        [3]             1072
                                        [4]             760
                                        [5]             2315
                                        [6]             474
                                        [7]             846
                                        [8]             39
                                        [9]             691
\end{DoxyCode}


\par


\par
\hypertarget{RC_mhttpd_Equipment_page_RC_mhttpd_Equipment_example4}{}\subparagraph{Event containing two Named Banks of different sizes}\label{RC_mhttpd_Equipment_page_RC_mhttpd_Equipment_example4}
If the equipment contains several banks that cannot share the same \char`\"{}Names\mbox{[}$\,$\mbox{]}\char`\"{} array, individual \char`\"{}Names\char`\"{} arrays can be set up for each bank. The banks then may be of different lengths. The following examples shows an equipment named \char`\"{}Target\char`\"{} that has two named banks, SCLR and TGT\_\- .

\par
\par
\par
 \begin{center}  Clicking on \char`\"{}Detector\char`\"{} Equipment Hyperlink shows SCLR and TGT\_\- bank contents each with defined Names \par
\par
\par
  \end{center}  \par
\par
\par


In this case, instead of one \char`\"{}Names\char`\"{} array, an array has been defined for each bank of the form {\itshape \char`\"{}Names  $<$bankname$>$\mbox{[}$<$Len$>$\mbox{]}\char`\"{}\/} where
\begin{DoxyItemize}
\item {\itshape $<$bankname$>$\/} is the {\bfseries name} of the bank and
\item {\itshape $<$Len$>$\/} is the {\bfseries length} of the bank.
\end{DoxyItemize}

Note that two extra hyperlinks appear on the {\bfseries \char`\"{}Groups\char`\"{}} line. These Group Hyperlinks are labelled as the name of each bank, and provide a shortcut to the top of the bank, useful when the bank is very large.

Shown below are the contents of the arrays {\bfseries  \char`\"{}Names TGT\_\-\char`\"{} }and {\bfseries \char`\"{}Names SCAL\char`\"{}} in /Equipment/target/settings: 
\begin{DoxyCode}
ls /Equipment/target/settings
Names TGT_
                                Time
                                Cryostat vacuum
                                Heat Pipe pressure
                                Target pressure
                                Target temperature
                                Shield temperature
                                Diode temperature
                                Diode current
                                Laser intensity
                                gas pressure
                                gas temperature
Names SCAL
                                LSeg0
                                LSeg1
                                LSeg2
                                LSeg3
                                RSeg0
                                RSeg1
                                RSeg2
                                RSeg3
[local:t2kgas:S]/Equipment>
\end{DoxyCode}
 and the contents of /Equipment/target/variables showing the two corresponding banks {\bfseries TGT\_\-} and {\bfseries SCAL} : 
\begin{DoxyCode}
[local:t2kgas:S]/Equipment>ls /Equipment/target/variables
SCAL
                                3453
                                2701
                                896
                                4351
                                2051
                                1467
                                1952
TGT_
                                114059
                                4.661
                                23.16
                                -0.498
                                22.888
                                82.099
                                40
                                14.2
                                9.871
                                -70.9
\end{DoxyCode}


\par


\par
\hypertarget{RC_mhttpd_Equipment_page_RC_mhttpd_Equipment_example5}{}\subparagraph{Names array Grouped for a large number of elements in a bank}\label{RC_mhttpd_Equipment_page_RC_mhttpd_Equipment_example5}
This example shows the main status page of an experiment which has a number of Equipments defined. The Equipment Hyperlink of the Equipment {\bfseries \char`\"{}cycle\_\-scalers\char`\"{}} ( \hyperlink{FE_bank_construction_FE_MIDAS_event_construction}{MIDAS format}) has been clicked. This hyperlink is circled in the following illustration. The resulting pages show the list of variables in the bank HSCL. This large bank has been divided into the groups:


\begin{DoxyItemize}
\item {\bfseries  All Back Front Scaler\_\-B General}
\end{DoxyItemize}

The names of the Groups appear in the Groups line of the Equipment pages. The example shows the complete bank (\char`\"{}All\char`\"{} which is the default) as well as the Groups of scalers which appears when each Group Hyperlink (circled) is pressed.

\label{RC_mhttpd_Equipment_page_RC_mhttpd_Equipment_image5}
\hypertarget{RC_mhttpd_Equipment_page_RC_mhttpd_Equipment_image5}{}
 \par
\par
\par
 \begin{center}  Clicking on the Equipment Hyperlink \char`\"{}cycle\_\-scalers\char`\"{} on Status page showing \char`\"{}All\char`\"{} scaler values, plus the defined Groups \par
\par
\par
  \end{center}  \par
\par
\par


\par


The illustration above shows how the mhttpd display combines the names of the variables in a \hyperlink{FE_bank_construction_FE_MIDAS_event_construction}{MIDAS format} event with the variables read out in the Equipment's databank. The {\bfseries \char`\"{}cycle\_\-scalers\char`\"{}} equipment has been set up in the ODB so that the scaler names are listed under the array {\bfseries \char`\"{}Names\char`\"{}} in /Equipment/Cycle\_\-Scalers/Settings. Since there are a great many scalers in this one bank, they have been divided into the groups \char`\"{}Back\char`\"{} \char`\"{}Front\char`\"{} \char`\"{}ScalerB\char`\"{} and \char`\"{}General\char`\"{} in the {\bfseries Names} array using a {\itshape \char`\"{}$<$group$>$\%\char`\"{}\/} construct as shown below:


\begin{DoxyCode}
[local:bnmr:S]>cd /Equipment/Cycle_Scalers/Settings
[local:bnmr:S]Settings>ls
Names
                                Back%BSeg00
                                Back%BSeg01
                                Back%BSeg02
                                Back%BSeg03
                                Back%BSeg04
                                Back%BSeg05
                                Back%BSeg06
                                Back%BSeg07
                                Back%BSeg08
                                Back%BSeg09
                                Back%BSeg10
                                Back%BSeg11
                                Back%BSeg12
                                Back%BSeg13
                                Back%BSeg14
                                Back%BSeg15
                                Front%FSeg00
                                Front%FSeg01
                                Front%FSeg02
                                Front%FSeg03
                                Front%FSeg04
                                Front%FSeg05
                                Front%FSeg06
                                Front%FSeg07
                                Front%FSeg08
                                Front%FSeg09
                                Front%FSeg10
                                Front%FSeg11
                                Front%FSeg12
                                Front%FSeg13
                                Front%FSeg14
                                Front%FSeg15
                                Scaler_B%SIS Ref pulse
                                Scaler_B%Fluor. mon 2
                                Scaler_B%Polariz Left
                                Scaler_B%Polariz Right
                                Scaler_B%Neutral Beam B1
                                Scaler_B%Neutral Beam B2
                                Scaler_B%Neutral Beam B3
                                Scaler_B%Neutral Beam B4
                                Scaler_B%Neutral Beam F1
                                Scaler_B%Neutral Beam F2
                                Scaler_B%Neutral Beam F3
                                Scaler_B%Neutral Beam F4
                                General%Back Userbit=0
                                General%Back Userbit=1
                                General%Back Userbit=2
                                General%Back Userbit=3
                                General%Front Userbit=0
                                General%Front Userbit=1
                                General%Front Userbit=2
                                General%Front Userbit=3
                                General%Back Cycle Sum
                                General%Front Cycle Sum
                                General%B/F Cycle
                                General%Asym Cycle
                                General%Pol Cycle Sum
                                General%Pol Cycle Asym
                                General%NeutBm Cycle Sum
                                General%NeutBm Cycle Asym
[local:bnmr:S]Settings> 
\end{DoxyCode}


\par


\par
\hypertarget{RC_mhttpd_Equipment_page_RC_mhttpd_Equipment_example6}{}\subparagraph{Fixed-\/Format event with named Variables}\label{RC_mhttpd_Equipment_page_RC_mhttpd_Equipment_example6}
An example of a \hyperlink{FE_bank_construction_FE_FIXED_event_readout}{FIXED format} event is shown below. The equipment {\bfseries \char`\"{}Info ODB\char`\"{}} is defined as FIXED-\/format. The variables are of different types, so cannot be output as a MIDAS event. In this example, there is no information listed in the subdirectory Settings for this equipment. The name of each element is listed in the \char`\"{}Variables\char`\"{} subtree. The Settings subtree is not defined.

\par
\par
\par
 \begin{center}  Clicking on the \char`\"{}Info ODB\char`\"{} Equipment Hyperlink on Status page showing the FIXED format event \par
\par
\par
  \end{center}  \par
\par
\par



\begin{DoxyCode}
[local:bnmr:S]Settings>ls "/Equipment/Info ODB/"
Common
Statistics
Variables
\end{DoxyCode}


The names of the variables for this FIXED-\/format event are in the /Equipment/Info ODB/Variables subtree, i.e.


\begin{DoxyCode}
[local:bnmr:S]>ls -lt "/Equipment/Info ODB/Variables"
Key name                        Type    #Val  Size  Last Opn Mode Value
---------------------------------------------------------------------------
helicity                        DWORD   1     4     8h   0   RWD  0
current cycle                   DWORD   1     4     8h   0   RWD  710
cancelled cycle                 DWORD   1     4     8h   0   RWD  9
current scan                    DWORD   1     4     8h   0   RWD  13
Ref HelUp thr                   DOUBLE  1     8     8h   0   RWD  6626873
Ref HelDown thr                 DOUBLE  1     8     8h   0   RWD  6626873
Current HelUp thr               DOUBLE  1     8     8h   0   RWD  6659381
Current HelDown thr             DOUBLE  1     8     8h   0   RWD  75
Prev HelUp thr                  DOUBLE  1     8     8h   0   RWD  6652944
Prev HelDown thr                DOUBLE  1     8     8h   0   RWD  133
RF state                        DWORD   1     4     8h   0   RWD  0
Fluor monitor counts            DWORD   1     4     8h   0   RWD  0
EpicsDev Set(V)                 FLOAT   1     4     8h   0   RWD  0
EpicsDev Read(V)                FLOAT   1     4     8h   0   RWD  0
Campdev set                     FLOAT   1     4     8h   0   RWD  0
Campdev read                    FLOAT   1     4     8h   0   RWD  0
Laser Power(V)                  FLOAT   1     4     8h   0   RWD  0
last failed thr test            DWORD   1     4     8h   0   RWD  0
cycle when last failed thr      DWORD   1     4     8h   0   RWD  710
last good hel                   DWORD   1     4     8h   0   RWD  1
ncycle sk tol                   DWORD   1     4     8h   0   RWD  2
hel_read                        DWORD   1     4     8h   0   RWD  9
[local:bnmr:S]>  
\end{DoxyCode}
\hypertarget{RC_mhttpd_Equipment_page_RC_mhttpd_Equipment_example7}{}\subparagraph{Fixed-\/Format event with defined Names array}\label{RC_mhttpd_Equipment_page_RC_mhttpd_Equipment_example7}
Alternatively, if the FIXED format event consists of an array , a \char`\"{}Names\char`\"{} array can be defined in the same way as for the MIDAS event. In this case, each element of the array under \char`\"{}Variables\char`\"{} will be referenced using the equivalent elements of the \char`\"{}Settings/Names\char`\"{} array, as shown in the following example: \par
\par
\par
 \begin{center}  FIXED format event with a \char`\"{}Names\char`\"{} array defined \par
\par
\par
  \end{center}  \par
\par
\par


This event is defined as follows:


\begin{DoxyCode}
[local:t2kgas:S]/>cd /Equipment/parameters

[local:t2kgas:S]/parameters>ls  -rlt
    ...............
    Variables                   DIR
        IODB                    STRING  5     32    26m  0   RWD
                                        [0]             123352
                                        [1]             pulse
                                        [2]             43.21
                                        [3]             DRT_9
                                        [4]             0.321
    settings                    DIR
        Names                   STRING  5     32    10m  0   RWD
                                        [0]             Rejected
                                        [1]             Type
                                        [2]             LTX Voltage
                                        [3]             Trig type
                                        [4]             PXV Current
\end{DoxyCode}
 \par




\par
 \label{index_end}
\hypertarget{index_end}{}
 \paragraph{Slow Control page}\label{RC_mhttpd_sc_page}
\label{RC_mhttpd_sc_page_idx_mhttpd_page_slow-control}
\hypertarget{RC_mhttpd_sc_page_idx_mhttpd_page_slow-control}{}
 \par




\par
 \hypertarget{RC_mhttpd_sc_page_RC_mhttpd_slow_control_intro}{}\subparagraph{Introduction}\label{RC_mhttpd_sc_page_RC_mhttpd_slow_control_intro}
The Slow Control page refers to the specific display of a {\bfseries Slow Control Equipment} (see \hyperlink{FE_Slow_Control_system}{Slow Control System}). This is a special \hyperlink{RC_mhttpd_Equipment_page}{Equipment page} that may be accessed from the \hyperlink{RC_mhttpd_Main_Status_page_RC_mhttpd_main_status}{main status page} by clicking on the \hyperlink{RC_mhttpd_Equipment_page_RC_mhttpd_Equipment_Hyperlink}{Equipment Hyperlink}. The Slow Control page will show a parameter table (such as those shown below). Parameters that are editable will be hyperlinked for parameter modification. This option is possible only if the parameter names have a particular {\bfseries name} {\bfseries syntax}.\hypertarget{RC_mhttpd_sc_page_RC_mhttpd_slow_control_name_syntax}{}\subparagraph{Name Syntax for Slow Control Page}\label{RC_mhttpd_sc_page_RC_mhttpd_slow_control_name_syntax}
The name syntax is similar to that already described on the \hyperlink{RC_mhttpd_Equipment_page}{Equipment Page}, where the Equipment variables are located in the /Equipment/$<$equipment-\/name$>$/Variables subtree with the matching information defined under the /Equipment/$<$equipment-\/name$>$/Settings subtree. Additinally, for a Slow Control Equipment each variable in the table may be {\bfseries editable} depending on the following rules :


\begin{DoxyItemize}
\item If the variable name is defined under the \char`\"{}Settings/\char`\"{} directory
\begin{DoxyEnumerate}
\item as one of the following names, it will be editable by default:
\begin{DoxyItemize}
\item in a {\bfseries \char`\"{}Demand\char`\"{}} array or
\item in an {\bfseries \char`\"{}Output\char`\"{}} array or
\item as {\bfseries \char`\"{}D\_\-$<$var\_\-name$>$\char`\"{}} 
\end{DoxyItemize}
\end{DoxyEnumerate}
\end{DoxyItemize}


\begin{DoxyEnumerate}
\item the variable is ALSO defined under the array\char`\"{}Settings/Editable\mbox{[}$\,$\mbox{]}\char`\"{}
\begin{DoxyItemize}
\item it will be editable.
\end{DoxyItemize}
\end{DoxyEnumerate}

This information will be combined to compose a table. 
\begin{DoxyCode}
[local:Default:S]/>cd Equipment/MSCB/Settings/
[local:Default:S]Settings>ls
[local:Default:S]Settings>ls
Names
                                Drift Voltage (KV)
                                Drift Current (uA)
                                uC Temperature (C)
DD
Offset
                                0
                                1
                                1
Gain
                                0
                                3
                                4
Editable                        Gain
\end{DoxyCode}


\par
\par
\par
 \begin{center}  Slow control Equipment page. \par
\par
\par
  \end{center}  \par
\par
\par


The following is an example of a Slow Control Page from another experiment, showing a High Voltage system for two crates, with multiple \hyperlink{structparameters}{parameters} set up.

\par
\par
\par
 \begin{center}  Slow control page for a High Voltage system. \par
\par
\par
  \end{center}  \par
\par
\par


\par




\par


\label{index_end}
\hypertarget{index_end}{}
 \paragraph{Message page}\label{RC_mhttpd_Message_page}
\label{RC_mhttpd_Message_page_idx_mhttpd_page_message}
\hypertarget{RC_mhttpd_Message_page_idx_mhttpd_page_message}{}
 \par




\par


This page (accessed by clicking the \hyperlink{RC_mhttpd_status_page_features_RC_mhttpd_status_Page_buttons}{Message button} \hyperlink{RC_mhttpd_status_page_features_RC_mhttpd_status_menu_buttons}{if present} on the \hyperlink{RC_mhttpd_Main_Status_page_RC_mhttpd_main_status}{main status page} ) displays the contents of the MIDAS System log file in blocks of 100 lines starting with the most recent messages. The MIDAS log file resides in the directory defined \hyperlink{F_Messaging_F_Log_File}{in the ODB}.

\par
\par
\par
 \begin{center}  Message page. \par
\par
\par
  \end{center}  \par
\par
\par


\par




\par
 \label{index_end}
\hypertarget{index_end}{}
 \paragraph{Elog page}\label{RC_mhttpd_Elog_page}
\label{RC_mhttpd_Elog_page_idx_mhttpd_page_elog}
\hypertarget{RC_mhttpd_Elog_page_idx_mhttpd_page_elog}{}
 \par




\par


\label{RC_mhttpd_Elog_page_Elog_200}
\hypertarget{RC_mhttpd_Elog_page_Elog_200}{}
\hypertarget{RC_mhttpd_Elog_page_RC_mhttpd_Elog_intro}{}\subparagraph{Introduction}\label{RC_mhttpd_Elog_page_RC_mhttpd_Elog_intro}
The ELOG page (accessed by clicking the \hyperlink{RC_mhttpd_status_page_features_RC_mhttpd_status_Page_buttons}{Elog button} \hyperlink{RC_mhttpd_status_page_features_RC_mhttpd_status_menu_buttons}{if present} on the \hyperlink{RC_mhttpd_Main_Status_page_RC_mhttpd_main_status}{main status page} ) provides access to an electronic logbook. This tool can replace the experimental logbook for daily entries. The main advantage of the Elog over a paper logbook is the possiblity to access it remotely, and provide a general knowledge of the experiment. On the other hand, the Elog is not limited strictly to experiments. Worldwide Elog implementations can be found on the internet.

Two different implementations of the Elog are available i.e
\begin{DoxyItemize}
\item \hyperlink{RC_mhttpd_Elog_page_RC_mhttpd_Internal_Elog}{Internal Elog} where the Elog is built in the mhttpd MIDAS web interface, or
\item \hyperlink{RC_mhttpd_Elog_page_RC_mhttpd_External_Elog}{External Elog} where the Elog runs independently from the experiment and mhttpd as well.
\end{DoxyItemize}

While the Internal implementation doesn't requires any setup, the External implementation requires a proper Elog installation which is fully described on the \href{http://midas.psi.ch/elog/}{\tt Elog} web site. The External Elog implementation also requires a dedicated entry in the ODB as shown in the code below. It also requires the package {\bfseries Elog} to be already installed, and properly configured. Once the ODB entry is present, the internal ELOG is disabled. \par
 The Elog is \hyperlink{F_Elog_F_Elog_Custom}{customized} through the ODB /Elog tree.



 \label{RC_mhttpd_Elog_page_idx_Elog_internal}
\hypertarget{RC_mhttpd_Elog_page_idx_Elog_internal}{}
 \hypertarget{RC_mhttpd_Elog_page_RC_mhttpd_Internal_Elog}{}\subparagraph{Internal Elog}\label{RC_mhttpd_Elog_page_RC_mhttpd_Internal_Elog}
By default the mhttpd provides the internal Elog. The entry destination directory is established by the logger key in ODB (see \hyperlink{F_Elog_F_Logger_Elog_Dir}{Elog Dir}). The Electronic Log page shows the most recent Log message recorded in the system. The top buttons allows you to either {\bfseries Create/Edit/Reply/Query/Show} a message

\par
\par
\par
 \begin{center}  main Elog page. \par
\par
\par
  \end{center}  \par
\par
\par


The format of the message log can be written in HTML format.

\par
\par
\par
 \begin{center}  HTML Elog message. \par
\par
\par
  \end{center}  \par
\par
\par
\hypertarget{RC_mhttpd_Elog_page_RC_mhttpd_Internal_Elog_shift_check}{}\subparagraph{The Shift Check button}\label{RC_mhttpd_Elog_page_RC_mhttpd_Internal_Elog_shift_check}
\label{RC_mhttpd_Elog_page_idx_elog_shift-check-form}
\hypertarget{RC_mhttpd_Elog_page_idx_elog_shift-check-form}{}



\begin{DoxyItemize}
\item A feature of the Elog entry page is the optional {\bfseries Shift Check} button, which permits the experimenter on shift to go through a checklist and record his/her findings in the Elog system. The checklist is user-\/defined and can be found in the ODB under the \hyperlink{F_Elog_F_ODB_Elog_Tree}{/Elog tree}. In the following example, a Shift Check button labelled {\itshape \char`\"{}Gas Handling\char`\"{}\/} has been created:
\end{DoxyItemize}

\par
\par
\par
 \begin{center}  HTML Elog message. \par
\par
\par
  \end{center}  \par
\par
\par



\begin{DoxyItemize}
\item The code below generates the above screen. The key {\itshape Gas Handling\/} contains all the information for a given form. There is no limit to the number of entries. By specifying an entry with the name {\itshape Attachment0\/},{\itshape Attachment1\/},... and filling it with a fixed file name, its content will be attached to the Elog entry for every shift report.
\end{DoxyItemize}


\begin{DoxyCode}
[local:myexpt:Running]/>cd /Elog/
[local:myexpt:Running]/Elog>mkdir Forms
[local:myexpt:Running]/Elog>cd Forms/
[local:myexpt:Running]Forms>mkdir "Gas Handling"
[local:myexpt:Running]Forms>cd "Gas Handling"
[local:myexpt:Running]Gas Handling>create string "N2 Pressure"
String length [32]: 
[local:myexpt:Running]Gas Handling>create string "Vessel Temperature"
String length [32]: 
[local:myexpt:Running]Gas Handling>ls
N2 pressure              
Vessel Temperature              
[local:myexpt:Running]Gas Handling>
[local:xenon:Running]Gas Handling>create string Attachment0 
String length [32]: 64
[local:xenon:Running]Gas Handling>set Attachment0 Gaslog.txt
\end{DoxyCode}
\hypertarget{RC_mhttpd_Elog_page_RC_mhttpd_Internal_Elog_runlog}{}\subparagraph{Runlog Button}\label{RC_mhttpd_Elog_page_RC_mhttpd_Internal_Elog_runlog}

\begin{DoxyItemize}
\item The {\bfseries runlog} button display the content of the file {\bfseries runlog.txt} which is expected to be in the data directory specified by the ODB key {\bfseries /Logger/Data Dir}. Regardless of its content, it will be displayed in the web page. Its common use is to {\bfseries append} {\bfseries lines} after every run. The client appending this run information can be any of the MIDAS applications. An {\bfseries  example } is available in the {\itshape \hyperlink{analyzer_8c}{examples/experiment/analyzer.c}\/} which will write some statistical information at end-\/of-\/run to the file runlog.txt
\end{DoxyItemize}

\par
\par
\par
 \begin{center}  Elog page, Runlog display. \par
\par
\par
  \end{center}  $\ast$ \par
\par
\par



\begin{DoxyItemize}
\item When composing a new entry into the Elog, several fields are available to specify the nature of the message i.e: Author, Type, System, Subject. Under Type and System a pulldown menu provides a choice of categories. These categories are user definable through the /Elog ODB tree under the keys \hyperlink{F_Elog_F_Types}{Types} and \$ref F\_\-Systems \char`\"{}Systems\char`\"{}. The number of categories is limited to 20 maximum. Any usused fields can be left empty.
\end{DoxyItemize}

\par
\par
\par
 \begin{center}  Elog page, New Elog entry form. \par
\par
\par
  \end{center}  \par
\par
\par




 \label{RC_mhttpd_Elog_page_idx_Elog_external}
\hypertarget{RC_mhttpd_Elog_page_idx_Elog_external}{}
 \hypertarget{RC_mhttpd_Elog_page_RC_mhttpd_External_Elog}{}\subparagraph{External Elog}\label{RC_mhttpd_Elog_page_RC_mhttpd_External_Elog}
The advantage of using the external Elog over the built-\/in version is its flexibility. This package is used worldwide and improvements are constantly being made. A full-\/features documentation and standalone installation can be found at the \href{http://midas.psi.ch/elog/}{\tt Elog} web site.

Installation requires requires several steps described below.


\begin{DoxyItemize}
\item Download the Elog package from the web site mentioned above. 
\begin{DoxyItemize}
\item Windows, Linux, Mac version can be found there. Simple installation procedures are also described. Its installation can be done at the system level or at the user level. The Elog can service multiple Electronic logbooks in parallel and therefore an extra entry in its configuration file can provide specific experimental elog in a similar fashion as the internal one.


\item You need to take note of several consideration for its installation. Several locations are required for the different files that elog deals with. 
\begin{DoxyItemize}
\item elog resource directory ( e.g. /elog\_\-installation\_\-dir where elog is installed) 
\item logbook directory (ex: /myexpt/logbook where the pwd and elog entries are stored). 
\end{DoxyItemize}
\item The pwd file uses encryption for the user password. 
\item As this Elog installation is tailored towards an experiment, a restriction applies i.e. {\itshape  ensure that the mhttpd and elog applications shares at least the same file system.\/} \par
This means that either 
\begin{DoxyItemize}
\item both applications runs on the same machine or 
\item a nsf mount provides file sharing. 
\end{DoxyItemize}
\item You need to know the node and ports for both applications. Like mhttpd, elogd also requires a port number for communication through the web (e.g. NodeA:mhttpd -\/p 8080, NodeB:elogd -\/p 8081).
\begin{DoxyEnumerate}
\item copy the default midas/src/elogd.cfg from the MIDAS distrbution to your operating directory.
\item modify the elogd.cfg to reflect your configuration 
\begin{DoxyCode}
  # This is a simple elogd configuration file to work with MIDAS
  # $Id: mhttpd.dox 4032 2007-11-02 17:13:52Z amaudruz $ 

  [global]
  ; port under which elogd should run
  port = 8081                             
  ; password file, created under 'logbook dir'
  password file = elog.pwd                
  ; directory under which elog was installed (themes etc.)
  resource dir = /elog_installation_dir     
  ; directory where the password file will end up
  logbook dir = /myexpt/logbook     
  ; anyone can create it's own account
  self register = 1                       
  ; URL under which elogd is accessible
  url = http://ladd00.triumf.ca:8081      
  ; the "main" tab will bring you back to mhttpd
  main tab = Xenon                        
  ; this is the URL of mhttpd which must run on a different port
  main tab url = http://NodeA:8080
  ; only needed for email notifications
  smtp host = your.smtp.host              
  ; Define one logbook for online use. Severl logbooks can be defined here
  [MyOnline]
  ; directory where the logfiles will be written to
  Data dir = /myexpt/logbook            
  Comment = My MIDAS Experiment Electronic Logbook
  ; mimic old mhttpd behaviour
  Attributes = Run number, Author, Type, System, Subject     
  Options Type = Routine, Shift Summary, Minor Error, Severe Error, Fix, Question
      , Info, Modification, Alarm, Test, Other, 
  Options System = General, DAQ, Detector, Electronics, Target, Beamline
  Extendable Options = Type, System
  ; This substitution will enter the current run number
  Preset Run number = $shell(odbedit -e myexpt -h NodeA -d Runinfo -c 'ls -v \"ru
      n number\"')    
  Preset Author = $long_name
  Required Attributes = Type, Subject
  ; Run number and Author cannot be changed
  Locked Attributes = Run number, Author  
  Page Title = ELOG - $subject
  Reverse sort = 1
  Quick filter = Date, Type, Author
  ; Don't send any emails
  Suppress email to users = 1             
\end{DoxyCode}

\item start the elog daemon. {\bfseries -\/x} is for the shell substitution of the command {\itshape Preset Run number = \$shell(...)\/} The argument invokes the odbedit remotely if needed to retrieve the current run number. You will have to ensure the proper path to the odbedit and the proper -\/e, -\/h argments for the experiment and host. You may want to verify this command from the console. 
\begin{DoxyCode}
  NodeB:~>/installation_elog_dir/elogd -c elogd.cfg -x
\end{DoxyCode}

\item start the mhttpd at its correct port and possibly in the daemon form. 
\begin{DoxyCode}
  NodeA:~>mhttpd -p 8080 -D 
\end{DoxyCode}

\item At this point the Elog from the MIDAS web page is accessing the internal Elog. To activate the external Elog, include in the ODB two entries such as: 
\begin{DoxyCode}
   NodeX:> odbedit -e myexpt -h NodeA
   [NodeX:myexpt:Running]/>cd elog
   [NodeX:myexpt:Running]/Elog>create string Url
   String length [32]: 64
   [NodeX:myexpt:Running]/Elog>set Url http://NodeB:8081/MyOnline
   [NodeX:myexpt:Running]
   [NodeX:myexpt:Running]/Elog>create string "Logbook Dir"
   String length [32]: 64
   [NodeX:myexpt:Running]/Elog>set "Logbook Dir" /myexpt/logbook

   [NodeX:myexpt:Running]/Elog>ls
Logbook Dir                     /home/myexpt/ElogBook
Url                             http://NodeB:8081/MyOnline
\end{DoxyCode}

\item Confirm proper operation of the external Elog by creating an entry. You will be prompted for a username and password. Click on New registration. Full control of these features are described in the Elog documentation.
\item Stop and restart the Elogd in the background. 
\begin{DoxyCode}
   NodeB:~>/installation_elog_dir/elogd -c elogd.cfg -x -D
\end{DoxyCode}

\item In the event you had a previous entry under the internal elog, you can convert the internal to external using the elconv tool. 
\begin{DoxyCode}
   NodeB:~> cp internal/elog_logbook/*.log /myexpt/logbook/.
   NodeB:~> cd /myexpt/logbook
   NodeB:~> /installation_elog_dir/elconv
\end{DoxyCode}

\end{DoxyEnumerate}


\end{DoxyItemize}
\end{DoxyItemize}\par




\par
 \label{index_end}
\hypertarget{index_end}{}
 \paragraph{Programs page}\label{RC_mhttpd_Program_page}
\label{RC_mhttpd_Program_page_idx_mhttpd_page_program}
\hypertarget{RC_mhttpd_Program_page_idx_mhttpd_page_program}{}
 \par




\par
 The {\bfseries Programs} page is displayed by clicking on the {\bfseries Programs} {\bfseries Button} ( \hyperlink{RC_mhttpd_status_page_features_RC_mhttpd_status_menu_buttons}{if present}) on the \hyperlink{RC_mhttpd_Main_Status_page}{Main Status Page} .

This page presents the current active list of the clients attached to the given experiment. On the right hand side, a dedicated button allows the user to stop a particular client. This is equivalent to the \hyperlink{RC_odbedit_examples_RC_odbedit_sh}{ODBedit shutdown} command. For example, pressing the {\bfseries \char`\"{}stop Speaker\char`\"{}} button (see example below) would be equivalent to 
\begin{DoxyCode}
odbedit> sh Speaker
\end{DoxyCode}


Clicking the client name hyperlink (on the left hand side of the page) pops up a new window pointing to the ODB {\bfseries /Programs} subdirectory related to that particular client (see \hyperlink{RC_customize_ODB_RC_ODB_Programs_Tree}{The ODB /Programs tree}). Example 1 shows the Programs page with a number of clients running. The hyperlink for the client {\bfseries ltnoRC} has been clicked, and the popup window shows the ODB path {\bfseries /Programs/ltnoRC}.

\par
\par
\par
 \begin{center}  Example 1: mhttpd Programs page. \par
\par
\par
   \end{center}  \par
\par
\par


The ODB structure for \hyperlink{RC_customize_ODB_RC_ODB_programs_client}{each client} should be customized by the user (see \hyperlink{RC_customize_ODB_RC_customize_Programs_tree}{Customize the ODB /Programs tree} ). \par
 \begin{DoxyNote}{Note}
The buttons and other information that appears on the {\bfseries Programs} mhttpd page depends on the settings of the fields of the /Programs/$<${\itshape client\/} $>$ subdirectory.
\end{DoxyNote}
\label{RC_mhttpd_Program_page_RC_mhttpd_Required}
\hypertarget{RC_mhttpd_Program_page_RC_mhttpd_Required}{}
 Example 2 shows the case where {\bfseries mlogger} is running, but is not {\bfseries Required} (i.e. \hyperlink{RC_customize_ODB_RC_programs_Required}{/Programs/logger/required} is false). When {\bfseries mlogger} is stopped by pressing the key {\bfseries \char`\"{}Stop Logger\char`\"{}} on the {\bfseries Programs} page, the {\bfseries Logger} program is no longer displayed. This is because  clients which are not {\bfseries Required} are not shown in the {\bfseries Programs} page unless they are running.

\par
\par
\par
 \begin{center}  Example 2: Programs page -\/ mlogger utility is not \hyperlink{RC_mhttpd_Program_page_RC_mhttpd_Required}{Required} \par
\par
\par
   \end{center}  \par
\par
\par


In Example 3, mlogger's \hyperlink{RC_mhttpd_Program_page_RC_mhttpd_Required}{Required} flag has been set true. In this case, when mlogger is stopped the line for the Logger program is retained. Because a \hyperlink{RC_customize_ODB_RC_programs_Start_command}{Start Command} has been supplied, the {\bfseries \char`\"{}Start Logger\char`\"{} button} is displayed. Pressing this will cause the {\bfseries Start Command} to be issued, and mlogger should start. The {\bfseries  /Programs/$<$ {\itshape client\/} $>$ tree } also provides for a client to be \hyperlink{RC_customize_ODB_RC_customize_Programs_tree}{automatically started or restarted}.

Note that an \hyperlink{RC_customize_ODB_RC_programs_Alarm_class}{Alarm Class} has also been supplied, giving rise to the {\bfseries Alarm banner} on the main status page. See \hyperlink{RC_customize_ODB_RC_Alarm_System}{MIDAS Alarm System} for details.

\par
\par
\par
\begin{center}  Example 3: Programs page -\/ mlogger utility {\bfseries is} \hyperlink{RC_customize_ODB_RC_programs_Required}{Required} and an alarm class has been specified \par
\par
\par
  \end{center}  \par
\par
\par


\par




\par
 \label{index_end}
\hypertarget{index_end}{}
 \paragraph{History page}\label{RC_mhttpd_History_page}
\label{RC_mhttpd_History_page_idx_mhttpd_page_history}
\hypertarget{RC_mhttpd_History_page_idx_mhttpd_page_history}{}
 \par




\par
 The {\bfseries History} page is displayed by clicking on the {\bfseries History} {\bfseries Button} on the \hyperlink{RC_mhttpd_Main_Status_page}{Main Status Page} ( \hyperlink{RC_mhttpd_status_page_features_RC_mhttpd_status_menu_buttons}{if present}).

This page reflects the \hyperlink{F_History_logging}{History System} settings.

It lists on the top of the page the possible group names containing a list of panels defined in the ODB. Next a series of buttons defines the time scale of the graph with predefined time window, {\bfseries  \char`\"{}$<$$<$\char`\"{},\char`\"{}$<$\char`\"{} \char`\"{}+\char`\"{} \char`\"{}-\/\char`\"{} \char`\"{}$>$\char`\"{} \char`\"{}$>$$>$\char`\"{} } buttons permit the shifting of the graph in the time direction. Other buttons will allow graph resizing, Elog attachment creation, configuration of the panel and custom time frame graph display. By default a single group is created \char`\"{}Default\char`\"{} containing the trigger rate for the \char`\"{}Trigger\char`\"{} equipment.

The configuration options for a given panel consists in:
\begin{DoxyItemize}
\item Zooming capability, run markers, logarithmic scale.
\item Data query in time.
\item Time scale in date format.
\item Web based page creation (button labelled \char`\"{}New\char`\"{}) for up to 10 history channels per page.
\end{DoxyItemize}

\par
\par
\par
 \begin{center}  History page. \par
\par
\par
  \end{center}  \par
\par
\par


\par
\par
\par
 \begin{center}  History channel selection Page. \par
\par
\par
  \end{center}  \par
\par
\par


\par




\par
 \label{index_end}
\hypertarget{index_end}{}
 \paragraph{Alarm page}\label{RC_mhttpd_Alarm_page}
\par
 \label{RC_mhttpd_Alarm_page_idx_mhttpd_page_alarm}
\hypertarget{RC_mhttpd_Alarm_page_idx_mhttpd_page_alarm}{}




\par


By clicking on the {\bfseries Alarms} button (\hyperlink{RC_mhttpd_status_page_features_RC_mhttpd_status_menu_buttons}{if present}) on the \hyperlink{RC_mhttpd_Main_Status_page}{Main Status Page}, the Alarms mhttpd page appears. This page presents information about the \hyperlink{RC_customize_ODB_RC_Alarm_System}{MIDAS Alarm System} from the \hyperlink{RC_customize_ODB_RC_ODB_Alarms_Tree}{ODB /Alarms Tree} . By clicking on the hyperlinks on this Alarms page, you can navigate directly to the alarm class and setup of a particular alarm under the /Alarms tree. In the example below, the name of the alarm that has gone off is \char`\"{}fepol\char`\"{} and the alarm class is \char`\"{}Caution\char`\"{}. Buttons are provided so that all or individual activated alarms can be reset.

\par
\par
\par
 \begin{center}  \par
\par
\par
  \end{center}  \par
\par
\par


\label{RC_mhttpd_Alarm_page_RC_mhttpd_alarm_banner}
\hypertarget{RC_mhttpd_Alarm_page_RC_mhttpd_alarm_banner}{}
 When an alarm has goes off, a banner will appear on the MIDAS Main Status page.

This following image shows the Main status page for the TRIUMF Pol experiment running with the alarm system enabled. The alarm on \char`\"{}fepol\char`\"{} has gone off, resulting in a large coloured banner with an alarm message.

\par
\par
\par
 \begin{center}  \par
\par
\par
  \end{center}  \par
\par
\par


\label{RC_mhttpd_Alarm_page_RC_odb_alarm_msg}
\hypertarget{RC_mhttpd_Alarm_page_RC_odb_alarm_msg}{}
 If running \hyperlink{RC_odbedit_utility}{odbedit}, a message appears on the screen as follows: 
\begin{DoxyCode}
[pol@isdaq01 ~]$ odb
Caution: Program fePOL is not running
[local:pol:S]/> 
\end{DoxyCode}


The alarm class in this case is \char`\"{}Caution\char`\"{} and the message is \char`\"{}Program fePOL is not running\char`\"{}. These fields have been set up in the \hyperlink{RC_customize_ODB_RC_ODB_Alarms_Tree}{ODB /Alarms Tree} ODB as described in the \hyperlink{RC_customize_ODB_RC_Alarm_System}{MIDAS Alarm System} .

A separate banner will appear for each alarm that is activated. Here three alarms of three different classes are activated. The classes have been set up with different colours for each class of alarm: \par
\par
\par
 \begin{center}  MIDAS Main Status page showing three alarm banners \par
\par
\par
  \end{center}  \par
\par
\par




\par
 \label{index_end}
\hypertarget{index_end}{}
 \paragraph{MSCB page}\label{RC_mhttpd_MSCB_page}
\par
 \label{RC_mhttpd_MSCB_page_idx_mhttpd_page_MSCB}
\hypertarget{RC_mhttpd_MSCB_page_idx_mhttpd_page_MSCB}{}




The MSCB (MIDAS Slow Control Bus) page is a new page that has been recently implemented in mhttpd (\hyperlink{NDF_ndf_dec_2009}{Dec 2009}) .

\par
 This allows web access to all devices in an \href{http://midas.psi.ch/mscb}{\tt MSCB system} and to their variables:

\par
\par
\par
 \begin{center}  An example of an MSCB page \par
\par
\par
  \end{center}  \par
\par
\par
 \par


In order to create the MSCB page, the flag \par
 {\bfseries -\/DHAVE\_\-MSCB} \par
 must be present in the Makefile for mhttpd. This is now the default in the Makefile from SVN, but it can be taken out for experiments not using MSCB. If the flag is present, mhttpd is linked against {\bfseries midas/mscb/mscb.c} and has direct access to all mscb ethernet submasters (USB access is currently disabled on purpose there). The presence of the flag {\bfseries -\/DHAVE\_\-MSCB} will cause the MSCB button to appear on the main status page by default, unless the \hyperlink{RC_mhttpd_status_page_features_RC_mhttpd_status_menu_buttons}{menu buttons} have been customized, in which case the MSCB key must be listed in the ODB Key /Experiment/Menu Buttons\par
e.g. 
\begin{DoxyCode}
/Experiment/Menu Buttons = Start, ODB, Messages, ELog, Alarms, Programs, History,
       MSCB, Config, Help
\end{DoxyCode}


The MSCB page uses the ODB Tree {\bfseries /MSCB/Submasters/...} to obtain a list of all available submasters:\par



\begin{DoxyCode}
[local:MEG:R]/MSCB>ls -r
MSCB
    Submaster
        mscb004
            Pwd                 xxxxx
            Comment             BTS
            Address             1
        mscb034
            Pwd                 xxxxx
            Comment             XEC HV & LED
            Address
                                0
                                1
                                2
\end{DoxyCode}


Each submaster tree contains an optional password needed by that submaster, an optional comment (which will be displayed on the 'Submaster' list on the web page), and an array of node addresses.\par
 \par
 These trees can be created by hand, but they are also created automatically by mhttpd if the /MSCB/Submaster entry is not present in the ODB. In this case, the equipment list is scanned and all MSCB devices and addresses are collected from locations such as \par
 /Equipment/$<$name$>$/Settings/Devices/Input/Device\par


or \par
 /Equipment/$<$name$>$/Settings/Devices/$<$name$>$/MSCB Device\par
 which are the locations for MSCB submasters used by the {\bfseries mscbdev.c} and {\bfseries mscbhvr.c} device drivers. Once the tree is created, it will not be touched again by mhttpd, so devices can be removed or reordered by hand.\par
 \par


\label{index_end}
\hypertarget{index_end}{}
  \paragraph{CAMAC Access page}\label{RC_mhttpd_CNAF_page}
\label{RC_mhttpd_CNAF_page_idx_mhttpd_page_CNAF}
\hypertarget{RC_mhttpd_CNAF_page_idx_mhttpd_page_CNAF}{}
 \par




\par
  By default, the CNAF page has been replaced by the MSCB Page (\hyperlink{NDF_ndf_dec_2009}{Dec 2009}) 

\par
 The CNAF page is accessed by clicking on the \hyperlink{RC_mhttpd_status_page_features_RC_mhttpd_status_menu_buttons}{CNAF button} (\hyperlink{RC_mhttpd_status_page_features_RC_mhttpd_status_menu_buttons}{if present}) on the Main Status page. \par
 It will only be active if one of the active equipments is a CAMAC-\/based data collector, it will be possible to remotely access CAMAC through this web-\/based CAMAC page, in which case the frontend is acting as a RPC CAMAC server for the request. \par
 The status of the connection is displayed in the top right hand side corner of the window.

\par
\par
\par
 \begin{center}  CAMAC command pages. \par
\par
\par
  \end{center}  \par
\par
\par


\par




\par


\label{index_end}
\hypertarget{index_end}{}
 \paragraph{mhttpd Alias page}\label{RC_mhttpd_Alias_page}
\label{RC_mhttpd_Alias_page_idx_mhttpd_page_alias}
\hypertarget{RC_mhttpd_Alias_page_idx_mhttpd_page_alias}{}
 \label{RC_mhttpd_Alias_page_idx_mhttpd_buttons_alias}
\hypertarget{RC_mhttpd_Alias_page_idx_mhttpd_buttons_alias}{}
 \par




\par
 \hypertarget{RC_mhttpd_Alias_page_RC_mhttpd_alias_buttons}{}\subparagraph{Alias-\/Buttons (or Hyperlinks)}\label{RC_mhttpd_Alias_page_RC_mhttpd_alias_buttons}
An Alias page is displayed by clicking on a user-\/defined  {\bfseries  Alias-\/button } ({\bfseries alias-\/hyperlink} prior to the \hyperlink{RC_mhttpd_status_page_redesign}{redesign})  on the mhttpd \hyperlink{RC_mhttpd_status_page_features_RC_mhpptd_optional_buttons}{main status page}.

Alias-\/buttons provide the user with a simple way to access other webpages (remote or local). They are often used to provide \char`\"{}shortcuts\char`\"{} from the main Status page to the documentation for an experiment, or to a particular ODB location. For example, if the experiment requires frequent reference to a list of \hyperlink{structparameters}{parameters} under /Equipment/TpcGasPlc/Common/, an {\bfseries  alias button } can be set up on the main status page to allow the user access to this tree with one click of the mouse.  Alias and \hyperlink{RC_mhttpd_Activate}{Custom} buttons are displayed on the \hyperlink{RC_mhttpd_status_page_redesign}{same line} on the status page. \par



\begin{DoxyItemize}
\item \hyperlink{RC_mhttpd_Alias_page_RC_mhttpd_alias_define}{How to create Alias-\/Buttons}
\item \hyperlink{RC_mhttpd_Alias_page_RC_odb_alias_tree}{The ODB /Alias Tree}
\end{DoxyItemize}

\label{RC_mhttpd_Alias_page_mhttpd_alias_image}
\hypertarget{RC_mhttpd_Alias_page_mhttpd_alias_image}{}
 \par
\par
\par
 \begin{center} Clicking on an alias-\/button displays the link contents   \end{center}  \par
\par
\par
 \par


\par
\hypertarget{RC_mhttpd_Alias_page_RC_mhttpd_alias_define}{}\subparagraph{How to create Alias-\/Buttons}\label{RC_mhttpd_Alias_page_RC_mhttpd_alias_define}
Alias-\/Buttons (hyperlinks prior to \hyperlink{NDF_ndf_dec_2009}{Dec 2009}) on the \hyperlink{RC_mhttpd_status_page_features_RC_mhpptd_optional_buttons}{main status page} are defined through the \hyperlink{RC_mhttpd_Alias_page_RC_odb_alias_tree}{ODB Alias tree}. \par


\label{RC_mhttpd_Alias_page_idx_ODB_tree_Alias}
\hypertarget{RC_mhttpd_Alias_page_idx_ODB_tree_Alias}{}
 \hypertarget{RC_mhttpd_Alias_page_RC_odb_alias_tree}{}\subparagraph{The ODB /Alias Tree}\label{RC_mhttpd_Alias_page_RC_odb_alias_tree}
\begin{DoxyNote}{Note}
The  /Alias  tree is applicable to \hyperlink{RC_mhttpd}{mhttpd} only, and ignored by \hyperlink{RC_odbedit}{odbedit}.
\end{DoxyNote}
This optional ODB tree provides the user with a way to access other webpages via {\bfseries buttons} placed on the mhttpd \hyperlink{RC_mhttpd_status_page_redesign}{main Status page} ({\bfseries hyperlinks} prior to \hyperlink{NDF_ndf_dec_2009}{Dec 2009}).

\par


The ODB /Alias key is not present until the user creates it. It is intended to contain a list of symbolic links to any desired ODB location. Any key created under /Alias will appear as a {\bfseries Button} or \hyperlink{RC_mhttpd_status_page_redesign}{Hyperlink} on the Main Status page, with the same name as the key, except where noted below (e.g. alias links spawned in the same frame). \par
 \par
 By default, the clicking of the alias-\/button in the web interface will spawn a {\bfseries new} {\bfseries frame}. To force the display of the alias page in the {\bfseries same} {\bfseries frame}, an {\bfseries \char`\"{}\&\char`\"{}} has to be appended to the name of the alias. The {\bfseries \&} is stripped off the alias name when it appears on the main status page.

The following code demonstrates creating alias-\/buttons linking to ODB keys using \hyperlink{RC_odbedit_utility}{odbedit} : 
\begin{DoxyCode}
odbedit
[local:t2kgas:S] mkdir Alias          ***  Create the optional /Alias directory
[local:t2kgas:S] cd Alias
[local:t2kgas:S] ln /Equipment/TpcGasPlc/Common/ "TPC Common"    *** New frame, n
      o &
[local:t2kgas:S] ln /Equipment/TpcGasPlc/Common/ "TPC Common"&"  *** Same frame, 
      with &
\end{DoxyCode}
  The items preceded by {\bfseries $\ast$$\ast$$\ast$} are comments

This would create two identical alias-\/buttons called \char`\"{}TPC Common\char`\"{}. Clicking on the first alias-\/button \char`\"{}TPC Common\char`\"{} would open the page in a {\bfseries new} frame as shown \hyperlink{RC_mhttpd_Alias_page_mhttpd_alias_image}{above}; clicking on the second would open it in the {\bfseries same} frame. (Two identical buttons have been created for demonstration purposes only. Normally all buttons would be created with unique names.)

The following code demonstrates how to make a link to an external webpage: 
\begin{DoxyCode}
[local:t2kgas:S] cd Alias
[local:t2kgas:S]create string triumf
String length [32]:
[local:t2kgas:S] set triumf "http://triumf.ca"
\end{DoxyCode}
 Clicking on the alias-\/button \char`\"{}triumf\char`\"{} will show the contents of the link in a new page (see \hyperlink{RC_mhttpd_Alias_page_mhttpd_alias_image}{above}).

\par
 \label{index_end}
\hypertarget{index_end}{}
 \par
 

\par
 \paragraph{mhttpd Logger page}\label{RC_mhttpd_Logger_page}
\label{RC_mhttpd_Logger_page_idx_mhttpd_page_logger}
\hypertarget{RC_mhttpd_Logger_page_idx_mhttpd_page_logger}{}
 \par


\hypertarget{RC_mhttpd_Logger_page_RC_mhttpd_Logger_mlogger}{}\subparagraph{mlogger settings information}\label{RC_mhttpd_Logger_page_RC_mhttpd_Logger_mlogger}
Provided the MIDAS logger \hyperlink{F_Logging_F_mlogger_utility}{mlogger} is running, the mhttpd main status page shows logger information and statistics. \par


mlogger channel information is displayed by clicking on a mlogger channel hyperlink on the mhttpd main status page (see \hyperlink{RC_mhttpd_Logger_page_RC_mhttpd_Logger_image}{below}). The subtree /Logger/Channels/$<$channel-\/number$>$/Settings is shown. See \hyperlink{F_Logging}{mlogger} for more information. In the example below, only one mlogger channel is defined (channel 0). If multiple logging channels are active, additional hyperlinks will be present on the main status page. \par
 \label{RC_mhttpd_Logger_page_RC_mhttpd_Logger_image}
\hypertarget{RC_mhttpd_Logger_page_RC_mhttpd_Logger_image}{}
 \begin{center} mlogger channel information on mhttpd main status page \par
\par
\par
  \end{center}  \par


\par


\par
\hypertarget{RC_mhttpd_Logger_page_RC_mhttpd_Logger_lazylogger}{}\subparagraph{lazylogger settings information}\label{RC_mhttpd_Logger_page_RC_mhttpd_Logger_lazylogger}
Provided the MIDAS lazylogger \hyperlink{F_LogUtil_F_lazylogger_utility}{lazylogger} is running, the mhttpd \hyperlink{RC_mhttpd_Logger_page_RC_mhttpd_Logger_image}{main status page} shows lazylogger information and statistics. \par


Lazylogger settings information is displayed by clicking on a lazylogger Label hyperlink on the mhttpd main status page (see \hyperlink{RC_mhttpd_Logger_page_RC_mhttpd_Logger_image}{above}). The subtree /Lazy/$<$label$>$/Settings is shown. See \hyperlink{F_LogUtil_F_lazylogger_utility}{lazylogger} for more information.

 \label{index_end}
\hypertarget{index_end}{}
 \paragraph{Config page}\label{RC_mhttpd_Config_page}


This page is displayed by pressing the {\bfseries Config Button} ( \hyperlink{RC_mhttpd_status_page_features_RC_mhttpd_status_menu_buttons}{if present}) on the \hyperlink{RC_mhttpd_Main_Status_page}{Main Status Page} . This allows the user to change the refresh period of the Status page.

\begin{center} mhttpd Config page \par
  \end{center} 

It is important to note that the {\bfseries refresh} of the Status Page is not \char`\"{}event driven\char`\"{} but is controlled by a timer whose rate is adjustable through the \hyperlink{RC_mhttpd_status_page_features_RC_mhttpd_Config_button}{Config button}. This means the information at any given time may reflect the experiment state of up to {\itshape  n \/} seconds in the past, where {\itshape  n \/} is the timer setting of the refresh parameter.

\label{index_end}
\hypertarget{index_end}{}
  \paragraph{Custom pages}\label{RC_mhttpd_Custom_page}
\par
  \label{RC_mhttpd_Custom_page_idx_mhttpd_page_custom}
\hypertarget{RC_mhttpd_Custom_page_idx_mhttpd_page_custom}{}
 \label{RC_mhttpd_Custom_page_idx_custom-see-mhttpd-page-custom}
\hypertarget{RC_mhttpd_Custom_page_idx_custom-see-mhttpd-page-custom}{}
 \par
 \par
 \hypertarget{RC_mhttpd_Custom_page_RC_mhttpd_custom_intro}{}\subparagraph{Introduction}\label{RC_mhttpd_Custom_page_RC_mhttpd_custom_intro}
Custom web pages provide the user with a means of creating secondary user-\/created web page(s) activated within the standard MIDAS web interface. These custom pages can contain specific links to the ODB, and therefore can present the essential \hyperlink{structparameters}{parameters} of the controlled experiment in a more compact way. A custom page may even \hyperlink{RC_mhttpd_Activate_RC_odb_custom_status}{replace} the default MIDAS Status Page. \par


\label{RC_mhttpd_Custom_page_idx_mhttpd_page_custom_example-pages}
\hypertarget{RC_mhttpd_Custom_page_idx_mhttpd_page_custom_example-pages}{}
 \hypertarget{RC_mhttpd_Custom_page_RC_Example_Custom_Webpages}{}\subparagraph{Examples of Custom Webpages}\label{RC_mhttpd_Custom_page_RC_Example_Custom_Webpages}
Examples of custom webpages :


\begin{DoxyItemize}
\item \hyperlink{RC_ROOT_analyzer_page}{Custom Page showing ROOT analyzer output} -\/ ROOT graphical output served through an mhttpd custom page
\item \hyperlink{RC_MEG_Gas_Page}{Custom Page for MEG gas system} -\/ complex gas system illustrating use of \char`\"{}fills\char`\"{} and \char`\"{}labels\char`\"{}
\item \hyperlink{RC_Ebit_custom_page}{Custom Pages for Ebit Experiment} -\/ page showing clickable input boxes, checkboxes
\item \hyperlink{RC_Mpet_custom_page}{Mpet Optics Custom Page} -\/ page showing labels
\item \hyperlink{RC_T2K_Gas_Page}{Custom Pages designed for T2K gas system} -\/ highly complex gas system showing popups, labels, fills, clickable areas
\item \hyperlink{RC_BNMQR_status}{Custom Status Pages for experiments BNMR and BNQR} -\/ custom status page replaces MIDAS standard status page \label{index_end}
\hypertarget{index_end}{}

\end{DoxyItemize}

\par


Two ways of implementing custom pages are available: 
\begin{DoxyEnumerate}
\item {\bfseries Internal} 
\begin{DoxyItemize}
\item The html code is fully stored in the Online Database (ODB). 
\item Limited capability, size restricted. 
\item The page is web editable. 
\end{DoxyItemize}
\item {\bfseries External} 
\begin{DoxyItemize}
\item ODB contains links to external html or Javascript document(s). 
\item Multiple custom pages are supported, 
\item The page can be edited with an external editor. 
\end{DoxyItemize}
\end{DoxyEnumerate}\par
 External pages are more commonly used, hence most of the {\bfseries examples in this section are those of external} pages. All the features described are available for both internal and external pages. An example of an {\bfseries internal} page is shown \hyperlink{RC_mhttpd_Internal}{here}.


\begin{DoxyItemize}
\item \hyperlink{RC_mhttpd_Activate}{How to activate custom page(s) in the ODB}
\item \hyperlink{RC_mhttpd_custom_js_lib}{JavaScript built-\/in library mhttpd.js}
\item \hyperlink{RC_mhttpd_custom_features}{Features available on custom pages}
\end{DoxyItemize}

\par
 \par
 \label{index_end}
\hypertarget{index_end}{}


 \subsubsection{Custom Page showing ROOT analyzer output}\label{RC_ROOT_analyzer_page}
\label{RC_ROOT_analyzer_page_idx_mhttpd_page_custom_examples_ROOT}
\hypertarget{RC_ROOT_analyzer_page_idx_mhttpd_page_custom_examples_ROOT}{}
 \par
 

Many MIDAS experiments work with ROOT based analyzers today. One problem there is that the graphical output of the root analyzer can only be seen through the X server and not through the web. At the MEG experiment, this problem was solved in an elegant way: The ROOT analyzer runs in the background, using a \char`\"{}virtual\char`\"{} X server called Xvfb. It plots its output (several panels) normally using this X server, then saves this panels every ten seconds into GIF files. These GIF files are then served through mhttpd using a custom page. The output looks like this:

\par
 \begin{center} Custom page showing ROOT Analyzer (MEG Experiment)  \end{center}  \par
 The buttons on the left sides are actually HTML buttons on that custom page overlaid to the GIF image, which in this case shows one of the 800 PMT channels digitized at 1.6 GSPS. With these buttons one can cycle through the different GIF images, which then automatically update ever ten seconds. Of course it is not possible to feed interaction back to the analyzer (i.e. the waveform cannot be fitted interactively) but for monitoring an experiment in production mode this tool is extremely helpful, since it is seamlessly integrated into mhttpd. All the magic is done with JavaScript, and the buttons are overlaid on the graphics using CSS with absolute positioning. The analysis ratio on the top right is also done with JavaScript accessing the required information from the ODB. \par


The custom page file is shown here:


\begin{DoxyItemize}
\item \hyperlink{RC_MEG_ROOT_code}{HTML code for the MEG ROOT Analyzer page}
\end{DoxyItemize}

For details using Xvfb server, please contact Ryu Sawada $<$\href{mailto:sawada@icepp.s.u-tokyo.ac.jp}{\tt sawada@icepp.s.u-\/tokyo.ac.jp}$>$.

\par
 \par
 \label{index_end}
\hypertarget{index_end}{}
  \paragraph{HTML code for the MEG ROOT Analyzer page}\label{RC_MEG_ROOT_code}


The following code is used for the \hyperlink{RC_ROOT_analyzer_page}{Custom Page showing ROOT analyzer output} : 
\begin{DoxyCode}
<html><head>
<meta http-equiv="content-type" content="text/html; charset=ISO-8859-1"><!-- $Id:
       analyzer.html 14662 2009-12-05 01:51:33Z ritt $ -->

  
    <title>Crates Status Page</title>
    <style type="text/css">
      <!--
      a:link     {
      text-decoration:none;
      color:#0000A0;
      }
      a:visited  {
      color:#0000A0;
      text-decoration:none;
      }
      body       {
      font-family:verdana,tahome,sans-serif;
      font-size:16px;
      line-height:16px;
      margin: 2px;
      }

      #i { position:relative; }
      #t { position:absolute; left:0px; top:140px; }
      #b { width:130px; }
      -->
    </style>
  <script type="text/javascript" src="MEG_analyzer_files/mhttpd.html"></script>
  <script type="text/javascript">
  
  var image_name = [  
    "eventdisplay2d.gif",
    "trgrate.gif",
    "trgsync.gif",
    "trgdaqrate.gif",
    "trgmonitor.gif",
    "drscount.gif",
    "-",
    "xec2d.gif",
    "xecwaveform.gif",
    "-",
    "dch2d.gif",
    "dch_hitmap.gif",
    "-",
    "tic2d.gif",
    "ticphit.gif",
  ];

  var image_title = [ 
    "Event Display 2D",
    "Trigger Scalers",
    "Trigger Sync",
    "Trigger Rates",
    "Trigger Monitor",
    "DRS Count",
    "-",
    "XEC 2D",
    "XEC Waveforms",
    "-",
    "DCH 2D",
    "DCH Hitmap",
    "-",
    "TIC 2D",
    "TIC Hits",
  ];

  var refreshID = null;

  function disp(i)
  {
    /* update image */
    var image = document.getElementById('img');
    var d = new Date();
    var s = d.toString();
    var t = document.getElementById('title_line');

    image.src = 'monitor/'+image_name[i]+'?'+d.getTime();
    if (navigator.appName == "Netscape")
      t.innerHTML = '<B>'+image_title[i]+'</B>'+'&nbsp;&nbsp;'+s.slice(16, 25)+' 
      CET';
    else
      t.innerHTML = '<B>'+image_title[i]+'</B>'+'&nbsp;&nbsp;'+s.slice(10, 19)+' 
      CET'; // mainly IE

    var n1 = ODBGet('/BGAnalyzer/Trigger/Statistics/Events received');
    var n2 = ODBGet('/Equipment/Trigger/Statistics/Events sent');
    document.getElementById('ratio').innerHTML = 'Analysis ratio: '+n1+'/'+n2;

    if (refreshID != null)
      clearInterval(refreshID);
    refreshID = setTimeout("disp("+i+")", 10000);
  }
  
  </script>
  </head><body onload="disp(0);">
    <form name="form1" method="GET" action="Crates">
      <table border="3" cellpadding="2">
        <tbody><tr>
          <td id="title_line" colspan="2" align="center" bgcolor="#a0a0ff"><b>Eve
      nt Display 2D</b>&nbsp;&nbsp;13:19:21  CET</td>
        </tr>
        <tr>
          <td bgcolor="#c0c0c0">
            <input name="cmd" value="ODB" type="submit">
            <input name="cmd" value="Alarms" type="submit">
            <input name="cmd" value="Status" type="submit">
          </td>
          <td id="ratio" nowrap="nowrap" width="200" align="center" bgcolor="#c0c
      0c0">
            Analysis ratio: 0/0
          </td>
        </tr>  
        <tr>
          <td colspan="2">
            <div id="i">
              <img id="img" src="MEG_analyzer_files/eventdisplay2d.html" alt="Ana
      lyzer Screendump" border="0">
              <table id="t">
                <tbody><tr>
                  <td nowrap="nowrap" valign="top" bgcolor="#c0c0c0">
                    <hr>
<script type="text/javascript">

  for (var i=0 ; i<image_name.length ; i++)
    if (image_name[i] == "-")
       document.writeln("<hr>");
    else
       document.writeln("<button type=\"button\" id=\"b\" onclick=\"disp("+i+");\
      ">"+image_title[i]+"</button><br>");

</script><button type="button" id="b" onclick="disp(0);">Event Display 2D</button
      ><br>
<button type="button" id="b" onclick="disp(1);">Trigger Scalers</button><br>
<button type="button" id="b" onclick="disp(2);">Trigger Sync</button><br>
<button type="button" id="b" onclick="disp(3);">Trigger Rates</button><br>
<button type="button" id="b" onclick="disp(4);">Trigger Monitor</button><br>
<button type="button" id="b" onclick="disp(5);">DRS Count</button><br>
<hr>
<button type="button" id="b" onclick="disp(7);">XEC 2D</button><br>
<button type="button" id="b" onclick="disp(8);">XEC Waveforms</button><br>
<hr>
<button type="button" id="b" onclick="disp(10);">DCH 2D</button><br>
<button type="button" id="b" onclick="disp(11);">DCH Hitmap</button><br>
<hr>
<button type="button" id="b" onclick="disp(13);">TIC 2D</button><br>
<button type="button" id="b" onclick="disp(14);">TIC Hits</button><br>

                  </td>
                </tr>
              </tbody></table>
            </div></td>
          
        </tr>
      </tbody></table>
    </form>
  </body></html>
\end{DoxyCode}


\label{index_end}
\hypertarget{index_end}{}
  \subsubsection{Custom Page for MEG gas system}\label{RC_MEG_Gas_Page}
\par




\par


This page from the MEG experiment at PSI shows a complex gas system. This shows the use of \char`\"{}fills\char`\"{} and \char`\"{}labels\char`\"{}. \par
 The valves are represented as green circles. If they are clicked, they close and become red (provided the user successfully supplied the correct password). \par
 \begin{center} MEG gas system  \end{center}  \par




\par
 \label{index_end}
\hypertarget{index_end}{}
 \subsubsection{Custom Pages for Ebit Experiment}\label{RC_Ebit_custom_page}
\par




\par


The following two custom pages are from the EBIT experiment at TRIUMF. They allow the experimenters easy access to the \hyperlink{structparameters}{parameters} for the TRIUMF Pulse Programmer (PPG) (i.e. pulse width, offset etc.) used to control the experiment.

Several different modes are available (each with its own custom page). Two of the custom pages are shown, for modes 1c and 1e. The experimenters change modes by clicking one of the customscript buttons (1a...1e). The mode script automatically loads a different set of \hyperlink{structparameters}{parameters} and a different custom page.

In the first example, the PPG \hyperlink{structparameters}{parameters} are changed by clicking on the links on the image. Each value is linked to a parameter in the ODB. Each line on the image represents an output from the Pulse Programmer. The x-\/axis ( {\bfseries not} to scale) represents the time from when the cycle is started (T0). The boxes on the left allow the polarity of the output signal to be reversed. \par
 \begin{center} EBIT experiment custom page (Mode 1c)\par
\par
  \end{center}  \par
\par


For the simplest mode (1e) a timing diagram is not necessary, and a table is used instead. To change the \hyperlink{structparameters}{parameters} the experimenters click on the links in the table. \par
\par
 \begin{center} EBIT experiment custom page (Mode 1e)\par
\par
  \end{center} 

\par
\par




\par
 \label{index_end}
\hypertarget{index_end}{}
 \subsubsection{Mpet Optics Custom Page}\label{RC_Mpet_custom_page}
\par




\par


This page shows the optics for the Mpet experiment at TRIUMF displaying the Lorenz steerer \hyperlink{structparameters}{parameters}.

\begin{center}  \end{center} 

\par




\par
 \label{index_end}
\hypertarget{index_end}{}
 \subsubsection{Custom Pages designed for T2K gas system}\label{RC_T2K_Gas_Page}
\par




\par
\hypertarget{RC_T2K_Gas_Page_T2K_example_1}{}\subparagraph{Example custom page: pop-\/up window, clickable boxes}\label{RC_T2K_Gas_Page_T2K_example_1}
Clicking on one of the valves produces a pop-\/up window (a flow-\/control valve A0FC2 is shown), allowing the set value for the flow rate to be changed, and the valve to be opened or closed. Navigation between the more than 20 different pages of this large gas system is done by clicking on one of the blue boxes. The pop-\/up windows are made using javascript. \char`\"{}Fills\char`\"{} are used to change the colour of the valves, and \char`\"{}Labels\char`\"{} are used to display the values of various \hyperlink{structparameters}{parameters}.

\begin{center} T2K Gas System: Gas Analyzer Page showing popup\par
  \end{center} 

\par
 \par
 \hypertarget{RC_T2K_Gas_Page_T2K_example_2}{}\subparagraph{Example custom page: hidden alias keys}\label{RC_T2K_Gas_Page_T2K_example_2}
The following image illustrates the use of hidden alias keys to load webpages from a clickable image map.

\begin{center} Custom pages written for the T2K experiment  \end{center}  \par
 

\par
 \label{index_end}
\hypertarget{index_end}{}
 \subsubsection{Custom Status Pages for experiments BNMR and BNQR}\label{RC_BNMQR_status}
\par
 

The webpages shown below replace the MIDAS default status pages for the BNMR and BNQR experiments at TRIUMF. A single file (custom\_\-status.html) is used for the custom status pages of both experiments. The colours are different to avoid confusion, since the two experiments may be run simultaneously with the main status page of each displayed on two adjacent consoles. Javascript is used extensively in the .html file. Colours are used to indicate the status of various \hyperlink{structparameters}{parameters}. Both Custom buttons and the regular buttons (Stop, Status, Messages etc.) are used.

In the image below the BNMR experiment is {\bfseries running}, while the BNQR experiment is {\bfseries stopped}. Note that with the BNQR run stopped, various PPG Mode buttons appear which are not available when running (compare with BNMR). These buttons allow the user to select the experimental mode. Various alias links allow the user to quickly access experimental \hyperlink{structparameters}{parameters}.

\begin{center} Custom Status Page for the BNMR experiment at TRIUMF  \par
\par
 Custom Status Page for the BNQR experiment at TRIUMF  \end{center} 

\par




\par
 \label{index_end}
\hypertarget{index_end}{}
 \subsubsection{How to activate custom page(s) in the ODB}\label{RC_mhttpd_Activate}
\par
  \par
 \par
 \hypertarget{RC_mhttpd_Activate_RC_odb_custom_intro}{}\subparagraph{Introduction}\label{RC_mhttpd_Activate_RC_odb_custom_intro}
 In order to activate a custom page from the within the MIDAS web interface, the pages must be listed in the ODB under the optional /Custom  tree as described below. This applies to \hyperlink{RC_mhttpd_Activate_RC_odb_custom_external_example}{external} and \hyperlink{RC_mhttpd_Activate_RC_odb_custom_internal_example}{internal} custom pages, as explained below.

The /Custom tree must be created by the user, e.g. using \hyperlink{RC_odbedit_utility}{odbedit} 
\begin{DoxyCode}
odbedit
[local:midas:S] mkdir Custom
\end{DoxyCode}
 \label{RC_mhttpd_Activate_idx_mhttpd_buttons_custom}
\hypertarget{RC_mhttpd_Activate_idx_mhttpd_buttons_custom}{}
 \label{RC_mhttpd_Activate_idx_ODB_tree_custom}
\hypertarget{RC_mhttpd_Activate_idx_ODB_tree_custom}{}
 \hypertarget{RC_mhttpd_Activate_RC_odb_custom_tree}{}\subparagraph{The /Custom ODB tree}\label{RC_mhttpd_Activate_RC_odb_custom_tree}
Features involving the /Custom tree is applicable only to \hyperlink{RC_mhttpd}{mhttpd}, and ignored by \hyperlink{RC_odbedit}{odbedit} , which has no web capabilities.

\par
 The optional ODB /Custom tree may contain
\begin{DoxyItemize}
\item links to {\bfseries local} {\bfseries external} custom web pages created by the user (see \hyperlink{RC_mhttpd_Activate_remote}{below} for {\bfseries remote} {\bfseries external} pages)
\item HTML content of an {\bfseries internal} custom web page created by the user.
\item {\bfseries images} subtree (explained \hyperlink{RC_mhttpd_Image_access_RC_mhttpd_custom_image}{later}) used to specify images for custom pages
\end{DoxyItemize}

If keys are defined in this tree, except where noted \hyperlink{RC_mhttpd_Activate_RC_odb_custom_keynames}{below}, the names of the keys will appear as custom-\/buttons in versions after \hyperlink{NDF_ndf_dec_2009}{Dec 2009} (or \hyperlink{RC_mhttpd_Alias_page}{custom-\/links} in earlier versions) on the  mhttpd \hyperlink{RC_mhttpd_Main_Status_page_RC_mhttpd_main_status}{Main Status page}. By clicking on one of these alias-\/buttons, the custom page will be visible in a new frame. \par
\hypertarget{RC_mhttpd_Activate_RC_odb_custom_keynames}{}\subparagraph{Keynames in the /Custom directory}\label{RC_mhttpd_Activate_RC_odb_custom_keynames}
There are two characters that have special meaning if they are the {\bfseries last} character of the keyname:
\begin{DoxyItemize}
\item The character {\bfseries \char`\"{}\&\char`\"{}} forces the page to be opened within the current frame. If this character is omitted, the page will be opened in a new frame (default).
\end{DoxyItemize}


\begin{DoxyItemize}
\item The character {\bfseries \char`\"{}!\char`\"{}} suppresses the link appearing on the main status page as a button (or link). This can be used to provide external webpages initially hidden from the user, such as code for pop-\/ups, or to access a file of Javascript functions needed by other custom pages.
\end{DoxyItemize}\hypertarget{RC_mhttpd_Activate_RC_odb_custom_external_example}{}\subparagraph{Create a link to an external webpage}\label{RC_mhttpd_Activate_RC_odb_custom_external_example}
If activating an external webpage, create a link for it in the ODB /Custom tree. For example, to create a link called Test\& that will open a  local  {\bfseries external} html file {\itshape  /home/t2ktpc/online/custom/try.html \/} in a new frame: 
\begin{DoxyCode}
odbedit
[local:pol:S] mkdir Custom   <-- if /Custom is not present
[local:pol:S] cd Custom
[local:pol:S]/>create string Test&
String length [32]: 80
[local:pol:S]/>set Test& "/home/t2ktpc/online/custom/try.html"
[local:pol:S]/>ls -lt my_page
Test&                            STRING  1     80    >99d 0   RWD  /home/t2ktpc/o
      nline/custom/try.html
\end{DoxyCode}
 \par
 \label{RC_mhttpd_Activate_remote}
\hypertarget{RC_mhttpd_Activate_remote}{}
 If the external HTML file is on a  remote  {\bfseries webserver}, the link should be placed under the \hyperlink{RC_mhttpd_Alias_page_RC_odb_alias_tree}{/Alias} tree rather than the /Custom tree, e.g.


\begin{DoxyCode}
[local:Default:Stopped]/>mkdir Alias  <-- if /Alias is not present
[local:Default:Stopped]/>cd alias
[local:Default:Stopped]/alias>create string WebDewpoint&
String length [32]: 256
[local:Default:Stopped]/alias>set WebDewpoint& "http://www.decatur.de/javascript/
      dew/index.html"
\end{DoxyCode}
\hypertarget{RC_mhttpd_Activate_RC_odb_custom_internal_example}{}\subparagraph{Import an internal webpage}\label{RC_mhttpd_Activate_RC_odb_custom_internal_example}
If creating an internal webpage, it must be imported into the ODB using the \hyperlink{RC_odbedit_examples_RC_odbedit_import}{odbedit import command}. This example shows a file {\bfseries mcustom.html} imported into the key /Custom/Test\&.


\begin{DoxyCode}
  Tue> odbedit                      
  [local:midas:Stopped]/>mkdir Custom    <-- create /Custom if it does not exist
  [local:midas:Stopped]/>cd Custom/
  [local:midas:Stopped]/Custom>import mcustom.html   <-- import an html file
  Key name: Test&              
  [local:midas:Stopped]/Custom>
\end{DoxyCode}


Example of an \hyperlink{RC_mhttpd_Internal}{Internal custom page} is shown here.\hypertarget{RC_mhttpd_Activate_RC_odb_custom_keys_examples}{}\subparagraph{Examples of /Custom keys}\label{RC_mhttpd_Activate_RC_odb_custom_keys_examples}
Here is an example of different keys in /Custom (all for {\bfseries external} webpages) 
\begin{DoxyCode}
GasMain&                        STRING  1     80    >99d 0   RWD  /home/t2ktpc/on
      line/custom_gas/GasMain.html
js_functions!                   STRING  1     64    >99d 0   RWD  /home/t2ktpc/on
      line/custom_gas/custom_functions.js
Purifier!                       STRING  1     132   >99d 0   RWD  /home/t2ktpc/on
      line/custom_gas/purifier.html
style0!                         STRING  1     45    >99d 0   RWD  /home/t2ktpc/on
      line/custom_gas/t0.css                    
\end{DoxyCode}



\begin{DoxyItemize}
\item The first key GasMain\& will produce an alias-\/link on the main status page called {\bfseries GasMain}. Clicking this alias-\/link will load the file {\bfseries GasMain.html} into the {\bfseries same} frame (the keyname ends in \char`\"{}\&\char`\"{}).
\end{DoxyItemize}

The other keys will not produce alias-\/links, since the keynames end with \char`\"{}!\char`\"{}).


\begin{DoxyItemize}
\item The key js\_\-functions! is used to load some javascript functions needed by the html code of the other custom pages.
\item The key Purifier! is a custom page that can be loaded by clicking on a box in the custom page GasMain.
\item The key style0! contains a style sheet for use with all the html custom pages.
\end{DoxyItemize}

The output is shown in \hyperlink{RC_T2K_Gas_Page_T2K_example_2}{this example}.\hypertarget{RC_mhttpd_Activate_RC_odb_custom_status}{}\subparagraph{How to replace the default Status page by a Custom Status page}\label{RC_mhttpd_Activate_RC_odb_custom_status}
By using a keyname of Status (no \char`\"{}\&\char`\"{}) in the link to a custom page, that page will replace the Default main status page. Clicking on the {\bfseries Status} button on any of the sub-\/pages will return to the {\bfseries Custom} status page.

\par
 See \hyperlink{RC_mhttpd_custom_status}{Custom Status page} for more information and examples.

\par
 

\label{index_end}
\hypertarget{index_end}{}
 \subsubsection{JavaScript built-\/in library mhttpd.js}\label{RC_mhttpd_custom_js_lib}


\label{RC_mhttpd_custom_js_lib_idx_JavaScript_built-in-library}
\hypertarget{RC_mhttpd_custom_js_lib_idx_JavaScript_built-in-library}{}
 \label{RC_mhttpd_custom_js_lib_idx_mhttpd-javascript-library}
\hypertarget{RC_mhttpd_custom_js_lib_idx_mhttpd-javascript-library}{}


Many of the supported features of MIDAS Custom web pages are available under HTML, but others rely on the inclusion of the mhttpd JavaScript (JS) library . This JS library relies on certain new commands which are built into mhttpd, and is therefore hardcoded into mhttpd.c .

The JS library can be viewed by entering in your browser location box: 
\begin{DoxyCode}
http://<mhttpd host>/mhttpd.js
\end{DoxyCode}


i.e. if running mhttpd on {\bfseries host} \char`\"{}midaspc\char`\"{} on {\bfseries port} 8081, you would enter : 
\begin{DoxyCode}
http://midaspc:8081/mhttpd.js
\end{DoxyCode}
 to view the JS library. Alternatively, a recent version is shown here:
\begin{DoxyItemize}
\item \hyperlink{RC_mhttpd_js}{Javascript Built-\/In library} Javascript Built-\/in Library code : mhttpd.js
\end{DoxyItemize}

\par


 \hypertarget{RC_mhttpd_custom_js_lib_RC_mhttpd_js_library_features}{}\subparagraph{List of functions in JS library}\label{RC_mhttpd_custom_js_lib_RC_mhttpd_js_library_features}
Use of the following functions in custom webpages require the JS library to be included:
\begin{DoxyItemize}
\item \hyperlink{RC_mhttpd_custom_ODB_access_RC_mhttpd_custom_odbget}{ODBGet}
\item \hyperlink{RC_mhttpd_custom_ODB_access_RC_mhttpd_custom_odbset}{ODBSet}
\item \hyperlink{RC_mhttpd_custom_ODB_access_RC_mhttpd_custom_odbedit}{ODBEdit}
\item \hyperlink{RC_mhttpd_custom_ODB_access_RC_mhttpd_custom_odbkey}{ODBKey}
\item \hyperlink{RC_mhttpd_custom_features_RC_mhttpd_custom_getmsg}{ODBGetMsg}
\item \hyperlink{RC_mhttpd_Image_access_RC_mhttpd_custom_getmouse}{getMouseXY}
\item \hyperlink{RC_mhttpd_custom_RPC_access}{ODBRpc\_\-rev0}
\end{DoxyItemize}



 \hypertarget{RC_mhttpd_custom_js_lib_RC_mhttpd_include_js_library}{}\subparagraph{Include the JS library mhttpd.js in a custom page}\label{RC_mhttpd_custom_js_lib_RC_mhttpd_include_js_library}
In order to use any of the {\bfseries  JavaScript built-\/in functions}, you must include the \hyperlink{RC_mhttpd_custom_js_lib}{library} in your custom page by putting the following statement inside the $<$head$>$...$<$/head$>$ tags of an HTML file:


\begin{DoxyCode}
<!DOCTYPE HTML PUBLIC "-//W3C//DTD HTML 4.0 TRANSITIONAL//EN">
<html><head>
<title> ODBEdit test</title>
<script type="text/javascript" src="../mhttpd.js"></script>
...
</head>
...
</html>
\end{DoxyCode}


\par
 \label{index_end}
\hypertarget{index_end}{}


 

 \paragraph{Javascript Built-\/In library}\label{RC_mhttpd_js}
 \par


The following code is the Javascript Built-\/in library {\bfseries mhttpd.js} (version 4505)


\begin{DoxyCode}
document.onmousemove = getMouseXY;

function getMouseXY(e)
{
   var x = e.pageX;
   var y = e.pageY;
   var p = 'abs: ' + x + '/' + y;
   i = document.getElementById('refimg');
   if (i == null)
      return false;
   document.body.style.cursor = 'crosshair';
   x -= i.offsetLeft;
   y -= i.offsetTop;
   while (i = i.offsetParent) {
      x -= i.offsetLeft;
      y -= i.offsetTop;
   }
   p += '   rel: ' + x + '/' + y;
   window.status = p;
   return true;
}

function XMLHttpRequestGeneric()
{
   var request;
   try {
      request = new XMLHttpRequest(); // Firefox, Opera 8.0+, Safari
   }
   catch (e) {
      try {
         request = new ActiveXObject('Msxml2.XMLHTTP'); // Internet Explorer
      }
      catch (e) {
         try {
            request = new ActiveXObject('Microsoft.XMLHTTP');
         }
         catch (e) {
           alert('Your browser does not support AJAX!');
           return undefined;
         }
      }
   }
  return request;
}

function ODBSet(path, value, pwdname)
{
   var value, request, url;

   if (pwdname != undefined)
      pwd = prompt('Please enter password', '');
   else
      pwd = '';

   request = XMLHttpRequestGeneric();

   url = '?cmd=jset&odb=' + path + '&value=' + value;

   if (pwdname != undefined)
      url += '&pnam=' + pwdname;

   request.open('GET', url, false);

   if (pwdname != undefined)
     request.setRequestHeader('Cookie', 'cpwd='+pwd);

   request.send(null);

   if (request.status != 200 || request.responseText != 'OK')
      alert('ODBSet error:\nPath: '+path+'\nHTTP Status: '+request.status+'\nMe
ssage: '+request.responseText+'\n'+document.location) ;
}

function ODBGet(path, format, defval, len, type)
{
   request = XMLHttpRequestGeneric();

   var url = '?cmd=jget&odb=' + path;
   if (format != undefined && format != '')
      url += '&format=' + format;
   request.open('GET', url, false);
   request.send(null);

   if (path.match(/[*]/)) {
      if (request.responseText == null)
         return null;
     if (request.responseText == '<DB_NO_KEY>') {
         url = '?cmd=jset&odb=' + path + '&value=' + defval + '&len=' + len + '
&type=' + type;

         request.open('GET', url, false);
         request.send(null);
         return defval;
      } else {
         var array = request.responseText.split('\n');
         return array;
      }
   } else {
      if ((request.responseText == '<DB_NO_KEY>' ||
           request.responseText == '<DB_OUT_OF_RANGE>') && defval != undefined)
 {
         url = '?cmd=jset&odb=' + path + '&value=' + defval + '&len=' + len + '
&type=' + type;

         request.open('GET', url, false);
         request.send(null);
         return defval;
      }
      return request.responseText;
   }
}

function ODBKey(path)
{
   request = XMLHttpRequestGeneric();

   var url = '?cmd=jkey&odb=' + path;
   request.open('GET', url, false);
   request.send(null);
   if (request.responseText == null)
      return null;
   var key = request.responseText.split('\n');
   this.name = key[0];
   this.type = key[1];
   this.num_values = key[2];
   this.item_size = key[3];
}

function ODBRpc_rev0(name, rpc, args)
{
   request = XMLHttpRequestGeneric();

   var url = '?cmd=jrpc_rev0&name=' + name + '&rpc=' + rpc;
   for (var i = 2; i < arguments.length; i++) {
     url += '&arg'+(i-2)+'='+arguments[i];
   };
   request.open('GET', url, false);
   request.send(null);
   if (request.responseText == null)
      return null;
   this.reply = request.responseText.split('\n');
}

function ODBGetMsg(n)
{
   request = XMLHttpRequestGeneric();

   var url = '?cmd=jmsg&n=' + n;
   request.open('GET', url, false);
   request.send(null);

   if (n > 1) {
     var array = request.responseText.split('\n');
      return array;
   } else
      return request.responseText;
}

function ODBEdit(path)
{
   var value = ODBGet(path);
   var new_value = prompt('Please enter new value', value);
   if (new_value != undefined) {
      ODBSet(path, new_value);
      window.location.reload();
   }
}

/* MIDAS type definitions */
var TID_BYTE = 1;
var TID_SBYTE = 2;
var TID_CHAR = 3;
var TID_WORD = 4;
var TID_SHORT = 5;
var TID_DWORD = 6;
var TID_INT = 7;
var TID_BOOL = 8;
var TID_FLOAT = 9;
var TID_DOUBLE = 10;
var TID_BITFIELD = 11;
var TID_STRING = 12;
var TID_ARRAY = 13;
var TID_STRUCT = 14;
var TID_KEY = 15;
var TID_LINK = 16;
\end{DoxyCode}
 \label{index_end}
\hypertarget{index_end}{}
  \par
 \subsubsection{Features available on custom pages}\label{RC_mhttpd_custom_features}
\par
 

\label{RC_mhttpd_custom_features_mhttpd_page_custom_create}
\hypertarget{RC_mhttpd_custom_features_mhttpd_page_custom_create}{}
 \hypertarget{RC_mhttpd_custom_features_RC_mhttpd_custom_create}{}\subparagraph{How to create a custom page}\label{RC_mhttpd_custom_features_RC_mhttpd_custom_create}
A custom page may be written in ordinary HTML code and/or JavaScript. It may include any of the following special features provided for use with MIDAS:

\label{RC_mhttpd_custom_features_idx_mhttpd_page_custom_features}
\hypertarget{RC_mhttpd_custom_features_idx_mhttpd_page_custom_features}{}



\begin{DoxyItemize}
\item \hyperlink{RC_mhttpd_custom_features_RC_mhttpd_custom_midas_buttons}{Access to the standard MIDAS Menu buttons}
\item \hyperlink{RC_mhttpd_custom_ODB_access}{Access to the ODB from a Custom page}
\item \hyperlink{RC_mhttpd_custom_ODB_access_features_RC_mhttpd_custom_pw_protection}{Password protection of ODB variables accessed from a custom page}
\item \hyperlink{RC_mhttpd_custom_features_RC_mhttpd_custom_script_buttons}{CustomScript Buttons}
\item \hyperlink{RC_mhttpd_custom_features_RC_mhttpd_custom_refresh}{Page refresh}
\item \hyperlink{RC_mhttpd_custom_features_RC_mhttpd_custom_alias}{Alias Buttons and Hyperlinks}
\item \hyperlink{RC_mhttpd_custom_features_RC_mhttpd_custom_getmsg}{Access to message log}
\item \hyperlink{RC_mhttpd_custom_ODB_access_features_RC_mhttpd_custom_checkboxes}{Including checkboxes on a custom page}
\item \hyperlink{RC_mhttpd_custom_ODB_access_features_RC_mhttpd_js_update_part}{Periodic update of parts of a custom page}
\item \hyperlink{RC_mhttpd_custom_RPC_access}{ODB RPC access}
\end{DoxyItemize}


\begin{DoxyItemize}
\item \hyperlink{RC_mhttpd_Image_access_RC_mhttpd_custom_image}{Image insertion into a Custom page}
\begin{DoxyItemize}
\item \hyperlink{RC_mhttpd_Image_access_RC_mhttpd_custom_history}{Inserting a history image in a custom page}
\item \hyperlink{RC_mhttpd_Image_access_RC_mhttpd_custom_imagemap}{Mapping active areas onto the image}
\begin{DoxyItemize}
\item \hyperlink{RC_mhttpd_Image_access_RC_mhttpd_custom_getmouse}{Display mouse position}
\end{DoxyItemize}
\item \hyperlink{RC_mhttpd_Image_access_RC_mhttpd_custom_Labels_Bars_Fills}{Superimposing Labels, Bars and Fills onto an image}
\item \hyperlink{RC_mhttpd_Image_access_RC_mhttpd_custom_edit_boxes}{Edit boxes floating on top of a graphic}
\end{DoxyItemize}
\end{DoxyItemize}

\par
 All these features will be described in the following sections. They apply to both {\bfseries external} and {\bfseries internal} web pages. \label{RC_mhttpd_custom_features_idx_mhttpd_page_custom_external}
\hypertarget{RC_mhttpd_custom_features_idx_mhttpd_page_custom_external}{}
 \label{RC_mhttpd_custom_features_mhttpd_custom_external}
\hypertarget{RC_mhttpd_custom_features_mhttpd_custom_external}{}
 Since external custom pages are more commonly used than internal, the examples in the following sections use {\bfseries external} file(s). Examples of {\bfseries internal} pages can be found \hyperlink{RC_mhttpd_Internal}{here}.

\par
 

\par
 \hypertarget{RC_mhttpd_custom_features_RC_mhttpd_custom_including_js}{}\subparagraph{Including JavaScript code in an HTML custom page}\label{RC_mhttpd_custom_features_RC_mhttpd_custom_including_js}
Javascript code can be inserted in the HTML file by enclosing it between the $<$script type=\char`\"{}text/javascript\char`\"{}$>$ ..... $<$/script$>$ HTML tags, for example: \par
 
\begin{DoxyCode}
<!-- include some js code -->
<script  type="text/javascript">
\code
var my_title="Gas Analyzer Custom Page"
var my_name="GasAnalyzer"
var my_num = 10;

document.write('<title>T2KGas: ' +my_title+ '(' +my_num+ ')</title>');
</script>
\end{DoxyCode}
 \par
 The $<$script ...$>$ HTML tag is also used to \hyperlink{RC_mhttpd_custom_js_lib_RC_mhttpd_include_js_library}{include} the \hyperlink{RC_mhttpd_custom_js_lib}{JavaScript built-\/in library mhttpd.js} in the custom page, to allow access to the ODB using the JavaScript (JS) \hyperlink{RC_mhttpd_custom_js_lib_RC_mhttpd_js_library_features}{built-\/in functions}. \par
Custom pages may be written entirely in JavaScript, if preferred. For example, a file of user-\/written JavaScript functions can be included into multiple HTML custom pages (provided links are created to these files in the ODB /Custom tree -\/ see \hyperlink{RC_mhttpd_Activate}{details here}).

In the following HTML code fragment, the \hyperlink{RC_mhttpd_custom_js_lib}{JavaScript built-\/in library mhttpd.js} is included, as well as two user-\/written Javascript files.


\begin{DoxyCode}
........
<html>
<head>
<!-- include mhttpd js library   -->
<script type="text/javascript"  src="/mhttpd.js">
</script>

<!-- js_constants! -> filename  names.js
List of device names and corresponding index into arrays
 -->
<script type="text/javascript"  src="js_constants!">
</script>

<!-- js_functions! -> filename  common_functions.js
 -->      
<script type="text/javascript"  src="js_functions!">
</script>
.......
\end{DoxyCode}




\hypertarget{RC_mhttpd_custom_features_RC_mhttpd_custom_key_access}{}\subparagraph{Access to the ODB from a Custom Page}\label{RC_mhttpd_custom_features_RC_mhttpd_custom_key_access}
To include the custom features such as Buttons, ODB editing etc. a custom page needs to have at least one form declared with the HTML  $<$form...$>$....$<$/form$>$  tags. Declarations for buttons, ODB editing etc. must be inserted between the enclosing HTML $<$form...$>$  tags, which are coded like this : 
\begin{DoxyCode}
<form method="GET" action="http://hostname.domain:port/CS/\<Custom_page_key\>">
.......
.......
</form>
\end{DoxyCode}
 where Custom\_\-page\_\-key is the name of the key that has been \hyperlink{RC_mhttpd_Activate_RC_odb_custom_keynames}{defined in the /Custom ODB directory} in order to \hyperlink{RC_mhttpd_Activate}{activate the custom page}.

\par
 For a {\bfseries remote} page defined using a key named Test\&, the HTML  $<$form...$>$  tag might be 
\begin{DoxyCode}
<form method="GET" action="http://hostname.domain:port/CS/Test&"\>
\end{DoxyCode}


For a {\bfseries local} page defined using a key named MyExpt\&, the $<$form...$>$ tag might be 
\begin{DoxyCode}
<form name="form1" method="Get" action="/CS/MyExpt&">
\end{DoxyCode}


\par






\label{RC_mhttpd_custom_features_idx_mhttpd_page_custom_refresh}
\hypertarget{RC_mhttpd_custom_features_idx_mhttpd_page_custom_refresh}{}
 \hypertarget{RC_mhttpd_custom_features_RC_mhttpd_custom_refresh}{}\subparagraph{Page refresh}\label{RC_mhttpd_custom_features_RC_mhttpd_custom_refresh}
The following $<$meta...$>$  tag included in the HTML header code will cause the whole custom page to refresh in 60 seconds :


\begin{DoxyCode}
<meta http-equiv="Refresh" content="60">
\end{DoxyCode}
 \par


It is also possible to \hyperlink{RC_mhttpd_custom_ODB_access_features_RC_mhttpd_js_update_part}{periodically update parts} of a custom page.



\hypertarget{RC_mhttpd_custom_features_RC_mhttpd_custom_midas_buttons}{}\subparagraph{Access to the standard MIDAS Menu buttons}\label{RC_mhttpd_custom_features_RC_mhttpd_custom_midas_buttons}
Access to the standard MIDAS Menu buttons can be provided with HTML $<$input...$>$ tags of the form:
\begin{DoxyItemize}
\item  $<$input name=\char`\"{}cmd\char`\"{} value=$<${\itshape button-\/name\/} $>$ type=\char`\"{}submit\char`\"{} $>$ 
\end{DoxyItemize}

Valid values are the standard MIDAS \hyperlink{RC_mhttpd_status_page_features_RC_mhttpd_status_menu_buttons}{Menu buttons} (Start, Pause, Resume, Stop, ODB, Elog, Alarms, History, Programs, etc). The $<$input...$>$ tags must be declared within enclosing HTML $<$form...$>$....$<$/form$>$  tags (see \hyperlink{RC_mhttpd_custom_features_RC_mhttpd_custom_key_access}{above}).

The following html fragment shows the inclusion of three of the standard buttons, giving access to the Main Status, ODB and Messages pages : 
\begin{DoxyCode}
<form name="form1" method="Get" action="/CS/MyExpt&">
<input name="cmd" value="Status" type="submit">
<input name="cmd" value="ODB" type="submit">
<input name="cmd" value="Messages" type="submit">
...
</form>
\end{DoxyCode}




\hypertarget{RC_mhttpd_custom_features_RC_mhttpd_custom_alias}{}\subparagraph{Alias Buttons and Hyperlinks}\label{RC_mhttpd_custom_features_RC_mhttpd_custom_alias}
Any hyperlink can easily be included on a custom page by using the standard HTML anchor $<$a...$>$ tag, e.g. 
\begin{DoxyCode}
<a href="http://ladd00.triumf.ca/~daqweb/doc/midas/html/">Midas Help</a>
\end{DoxyCode}


Links on a custom page equivalent to \hyperlink{RC_mhttpd_status_page_features_RC_mhttpd_status_Alias_buttons}{alias-\/buttons} can also be made e.g.


\begin{DoxyCode}
<button type="button" onclick="document.location.href='/Alias/alias&';">alias</bu
      tton>
\end{DoxyCode}
\hypertarget{RC_mhttpd_custom_features_RC_mhttpd_custom_simple_example}{}\subparagraph{Simple Example of a custom page in HTML}\label{RC_mhttpd_custom_features_RC_mhttpd_custom_simple_example}
Here is a simple example of an HTML custom page demonstrating the features described above, including access to
\begin{DoxyItemize}
\item the standard MIDAS Menu buttons
\begin{DoxyItemize}
\item \char`\"{}Status\char`\"{} and \char`\"{}ODB\char`\"{}
\end{DoxyItemize}
\item alias buttons
\begin{DoxyItemize}
\item \char`\"{}TPC Button\char`\"{} a local link through the /alias ODB tree
\item \char`\"{}triumf\char`\"{} an external link
\end{DoxyItemize}
\item alias links
\begin{DoxyItemize}
\item \char`\"{}TpcGasPlc\char`\"{} local link to /Equipment/TpcGasPlc
\item \char`\"{}TPC Alias-\/link\char`\"{} local link through the /alias ODB tree (to same area as \char`\"{}TPC Button\char`\"{})
\end{DoxyItemize}
\end{DoxyItemize}

\par
 \begin{center} A simple custom page  \end{center}  \par


The code for this page is shown below: 
\begin{DoxyCode}
 <!DOCTYPE html PUBLIC "-//w3c//dtd html 4.0 transitional//en">
<html>
<head>
<title>simple custom page</title>
</head>
<body>
<form name="form1" method="Get" action="/CS/Colour&">
<table style="text-align: center; width: 40%;" border="1" cellpadding="2" cellspa
      cing="2">
<tr>
<td colspan="2"
style="vertical-align: top; background-color: mediumslateblue; color: white; text
      -align: center;">
Custom Page for experiment <odb src="/Experiment/Name">
</td>
</tr>
<td
style="vertical-align: top; background-color: lightyellow;  text-align: center;">
      
Run Control Buttons:</td>
<td
style="vertical-align: top; background-color: lightyellow;  text-align: center;">
      
<input name="cmd" value="Status" type="submit">
<input name="cmd" value="ODB" type="submit">
</td>
</tr>
<tr>
<td
style="vertical-align: top; background-color: seagreen; color: white; text-align:
       center;">Alias Buttons:</td>
<td
style="vertical-align: top; background-color: seagreen;  text-align: center;">
<button type="button" onclick="document.location.href='/Alias/TPC Common&';">TPC 
      Button</button>
<button type="button" onclick="window.open('http://triumf.ca');">triumf</button>
</td>
</tr>
<tr>
<td
style="vertical-align: top; background-color: fuschia;  text-align: center;">Link
      s:</td>
<td
style="vertical-align: top; background-color: fuschia;  text-align: center;">
<a href="/SC/TpcGasPlc">TpcGasPlc</a>
<a href="/Alias/TPC Common&">TPC Alias-link</a>
</td>
</tr>
</table>
</form>
</body>
\end{DoxyCode}


To create this page, you need an existing MIDAS experiment. To have all the buttons/links work, you need the ODB keys
\begin{DoxyItemize}
\item /Equipment/$<$eqp-\/name$>$ ({\itshape $<$eqp-\/name$>$\/} is \char`\"{}TpcGasPlc\char`\"{} in the example)
\item /Alias/$<$alias-\/name$>$ ({\itshape $<$alias-\/name$>$\/} is \char`\"{}TPC Common\&\char`\"{} in the example)
\end{DoxyItemize}


\begin{DoxyItemize}
\item create a file {\itshape custom.html\/} containing the above code. In this example, the code is in path {\itshape /home/mydir/custom.html\/}
\item in {\itshape custom.html\/}
\begin{DoxyItemize}
\item change {\itshape TPC Common\&\/} to your {\itshape  $<$alias-\/name$>$ \/} if different
\item change {\itshape TpcGasPlc\/} to your {\itshape $<$eqp-\/name$>$ \/} if different
\end{DoxyItemize}
\item in the ODB for your experiment, create a key /custom/test as shown below: 
\begin{DoxyCode}
     $ odb
     [local:customgas:S]/>cd /custom
     [local:customgas:S]/Custom>create string test
     String length [32]:
     [local:customgas:S]/Custom>set test "/home/mydir/custom.html"
\end{DoxyCode}

\item on the mhttpd status page for the experiment, there should now be a custom-\/button labelled \char`\"{}custom\char`\"{}
\item click on this \char`\"{}custom\char`\"{} button to see the custom page
\end{DoxyItemize}

\par


\par
 \label{RC_mhttpd_custom_features_idx_mhttpd_buttons_customscript}
\hypertarget{RC_mhttpd_custom_features_idx_mhttpd_buttons_customscript}{}
 \label{RC_mhttpd_custom_features_idx_mhttpd_custom_script-button}
\hypertarget{RC_mhttpd_custom_features_idx_mhttpd_custom_script-button}{}
 \hypertarget{RC_mhttpd_custom_features_RC_mhttpd_custom_script_buttons}{}\subparagraph{CustomScript Buttons}\label{RC_mhttpd_custom_features_RC_mhttpd_custom_script_buttons}
CustomScript buttons can be provided on custom webpages. These buttons are equivalent to \hyperlink{RC_mhttpd_status_page_features_RC_mhttpd_status_script_buttons}{Optional Script buttons} on the MIDAS Main Status page, and allow a particular action to be performed when the button is pressed.

If the user defines a new tree in ODB named /CustomScript , then any key created in the /CustomScript tree will appear as a script-\/button of that name on a custom page that includes an HTML $<$input...$>$ tag of the form: 
\begin{DoxyCode}
        <input type=submit name=customscript value="my button">
\end{DoxyCode}
 where the action of the button {\itshape  \char`\"{}my button\char`\"{}\/} will be found in the /customscript/my button subdirectory.

\label{RC_mhttpd_custom_features_idx_ODB_tree_customscript}
\hypertarget{RC_mhttpd_custom_features_idx_ODB_tree_customscript}{}
 \hypertarget{RC_mhttpd_custom_features_RC_odb_customscript_tree}{}\subparagraph{ODB /CustomScript tree}\label{RC_mhttpd_custom_features_RC_odb_customscript_tree}
\begin{DoxyNote}{Note}
The optional /CustomScript tree is applicable only to \hyperlink{RC_mhttpd}{mhttpd}, and ignored by \hyperlink{RC_odbedit}{odbedit}.
\end{DoxyNote}
The syntax of CustomScript buttons in the  /customscript  directory is identical to that of the \hyperlink{RC_mhttpd_status_page_features_RC_mhttpd_status_script_buttons}{Optional Script buttons} under the /Script ODB directory, i.e.


\begin{DoxyItemize}
\item each sub-\/directory ( /CustomScript/$<$button name$>$/) should contain at least one string key which is the custom script command to be executed.
\item Further keys will be passed as {\bfseries  arguments } to the script.
\item MIDAS symbolic links are permitted.
\end{DoxyItemize}

\par
 \hypertarget{RC_mhttpd_custom_features_RC_mhttpd_customscript_example}{}\subparagraph{Example of CustomScript buttons and corresponding /CustomScript tree}\label{RC_mhttpd_custom_features_RC_mhttpd_customscript_example}
The following JavaScript fragment shows customscript buttons that appear only when the run is stopped. The button labelled \char`\"{}tri\_\-config\char`\"{} and, depending on the current experimental mode (given by variable {\itshape  \char`\"{}ppg\_\-mode\char`\"{} \/}), three of the mode buttons labelled \char`\"{}1a\char`\"{},\char`\"{}1b\char`\"{},\char`\"{}1c\char`\"{} or \char`\"{}1d\char`\"{} . 
\begin{DoxyCode}
 if (rstate == state_stopped) // run stopped
{
 document.write('<input name="customscript" value="tri_config" type="submit">');
if(ppg_mode != '1a')
  document.write('<input name="customscript" value="1a" type="submit">');
if(ppg_mode != '1b')
  document.write('<input name="customscript" value="1b" type="submit">');
if(ppg_mode != '1c')
 document.write('<input name="customscript" value="1c" type="submit">');
if(ppg_mode != '1d')
 document.write('<input name="customscript" value="1d" type="submit">');
}
\end{DoxyCode}


The corresponding entry under /customscript is as shown below. The first button (tri\_\-config) when pressed will cause a user-\/written program {\itshape  tri\_\-config \/} to be executed. The other buttons \char`\"{}1a\char`\"{},\char`\"{}1b\char`\"{} etc. when pressed will cause the current experimental mode to be changed to that mode. Only the entry for \char`\"{}1a\char`\"{} is shown. 
\begin{DoxyCode}
[local:ebit:S]/>ls /customscript/tri_config
cmd                             /home/ebit/online/ppg/perl/exec.pl
include path                    /home/ebit/online/ppg/perl/
experiment name -> /experiment/Name
                                ebit
execute                         '/home/ebit/online/ppg/tri_config -s'
beamline                        ebit

[local:ebit:S]/>ls /customscript/1a
cmd                             /home/ebit/online/ppg/perl/change_mode.pl
include path                    /home/ebit/online/ppg/perl/
experiment name -> /experiment/Name
                                ebit
ppg_mode                        1a
modefile                        defaults
\end{DoxyCode}


These buttons are illustrated in the example \hyperlink{RC_Ebit_custom_page}{Custom Pages for Ebit Experiment}

\par


\par


\label{RC_mhttpd_custom_features_idx_ODBGetMsg-JavaScript-function}
\hypertarget{RC_mhttpd_custom_features_idx_ODBGetMsg-JavaScript-function}{}
 \hypertarget{RC_mhttpd_custom_features_RC_mhttpd_custom_getmsg}{}\subparagraph{Access to message log}\label{RC_mhttpd_custom_features_RC_mhttpd_custom_getmsg}
The message log can be accessed from a custom page using a call to the \hyperlink{RC_mhttpd_custom_js_lib}{JavaScript library function}  ODBGetMsg  (provided the JS library is \hyperlink{RC_mhttpd_custom_js_lib_RC_mhttpd_include_js_library}{included}). There is no HTML equivalent to this JS function. \par
 \begin{table}[h]\begin{TabularC}{3}
\hline
JavaScript Function  &Purpose  &Parameters  

\\\cline{1-3}

\begin{DoxyCode}
 ODBGetMsg(n)
\end{DoxyCode}
  &Get the most recent {\bfseries n} lines from the system message log  &{\bfseries n} number of lines required   \\\cline{1-3}
\end{TabularC}
\centering
\caption{Above: ODB Message access from JavaScript }
\end{table}
\par
 This allows the inclusion of the \char`\"{}Last Midas message\char`\"{} on a custom page. e.g. 
\begin{DoxyCode}
<script>
var message;
message= ODBGetMsg(1);
document.write('Last message:'+message+'<br>');
</script>
\end{DoxyCode}


\par
 \begin{center} A simple custom page with a call to ODBGetMsg()  \end{center}  \par


The Javascript call has been added to the \hyperlink{RC_mhttpd_custom_features_RC_mhttpd_custom_simple_example}{Simple Example of a custom page in HTML} above, by
\begin{DoxyItemize}
\item adding access to the standard MIDAS Javascript library to the header
\item adding the call to ODBGetMsg()
\end{DoxyItemize}

as detailed below: 
\begin{DoxyCode}
<!DOCTYPE html PUBLIC "-//w3c//dtd html 4.0 transitional//en">
<html>
<head>
<title> table test</title>
<script type="text/javascript" src="../mhttpd.js"></script>
</head>
............
............
</table>

<script>
var message;
message= ODBGetMsg(1);
document.write('<h2>Last message:</h2>'+message+'<br>');
</script>
</form>
</body>
\end{DoxyCode}


\par


\par



\begin{DoxyItemize}
\item \hyperlink{RC_mhttpd_custom_ODB_access}{Access to the ODB from a Custom page}
\item \hyperlink{RC_mhttpd_custom_RPC_access}{ODB RPC access}
\item \hyperlink{RC_mhttpd_Image_access}{Inserting an Image into a Custom page}
\item \hyperlink{RC_mhttpd_custom_status}{Custom Status page}
\end{DoxyItemize}

\par
 \par
 \label{index_end}
\hypertarget{index_end}{}
  \paragraph{Access to the ODB from a Custom page}\label{RC_mhttpd_custom_ODB_access}
\par




\par


Access to the ODB is available \hyperlink{RC_mhttpd_custom_ODB_access_RC_mhttpd_custom_odb_html}{using HTML tags} and using \hyperlink{RC_mhttpd_custom_ODB_access_RC_mhttpd_custom_odb_js}{JavaScript functions} with the \hyperlink{RC_mhttpd_custom_js_lib}{JavaScript built-\/in library mhttpd.js} . Both methods are described in the following sections:


\begin{DoxyItemize}
\item \hyperlink{RC_mhttpd_custom_ODB_access_RC_mhttpd_custom_odb_html}{ODB access using HTML tags}
\item \hyperlink{RC_mhttpd_custom_ODB_access_RC_mhttpd_custom_odb_js}{ODB Access using mhttpd JavaScript built-\/in functions}
\item \hyperlink{RC_mhttpd_custom_ODB_access_examples}{Examples of accessing ODB from a Custom page}
\item \hyperlink{RC_mhttpd_custom_ODB_access_features}{Features using ODB access from a Custom page}
\end{DoxyItemize}

\label{RC_mhttpd_custom_ODB_access_idx_odb-HTML-tag}
\hypertarget{RC_mhttpd_custom_ODB_access_idx_odb-HTML-tag}{}
 \hypertarget{RC_mhttpd_custom_ODB_access_RC_mhttpd_custom_odb_html}{}\subparagraph{ODB access using HTML tags}\label{RC_mhttpd_custom_ODB_access_RC_mhttpd_custom_odb_html}
The $<$odb...$>$ tag has been defined for read/write access to the ODB under HTML. Also shown in the table below is the equivalent JavaScript function.

that the $<$odb...$>$ tags and JavaScript equivalent must be declared within enclosing HTML $<$form...$>$....$<$/form$>$  tags (see \hyperlink{RC_mhttpd_custom_features_RC_mhttpd_custom_key_access}{above}).

\begin{table}[h]\begin{TabularC}{3}
\hline
HTML ODB tag  &Meaning  &Equivalent JS function  

\\\cline{1-3}
 $<$odb src=\char`\"{}odb path\char`\"{}$>$   &Display ODB field (read only)  & ODBGet  

\\\cline{1-3}
\label{RC_mhttpd_custom_ODB_access_odb_edit_tag}
\hypertarget{RC_mhttpd_custom_ODB_access_odb_edit_tag}{}
  $<$odb src=\char`\"{}odb path\char`\"{} edit=1 pwd=\char`\"{}CustomPwd\char`\"{}$>$   &Display an Editable ODB field (inline style). Optional \hyperlink{RC_mhttpd_custom_ODB_access_features_RC_mhttpd_custom_pw_protection}{password protection} with {\bfseries pwd} .  &\par
 

\\\cline{1-3}
 $<$odb src=\char`\"{}odb path\char`\"{} edit=2 pwd=\char`\"{}CustomPwd\char`\"{} $>$   &Display an Editable ODB field (popup style). Optional \hyperlink{RC_mhttpd_custom_ODB_access_features_RC_mhttpd_custom_pw_protection}{password protection} with {\bfseries pwd} .  & ODBEdit   \\\cline{1-3}
\end{TabularC}
\centering
\caption{Above: Access to ODB from HTML }
\end{table}


{\bfseries Usage:} 
\begin{DoxyCode}
Experiment Name: <odb src="/Experiment/Name">
Run Number: <odb src="/runinfo/run number" edit=1>
\end{DoxyCode}


\label{RC_mhttpd_custom_ODB_access_odb_tag_ex1}
\hypertarget{RC_mhttpd_custom_ODB_access_odb_tag_ex1}{}
 {\bfseries Examples} 
\begin{DoxyEnumerate}
\item The following image shows the status of the ODB key /logger/write data:\par
 \begin{center} ODB access using $<$odb..$>$ tag \par
  \end{center} 

The HTML code fragment producing the image above is shown below:


\begin{DoxyCode}
<table style="text-align: center; width: 40%;" border="1" cellpadding="2"
cellspacing="2">
<tr><td style="vertical-align: top; background-color: lightyellow; text-align: ce
      nter;">
Logging data</td>
<td><odb src="/logger/write data">
</td></tr</table>
\end{DoxyCode}



\item \hyperlink{RC_mhttpd_custom_ODB_access_examples_RC_mhttpd_js_example1}{Example of ODB access with HTML and JavaScript equivalent} 
\end{DoxyEnumerate}\par


\par
\hypertarget{RC_mhttpd_custom_ODB_access_RC_mhttpd_custom_odb_js}{}\subparagraph{ODB Access using mhttpd JavaScript built-\/in functions}\label{RC_mhttpd_custom_ODB_access_RC_mhttpd_custom_odb_js}
The following \hyperlink{RC_mhttpd_custom_js_lib}{mhttpd JS built-\/in functions} are defined for ODB access:
\begin{DoxyItemize}
\item \hyperlink{RC_mhttpd_custom_ODB_access_RC_mhttpd_custom_odbget}{ODBGet}
\item \hyperlink{RC_mhttpd_custom_ODB_access_RC_mhttpd_custom_odbedit}{ODBEdit}
\item \hyperlink{RC_mhttpd_custom_ODB_access_RC_mhttpd_custom_odbset}{ODBSet}
\item \hyperlink{RC_mhttpd_custom_ODB_access_RC_mhttpd_custom_odbkey}{ODBKey}
\end{DoxyItemize}

{\bfseries Examples:} 
\begin{DoxyEnumerate}
\item As in the HTML example \hyperlink{RC_mhttpd_custom_ODB_access_odb_tag_ex1}{above}, the status of the ODB key /logger/write data is displayed in the following image, but in this case the background colour is changed (using Javascript) depending on the value of the key:

\begin{center} ODB access using ODBGet showing colour change depending on state of ODB variable  \par
  \end{center}  \par
 The code fragment for the above images is shown below: 
\begin{DoxyCode}
<script>
var wd= ODBGet('/logger/write data')
alert ('wd = '+wd);
</script>
<table style="text-align: center; width: 40%;" border="1" cellpadding="2"
cellspacing="2">
<tr>
<td style="vertical-align: top; background-color:  lightyellow; text-align: cente
      r;">Logging data</td>
<script>
if (wd == "y")
   document.write('<td style="vertical-align: top; background-color: lime; text-a
      lign: center;">'+wd);
else
   document.write('<td style="vertical-align: top; background-color: red; text-al
      ign: center;">'+wd);
</script>
</td></tr></table>
\end{DoxyCode}



\item \hyperlink{RC_mhttpd_custom_ODB_access_examples_RC_mhttpd_js_example1}{Example of ODB access with HTML and JavaScript equivalent} 
\item \hyperlink{RC_mhttpd_custom_ODB_access_examples_RC_mhttpd_js_example2}{Example of ODB access with JavaScript functions ODBSet and ODBKey} 
\end{DoxyEnumerate}

\begin{DoxyNote}{Note}
The built-\/in library must be \hyperlink{RC_mhttpd_custom_js_lib_RC_mhttpd_include_js_library}{included} in your custom page when using any of the JS built-\/in functions.
\end{DoxyNote}
\par


\par


\label{RC_mhttpd_custom_ODB_access_idx_ODBGet-JavaScript-function}
\hypertarget{RC_mhttpd_custom_ODB_access_idx_ODBGet-JavaScript-function}{}
 \subparagraph{Examples of accessing ODB from a Custom page}\label{RC_mhttpd_custom_ODB_access_examples}
\par




\par
 \hypertarget{RC_mhttpd_custom_ODB_access_examples_RC_mhttpd_js_example1}{}\subparagraph{Example of ODB access with HTML and JavaScript equivalent}\label{RC_mhttpd_custom_ODB_access_examples_RC_mhttpd_js_example1}
The following simple HTML code shows ODB access using JavaScript (ODBGet, ODBEdit) and using the HTML  $<$odb$>$ tag . The output produced by this code is shown below. 
\begin{DoxyCode}
<!DOCTYPE HTML PUBLIC "-//W3C//DTD HTML 4.0 TRANSITIONAL//EN">
<html><head>
<title> ODBEdit test</title>
<!-- include the mhttpd JS library -->
\htmlonly <script src="/js/mhttpd.js" type="text/javascript"></script> \endhtmlon
      ly

\htmlonly <script type="text/javascript">
var my_action = '"/CS/try&"'
var rn
var path
var my_expt="midas";

document.write('</head><body>')
document.write('<form method="get" name="form2" action='+my_action+'> ')
document.write('<input name="exp" value="'+my_expt+'" type="hidden">');

document.write('Using Javascript and ODBEdit:<br>')
path='/runinfo/run number'
rn = ODBGet(path,"Run Number with format: %d")
document.writeln('Run Number: '+rn+'<br>')
document.writeln('Edit Run Number:')
document.writeln('<a href="#" onclick="ODBEdit(path)" >')
document.writeln(rn)
document.writeln('</a>');
</script> \endhtmlonly
<br>
Using HTML :
<br>
Using edit=2 ...  Run Number:
<odb src="/runinfo/run number" edit=2>
<br>
Using edit=1 ...  Run Number:
<odb src="/runinfo/run number" edit=1>
<br>
</form>
</html>
\end{DoxyCode}
 \par


This code produces the output shown in Figures 1 and 2 below. In Figure 1, a value has been entered using the hyperlink created by the {\bfseries Javascript} function ODBEdit. A {\bfseries popup} box appears in which the user may enter a new value.

\par
\par
\par
 \begin{center} Figure 1: ODB tags under html and javascript -\/ entering an ODB value using Javascript \par
\par
\par
  \end{center}  \par
\par
\par


Figure 2 shows entering a value using the {\bfseries HTML} tags. The two different styles are shown.
\begin{DoxyItemize}
\item {\bfseries edit=2} type produces a pop-\/up box as in the Javascript version
\item {\bfseries edit=1} type produces an in-\/line input box
\end{DoxyItemize}

\par
\par
\par
 \begin{center} Figure 2: ODB tags under html and javascript -\/ entering an ODB value using HTML \par
\par
\par
  \par
\par
\par
 \end{center} \hypertarget{RC_mhttpd_custom_ODB_access_examples_RC_mhttpd_js_example2}{}\subparagraph{Example of ODB access with JavaScript functions ODBSet and ODBKey}\label{RC_mhttpd_custom_ODB_access_examples_RC_mhttpd_js_example2}
The following HTML code shows an example using the JavaScript functions ODBSet and ODBKey. There is no equivalent to these functions available in HTML. The output from this example is shown in Figure 3.


\begin{DoxyCode}
<!DOCTYPE HTML PUBLIC "-//W3C//DTD HTML 4.0 TRANSITIONAL//EN">
<html><head>
<title> ODBEdit test</title>
\htmlonly <script src="/js/mhttpd.js" type="text/javascript"></script> \endhtmlon
      ly

\htmlonly <script type="text/javascript">
var my_action = '"/CS/try&"'
var rn,ival,irn
var path="/runinfo/run number";
var my_expt="midas";
var message;

function test(path,value)
{
var pattern=/DB_NO_KEY/;
var ival,key

document.write('Function test starting with path: '+path+' value: '+value+'<br>')
      ;
document.write('ODBGet with a format parameter:  <br>');
ival = ODBGet(path,"read:%4.4d");
document.write(ival+'<br>');

document.write('<br>Now using ODBSet to set a value: <br>');
document.write('Setting '+path+' to '+value+' with ODBSet<br>') ;
ODBSet(path,value);
ival = ODBGet(path)
document.writeln('Value: '+ival+'<br>')

document.write('<br>Now using ODBKey to get a key using path: '+path+' <br>');
key = ODBKey(path);
document.write('<br>Testing response for the pattern: '+pattern+'...');
 if ( pattern.test(key))
      document.write('test is TRUE <br>');
 else
      document.write('test is FALSE<br>');
document.write('key array : '+key+'<br>');
document.write('done<br>');
return;
}


document.write('</head><body>')
document.write('<form method="get" name="form2" action='+my_action+'> ')
document.write('<input name="exp" value="'+my_expt+'" type="hidden">');

irn=ODBGet(path); // remember initial run number
ODBSet(path,70); // initialize the run number to 70

document.write('Example showing use of ODBGet, ODBSet, ODBKey, ODBGetMsg <br>');
document.write('First with a good path...<br>');
document.write('<span style= "color: green;">')
test("/runinfo/run number", 76);
document.write('</span>')
document.write('<br>Then with bad path to show the difference....<br>');
document.write('<span style= "color: red;">')
test("/nopath/nokey", 79);
document.write('</span>')
message= ODBGetMsg(1);
document.write('Last message:'+message+'<br>');

ODBSet(path,irn); // rewrite initial run number
</script> \endhtmlonly
</form>
</html>
\end{DoxyCode}


\par
\par
\par
 \begin{center} Figure 3 Output from above example code showing ODB access with JS built-\/in functions \par
\par
\par
  \par
\par
\par
 \end{center} \hypertarget{RC_mhttpd_custom_ODB_access_examples_RC_mhttpd_js_example3}{}\subparagraph{Example of ODB access with arrays}\label{RC_mhttpd_custom_ODB_access_examples_RC_mhttpd_js_example3}
Accessing ODB values can slow the page update considerably where there are many values to access. The access time can be cut considerably by having most of the input and output data in arrays.

 Note that writing arrays with ODBSet has been supported since \hyperlink{NDF_ndf_may_2010}{May 2010} . 

In the following example, the raw data is provided in two large arrays. Some of this data is used in logical calculations (done in JavaScript) to determine the state of various devices, and the result is output into an array in the ODB in order to colour various items with the use of \char`\"{}fills\char`\"{} on the image pages. \par
 In this example, the arrays PLCR,PLCA in the odb are read into arrays in JavaScript in the function get\_\-PLC\_\-arrays in the file custom\_\-functions.js. Calculated data stored as an array in the odb are read into an array CAL. 
\begin{DoxyCode}
// custom_fuctions.js
// globals
var equipment_path='/Equipment/TpcGasPlc/';
var gascalc_array = equipment_path + 'GasCalc/Variables/Calculated[*]';
var variables_path = equipment_path + 'Variables/';
var plcr_path = variables_path + 'PLCR'; // indices of these PLC arrays are in na
      mes.js
var plca_path = variables_path + 'PLCA';

var PLCR=[];
var PLCA=[];
var CAL=[];

function get_PLC_arrays()
{  // get the arrays in one go
   // returns 0=success or 1=failure
 
  var pattern1=/DB_NO_KEY/;
  var pattern2=/undefined/;

  var i,idx;
    
  PLCR =     ODBGet(plcr_path+ '[*]');
  if ( pattern1.test(PLCR) ||  pattern2.test(PLCR)  )
  {
      alert ('get_PLCR_array: ERROR '+PLCR+' from ODBGet('+plcr_path+'[*])' );
      return 1;
  } 
  
   PLCA = ODBGet(plca_path+ '[*]', "%9.5f"); // the required values are float
   if ( pattern1.test(PLCA) ||  pattern2.test(PLCA)  )
   {
      alert ('get_PLCA_array: ERROR '+PLCA+' from ODBGet('+plca_path+'[*])' );
      return 1;
   }
              
// get Calculated array
   CAL = ODBGet(gascalc_array, "%d"); // the required values are INT
   if ( pattern1.test(CAL) ||  pattern2.test(CAL)  )
   {
      alert ('get_CAL_array: ERROR '+CAL+' from ODBGet('+gascalc_array+')' );
      return 1;
   }

   return 0; // success
}

..........
\end{DoxyCode}


For each of the gas pages, various items are calculated and the CAL array is updated for each item. At the end of all calculations, the CAL array is written back into the ODB.


\begin{DoxyCode}
<!-- GasPage.html -->
.......

<!-- js_functions!   custom_functions.js defined by  ODB key  /custom/js_function
      s!  -->
\htmlonly <script type="text/javascript"  src="js_functions!">
</script> \endhtmlonly
</head><body>


\htmlonly <script>
//Read all the arrays from the ODB
var plc_error = get_PLC_arrays();
.....
calculate_device(G2VA1_STAT,G2VA1,plc_error); // saves result to CAL array
......
calculate_logical(17,PU_Box,plc_error); // saves result to CAL array
......
ODBSet(gascalc_array, CAL); // write CAL array into ODB after all calculations
</script> \endhtmlonly
</body>
</html>
\end{DoxyCode}




\par
 \label{index_end}
\hypertarget{index_end}{}
 \subparagraph{Features using ODB access from a Custom page}\label{RC_mhttpd_custom_ODB_access_features}
\par




\par
 This page describes several features with ODB access on a custom page.


\begin{DoxyItemize}
\item \hyperlink{RC_mhttpd_custom_ODB_access_features_RC_mhttpd_custom_checkboxes}{Including checkboxes on a custom page}
\item \hyperlink{RC_mhttpd_custom_ODB_access_features_RC_mhttpd_js_update_part}{Periodic update of parts of a custom page}
\item \hyperlink{RC_mhttpd_custom_ODB_access_features_RC_mhttpd_custom_pw_protection}{Password protection of ODB variables accessed from a custom page}
\end{DoxyItemize}\hypertarget{RC_mhttpd_custom_ODB_access_features_RC_mhttpd_custom_checkboxes}{}\subparagraph{Including checkboxes on a custom page}\label{RC_mhttpd_custom_ODB_access_features_RC_mhttpd_custom_checkboxes}
The function ODBSet can be used when one clicks on an {\bfseries checkbox} for example: 
\begin{DoxyCode}
  <input type="checkbox" onClick="ODBSet('/Logger/Write data',this.checked?'1':'0
      ')">
\end{DoxyCode}


If used as above, the state of the checkbox must be initialized when the page is loaded. This can be done with some JavaScript code called on initialization, which then uses \hyperlink{RC_mhttpd_custom_ODB_access_RC_mhttpd_custom_odbset}{ODBSet JavaScript function} as described above.

Alternatively, the checkbox can be created using an  $<$odb...$>$  \hyperlink{RC_mhttpd_custom_ODB_access_RC_mhttpd_custom_odb_html}{tag} as follows: 
\begin{DoxyCode}
  <odb src="/Logger/Write data" type="checkbox" edit="2" onclick="ODBSet('/Logger
      /Write data',this.checked?'1':'0')">
\end{DoxyCode}


The special code {\bfseries edit=\char`\"{}2\char`\"{}} instructs mhttpd not to put any JavaScript code into the checkbox tag, since setting this value in the ODB is now handled by the user-\/supplied ODBSet() code.\hypertarget{RC_mhttpd_custom_ODB_access_features_RC_mhttpd_js_example_3}{}\subparagraph{Example of Checkboxes using JavaScript and HTML}\label{RC_mhttpd_custom_ODB_access_features_RC_mhttpd_js_example_3}

\begin{DoxyCode}
<!DOCTYPE HTML PUBLIC "-//W3C//DTD HTML 4.0 TRANSITIONAL//EN">
<html><head>
<title> ODBEdit test</title>
<!-- include the mhttpd JS library -->
\htmlonly <script src="/js/mhttpd.js" type="text/javascript"></script> \endhtmlon
      ly

\htmlonly <script type="text/javascript">

var my_action = '"/CS/try&"'
var ival;
var my_expt="midas";
</script> \endhtmlonly
</head><body>
<form method="get" name="form2" action='+my_action+'>
<input name="exp" value="'+my_expt+'" type="hidden">
Write data: <odb src="/Logger/Write data"><br>
JS Checkbox ... Write Data:
<input  name="mybox"  type="checkbox"   onClick="ODBSet('/Logger/Write data',this
      .checked?'1':'0')">
\htmlonly <script>
if( ODBGet('/Logger/Write data') =='y')
  ival=1;
else
  ival=0;
document.write('<br>ival='+ival+'<br>');
document.form2.mybox.checked=ival  // initialize to the correct value
</script> \endhtmlonly

<br>HTML checkbox... Write Data:
  <odb src="/Logger/Write data" type="checkbox" edit="2" onclick="ODBSet('/Logger
      /Write data',this.checked?'1':'0')">
<br>
</form>
</html>
\end{DoxyCode}


\par
\par
\par
 \begin{center} Figure 4 Output from above code: checkboxes \par
\par
\par
  \par
\par
\par
 \end{center} 

\par


\par


\label{RC_mhttpd_custom_ODB_access_features_idx_mhttpd_page_custom_refresh_partial}
\hypertarget{RC_mhttpd_custom_ODB_access_features_idx_mhttpd_page_custom_refresh_partial}{}
 \hypertarget{RC_mhttpd_custom_ODB_access_features_RC_mhttpd_js_update_part}{}\subparagraph{Periodic update of parts of a custom page}\label{RC_mhttpd_custom_ODB_access_features_RC_mhttpd_js_update_part}
The functionality of ODBGet together with the
\begin{DoxyItemize}
\item {\bfseries window.setInterval()} function
\end{DoxyItemize}

can be used to update parts of the web page periodically. \par
 For example the Javascript fragment below contains a function which updates the current run number every 10 seconds in the background : 
\begin{DoxyCode}
  window.setInterval("Refresh()", 10000);

  function Refresh() {
    document.getElementById("run_number").innerHTML = ODBGet('/Runinfo/Run number
      ');
  }
\end{DoxyCode}


The custom page has to contain an element with id=\char`\"{}run\_\-number\char`\"{}, such as 
\begin{DoxyCode}
  <td id="run_number"></td>
\end{DoxyCode}
 \par
\par
\hypertarget{RC_mhttpd_custom_ODB_access_features_RC_mhttpd_custom_pw_protection}{}\subparagraph{Password protection of ODB variables accessed from a custom page}\label{RC_mhttpd_custom_ODB_access_features_RC_mhttpd_custom_pw_protection}
Being able to control an experiment through a web interface of course raises the question of safety. This is not so much about external access (for which there are other protection schemes like host lists etc.) but it's about accidental access by the normal shift crew. If a single click on a web page opens a critical valve, this might be a problem. In order to restrict access to some \char`\"{}experts\char`\"{}, an additional password can be chosen for all or some controls on a custom page.

Password protection is optional, and must be set up by the user. The {\itshape password\/} must be defined as an ODB entry of the form  /Custom/Pwd/$<$password$>$ . If the password is {\itshape CustomPwd\/}, the ODB key /Custom/Pwd/CustomPwd  would be defined.

By using an explicit name, one can use a single password for all controls on a page, or one could use several passwords on the same page. For example, a shift crew password for the less severe controls ({\itshape ShiftPwd\/}), and an \char`\"{}expert\char`\"{} password ({\itshape ExpertPwd\/}) for the critical things.

The ODB would have two passwords defined, i.e.\par
  /Custom/Pwd/ExpertPwd\par
 /Custom/Pwd/ShiftPwd\par


The password is of course not secure in the sense that it's placed in plain text into the ODB, but its purpose is to prevent accidental modification, rather than malicious interference.

\par
 Password protection is available for
\begin{DoxyItemize}
\item \hyperlink{RC_mhttpd_Image_access_RC_mhttpd_custom_pw}{Password protection of Edit Boxes}
\item \hyperlink{RC_mhttpd_custom_ODB_access_RC_mhttpd_custom_odbset}{ODBSet JavaScript function}
\item \hyperlink{RC_mhttpd_Image_access_RC_mhttpd_custom_imagemap_pw}{Area map with password check}
\end{DoxyItemize}

If password protection {\bfseries is} set up, mhttpd will check the supplied password against the ODB entry, and show an error if they don't match.

\label{index_end}
\hypertarget{index_end}{}


 \paragraph{ODB RPC access}\label{RC_mhttpd_custom_RPC_access}


\par


 RPC access has been available from custom webpages since \hyperlink{NDF_ndf_jan_2009}{Jan 2009} . 

The \hyperlink{RC_mhttpd_custom_js_lib}{JavaScript library function} {\bfseries ODBRpc} is defined for RPC access (the library must be \hyperlink{RC_mhttpd_custom_js_lib_RC_mhttpd_include_js_library}{included}) : This permits buttons on MIDAS \char`\"{}custom\char`\"{} web pages to invoke RPC calls directly into user frontend programs, for example to turn hardware modules on or off.

\begin{Desc}
\item[\hyperlink{todo__todo000015}{Todo}]Documentation and examples needed -\/ ODBRpc\end{Desc}
\par
 \begin{table}[h]\begin{TabularC}{3}
\hline
JavaScript Function  &Purpose  &Parameters  

\\\cline{1-3}

\begin{DoxyCode}
 ODBRpc_rev0(name, rpc, args)
\end{DoxyCode}
  &to be filled  &\begin{TabularC}{2}
\hline
name &\par
  

\\\cline{1-2}
rpc &\par
   \\\cline{1-2}
args &\par
   \\\cline{1-2}
\end{TabularC}
\\\cline{1-2}
\end{TabularC}
\centering
\caption{Above: ODB RPC access from JavaScript }
\end{table}




\par
 \label{index_end}
\hypertarget{index_end}{}
 \paragraph{Inserting an Image into a Custom page}\label{RC_mhttpd_Image_access}
\par


\hypertarget{RC_mhttpd_Image_access_RC_mhttpd_custom_history}{}\subparagraph{Inserting a history image in a custom page}\label{RC_mhttpd_Image_access_RC_mhttpd_custom_history}
In the special case where the image to be inserted is a {\bfseries History} image, it can be inserted into a custom page using an HTML $<$$<$img...$>$ tag of the following form: 
\begin{DoxyCode}
<img src="http://hostname.domain:port/HS/Meterdis.gif&scale=12h&width=300">
\end{DoxyCode}


{\bfseries Examples} See \hyperlink{RC_mhttpd_custom_demo}{Demo of custom image page} and \hyperlink{RC_mhttpd_custom_demo_example_image_all}{image} created by the demo.

\par


\par
 \label{RC_mhttpd_Image_access_idx_mhttpd_page_custom_image_insert-image}
\hypertarget{RC_mhttpd_Image_access_idx_mhttpd_page_custom_image_insert-image}{}
 \hypertarget{RC_mhttpd_Image_access_RC_mhttpd_custom_image}{}\subparagraph{Image insertion into a Custom page}\label{RC_mhttpd_Image_access_RC_mhttpd_custom_image}
\label{RC_mhttpd_Image_access_idx_mhttpd_page_custom_image_history-image}
\hypertarget{RC_mhttpd_Image_access_idx_mhttpd_page_custom_image_history-image}{}

\begin{DoxyItemize}
\item An image may be inserted into a custom webpage by using an HTML $<$img...$>$ tag. The image file can be in any format supported by the browser (e.g. ~ {\bfseries gif},~ {\bfseries png},~ {\bfseries jpg} etc.).
\end{DoxyItemize}


\begin{DoxyItemize}
\item Alternatively, by defining the image in the subdirectory \char`\"{}images\char`\"{} in the /Custom ODB tree  features such as labels, bars and fills can be superimposed on the image. \par
 To make an image {\itshape \char`\"{}myexpt.gif\char`\"{}\/} available in a Custom Page, follow these steps:
\end{DoxyItemize}


\begin{DoxyEnumerate}
\item the subdirectory Images must be created in the ODB /Custom directory, e.g. using \hyperlink{RC_odbedit_utility}{odbedit} :


\begin{DoxyCode}
[local:Default:Stopped]/>cd Custom
[local:Default:Stopped]/Custom>mkdir Images  <-- if Images does not exist
[local:Default:Stopped]/Custom>cd Images/
\end{DoxyCode}



\item Under the subdirectory Images, create another subdirectory with the name of the image you are going to use (in this example, {\bfseries myexpt.gif}), i.e.


\begin{DoxyCode}
[local:Default:Stopped]Images>mkdir myexpt.gif <-- make a subdirectory with the i
      mage name
[local:Default:Stopped]Images>cd myexpt.gif/
\end{DoxyCode}



\item Under the subdirectory  myexpt.gif , create the STRING key  Background : 
\begin{DoxyCode}
[local:Default:Stopped]myexpt.gif>create string Background  <-- create key "Backg
      round" 
String length [32]: 256
\end{DoxyCode}



\item Set the key  Background to contain the name of the image-\/file. 
\begin{DoxyCode}
[local:Default:Stopped]myexpt.gif>set Background \midas\examples\custom\myexpt.gi
      f
\end{DoxyCode}
 
\end{DoxyEnumerate}

This image must be referenced in a custom HTML file, such as \hyperlink{myexpt_8html}{myexpt.html} . This file also includes other features such as {\bfseries active clickable areas, labels, bars and fills} superimposed on the image. These are explained in the sections \hyperlink{RC_mhttpd_Image_access_RC_mhttpd_custom_Labels_Bars_Fills}{below}.

Before it can be accessed from mhttpd, this file (\hyperlink{myexpt_8html}{myexpt.html}) must be \hyperlink{RC_mhttpd_Activate}{activated} by being defined in the ODB under /Custom. This is demonstrated in the custom demo \hyperlink{RC_mhttpd_custom_demo}{example} and the \hyperlink{RC_mhttpd_custom_demo_example_image_all}{image} created by the demo.

\label{RC_mhttpd_Image_access_idx_ODB_tree_custom_images}
\hypertarget{RC_mhttpd_Image_access_idx_ODB_tree_custom_images}{}
 \subparagraph{Demo of custom image page}\label{RC_mhttpd_custom_demo}
\par




\par


This demo will show you how to make a custom page containing an image, and superimpose edit boxes, clickable areas, labels, fills etc.

The HTML document \hyperlink{myexpt_8html}{myexpt.html} can be found in the examples/custom directory. This code forms part of a custom demo. For the full operation of this demo, you'll need to have the frontend {\bfseries \char`\"{}sample frontend\char`\"{}} (midas/example/experiment/frontend.c), mlogger, mhttpd running.

The code \hyperlink{myexpt_8html}{myexpt.html} is shown below for convenience: 
\begin{DoxyCode}
<html>
  <head>
   <title>MyExperiment Demo Status</title>
   <meta http-equiv="Refresh" content="30">
  </head>
 <body>
  <form name="form1" method="Get" action="/CS/MyExpt&">
     <table border=3 cellpadding=2>
          <tr><th bgcolor="#A0A0FF">Demo Experiment<th bgcolor="#A0A0FF">Custom M
      onitor/Control</tr> 
          <tr><td> <b><font color="#ff 0">Actions: </font></b><input
                      value="Status" name="cmd" type="submit"> <input type="submi
      t"
                      name="cmd" value="Start"><input type="submit" name="cmd" va
      lue="Stop">
           </td><td>
           <center> <a href="http://midas.triumf.ca/doc/html/index.html"> Help </
      a></center>
           </td></tr>
           <td>Current run #: <b><odb src="/Runinfo/run number"></b></td>
           <td>#events: <b><odb src="/Equipment/Trigger/Statistics/Events sent"><
      /b></td>
           </tr><tr>
           <td>Event Rate [/sec]: <b><odb src="/Equipment/Trigger/Statistics/Even
      ts per sec."></b></td>
           <td>Data Rate [kB/s]: <b><odb src="/Equipment/Trigger/Statistics/kByte
      s per sec."></b></td>
            </tr><tr>
            <td>Cell Pressure: <b><odb src="/Equipment/NewEpics/Variables/CellPre
      ssure"></b></td>
           <td>FaradayCup   : <b><odb src="/Equipment/NewEpics/Variables/ChargeFa
      radayCup"></b></td>
           </tr><tr>
           <td>Q1 Setpoint: <b><odb src="/Equipment/NewEpics/Variables/EpicsVars[
      17]" edit=1></b></td>
          <td>Q2 Setpoint: <b><odb src="/Equipment/NewEpics/Variables/EpicsVars[1
      9]" edit=1></b></td>
          </tr><tr>
          <th> <img src="http://localhost:8080/HS/Default/Trigger%20rate.gif?
                          exp=default&amp;scale=12h&amp;width=250">
          </th>
          <th> <img src="http://localhost:8080/HS/Default/Scaler%20rate.gif?
                          exp=default&amp;scale=10m&amp;width=250"></th>
          </tr>
          <tr><td colspan=2>
          <map name="myexpt.map">
          <area shape=rect coords="140,70, 420,170" 
                  href="http://midas.triumf.ca/doc/html/index.html" title="Midas 
      Doc">
          <area shape=rect coords="200,200,400,400"
                  href="http://localhost:8080" title="Switch pump">
       <area shape=rect coords="230,515,325,600"
              href="http://localhost:8080" title="Logger in color level (using Fi
      ll)">
        <img src="myexpt.gif" border=1 usemap="#myexpt.map">
          </map>
          </td></tr>
     </table></form>
   </body>
  </html>  
\end{DoxyCode}


To \hyperlink{RC_mhttpd_Activate}{activate} this HTML document, it has to be defined in the ODB as follow: 
\begin{DoxyCode}
[local:Default:Stopped]/>cd /Custom
[local:Default:Stopped]/Custom>create string Myexpt&
String length [32]: 256
[local:Default:Stopped]/Custom>set Myexpt& /midas/examples/custom/myexpt.html
\end{DoxyCode}
 After refresh, the alias-\/link {\bfseries Myexpt} should be visible on the Main Status Page. If you have not already inserted the image file name {\bfseries myexpt.gif} into the Custom page, do so now by following the instructions to \hyperlink{RC_mhttpd_Image_access_RC_mhttpd_custom_image}{insert the image}.

Once the image is inserted, after refresh the image should be visible by clicking on the alias-\/link {\bfseries Myexpt}, and the mapping active.

\label{RC_mhttpd_custom_demo_mapping_demo}
\hypertarget{RC_mhttpd_custom_demo_mapping_demo}{}
 The mapping based on myexpt.map is active, hovering the mouse over the boxes will display the associated titles (Midas Doc, Switch pump, etc), By clicking on either box the browser will go to the defined html page specified by the map.

\par
\par
\par
 \begin{center}  Figure 1 : Demo Custom web page with external reference to html document. \par
\par
\par
  \end{center}  \par
\par
\par


In addition to these initial features, multiple ODB values can be superimposed at define location on the image. Each entry will have a ODB tree associated to it defining the ODB variable, X/Y position, color, etc...

Make the {\bfseries Rate} label as explained \hyperlink{RC_mhttpd_Image_access_RC_mhttpd_custom_labels}{above}. After refreshing the web page, you will see the error message below:


\begin{DoxyCode}
>>>>>>>> Refresh web page <<<<<<<<

12:32:38 [mhttpd] [mhttpd.c:5508:show_custom_gif] Empty Src key for label "Rate"
\end{DoxyCode}


The keys created in the Labels/Rate subtree are explained \hyperlink{RC_mhttpd_Image_access_RC_mhttpd_labels_tree}{here}. Customize this label by assigning the {\bfseries Src} key to a valid ODB Key variable, and the X,Y fields to position top-\/left corner of the label, e.g. 
\begin{DoxyCode}
[local:Default:Stopped]Rate>set src "/Equipment/Trigger/statistics/kbytes per sec
      ."
[local:Default:Stopped]Rate>set x 330
[local:Default:Stopped]Rate>set y 250 
[local:Default:Stopped]Rate>set format "Rate:%1.1f kB/s"
\end{DoxyCode}


Once the initial label is created, the simplest way to extent to multiple labels is to copy the existing label sub-\/tree and modify the label \hyperlink{structparameters}{parameters}. 
\begin{DoxyCode}
[local:Default:Stopped]Labels>cd .. 
[local:Default:Stopped]Labels>copy Rate Event
[local:Default:Stopped]Labels>cd Events/
[local:Default:Stopped]Event>set src "/Equipment/Trigger/statistics/events per se
      c."
[local:Default:Stopped]Event>set Format "Rate:%1.1f evt/s"
[local:Default:Stopped]Event>set y 170
[local:Default:Stopped]Event>set x 250
\end{DoxyCode}
 You will now have two {\bfseries Labels}, named \char`\"{}Rate\char`\"{} and \char`\"{}Event\char`\"{}, both subtrees under ../Labels.

In the same manner, you can create \hyperlink{RC_mhttpd_Image_access_RC_mhttpd_custom_bars}{bars} used for level representation. The keys in the Bars subdirectory are explained \hyperlink{RC_mhttpd_Image_access_RC_mhttpd_bars_tree}{above}.

This code will setup two ODB values defined by the fields src. 
\begin{DoxyCode}
[local:Default:Stopped]myexpt.gif>pwd
/Custom/Images/myexpt.gif
[local:Default:Stopped]myexpt.gif>mkdir Bars
[local:Default:Stopped]myexpt.gif>cd bars/
[local:Default:Stopped]Labels>mkdir Rate

>>>>>>>> Refresh web page <<<<<<<<

14:05:58 [mhttpd] [mhttpd.c:5508:show_custom_gif] Empty Src key for bars "Rate"
[local:Default:Stopped]Labels>cd Rate/
[local:Default:Stopped]Rate>set src "/Equipment/Trigger/statistics/kbytes per sec
      ."
[local:Default:Stopped]Rate>set x 4640
[local:Default:Stopped]Rate>set y 210 
[local:Default:Stopped]Rate>set max 1e6 
[local:Default:Stopped]Labels>cd .. 
[local:Default:Stopped]Labels>copy Rate Events
[local:Default:Stopped]Labels>cd Events/
[local:Default:Stopped]Event>set src "/logger/channles/0/statistics/events writte
      n"
[local:Default:Stopped]Event>set direction 1
[local:Default:Stopped]Event>set y 240
[local:Default:Stopped]Event>set x 450
[local:Default:Stopped]Rate>set max 1e6 
\end{DoxyCode}


You will now have two {\bfseries Bars}, also named \char`\"{}Rate\char`\"{} and \char`\"{}Event\char`\"{}, both subtrees under ../Bars.

The last feature to be added is a \hyperlink{RC_mhttpd_Image_access_RC_mhttpd_custom_fills}{Fill} (where an area can be filled with different colors depending on the given ODB value). These have to be entered by hand. See instructions in \hyperlink{RC_mhttpd_Image_access_RC_mhttpd_custom_fills}{fills}, which shows you how to create a {\bfseries Filled} area named \char`\"{}Level\char`\"{} (a subtree under ../Fills).

Once all these features have been added, the custom page will look as Figure 2: \label{RC_mhttpd_custom_demo_example_image_all}
\hypertarget{RC_mhttpd_custom_demo_example_image_all}{}


\par
\par
\par
 \begin{center}  Figure 2 : Demo Custom web page with labels,bars,fills and history plots \par
\par
\par
  \end{center}  \par
\par
\par




\label{index_end}
\hypertarget{index_end}{}
 \subparagraph{Internal custom page}\label{RC_mhttpd_Internal}
\par




An {\bfseries internal} custom page (written in HTML and/or JavaScript) may be imported under a given /Custom/ ODB key. The name of this key will appear in the Main Status page as an \hyperlink{RC_mhttpd_Alias_page}{alias-\/links} (or alias-\/button -\/ \hyperlink{NDF_ndf_dec_2009}{Dec 2009}). By clicking on this link/button, the contents of this key is interpreted as html content.

The insertion of a new Custom page requires the following steps:
\begin{DoxyItemize}
\item Create an initial html file using your favorite HTML editor (see \hyperlink{RC_mhttpd_custom_features_RC_mhttpd_custom_create}{How to create a custom page})
\item \hyperlink{RC_mhttpd_Activate_RC_odb_custom_internal_example}{Import} this file
\end{DoxyItemize}

\begin{DoxyNote}{Note}

\begin{DoxyItemize}
\item Once the file is imported into ODB, you can {\bfseries ONLY} edit it through the web (as long as mhttpd is active) by clicking on the {\bfseries ODB(button)} ... Custom(Key) ... Edit (Hyperlink at the bottom of the key).
\end{DoxyItemize}
\end{DoxyNote}

\begin{DoxyItemize}
\item The Custom page can also be exported back to a ASCII file using the odbedit command \hyperlink{RC_odbedit_examples_RC_odbedit_export}{export}, e.g. 
\begin{DoxyCode}
  [local:midas:Stopped]/>cd Custom/
  [local:midas:Stopped]/Custom>export test&
  File name: mcustom.html
  [local:midas:Stopped]/Custom>
\end{DoxyCode}

\end{DoxyItemize}

Figure 1 shows an {\bfseries internal} custom page which has been imported into the ODB at key /Custom/Overview\& as shown in Figure 2.

\par
\par
\par
 \begin{center}  Figure 1 : Internal custom web page with history graph. \par
\par
\par
  \end{center}  \par
\par
\par


\par
\par
\par
 \begin{center}  Figure 2 : Internal custom web page loaded into the ODB. \par
\par
\par
  \end{center}  \par
\par
\par


\par


\par
 \par




\label{index_end}
\hypertarget{index_end}{}
 \paragraph{Custom Status page}\label{RC_mhttpd_custom_status}
\par
 

{\bfseries A Custom Status page} is a custom page that {\bfseries replaces the default MIDAS status page.} \par
\par
 A custom Status page is activated as described \hyperlink{RC_mhttpd_Activate_RC_odb_custom_status}{previously}. Features such as the standard \hyperlink{RC_mhttpd_custom_features_RC_mhttpd_custom_midas_buttons}{Menu buttons}, \hyperlink{RC_mhttpd_custom_features_RC_mhttpd_custom_alias}{Alias Buttons and Hyperlinks} etc. can be provided in the same way as for any other custom page. When replacing the default Status page by a Custom Status page, any \hyperlink{RC_mhttpd_status_page_features_RC_mhttpd_status_script_buttons}{script-\/buttons} defined on the original MIDAS Status page can be easily turned into \hyperlink{RC_mhttpd_custom_features_RC_mhttpd_custom_script_buttons}{CustomScript Buttons} on the Custom Status page by placing links in the  /customscript  tree to any existing script-\/buttons under the  /Script  tree.


\begin{DoxyCode}
[local:bnmr:S]/>ls /customscript
Hold      -> /Script/BNMR Hold/
Continue  -> /script/Continue 
Scalers   -> /script/Scalers
\end{DoxyCode}
 See \hyperlink{RC_mhttpd_custom_status_RC_mhttpd_status_toggle}{example} of the above defined buttons.\hypertarget{RC_mhttpd_custom_status_RC_mhttpd_status_toggle}{}\subparagraph{Toggle between Default and Custom Status pages}\label{RC_mhttpd_custom_status_RC_mhttpd_status_toggle}
In the following example, the main Status page can be toggled between the Custom version and the default MIDAS status page.

The Custom status page contains a button {\bfseries ToggleStatusPage} that will toggle between the two cases. A custom-\/link has been provided on the Default status page in order to access the custom status page when using the Default status page.

The following shows the /Custom directory when the Custom status page is running. The /Custom directory contains the key {\bfseries CustomStatus\&}, a link to the custom Status page. The key {\bfseries Status} is set as a link to {\bfseries CustomStatus\&}. 
\begin{DoxyCode}
[local:bnmr:S]/>ls /custom
CustomStatus&                   /home/bnmr/online/custom/Custom_Status.html
Status                          /custom/CustomStatus& -> /home/bnmr/online/custom
      /Custom_Status.html
\end{DoxyCode}


A script-\/button is set up in the custom status page (see \hyperlink{RC_mhttpd_custom_features_RC_mhttpd_custom_script_buttons}{CustomScript Buttons}) as follows


\begin{DoxyCode}
[local:bnmr:S]/>ls /customscript/ToggleStatusPage
cmd                             /home/bnmr/online/perl/set_status.pl
include path                    /home/bnmr/online/perl
experiment name                 bnmr
beamline                        bnmr
\end{DoxyCode}


The script /home/bnmr/online/perl/set\_\-status.pl will toggle between the status page being set to the default or to the custom, by removing or creating the key \char`\"{}Status\char`\"{} in the /Custom subdirectory.

The following shows the /custom directory when the default Status page is selected: 
\begin{DoxyCode}
[local:bnmr:S]/>ls /custom
CustomStatus&                   /home/bnmr/online/custom/Custom_Status.html
Status                          /custom/CustomStatus& -> /home/bnmr/online/custom
      /Custom_Status.html
\end{DoxyCode}


\begin{center} Custom and default status pages for the BNMR experiment at TRIUMF  \end{center} 

The script file {\itshape  set\_\-status.pl \/} is as follows:


\begin{DoxyCode}
#!/usr/bin/perl 
# above is magic first line to invoke perl
# or for debug
###  !/usr/bin/perl -d
#
#   Normally invoked from set_status button on custom Web page Custom_Status.html
      
# 
# invoke this script with cmd  e.g.
#              include                    experiment        beamline
#                path                                     
# set_status.pl /home/bnmr/online/perl       bnmr             bnmr
# set_status.pl /home/bnqr/online/perl       bnqr             bnqr
#
# toggles Status page between std MIDAS page and Custom Status page
# by either deleting odb key /Custom/Status
# or creating odb key /Custom/Status by loading a file /home/bn<mq>r/online/custo
      m/status.odb
#
use strict;
##################### G L O B A L S ####################################
our  @ARRAY;
our $FALSE=0;
our $FAILURE=0;
our $TRUE=1;
our $SUCCESS=1;
our $ODB_SUCCESS=0;   # status = 0 is success for odb
our $DEBUG=$FALSE;    # set to 1 for debug, 0 for no debug
our $EXPERIMENT=" ";
our $ANSWER=" ";      # reply from odb_cmd
our $COMMAND=" ";     # copy of command sent be odb_cmd (for error handling)
our $STATE_STOPPED=1; # Run state is stopped
our $STATE_RUNNING=3; # Run state is running
# for odb  msg cmd:
our $MERROR=1; # error
our $MINFO=2;  # info
our $MTALK=32; # talk
# constants for print_3
our $DIE = $TRUE;  # die after print_3
our $CONT = $FALSE; # do not die after print_3 (set_status)
#e.g.    print_3($name,  "ERROR: no path supplied",$MERROR,$DIE);
#    or   print_3($name,  "INFO: run number has not changed",$MINFO,$CONT);
# print_3 is a subroutine that prints the message to three places:
#     the output file, standard output, MIDAS message log
# print_2 is a subroutine that prints the message to two places:
#     the output file and standard output

#######################################################################
#  parameters needed by init_check.pl (required code common to perlscripts) :
# init_check uses $inc_dir, $expt, $beamline from the input parameters
our ($inc_dir,$expt, $beamline ) = @ARGV;
our $len =  $#ARGV; # array length
our $name = "set_status"; # same as filename
our $nparam=3;
our $outfile = "set_status.txt"; #path supplied by file open
our $parameter_msg = "include_path, experiment;  beamline\n";
#######################################################################
# local variables
my ($path, $key, $status);
my $debug=$FALSE;
my $filename;
#######################################################################
#
$|=1; # flush output buffers


# Inc_dir needed because when script is invoked by browser it can't find the
# code for require
unless ($inc_dir) { die "$name: No include directory path has been supplied\n";}
$inc_dir =~ s/\/$//;  # remove any trailing slash
require "$inc_dir/odb_access.pl"; 

# Init_check.pl checks:
#   one copy of script running
#   no. of input parameters
#   opens output file $outfile on file handle FOUT
#
require "$inc_dir/init_check.pl";

#
# Output will be sent to file $outfile 
# because this is for use with the browser and STDOUT and STDERR get set to null

print FOUT  "$name starting with parameters:  \n";
print FOUT  "Experiment = $expt; \n";
#
#
#

if( $beamline =~ /bn[qm]r/i )
{
# BNM/QR experiments use custom page
}
else
{ # no action
exit;
}

# Status is now a link; supply the expected name of the link or odb_cmd will flag
       and error 
($status) = odb_cmd ( "ls","/Custom","Status","","","CustomStatus&");

if  ($status)
{
    ($status)=odb_cmd ( "rm","/Custom","Status","", "" ) ;  
}
else
{
# load required link from a saved file
    $filename = "/home/$beamline/online/custom/status.odb"; 
    unless (-e $filename)
    {
        print_3 ($name,"No such file as $filename",$DIE);
    }

    ($status)=odb_cmd ( "load","$filename","","", "" ) ;
    if  ($status)
    {
        
        print_2($name, "loaded file to create /Custom/Status",$CONT);
    }
    else {
        print_2 ($name,"$name: after odb_cmd, ANSWER=$ANSWER ",$CONT);}

}    
exit;
\end{DoxyCode}
 \par


\par
  \label{index_end}
\hypertarget{index_end}{}
 \subsection{Monitoring the Experiment}\label{RC_Monitor}
\par
  \par


The most powerful tool for {\bfseries monitoring} a MIDAS experiment is of course \hyperlink{RC_mhttpd}{mhttpd}, used for both Run Control and Monitoring. However, it does not (yet) cover every feature needed, and the MIDAS system also provides several other utilities for monitoring aspects of the data acquistion system. These include \hyperlink{RC_Monitor_RC_mstat_utility}{mstat} to display the run statistics, and \hyperlink{RC_Monitor_RC_mdump_utility}{mdump}, to dump the raw data, which is particularly useful for debugging. These utilities do provide information without having to use a browser.


\begin{DoxyItemize}
\item \hyperlink{RC_mhttpd}{Monitoring and Run Control using the MIDAS Web Server}
\item \hyperlink{RC_Monitor_RC_mstat_utility}{mstat -\/ monitor application display}
\item \hyperlink{RC_Monitor_RC_mdump_utility}{mdump -\/ displays event bank contents (online or offline)}
\item \hyperlink{RC_Monitor_RC_rmidas_utility}{rmidas -\/ ROOT Midas application for histograms/run control}
\item \hyperlink{RC_Monitor_RC_hvedit_utility}{hvedit -\/ High Voltage editor and GUI} \par

\end{DoxyItemize}

\par


\par
 \label{RC_Monitor_idx_mstat-utility}
\hypertarget{RC_Monitor_idx_mstat-utility}{}
 \hypertarget{RC_Monitor_RC_mstat_utility}{}\subsubsection{mstat -\/ monitor application display}\label{RC_Monitor_RC_mstat_utility}
{\bfseries mstat} is a simple ASCII status display. It presents in a compact form the most valuable information of the current condition of the MIDAS Acquisition system. The display is composed at the most of 5 sections depending on the current status of the experiment. The section displayed in order from top to bottom refer to:
\begin{DoxyItemize}
\item Run information.
\item Equipment listing and statistics if any \hyperlink{FrontendOperation_FE_frontend_utility}{frontend} is active.
\item Logger information and statistics if \hyperlink{F_Logging_F_mlogger_utility}{mlogger} is active.
\item Lazylogger status if \hyperlink{F_LogUtil_F_lazylogger_utility}{lazylogger} is active.
\item Client listing. lock
\item {\bfseries  Arguments }
\begin{DoxyItemize}
\item \mbox{[}-\/h \mbox{]} : help
\item \mbox{[}-\/h hostname \mbox{]} : host name (see \hyperlink{RC_odbedit_utility}{odbedit -\/ ODB Editor and run control utility})
\item \mbox{[}-\/e exptname \mbox{]} : experiment name (see \hyperlink{RC_odbedit_utility}{odbedit -\/ ODB Editor and run control utility})
\item \mbox{[}-\/l \mbox{]} : loop. Forces mstat to remain in a display loop. Enter \char`\"{}!\char`\"{} to terminate the command.
\item \mbox{[}-\/w time \mbox{]} : refresh rate in second. Specifies the delay in second before refreshing the screen with up to date information. Default: 5 seconds. Has to be used in conjunction with -\/l switch. Enter \char`\"{}R\char`\"{} to refresh screen on next update.
\end{DoxyItemize}
\end{DoxyItemize}


\begin{DoxyItemize}
\item {\bfseries  Usage } 
\begin{DoxyCode}
 >mstat -l
*-v1.8.0- MIDAS status page -------------------------Mon Apr  3 11:52:52 2000-* 
Experiment:chaos       Run#:8699    State:Running          Run time :00:11:34   
Start time:Mon Apr  3 11:41:18 2000                                             
                                                                                
FE Equip.   Node              Event Taken    Event Rate[/s] Data Rate[Kb/s]     
B12Y        pcch02            67             0.0            0.0                 
CUM_Scaler  vwchaos           23             0.2            0.2                 
CHV         pcch02            68             0.0            0.0                 
KOS_Scalers vwchaos           330            0.4            0.6                 
KOS_Trigger vwchaos           434226         652.4          408.3               
KOS_File    vwchaos           0              0.0            0.0                 
Target      pcch02            66             0.0            0.0 
                                                                                
Logger Data dir: /scr0/spring2000            Message File: midas.log            
Chan.   Active Type      Filename            Events Taken   KBytes Taken        
  0     Yes    Disk      run08699.ybs        434206          4.24e+06           
                                                                                
Lazy Label     Progress  File name           #files         Total               
cni-53         100[%]    run08696.ybs        15             44.3[%]             
                                                                                
Clients:  MStatus/koslx0         Logger/koslx0          Lazy_Tape/koslx0        
          CHV/pcch02             MChart1/umelba         ODBEdit/koslx0          
          CHAOS/vwchaos          ecl/koslx0             Speaker/koslx0          
          MChart/umelba          targetFE/pcch02        HV_MONITOR/umelba       
          SUSI/koslx0            History/kosal2         MStatus1/dasdevpc  
     
*------------------------------------------------------------------------------*
\end{DoxyCode}

\end{DoxyItemize}





\label{RC_Monitor_idx_mdump-utility}
\hypertarget{RC_Monitor_idx_mdump-utility}{}
 \hypertarget{RC_Monitor_RC_mdump_utility}{}\subsubsection{mdump       -\/ displays event bank contents (online or offline)}\label{RC_Monitor_RC_mdump_utility}
This application allows the experimenter to \char`\"{}peep\char`\"{} into the data flow in order to display a snap-\/shot of the event. Its use is particularly powerful during experimental setup. \par
In addition mdump has the capability to operate on data save-\/set files stored on disk or tape. The main {\bfseries mdump} restriction is the fact that it works only for events formatted in \hyperlink{FE_bank_construction}{banks} (i.e. MIDAS banks).

mdump can be built with {\bfseries zlib.a} in order to gain direct access to the data within a file with extension {\bfseries mid.gz} or {\bfseries ybs.gz}. See \hyperlink{BuildingOptions_BO_NEED_ZLIB}{NEED\_\-ZLIB} Building Option.


\begin{DoxyItemize}
\item {\bfseries  Arguments } for Online use
\begin{DoxyItemize}
\item \mbox{[}-\/h \mbox{]} : help for online use.
\item \mbox{[}-\/h hostname \mbox{]} : Host name.
\item \mbox{[}-\/e exptname \mbox{]} : Experiment name.
\item \mbox{[}-\/b bank name\mbox{]} : Display event containg only specified bank name.
\item \mbox{[}-\/c compose\mbox{]} : Retrieve and compose file with either Add run\# or Not (def:N).
\item \mbox{[}-\/f format\mbox{]} : Data representation (x/d/ascii) def:hex.
\item \mbox{[}-\/g type \mbox{]} : Sampling mode either Some or All (def:S). $>$$>$$>$ in case of -\/c it is recommented to used -\/g all.
\item \mbox{[}-\/i id \mbox{]} : Event Id.
\item \mbox{[}-\/j \mbox{]} : Display bank header only.
\item \mbox{[}-\/k id \mbox{]} : Event mask.
\item \mbox{[}-\/l number \mbox{]} : Number of consecutive event to display (def:1).
\item \mbox{[}-\/m mode\mbox{]} : Display mode either Bank or Raw (def:B)
\end{DoxyItemize}
\end{DoxyItemize}


\begin{DoxyItemize}
\item \mbox{[}-\/s \mbox{]} : Data transfer rate diagnositic.
\item \mbox{[}-\/w time\mbox{]} : Insert wait in \mbox{[}sec\mbox{]} between each display.
\item \mbox{[}-\/x filename \mbox{]} : Input channel. data file name of data device. (def:online)
\item \mbox{[}-\/y \mbox{]} : Display consistency check only.
\item \mbox{[}-\/z buffer name\mbox{]} : Midas buffer name to attach to (def:SYSTEM)
\end{DoxyItemize}


\begin{DoxyItemize}
\item Additional {\bfseries  Arguments } for Offline
\begin{DoxyItemize}
\item \mbox{[}-\/x -\/h \mbox{]} : help for offline use.
\item \mbox{[}-\/t type \mbox{]} : Bank format (MIDAS).
\item \mbox{[}-\/w what\mbox{]} : Header, Record, Length, Event, Jbank\_\-list (def:E)
\end{DoxyItemize}
\end{DoxyItemize}


\begin{DoxyItemize}
\item {\bfseries  Usage } mdump can operate on either data stream (online) or on save-\/set data files. Specific help is available for each mode. 
\begin{DoxyCode}
 > mdump -h
 > mdump -x -h
\end{DoxyCode}

\end{DoxyItemize}

\par


\par
 \hypertarget{RC_Monitor_RC_mdump_ex1}{}\paragraph{Example 1 mdump in offline mode}\label{RC_Monitor_RC_mdump_ex1}
The example below shows mdump operating on a file of saved data in \hyperlink{FE_Data_format_FE_Midas_format}{MIDAS format} : 
\begin{DoxyCode}
Tue> mdump -x run37496.mid | more
------------------------ Event# 0 --------------------------------
------------------------ Event# 1 --------------------------------
Evid:0001- Mask:0100- Serial:1- Time:0x393c299a- Dsize:72/0x48

#banks:2 - Bank list:-SCLRRATE-

Bank:SCLR Length: 24(I*1)/6(I*4)/6(Type) Type:Integer*4
   1-> 0x00000000 0x00000000 0x00000000 0x00000000 0x00000000 0x00000000 

Bank:RATE Length: 24(I*1)/6(I*4)/6(Type) Type:Real*4 (FMT machine dependent)
   1-> 0x00000000 0x00000000 0x00000000 0x00000000 0x00000000 0x00000000 
------------------------ Event# 2 --------------------------------
Evid:0001- Mask:0004- Serial:1- Time:0x393c299a- Dsize:56/0x38
#banks:2 - Bank list:-MMESMMOD-

Bank:MMES Length: 24(I*1)/6(I*4)/6(Type) Type:Real*4 (FMT machine dependent)
   1-> 0x3de35788 0x3d0b0e29 0x00000000 0x00000000 0x3f800000 0x00000000 

Bank:MMOD Length: 4(I*1)/1(I*4)/1(Type) Type:Integer*4


   1-> 0x00000001 
------------------------ Event# 3 --------------------------------
Evid:0001- Mask:0008- Serial:1- Time:0x393c299a- Dsize:48/0x30
#banks:1 - Bank list:-BMES-

Bank:BMES Length: 28(I*1)/7(I*4)/7(Type) Type:Real*4 (FMT machine dependent)
   1-> 0x443d7333 0x444cf333 0x44454000 0x4448e000 0x43bca667 0x43ce0000 0x43f980
      00 
------------------------ Event# 4 --------------------------------
Evid:0001- Mask:0010- Serial:1- Time:0x393c299a- Dsize:168/0xa8
#banks:1 - Bank list:-CMES-

Bank:CMES Length: 148(I*1)/37(I*4)/37(Type) Type:Real*4 (FMT machine dependent)
   1-> 0x3f2f9fe2 0x3ff77fd6 0x3f173fe6 0x3daeffe2 0x410f83e8 0x40ac07e3 0x3f6ebf
      d8 0x3c47ffde 
   9-> 0x3e60ffda 0x00000000 0x00000000 0x00000000 0x00000000 0x00000000 0x000000
      00 0x3f800000 
  17-> 0x00000000 0x00000000 0x00000000 0x00000000 0x00000000 0x00000000 0x000000
      00 0x00000000 

  25-> 0x3f800000 0x3f800000 0x3f800000 0x00000000 0x3f800000 0x00000000 0x3f8000
      00 0x3f800000 
  33-> 0x3f800000 0x3f800000 0x3f800000 0x3f800000 0x00000000 
------------------------ Event# 5 --------------------------------
Evid:0001- Mask:0020- Serial:1- Time:0x393c299a- Dsize:32/0x20
#banks:1 - Bank list:-METR-

Bank:METR Length: 12(I*1)/3(I*4)/3(Type) Type:Real*4 (FMT machine dependent)
   1-> 0x00000000 0x39005d87 0x00000000 
...
\end{DoxyCode}


\par


\par
 \hypertarget{RC_Monitor_RC_mdump_ex2}{}\paragraph{Example 2 : mdump in online mode}\label{RC_Monitor_RC_mdump_ex2}
The examples below shows mdump operating in online mode (data is in \hyperlink{FE_Data_format_FE_Midas_format}{MIDAS format}).
\begin{DoxyItemize}
\item {\bfseries  Example 1 : dump the bankheaders} 
\begin{DoxyCode}
> mdump -j
[pol@isdaq01 pol]$ mdump -j
-4506 -- Enter <!> to Exit ------- Midas Dump ---
------------------------ Event# 1 ------------------------
Evid:0002- Mask:0001- Serial:2- Time:0x4c9a4c2b- Dsize:832/0x340
#banks:2 Bank list:-HI00HI01-
\end{DoxyCode}

\end{DoxyItemize}


\begin{DoxyItemize}
\item {\bfseries  Example 2 : dump the bank CYCI in decimal format } 
\begin{DoxyCode}
[pol@isdaq01 pol]$ mdump -b CYCI -l 2 -f d
-4506 -- Enter <!> to Exit ------- Midas Dump ---
------------------------ Event# 1 ------------------------
Bank -CYCI- not found (2) in #banks:2 Bank list:-HI00HI01-
------------------------ Event# 2 ------------------------
#banks:2 Bank list:-CYCIHSCL-
Bank:CYCI Length: 36(I*1)/9(I*4)/9(Type) Type:Unsigned Integer*4
   1->       16       16        1        0        4        0        0        0
   9->        0
\end{DoxyCode}

\end{DoxyItemize}

\par


\par
\hypertarget{RC_Monitor_RC_rmidas_utility}{}\subsubsection{rmidas       -\/ ROOT Midas application for histograms/run control}\label{RC_Monitor_RC_rmidas_utility}
\label{RC_Monitor_idx_rmidas-utility}
\hypertarget{RC_Monitor_idx_rmidas-utility}{}
 Root/Midas remote GUI application for root histograms and possible run control under the ROOT. environment.

 Users of this utility may also be interested in a \hyperlink{RC_ROOT_analyzer_page}{Custom Page showing ROOT analyzer output} written for \hyperlink{RC_mhttpd}{mhttpd}, (Dec 2009) 


\begin{DoxyItemize}
\item {\bfseries  Arguments }
\begin{DoxyItemize}
\item \mbox{[}-\/h \mbox{]} : help
\item \mbox{[}-\/h hostname \mbox{]} : host name
\item \mbox{[}-\/e exptname \mbox{]} : experiment name
\end{DoxyItemize}
\item {\bfseries  Usage } to be written.
\item {\bfseries  Example } 
\begin{DoxyCode}
 >rmidas midasserver.domain
\end{DoxyCode}

\end{DoxyItemize}

\begin{center} rmidas display sample. Using the example/experiment/ demo setup.  \end{center} 



\hypertarget{RC_Monitor_RC_hvedit_utility}{}\subsubsection{hvedit       -\/ High Voltage editor and GUI}\label{RC_Monitor_RC_hvedit_utility}
\label{RC_Monitor_idx_hvedit-utility}
\hypertarget{RC_Monitor_idx_hvedit-utility}{}
 High Voltage editor, graphical interface to the Slow Control System. Originally for Windows machines, but recently ported on Linux under Qt by Andreas Suter.


\begin{DoxyItemize}
\item {\bfseries  Arguments }
\begin{DoxyItemize}
\item \mbox{[}-\/h \mbox{]} : help
\item \mbox{[}-\/h hostname \mbox{]} : host name
\item \mbox{[}-\/e exptname \mbox{]} : experiment name
\item \mbox{[}-\/D \mbox{]} : start program as a daemon
\end{DoxyItemize}
\item {\bfseries  Usage }: To control the high voltage system, the program HVEdit can be used under Windows 95/NT. It can be used to set channels, save and load values from disk and print them. The program can be started several times even on different computers. Since they are all linked to the same ODB arrays, the demand and measured values are consistent among them at any time. HVEdit is started from the command line:
\item {\bfseries  Example } 
\begin{DoxyCode}
 >hvedit
\end{DoxyCode}

\end{DoxyItemize}

\par
  \par


\label{index_end}
\hypertarget{index_end}{}
 \subsection{Customizing the Experiment}\label{RC_customize_ODB}
\par
  \par
\hypertarget{RC_customize_ODB_RC_customize_intro}{}\subsubsection{Introduction}\label{RC_customize_ODB_RC_customize_intro}
When the MIDAS system is installed, the ODB is loaded with default values, which will provide the user with a working system. However, by customizing the ODB, the user can enable powerful features, thus providing him or herself with very sophisticated DAQ system to suit almost every experiment. The user may even wish to customize their experiment further by providing \hyperlink{RC_mhttpd_Custom_page}{custom web-\/pages} for their particular application.

Customizing the ODB for Run Control involves some or all of:
\begin{DoxyItemize}
\item setting up the \hyperlink{RC_customize_ODB_RC_Edit_On_Start}{Edit-\/on-\/Start} \hyperlink{structparameters}{parameters}
\item setting up \hyperlink{RC_customize_ODB_RC_Access_Control}{Access control (web security)}
\item setting up \hyperlink{RC_customize_ODB_RC_Lock_when_Running}{write-\/protection when running}
\item setting up \hyperlink{RC_Hot_Link}{hot-\/links}
\item setting up the \hyperlink{F_Logging_Data}{data logging}
\item setting up the \hyperlink{F_History_logging}{history logging}
\item setting up scripts to act at end and begin of run
\item setting up the \hyperlink{RC_customize_ODB_RC_Alarm_System}{MIDAS Alarm System}
\item setting up \hyperlink{RC_mhttpd_Alias_page}{alias links}
\item setting up the \hyperlink{F_Elog}{elog}
\item changing the size of the Midas buffers
\end{DoxyItemize}

Other customization may include
\begin{DoxyItemize}
\item \hyperlink{RC_customize_ODB_RC_starting_clients}{Customize the scripts that start up and shut down the clients.}
\end{DoxyItemize}

Some of this customization has already been described, such as setting up the data and history logging and the electronic logbook (see \hyperlink{Features}{SECTION 4: Features}). Further customization will be described in the following sections.





\label{RC_customize_ODB_idx_ODB_tree_Experiment}
\hypertarget{RC_customize_ODB_idx_ODB_tree_Experiment}{}
 \hypertarget{RC_customize_ODB_RC_ODB_Experiment_Tree}{}\subsubsection{The ODB /Experiment tree}\label{RC_customize_ODB_RC_ODB_Experiment_Tree}
When initially created by the MIDAS installation, the ODB /Experiment tree contains the following keys:


\begin{DoxyCode}
[local:t2kgas:S]/>ls -rlt /experiment
Key name                        Type    #Val  Size  Last Opn Mode Value
---------------------------------------------------------------------------
Experiment                      DIR
    Name                        STRING  1     32    13m  0   RWD  t2kgas
    Buffer sizes                DIR
        SYSMSG                  DWORD   1     4     25h  0   RWD  100000
    Menu Buttons                STRING  1     256   25h  0   RWD  Start, ODB, Mes
      sages, ELog, Alarms, Programs, History, MSCB, Config, Help
\end{DoxyCode}


The meaning of these keys is explained \hyperlink{RC_customize_ODB_RC_Experiment_tree_keys}{below}.

\par
 \hypertarget{RC_customize_ODB_RC_Experiment_tree_keys}{}\paragraph{Explanation of the keys in the ODB /Experiment tree}\label{RC_customize_ODB_RC_Experiment_tree_keys}
\label{RC_customize_ODB_RC_customize_mhttpd_run_buttons}
\hypertarget{RC_customize_ODB_RC_customize_mhttpd_run_buttons}{}
 This table also includes some of the {\bfseries optional keys} that the user may create.

that \hyperlink{structparameters}{parameters} Menu Buttons and Hide Run Buttons allow the customization of the run control buttons appearing on the \hyperlink{RC_mhttpd_Main_Status_page}{Main Status Page} . They have no effect when running \hyperlink{RC_odbedit_utility}{odbedit};

\begin{table}[h]\begin{TabularC}{4}
\hline
\multirow{1}{\linewidth}{Keys in /Experiment ODB tree\par
   }\\\cline{1-1}
Key\par
  &Type\par
  &Explanation\par
  

\\\cline{1-3}
Name\par
  &\par
  &STRING\par
  &Contains the name of the experiment. Filled automatically when the ODB is created. 

\\\cline{1-4}
Buffer Sizes\par
  &\par
  &DIR\par
 

&Contains the sizes of the {\bfseries Midas} {\bfseries Buffers} for the experiment. Created with default values. The sizes can be changed to optimize the memory usage. See \hyperlink{FE_event_buffer_size}{Increase the Event Buffer Size(s)} for details.  

\\\cline{1-4}
\par
  &SYSMSG  &DWORD  &Size of SYSMSG buffer 

\\\cline{1-4}
\par
  &SYSTEM  &DWORD  &Size of SYSTEM buffer  

\\\cline{1-4}
\label{RC_customize_ODB_experiment_menu_buttons}
\hypertarget{RC_customize_ODB_experiment_menu_buttons}{}
 Menu Buttons\par
  &\par
  &STRING\par
 

&This key added automatically by mhttpd  (since \hyperlink{NDF_ndf_dec_2009}{Dec 2009}).  to allow the \hyperlink{RC_mhttpd_status_page_features_RC_mhttpd_status_menu_buttons}{Menu buttons} that appear on the mhttpd \hyperlink{RC_mhttpd_Main_Status_page}{Main Status Page} to be customized. The default set of buttons is {\bfseries Start, ODB, Messages, ELog, Alarms, Programs, History, MSCB Config, Help}



\\\cline{1-4}
\label{RC_customize_ODB_experiment_hide_run_buttons}
\hypertarget{RC_customize_ODB_experiment_hide_run_buttons}{}
 Hide Run Buttons\par
  &\par
  &BOOL\par
  &Optional key -\/ hides the \hyperlink{RC_mhttpd_status_page_features_RC_mhttpd_status_RC_buttons}{Run Control buttons} on the mhttpd Main Status Page if set to \char`\"{}y\char`\"{}  (since \hyperlink{NDF_ndf_nov_2009}{Nov 2009} ).   

\\\cline{1-4}
Edit on Start\par
  &\par
  &DIR\par
  &This subdirectory may contain user-\/defined \hyperlink{structparameters}{parameters} which will be displayed for editing at the beginning of each run. See \hyperlink{RC_customize_ODB_RC_Edit_On_Start}{Defining Edit-\/on-\/start Parameters} for details.  

\\\cline{1-4}
\par
  &Edit run number  &BOOL  &Optional key. If present and set to \char`\"{}y\char`\"{} will \hyperlink{RC_mhttpd_Start_page_RC_Prevent_Edit_RN}{Prevent the run number being edited at Run Start} . Active for {\bfseries mhttpd} only (ignored by {\bfseries odbedit})  

\\\cline{1-4}
\par
  &\par
  &\par
 

&Any number of user-\/defined keys or links may be defined (i.e. the edit-\/on-\/start \hyperlink{structparameters}{parameters}). See example \hyperlink{RC_customize_ODB_RC_Experiment_Tree_Example}{below}  

\\\cline{1-4}
Lock when running\par
  &\par
  &DIR\par
  &An optional subdirectory that may be created by the user  

\\\cline{1-4}
\par
  &\par
  &\par
 

&may contain user-\/defined links to keys that are to be write-\/protected when a run is in progress. See \hyperlink{RC_customize_ODB_RC_Lock_when_Running}{Lock when Running} for details. 

\\\cline{1-4}
Parameter Comments\par
  &\par
  &DIR\par
  &Optional subdirectory may be created by the user.\par
  

\\\cline{1-4}
\par
  &\par
  &\par
  &Optional keys may be created by the user to contain parameter comments associated with the \hyperlink{RC_customize_ODB_RC_Edit_On_Start}{Edit on Start} Parameters. See \hyperlink{RC_mhttpd_Start_page_RC_Edit_PC}{Edit-\/on-\/start Parameter Comments} for details. Active for {\bfseries mhttpd} only (ignored by {\bfseries odbedit})   \\\cline{1-4}
Security\par
  &\par
  &DIR\par
  &subdirectory for user to set up security features. See \hyperlink{RC_customize_ODB_RC_Access_Control}{Security} for details   \\\cline{1-4}
\end{TabularC}
\centering
\caption{Above: Explanation of keys in /Experiment ODB tree}
\end{table}
\hypertarget{RC_customize_ODB_RC_Experiment_Tree_Example}{}\paragraph{Example of an  ODB /Experiment tree with optional subdirectories}\label{RC_customize_ODB_RC_Experiment_Tree_Example}
An example of a typical /Experiment tree containing various optional keys is shown below.


\begin{DoxyCode}
[local:midas:S]/>ls -rlt /experiment
Key name                        Type    #Val  Size  Last Opn Mode Value
---------------------------------------------------------------------------
Experiment                      DIR
    Name                        STRING  1     32    14s  0   RWD  midas
    Hide Run Buttons            BOOL    1     4     3h   0   RWD  n
    Buffer sizes                DIR
        SYSMSG                  DWORD   1     4     56h  0   RWD  100000
        SYSTEM                  DWORD   1     4     56h  0   RWD  8388608
    Menu Buttons                STRING  1     256   25h  0   RWD  Start, ODB, Mes
      sages, ELog, Alarms, Programs, History, MSCB, Config
    Run Parameters              DIR
        Comment                 STRING  1     32    3m   0   RWD   no beam, test 
      only
        Run Description         STRING  1     32    3m   0   RWD  28.2keV resonan
      t energy 7Li
        Sample                  STRING  1     15    3h   0   RWD  NA
    Edit on Start               DIR
        run_title -> /Experiment/run parameters/Run Description
                                STRING  1     32    3m   0   RWD  28.2keV resonan
      t energy 7Li
        run_comment -> /Experiment/run parameters/comment
                                STRING  1     32    3m   0   RWD   no beam, test 
      only
        experiment number       DWORD   1     4     3h   0   RWD  12
        experimenter            STRING  1     32    3h   0   RWD  Douglas,Thomas,
      Minnie
        sample ->  /Experiment/run parameters/Sample
                                STRING  1     15    3h   0   RWD  NA
        orientation             STRING  1     15    3h   0   RWD
        temperature             STRING  1     15    3h   0   RWD
        field                   STRING  1     15    3h   0   RWD
        Element                 STRING  1     24    3h   0   RWD  li
        Mass                    INT     1     4     3h   0   RWD  7
        DC offset(V)            INT     1     4     3h   0   RWD  0
        Ion source (kV)         DOUBLE  1     8     3h   0   RWD  28
        Laser wavelength (nm)   DOUBLE  1     8     3h   0   RWD  14859.952
        write data -> /Logger/Channels/0/Settings/Active
                                BOOL    1     4     3h   0   RWD  y
        Number of scans -> /Equipment/FIFO_acq/sis mcs/hardware/num scans
                                INT     1     4     3h   0   RWD  10
        Source HV Bias          STRING  1     12    3h   0   RWD  ISAC-WEST
        Edit run number         BOOL    1     4     3h   0   RWD  n
        Pedestals run           BOOL    1     4     3h   0   RWD  n
    Lock when running           DIR
        dis_rn_check -> /Equipment/FIFO_acq/mdarc/disable run number check
                                BOOL    1     4     56h  0   RWD  n
        SIS test mode -> /Equipment/FIFO_acq/sis mcs/sis test mode
                                KEY     1     12    >99d 0   RWD  <subdirectory>
        PPGinput -> /Equipment/FIFO_acq/sis mcs/Input
                                KEY     1     12    >99d 0   RWD  <subdirectory>
        SIS ref A -> /Equipment/FIFO_acq/sis mcs/Hardware/Enable SIS ref ch1 
      scaler A
                                BOOL    1     4     56h  0   RWD  n
        SIS ref B -> /Equipment/FIFO_acq/sis mcs/Hardware/Enable SIS ref ch1 
      scaler B
                                BOOL    1     4     56h  0   RWD  y
    Parameter Comments          DIR
        Active                  STRING  1     36    56h  0   RWD  <i>Enter y to s
      ave data to disk</i>
        Num scans               STRING 1     80    56h  0   RWD  <i>Stop run afte
      r num scans is reached. Enter 0 to disable (free running)</i>
        Num cycles              STRING  1     80    56h  0   RWD  <i>Stop run aft
      er num cycles is reached. Enter 0 to disable (freerunning)</i>
        Source HV Bias          STRING  1     80    56h  0   RWD  <i>Enter one of
       'OLIS', 'ISAC-WEST' or 'ISAC-EAST' </i>
    Transition debug flag       INT     1     4     21h  0   RWD  0
    Transition connect timeout  INT     1     4     21h  0   RWD  10000
    Transition timeout          INT     1     4     21h  0   RWD  120000
    security                    DIR
        Web Password            STRING  1     32    15h  0   RWD  mim2Q41CV.GW3
\end{DoxyCode}


The meaning of these keys is described \hyperlink{RC_customize_ODB_RC_Experiment_tree_keys}{above} and in the following sections.

\label{RC_customize_ODB_idx_ODB_tree_Experiment_Customize}
\hypertarget{RC_customize_ODB_idx_ODB_tree_Experiment_Customize}{}
 \label{RC_customize_ODB_idx_ODB_tree_Experiment_Parameter-comments}
\hypertarget{RC_customize_ODB_idx_ODB_tree_Experiment_Parameter-comments}{}
\hypertarget{RC_customize_ODB_RC_customize_experiment_tree}{}\paragraph{Customizing parameters under the ODB /Experiment tree}\label{RC_customize_ODB_RC_customize_experiment_tree}
The user can optionally create the following subdirectories under the /Experiment tree if they do not exist already. These must have the names {\bfseries 
\begin{DoxyItemize}
\item /Experiment/
\begin{DoxyItemize}
\item \hyperlink{RC_customize_ODB_RC_Edit_On_Start}{Edit on Start}
\item \hyperlink{RC_customize_ODB_RC_parameter_comments}{Parameter Comments}
\item \hyperlink{RC_customize_ODB_RC_Run_Parameters}{Run Parameters}
\item \hyperlink{RC_customize_ODB_RC_Lock_when_Running}{Lock when running}
\item \hyperlink{RC_customize_ODB_RC_Access_Control}{Security}
\end{DoxyItemize}
\end{DoxyItemize}}

{\bfseries } These keynames have a particular meaning for the MIDAS system which will be described below.

\par


\par
 \label{RC_customize_ODB_idx_run_start_parameters}
\hypertarget{RC_customize_ODB_idx_run_start_parameters}{}
\hypertarget{RC_customize_ODB_RC_Edit_On_Start}{}\subparagraph{Defining Edit-\/on-\/start Parameters}\label{RC_customize_ODB_RC_Edit_On_Start}
With {\bfseries no} optional \char`\"{}edit-\/on-\/start\char`\"{} \hyperlink{structparameters}{parameters} set up, when a run is started,
\begin{DoxyItemize}
\item either by the \hyperlink{RC_odbedit_utility}{odbedit} command \hyperlink{RC_odbedit_examples_RC_odbedit_start}{start}
\item or by clicking the \hyperlink{RC_mhttpd_utility}{mhttpd} start button (see \hyperlink{RC_mhttpd_status_page_features_RC_mhttpd_status_menu_buttons}{menu buttons} ) on the mhttpd main status page
\end{DoxyItemize}

a {\bfseries  TR\_\-START transition } is received, and the {\bfseries run number} of the upcoming run will be displayed for editing. It will have been {\bfseries automatically incremented} relative to the last run.

The following example shows a run started using  odbedit. See (see \hyperlink{RC_mhttpd_Start_page}{Start page}) ,

The default run number of the next run is 30499. The user has changed this to 500. 
\begin{DoxyCode}
[local:bnmr:S]/Experiment>start
Run number [30499]: 500
Are the above parameters correct? ([y]/n/q): 
\end{DoxyCode}


\label{RC_customize_ODB_idx_edit-on-start}
\hypertarget{RC_customize_ODB_idx_edit-on-start}{}
 \label{RC_customize_ODB_idx_ODB_tree_Experiment_Edit-on-Start-parameters}
\hypertarget{RC_customize_ODB_idx_ODB_tree_Experiment_Edit-on-Start-parameters}{}


It is often convenient to {\bfseries display and edit additional \hyperlink{structparameters}{parameters} } at this time. These \hyperlink{structparameters}{parameters} are known as {\bfseries Edit-\/on-\/start} \hyperlink{structparameters}{parameters} since they automatically appear every time a run starts, and they are editable by the user. {\bfseries Edit-\/on-\/start} \hyperlink{structparameters}{parameters} are defined by creating them or linking to them in a special subdirectory named \char`\"{}Edit on start\char`\"{} that the user may create in the ODB \hyperlink{RC_customize_ODB_RC_ODB_Experiment_Tree}{/Experiment tree}. Links can point to any ODB key including the logger settings. It is often convenient to create a link to the logger setting which enables/disables writing of data. A quick test run can then be made without data logging, for example: 
\begin{DoxyCode}
[local:bnmr:S]/Experiment>start
Write data : n
Run number [30499]:
Are the above parameters correct? ([y]/n/q):  
\end{DoxyCode}


The first step to setting up the {\bfseries Edit on Start} \hyperlink{structparameters}{parameters} is to {\bfseries create the subdirectory \char`\"{}Edit on Start\char`\"{}} in the ODB under {\bfseries /Experiment} as follows: 
\begin{DoxyCode}
$odbedit
[local:Default:S]/>cd /experiment
[local:Default:S]/Experiment>mkdir "Edit on start"
[local:Default:S]/Experiment>cd "Edit on start"
[local:Default:S]/Edit on start>
\end{DoxyCode}


\label{RC_customize_ODB_RC_rp}
\hypertarget{RC_customize_ODB_RC_rp}{}
 Then the user either {\bfseries creates the required \hyperlink{structparameters}{parameters}}, or, if the \hyperlink{structparameters}{parameters} already exist elsewhere in the ODB, {\bfseries creates links to the \hyperlink{structparameters}{parameters}} in the Edit on start subdirectory. Many users find it convenient to create their run \hyperlink{structparameters}{parameters} in a subdirectory of /Experiment named {\bfseries \char`\"{}Run Parameters\char`\"{}}, and create links to them in the Edit on start subdirectory.

\par
\par
The example below shows the creation of three \hyperlink{structparameters}{parameters} in the Edit on start subdirectory: 
\begin{DoxyEnumerate}
\item a {\bfseries parameter} to contain the title of the run (called \char`\"{}run\_\-title\char`\"{}) \par
 The user will be able to enter the title of each run before it starts 
\item a {\bfseries link} to the ODB parameter /Logger/Write data \par
 It can make sense to create a link to the logger setting which enables/disables writing of data. A quick test run can then be made without data logging 
\item a {\bfseries link} to the ODB parameter /Equipment/FIFO\_\-acq/hardware/num scans which has been previously created by the user 
\item a {\bfseries link} to the ODB parameter /Equipment/Run Parameters/Sample which has been previously created by the user 
\end{DoxyEnumerate}


\begin{DoxyCode}
[local:Default:S]/Edit on start>
[local:Default:S]/Edit on start>create string run_title
String length [32]:128
[local:Default:S]/Edit on start>ln "/Logger/Write data" "write data"
[local:Default:S]/Edit on start>ln "/Equipment/FIFO_acq/hardware/num scans" "Numb
      er of scans"
[local:Default:S]/Edit on start>ln "/Experiment/Run Parameters/Sample" "sample"
\end{DoxyCode}


Here is an example of the {\bfseries Edit on start} \hyperlink{structparameters}{parameters} from an experiment:


\begin{DoxyCode}
[local:Default:S]Edit on start>ls -lt
Key name                        Type    #Val  Size  Last Opn Mode Value
---------------------------------------------------------------------------
run_title                       STRING  1     128   3h   0   RWD  2e test
experiment number               DWORD   1     4     3h   0   RWD  9999
experimenter                    STRING  1     32    3h   0   RWD  gdm
sample                          STRING  1     23    3h   0   RWD  /Experiment/Run
       Parameters/Sample -> NA
orientation                     STRING  1     15    11h  0   RWD  
temperature                     STRING  1     32    3h   0   RWD  285.12K
field                           STRING  1     32    3h   0   RWD  0G
Number of scans                 INT     1     4     11h  0   RWD  /Equipment/FIFO
      _acq/hardware/num scans -> 0
write data                      BOOL    1     4     7s   0   RWD  /Logger/Write d
      ata -> y
Edit run number                 BOOL    1     4     11h  0   RWD  n
\end{DoxyCode}


that the \char`\"{}Edit run number\char`\"{} feature (that prevents editing the run number) is available for mhttpd only. See \hyperlink{RC_mhttpd_Start_page_RC_Prevent_Edit_RN}{Prevent the run number being edited at Run Start} .

{\bfseries Examples} of starting a run with {\bfseries  edit on start } \hyperlink{structparameters}{parameters} are shown here:
\begin{DoxyItemize}
\item using \hyperlink{RC_odbedit_examples_RC_EOS_example2}{odbedit} or
\item using \hyperlink{RC_mhttpd_Start_page_RC_mhttpd_Edit_On_Start}{mhttpd} .
\end{DoxyItemize}\hypertarget{RC_customize_ODB_RC_parameter_comments}{}\subparagraph{Parameter Comments subdirectory}\label{RC_customize_ODB_RC_parameter_comments}
Optional parameter comments (in the /Experiment/Parameter Comments subdirectory) can be set up to give more information about the Edit-\/on-\/Start \hyperlink{structparameters}{parameters}. This feature is only active when using mhttpd. Setting up \hyperlink{RC_mhttpd_Start_page_RC_Edit_PC}{Edit-\/on-\/start Parameter Comments} is described in the mhttpd \hyperlink{RC_mhttpd_Start_page}{Start page} section. \par


\par
\hypertarget{RC_customize_ODB_RC_Run_Parameters}{}\subparagraph{Run Parameters subdirectory}\label{RC_customize_ODB_RC_Run_Parameters}
Users often create a subdirectory /Experiment/Run Parameters to contain any \hyperlink{structparameters}{parameters} needed for the run. These may be the subject of links from the Edit on start subdirectory (see \hyperlink{RC_customize_ODB_RC_rp}{example above}). \par
 In the case of mhttpd, two keys in this directory have special meaning. If the keys {\bfseries \char`\"{}Comment\char`\"{}} or {\bfseries \char`\"{}Run Description\char`\"{}} are created in the \char`\"{}Run Parameter\char`\"{} subdirectory, the contents of each key will be displayed on an extra line on the mhttpd main status page. See \hyperlink{RC_mhttpd_status_page_features_RC_Edit_RP}{Comment and Run Description} for details. \par


\par


\label{RC_customize_ODB_idx_ODB_tree_Experiment_Lock-when-Running}
\hypertarget{RC_customize_ODB_idx_ODB_tree_Experiment_Lock-when-Running}{}
 \hypertarget{RC_customize_ODB_RC_Lock_when_Running}{}\subparagraph{Lock when Running}\label{RC_customize_ODB_RC_Lock_when_Running}
Often it is desirable that various experimental ODB \hyperlink{structparameters}{parameters} should not to be changed when a run is in progress, i.e. that they are set to a \char`\"{}read-\/only\char`\"{} mode while running. This can be done by creating logical links to these ODB keys in the optional directory {\bfseries  \char`\"{}Lock when Running\char`\"{} } in the \hyperlink{RC_customize_ODB_RC_ODB_Experiment_Tree}{The ODB /Experiment tree}. \par
 In the example below, all the \hyperlink{structparameters}{parameters} under the declared tree will be switched to read-\/only, thus preventing any modification of these \hyperlink{structparameters}{parameters} during the run. 
\begin{DoxyCode}
  [local]/>create key "/Experiment/Lock when running"
  [local]/>cd "/Experiment/Lock when running"
  [local]/>ln "/Experiment/Run parameters" "Run parameter"
  [local]/>ln "/Logger/Write Data" "Write Data?"
\end{DoxyCode}


In the following example, the user attempts to change one of the read-\/only \hyperlink{structparameters}{parameters}: 
\begin{DoxyCode}
[local:bnmr:R]/>set "/Logger/Write Data" y
Write access not allowed
\end{DoxyCode}


\par


\par


\label{RC_customize_ODB_idx_access-control_ODB}
\hypertarget{RC_customize_ODB_idx_access-control_ODB}{}
 \label{RC_customize_ODB_idx_ODB-access-control}
\hypertarget{RC_customize_ODB_idx_ODB-access-control}{}
 \hypertarget{RC_customize_ODB_RC_Access_Control}{}\subparagraph{Access Control (Security) using the ODB}\label{RC_customize_ODB_RC_Access_Control}
\begin{DoxyNote}{Note}
To prevent access by determined or malicious hackers, a {\bfseries firewall} and/or {\bfseries restrictions on off-\/site access} should be implemented. This kind of security can be provided by setting up \hyperlink{RC_mhttpd_utility_RC_mhttpd_proxy}{Proxy Access to mhttpd} . \par

\end{DoxyNote}
By default, there is no restriction for any user to connect locally or remotely to a given experiment. MIDAS provides a means to setup access restrictions using the ODB in order to protect the experiment from accidental or unauthorized access.

There are two levels of access restriction available each of which can be enabled independently:
\begin{DoxyItemize}
\item To require a password before MIDAS clients can start running on the host.
\item To restrict write access via the web by requiring a password before any parameter can be changed.
\end{DoxyItemize}

The user can select {\bfseries either} or {\bfseries both} of these security features.

Note that other forms of ODB access control independent of these security features is also available:


\begin{DoxyItemize}
\item Write access can be restricted during a run (see \hyperlink{RC_customize_ODB_RC_Lock_when_Running}{Lock when Running} )
\item Individual keys or subdirectories in the experiment's ODB can be set \char`\"{}read only\char`\"{} with the odbedit command \hyperlink{RC_odbedit_examples_RC_odbedit_chmod}{chmod}.
\end{DoxyItemize}

\par


\par


\label{RC_customize_ODB_idx_ODB_tree_Experiment_Security}
\hypertarget{RC_customize_ODB_idx_ODB_tree_Experiment_Security}{}
\hypertarget{RC_customize_ODB_RC_Setup_Security}{}\subparagraph{How to Setup Client Access Restrictions}\label{RC_customize_ODB_RC_Setup_Security}
\label{RC_customize_ODB_idx_access-control_client}
\hypertarget{RC_customize_ODB_idx_access-control_client}{}
 In order to \hyperlink{RC_customize_ODB_RC_Access_Control}{restrict access} to the experiment, a password mechanism needs to be defined. This is provided by the \char`\"{}Security\char`\"{} subdirectory in odb. This subdirectory is automatically created (if not already present) when the \hyperlink{RC_odbedit_utility}{odbedit} command {\bfseries passwd} is issued as follows: 
\begin{DoxyCode}
  C:\online>odbedit
  [local:Default:S]/>cd Experiment/
  [local]/>passwd
  Password:<xxxx>
  Retype password:<xxxx>
\end{DoxyCode}
 After running the odb command \char`\"{}passwd\char`\"{}, three new sub-\/fields (odb keys) will be present under the /Experiment/Security subtree.

\begin{table}[h]\begin{TabularC}{4}
\hline
\multirow{1}{\linewidth}{Keys in ODB /Experiment/Security subtree\par
   }\\\cline{1-1}
Key\par
  &Type\par
  &Explanation\par
  

\\\cline{1-3}
Password  &\par
  &STRING\par
  &Contains the encrypted password. Key is created when \hyperlink{RC_odbedit_utility}{odbedit} command {\bfseries passwd} is issued.  

\\\cline{1-4}
Allowed hosts\par
  &\par
  &DIR  &This key is created when \hyperlink{RC_odbedit_utility}{odbedit} command {\bfseries passwd} is issued. Subdirectory may contain names of remote hosts allowed to have free access (i.e. without password) to the current experiment. See \hyperlink{RC_customize_ODB_RC_security_allowed_hosts}{Allowed Hosts} .  

\\\cline{1-4}
\par
  &pierre.triumf.ca  &\par
INT  &\par
Example -\/ name of a host allowed password-\/less access to experiment (key created by the user).  

\\\cline{1-4}
Allowed programs  &\par
  &DIR  &This key is created when \hyperlink{RC_odbedit_utility}{odbedit} command {\bfseries passwd} is issued. Subdirectory may contain names of clients allowed to have free access (i.e. without password) to the current experiment. See \hyperlink{RC_customize_ODB_RC_security_allowed_programs}{Allowed programs} .  

\\\cline{1-4}
\par
  &mstat  &\par
INT  &\par
Example -\/ name of a client (run from any host) allowed password-\/less access to the experiment (key created by the user).  

\\\cline{1-4}
Web Password  &\par
  &STRING\par
  &This key specifies a separate encrypted password for Web server access. Key is created when \hyperlink{RC_odbedit_utility}{odbedit} command {\bfseries webpasswd} is issued. See \hyperlink{RC_customize_ODB_RC_Setup_Web_Security}{Web Access restriction}.  

\\\cline{1-4}
\end{TabularC}
\centering
\caption{Above: Explanation of keys in ODB /Experiment/Security subtree}
\end{table}
\hypertarget{RC_customize_ODB_RC_security_allowed_hosts}{}\subparagraph{Allowed Hosts}\label{RC_customize_ODB_RC_security_allowed_hosts}
This key is a fixed directory name where names of remote hosts can be defined for free access to the current experiment. While the access restriction can make sense to deny access to outsider to a given experiment, it can be annoying for the people working directly at the back-\/end computer or for the automatic frontend reloading mechanism (MS-\/DOS, VxWorks configuration). To address this problem, specific hosts can be exempt from having to supply a password before being granted of full access.


\begin{DoxyCode}
  [local]/>cd "/Experiment/Security/Allowed hosts"
  [local]rhosts>create int myhost.domain
  [local]rhosts>
\end{DoxyCode}
 where $<$myHost.domain$>$ is to be replaced by the full IP address of the host requesting full clearance, e.g \char`\"{}pierre.triumf.ca\char`\"{}.\hypertarget{RC_customize_ODB_RC_security_allowed_programs}{}\subparagraph{Allowed programs}\label{RC_customize_ODB_RC_security_allowed_programs}
This key is a fixed directory name where a list of programs can be defined that have full access to the ODB {\itshape regardless of the node they are running on.\/} 
\begin{DoxyCode}
  [local]/>cd "/Experiment/Security/Allowed programs"
  [local]:S>create int mstat
  [local]:S>
\end{DoxyCode}
 \par
\par


\par


\par
 \label{RC_customize_ODB_idx_ODB_tree_Experiment_Security_Restrict-Web-access}
\hypertarget{RC_customize_ODB_idx_ODB_tree_Experiment_Security_Restrict-Web-access}{}
 \hypertarget{RC_customize_ODB_RC_Setup_Web_Security}{}\subparagraph{How to Setup Web Access Restriction}\label{RC_customize_ODB_RC_Setup_Web_Security}
\label{RC_customize_ODB_idx_access-control_web}
\hypertarget{RC_customize_ODB_idx_access-control_web}{}


This section is only applicable to access using mhttpd (i.e. web access). Access with odbedit is unaffected.

The ODB /Experiment/Security subtree can also be used to \hyperlink{RC_customize_ODB_RC_Access_Control}{restrict access} to the experiment via the Web. This subtree is automatically created (if not already present) when the odbedit command {\bfseries webpasswd} is issued as follows: 
\begin{DoxyCode}
  C:\online>odbedit
  [local:Default:S]/>cd Experiment/
  [local]/>webpasswd
  Password:<xxxx>
  Retype password:<xxxx>
\end{DoxyCode}


After running {\bfseries webpasswd}, one new sub-\/field (odb key) i.e. {\bfseries \char`\"{}Web Password\char`\"{}} will be present under the Security tree. If, of course, \hyperlink{RC_customize_ODB_RC_Setup_Security}{client security} is ALSO enabled, there will now be a total of four keys present.\hypertarget{RC_customize_ODB_RC_security_web_pw}{}\subparagraph{Web Password}\label{RC_customize_ODB_RC_security_web_pw}
This key specifies a separate password for the Web server access via \hyperlink{RC_mhttpd_utility}{mhttpd}. If this field is active, the user will be requested to provide the \char`\"{}Web Password\char`\"{} when accessing the requested experiment in \char`\"{}Write Access\char`\"{} mode (see \hyperlink{RC_mhttpd_ODB_page}{example}). The \char`\"{}Read Only Access\char`\"{} mode is still available to all users. 
\begin{DoxyCode}
[local:bnqr:S]/Experiment>ls Security/
Web Password                    pon4@#@%SSDF2
\end{DoxyCode}




\hypertarget{RC_customize_ODB_RC_Example_security}{}\subparagraph{Examples of Access Control using the ODB Security subtree}\label{RC_customize_ODB_RC_Example_security}
The following examples illustrate \hyperlink{RC_customize_ODB_RC_Access_Control}{access control} using the ODB Security features:\hypertarget{RC_customize_ODB_RC_Example_full_security}{}\subparagraph{Example of Full access control  setup for an experiment}\label{RC_customize_ODB_RC_Example_full_security}
The following example shows the odb when {\bfseries both client and web security} have been setup. If client security {\bfseries only} is enabled, the key \char`\"{}Web Password\char`\"{} would not be present.


\begin{DoxyCode}
Key name                        Type    #Val  Size  Last Opn Mode Value
---------------------------------------------------------------------------
Experiment                      DIR
    Security                    DIR
        Password                STRING  1     32    16h  0   RWD  #@D&%F56
        Allowed hosts           DIR
            host.sample.domain  INT     1     4     >99d 0   RWD  0
            pierre.triumf.ca    INT     1     4     >99d 0   RWD  0
            pcch02.triumf.ca    INT     1     4     >99d 0   RWD  0
            koslx1.triumf.ca    INT     1     4     >99d 0   RWD  0
            koslx2.triumf.ca    INT     1     4     >99d 0   RWD  0
            vwchaos.triumf.ca   INT     1     4     >99d 0   RWD  0
            koslx0.triumf.ca    INT     1     4     >99d 0   RWD  0
        Allowed programs        DIR
            mstat               INT     1     4     >99d 0   RWD  0
            fechaos             INT     1     4     >99d 0   RWD  0
        Web Password            STRING  1     32    16h  0   RWD  pon4@#@%SSDF2
\end{DoxyCode}


\par


\par
\hypertarget{RC_customize_ODB_RC_Example_web_security}{}\subparagraph{Example of Web-\/Only access control setup for an experiment}\label{RC_customize_ODB_RC_Example_web_security}
The following example shows the odb when {\bfseries web security only} has been enabled. See also \hyperlink{RC_customize_ODB_RC_Example_full_security}{Example of Full access control setup for an experiment} 
\begin{DoxyCode}
Key name                        Type    #Val  Size  Last Opn Mode Value
---------------------------------------------------------------------------
Experiment                      DIR
    Security                    DIR
        Web Password            STRING  1     32    16h  0   RWD  pon4@#@%SSDF2
\end{DoxyCode}




 \label{RC_customize_ODB_idx_ODB_tree_Experiment_Security_Remove-Access-restrictions}
\hypertarget{RC_customize_ODB_idx_ODB_tree_Experiment_Security_Remove-Access-restrictions}{}
 \hypertarget{RC_customize_ODB_RC_Remove_Security}{}\subparagraph{To Remove Access Restrictions}\label{RC_customize_ODB_RC_Remove_Security}
\label{RC_customize_ODB_idx_access-control_remove}
\hypertarget{RC_customize_ODB_idx_access-control_remove}{}
 To remove the full password checking mechanism completely, the ODB security sub-\/tree has to be entirely deleted using the following command: 
\begin{DoxyCode}
  [local]/>rm /Experiment/Security
  Are you sure to delete the key
  "/Experiment/Security"
  and all its subkeys? (y/[n]) y
\end{DoxyCode}


To partially remove access restrictions, remove only those keys relevent to web or client security. i.e. to remove web access restriction only, remove the key \char`\"{}Web Password\char`\"{}


\begin{DoxyCode}
  [local]/>cd /Experiment/Security
[local:bnmr:S]Security>ls
Password                        #@D&%F56
Allowed hosts
Allowed programs
Web Password                    pon4@#@%SSDF2

[local:bnqr:S]Security>rm "Web Password"
Are you sure to delete the key
"/Experiment/Security/Web Password"
(y/[n]) y
\end{DoxyCode}


Client security is retained: 
\begin{DoxyCode}
[local:bnmr:S]Security>ls
Password                      #@D&%F56  
Allowed hosts
Allowed programs
\end{DoxyCode}


Alternatively, to retain web security only, delete the keys Password, Allowed hosts and Allowed Programs 
\begin{DoxyCode}
  [local]/>cd /Experiment/Security
[local:bnmr:S]Security>ls
Password                        #@D&%F56
Allowed hosts
Allowed programs
Web Password                    pon4@#@%SSDF2

[local:bnqr:S]Security>rm "Password"
[local:bnqr:S]Security>rm "Allowed Hosts"
[local:bnqr:S]Security>rm "Allowed programs"
\encode

So that the only key remaining is the Web Password key:
\code
[local:bnmr:S]Security>ls
Web Password                    pon4@#@%SSDF2
\end{DoxyCode}


\par
 

 \par
\hypertarget{RC_customize_ODB_RC_starting_clients}{}\subsubsection{Customize the scripts that start up and shut down the clients.}\label{RC_customize_ODB_RC_starting_clients}
\label{RC_customize_ODB_idx_startup_script}
\hypertarget{RC_customize_ODB_idx_startup_script}{}
 \label{RC_customize_ODB_idx_shutdown_script}
\hypertarget{RC_customize_ODB_idx_shutdown_script}{}
 Before a run can be started, all the clients necessary to the experiment must be started. We have already seen in \hyperlink{Quickstart}{SECTION 3: Quick Start} that templates of scripts to \hyperlink{RC_customize_ODB_start-all}{start and kill} the required clients are provided in the MIDAS package. These must be customized for your experiment. The following example shows the template start-\/up script {\bfseries start\_\-daq.sh} that starts
\begin{DoxyItemize}
\item a frontend
\item an analyzer,
\item \hyperlink{F_Logging_F_mlogger_utility}{MIDAS logger} to save the data
\item the run-\/control/monitoring program \hyperlink{RC_mhttpd_utility}{mhttpd} and the template kill script {\bfseries kill\_\-daq.sh} that shuts then all down.
\end{DoxyItemize}

\begin{table}[h]\begin{TabularC}{2}
\hline
start\_\-daq.sh &kill\_\-daq.sh \\\cline{1-2}

\begin{DoxyCode}
#!/bin/sh

. setup.sh


./kill_daq.sh

odbedit -c clean

mhttpd -p 8081 -D
sleep 2
xterm -e ./frontend &
xterm -e ./analyzer &
mlogger -D


echo Please point your web browser to http://localhost:8081
echo Or run: firefox http://localhost:8081 &
echo To look at live histograms, run: roody -Hlocalhost

#end file
\end{DoxyCode}
  &\par
 
\begin{DoxyCode}
#!/bin/sh

killall mlogger
killall mhttpd
killall frontend
killall analyzer
sleep 1

#end file
\end{DoxyCode}
   \\\cline{1-2}
\end{TabularC}
\centering
\caption{Above: Template scripts to start and kill the clients }
\end{table}
\hypertarget{RC_customize_ODB_RC_start_all_example}{}\paragraph{Example of a start-\/all script}\label{RC_customize_ODB_RC_start_all_example}
The following is a start-\/all script from an experiment at TRIUMF. It can be run at any time to restart any clients that have stopped. Note that it also starts \hyperlink{RC_customize_ODB_RC_mserver_utility}{mserver}, because the frontend is run on a remote host.


\begin{DoxyCode}
#!/bin/sh
# Host based
if [ $HOST == "lxebit.triumf.ca" ]; then
 echo "run start-all from titan04 only "
else
#. setup.sh


odbedit -c clean
#odbedit -c "rm /Analyzer/Trigger/Statistics"
#odbedit -c "rm /Analyzer/Scaler/Statistics"


ps -ef > ~/temp
grep --silent "mserver" ~/temp
if [ "$?" != "0" ];  then
    echo "Starting mserver "
    $MIDASSYS/linux/bin/mserver -D
else
 echo "mserver is already running"
fi
rm ~/temp

# Start the http MIDAS server
ps -ef > ~/temp
grep --silent "mhttpd -p 8089" ~/temp
if [ "$?" != "0" ] ; then
    echo "Starting mhttpd"
     $MIDASSYS/linux/bin/mhttpd -p 8089 -D
  else
 echo mhttpd is already running
fi
rm ~/temp

sleep 2
#xterm -e ./frontend &
#xterm -e ./analyzer &

# The  MIDAS logger
# Start the logger
$MIDASSYS/linux/bin/odbedit -c scl | grep --silent  Logger
if [ "$?" != "0" ] ; then
    echo "Starting mlogger"
    $MIDASSYS/linux/bin/mlogger  -D
  else
 echo mlogger is already running
fi


# start the feebit program via remote login on lxebit
ssh lxebit ~/online/bin/start_feebit_ppg


echo Please point your web browser to http://localhost:8089
#echo Or run: mozilla http://localhost:8089 &
#echo To look at live histograms, run: roody -Hlocalhost
fi
#end file
\end{DoxyCode}
\hypertarget{RC_customize_ODB_RC_kill_all_example}{}\paragraph{Example of a kill-\/all script}\label{RC_customize_ODB_RC_kill_all_example}
This is the kill-\/all script for the ebit experiment at TRIUMF.


\begin{DoxyCode}
#!/bin/sh
# Host based
if [ $HOST == "lxebit.triumf.ca" ] ; then
 echo "run kill-all from titan04 only "
else
  killall mserver
  killall mlogger
  killall mhttpd
  killall feebit_ppg
  killall analyzer
  sleep 1
fi
#end file
\end{DoxyCode}


\par


\par
 \label{RC_customize_ODB_idx_access-control_remote}
\hypertarget{RC_customize_ODB_idx_access-control_remote}{}
 \label{RC_customize_ODB_idx_mserver-utility}
\hypertarget{RC_customize_ODB_idx_mserver-utility}{}
 \hypertarget{RC_customize_ODB_RC_mserver_utility}{}\subsubsection{mserver      -\/ MIDAS Remote server}\label{RC_customize_ODB_RC_mserver_utility}
mserver provides remote access to any MIDAS client. It is needed when {\bfseries one or more of the MIDAS clients for an experiment are running on a different host}. In this case, an mserver client must be started on the host where the experiment resides.

For example, if there is no mserver client is running on host dasdevpc2, then an attempt to run a client on dasdevpc2 from a remote computer (isdaq01) will result in an error message: 
\begin{DoxyCode}
[bnmr@isdaq01 ~/online]$ odb -e t2kgas -h dasdevpc2
Cannot connect to remote host
\end{DoxyCode}
 If an mserver client is now started on host dasdevpc2, 
\begin{DoxyCode}
[suz@dasdevpc2 ~]$ mserver -D
mserver started interactively
Becoming a daemon...
\end{DoxyCode}
 Now one can connect to the remote experiment : 
\begin{DoxyCode}
[bnmr@isdaq01 ~/online]$ odb -e t2kgas -h dasdevpc2
[dasdevpc2:t2kgas:S]/>quit
\end{DoxyCode}


The mserver utility usually runs in the background and doesn't need to be modified. In the case where debugging is required, the mserver can be started with the -\/d flag which will write an entry for each transaction to a log file {\itshape  /tmp/mserver.log \/} . The log entry contains the time stamp and RPC call request. \par


More than one mserver can be started on a system, provided they use different tcp ports. This is useful if, for example, different versions of MIDAS are in use on a single host at the same time. To start a version of mserver on a different port, use the -\/p argument, e.g. mserver -\/p XXXX -\/D \par
 To connect a client to this version of mserver, use the format \char`\"{}hostname:port\char`\"{}, e.g. 
\begin{DoxyCode}
   fe_test -h lin08:7066 -e expt
\end{DoxyCode}
 \par
\hypertarget{RC_customize_ODB_RC_mserver_arguments}{}\paragraph{mserver arguments}\label{RC_customize_ODB_RC_mserver_arguments}

\begin{DoxyItemize}
\item {\bfseries  Arguments }
\begin{DoxyItemize}
\item \mbox{[}-\/h \mbox{]} : help
\item \mbox{[}-\/s \mbox{]} : Single process server
\item \mbox{[}-\/t \mbox{]} : Multi thread server
\item \mbox{[}-\/m \mbox{]} : Multi process server (default)
\item \mbox{[}-\/p \mbox{]} : Port number; listen for connections on non-\/default tcp port
\item \mbox{[}-\/d \mbox{]} : Write debug info to /tmp/mserver.log
\item \mbox{[}-\/D \mbox{]} : Become a Daemon
\end{DoxyItemize}
\end{DoxyItemize}

\label{RC_customize_ODB_idx_ODB_tree_Programs}
\hypertarget{RC_customize_ODB_idx_ODB_tree_Programs}{}
\hypertarget{RC_customize_ODB_RC_ODB_Programs_Tree}{}\subsubsection{The ODB /Programs tree}\label{RC_customize_ODB_RC_ODB_Programs_Tree}
The ODB {\bfseries /Programs} tree is created by the system. It contains


\begin{DoxyItemize}
\item key {\bfseries Execute on start run}
\item key {\bfseries Execute on stop run}
\item a {\bfseries subdirectory for each client} that runs on the experiment.
\end{DoxyItemize}

The subdirectory is created by the system the first time a client runs. The following is an example of the {\bfseries /Programs} tree from an experiment:


\begin{DoxyCode}
[local:bnmr:S]/>ls -lt /programs
Key name                        Type    #Val  Size  Last Opn Mode Value
---------------------------------------------------------------------------
Execute on start run            STRING  1     256   18h  0   RWD  /home/bnmr/onli
      ne/bnmr/bin/at_start_run.csh
Execute on stop run             STRING  1     256   18h  0   RWD  /home/bnmr/onli
      ne/bnmr/bin/at_end_run.csh
ODBEdit                         DIR
Logger                          DIR
Epics                           DIR
rf_config                       DIR
mheader                         DIR
Mdarc                           DIR
autorun                         DIR
feBNMR                          DIR
camplog                         DIR
Lcrplot                         DIR
mhttpd                          DIR
mdump                           DIR
Speaker                         DIR
mdarc_cleanup                   DIR
\end{DoxyCode}
\hypertarget{RC_customize_ODB_RC_ODB_programs_client}{}\paragraph{The ODB /Programs/$<$client$>$ subtree}\label{RC_customize_ODB_RC_ODB_programs_client}
The subdirectory for each client contains system information as well as task-\/specific characteristics, such as the watchdog timeout, optionally a command to restart the task, the optional \hyperlink{RC_customize_ODB_RC_Alarm_System}{alarm condition} etc. The following example shows the subdirectory for the \hyperlink{F_Logging_F_mlogger_utility}{mlogger} client:


\begin{DoxyCode}
[local:bnmr:S]/>ls -lt /programs/logger
Key name                        Type    #Val  Size  Last Opn Mode Value
---------------------------------------------------------------------------
Required                        BOOL    1     4     4h   0   RWD  y
Watchdog timeout                INT     1     4     4h   0   RWD  10000
Check interval                  DWORD   1     4     4h   0   RWD  180000
Start command                   STRING  1     256   4h   0   RWD  mlogger -D
Auto start                      BOOL    1     4     4h   0   RWD  n
Auto stop                       BOOL    1     4     4h   0   RWD  n
Auto restart                    BOOL    1     4     4h   0   RWD  y
Alarm class                     STRING  1     32    4h   0   RWD  Caution
First failed                    DWORD   1     4     4h   0   RWD  1259294464
\end{DoxyCode}


The fields of the /Programs tree are explained below.\hypertarget{RC_customize_ODB_RC_customize_Programs_tree}{}\paragraph{Customize the ODB /Programs tree}\label{RC_customize_ODB_RC_customize_Programs_tree}
The \hyperlink{structparameters}{parameters} of each client in the {\bfseries /Programs} tree should be customized for the experiment. The meaning of the fields is explained below. This may involve adjusting the watchdog timer, turning on an alarm if the client dies, starting the client automatically using a supplied start command etc. Customizing is done individually for each client.

The fields {\bfseries Execute on start run} and {\bfseries Execute on stop run} may be filled by the user with a command to be executed on the appropriate run transition, for example
\begin{DoxyItemize}
\item an \hyperlink{RC_odbedit_utility}{odbedit} command, to set or clear an odb parameter e.g \par
 {\bfseries  odb -\/c 'set \char`\"{}/Equipment/fifo\_\-acq/client flags/client alarm\char`\"{} 0' }
\item or the {\bfseries  name and path of a script} as shown in the example above.
\end{DoxyItemize}

\begin{table}[h]\begin{TabularC}{5}
\hline
Keys in the ODB tree /Programs   \\\cline{1-1}
ODB Key  &Explanation  

\\\cline{1-2}
Programs  &\par
 &\par
 &\par
 &\par
 

\\\cline{1-5}
\par
 &Execute on start run  &\par
 &STRING &Contains optional command or script to be executed on START transition.  

\\\cline{1-5}
\par
 &Execute on stop run  &\par
 &STRING &Contains optional command or script to be executed on STOP transition (see \hyperlink{RC_odbedit_examples_RC_example_script_1}{example})  

\\\cline{1-5}
\par
 &Logger &\par
 &STRING &Name of client. There will be a subdirectory created for each client named with the client-\/name. This example shows the subdirectory for the client \char`\"{}Logger\char`\"{}.  

\\\cline{1-5}
\par
 &\par
 &\label{RC_customize_ODB_RC_programs_Required}
\hypertarget{RC_customize_ODB_RC_programs_Required}{}
 Required &BOOL &If set to \char`\"{}y\char`\"{}, a run will be prevented from starting if this client is not running. Set to \char`\"{}y\char`\"{} for essential clients only. It should be combined with setting \hyperlink{RC_customize_ODB_RC_programs_Auto_start}{Auto start} and/or \hyperlink{RC_customize_ODB_RC_programs_Auto_restart}{Auto restart} to \char`\"{}y\char`\"{}, and supplying a \hyperlink{RC_customize_ODB_RC_programs_Start_command}{Start command}. If set to \char`\"{}n\char`\"{}, the run will start successfully without this client running. Note that this field also changes the display in the  \hyperlink{RC_mhttpd_utility}{mhttpd} \hyperlink{RC_mhttpd_Program_page}{Programs page}.  

\\\cline{1-5}
\par
 &\par
 &\label{RC_customize_ODB_RC_programs_Watchdog_timeout}
\hypertarget{RC_customize_ODB_RC_programs_Watchdog_timeout}{}
 Watchdog timeout &INT &This value is the watchdog timeout set in milliseconds. A watchdog runs automatically checking (every \hyperlink{RC_customize_ODB_RC_programs_Check_interval}{Check interval} ms) whether the client responds. If the client has not been responded for {\bfseries Watchdog} {\bfseries timeout} ms, the client will be assumed to have timed out, and it will be killed. The watchdog time for each client should be adjusted as required. For example, clients that contact external hardware that is slow to respond should have a longer time set, or they may timeout before the operation is complete.  

\\\cline{1-5}
\par
 &\par
 &\label{RC_customize_ODB_RC_programs_Check_interval}
\hypertarget{RC_customize_ODB_RC_programs_Check_interval}{}
 Check interval &INT &This value is the time interval in milliseconds that the Watchdog checks the client to see if it is responding. See \hyperlink{RC_customize_ODB_RC_programs_Watchdog_timeout}{Watchdog timeout} .  

\\\cline{1-5}
\par
 &\par
 &\label{RC_customize_ODB_RC_programs_Start_command}
\hypertarget{RC_customize_ODB_RC_programs_Start_command}{}
 Start Command &STRING &Contains the command used to restart the client. It is required if either \hyperlink{RC_customize_ODB_RC_programs_Auto_start}{Auto start}, or \hyperlink{RC_customize_ODB_RC_programs_Auto_restart}{Auto restart} is set to \char`\"{}y\char`\"{}, or the user wishes to start or restart the client using the \hyperlink{RC_mhttpd_Program_page}{mhttpd restart button}). If no start command is supplied, the user can restart the client by hand, or by using the (customized) script \label{RC_customize_ODB_start-all}
\hypertarget{RC_customize_ODB_start-all}{}
 \char`\"{}start\_\-daq.sh\char`\"{} .  

\\\cline{1-5}
\par
 &\par
 &\label{RC_customize_ODB_RC_programs_Auto_start}
\hypertarget{RC_customize_ODB_RC_programs_Auto_start}{}
 \label{RC_customize_ODB_idx_Auto_client_start}
\hypertarget{RC_customize_ODB_idx_Auto_client_start}{}
 Auto start &BOOL &If set to \char`\"{}y\char`\"{} the client will be started automatically using the \hyperlink{RC_customize_ODB_RC_programs_Start_command}{Start Command}. This will occur when an experiment is first started after killing all clients. If the client then dies (or times out -\/ see \hyperlink{RC_customize_ODB_RC_programs_Watchdog_timeout}{Watchdog time out}) the client will {\bfseries not} be restarted unless \hyperlink{RC_customize_ODB_RC_programs_Auto_restart}{Auto restart} is set to \char`\"{}y\char`\"{}.  

\\\cline{1-5}
\par
 &\par
 &\label{RC_customize_ODB_RC_programs_Auto_stop}
\hypertarget{RC_customize_ODB_RC_programs_Auto_stop}{}
 \label{RC_customize_ODB_idx_Auto_client_stop}
\hypertarget{RC_customize_ODB_idx_Auto_client_stop}{}
 Auto stop &BOOL &If this is set to \char`\"{}y\char`\"{} ...  

\\\cline{1-5}
\par
 &\par
 &\label{RC_customize_ODB_RC_programs_Auto_restart}
\hypertarget{RC_customize_ODB_RC_programs_Auto_restart}{}
 \label{RC_customize_ODB_idx_Auto_client_restart}
\hypertarget{RC_customize_ODB_idx_Auto_client_restart}{}
 Auto restart &BOOL &If set to \char`\"{}y\char`\"{} the client will be restarted automatically using the \hyperlink{RC_customize_ODB_RC_programs_Start_command}{Start Command}. This will occur if the client dies or times out (see \hyperlink{RC_customize_ODB_RC_programs_Watchdog_timeout}{Watchdog time out}). If set to \char`\"{}n\char`\"{}, the client must be restarted by the user (see \hyperlink{RC_customize_ODB_RC_programs_Start_command}{Start command}). See also \hyperlink{RC_customize_ODB_RC_programs_Auto_start}{Auto start}.  

\\\cline{1-5}
\par
 &\par
 &\label{RC_customize_ODB_RC_programs_Alarm_class}
\hypertarget{RC_customize_ODB_RC_programs_Alarm_class}{}
 Alarm class &STRING &If this field is set to one of the existing \hyperlink{RC_customize_ODB_RC_alarm_classes}{alarm classes}, an entry in the {\bfseries /Alarms/alarms} tree will be automatically created for this program. The \hyperlink{RC_customize_ODB_RC_alarm_type}{Alarm Type} will be {\itshape Program\/} {\itshape Alarm\/} . This will cause an alarm to go off if the program is not running (provided both the \hyperlink{RC_customize_ODB_RC_alarm_system_active}{alarm system} and the \hyperlink{RC_customize_ODB_RC_active}{individual alarm} are enabled).  

\\\cline{1-5}
\par
 &\par
 &\label{RC_customize_ODB_RC_programs_First_failed}
\hypertarget{RC_customize_ODB_RC_programs_First_failed}{}
 First\_\-failed &DWORD &Value filled by the System to indicate when client first failed.  

\\\cline{1-5}
\end{TabularC}
\centering
\caption{Above: Meaning of keys in the ODB /Programs tree. }
\end{table}


\par


\label{RC_customize_ODB_idx_alarm_system}
\hypertarget{RC_customize_ODB_idx_alarm_system}{}
 

 \hypertarget{RC_customize_ODB_RC_Alarm_System}{}\subsubsection{MIDAS Alarm System}\label{RC_customize_ODB_RC_Alarm_System}
MIDAS provides an alarm system, which by default is turned off. When the alarm system is \hyperlink{RC_customize_ODB_RC_alarm_system_active}{activated} and an alarm condition is detected, alarms messages are sent by the system which appear as an \hyperlink{RC_mhttpd_Alarm_page_RC_mhttpd_alarm_banner}{alarm banner} on the \hyperlink{RC_mhttpd_utility}{mhttpd} main status page, and as a \hyperlink{RC_mhttpd_Alarm_page_RC_odb_alarm_msg}{message} on any windows running \hyperlink{RC_odbedit_utility}{odbedit} clients. The alarm system is flexible and can be extensively customized for each experiment.

The MIDAS alarm system is built-\/in and part of the main experiment scheduler. This means no separate task is necessary to benefit from the alarm system. The Alarm feature is active during {\bfseries ONLINE} mode {\bfseries ONLY}.  Alarm setup and activation is done through the Online DataBase (ODB). The alarm system includes several other features such as sequencing and control of the experiment. The alarm capabilities are:
\begin{DoxyItemize}
\item Alarm setting on any ODB variable against a threshold parameter.
\item Alarm triggered by \hyperlink{RC_customize_ODB_RC_evaluated_alarm_condition}{evaluated condition}
\item Selection of Alarm check frequency
\item Selection of Alarm trigger frequency
\item Customizable alarm scheme; under this scheme multiple choices of alarm type can be selected.
\item Selection of alarm message destination ( to system message log or to elog)
\item \hyperlink{RC_customize_ODB_RC_alarms_email}{email or SMS alerts} can be sent
\item Program control on run transition.
\end{DoxyItemize}

The alarm system can be customized through \hyperlink{RC_odbedit_utility}{ODBEdit} or the \hyperlink{RC_mhttpd_ODB_page}{mhttpd Alarm page}. Some of the features (such as colour) are applicable only to {\bfseries mhttpd}.

The following sections describe how to use the MIDAS Alarm System.

See also \hyperlink{RC_customize_ODB_RC_ODB_Alarm_system_implementation}{Implementation of the MIDAS Alarm System}

\label{RC_customize_ODB_idx_alarm_classes}
\hypertarget{RC_customize_ODB_idx_alarm_classes}{}
 \label{RC_customize_ODB_idx_alarm_type}
\hypertarget{RC_customize_ODB_idx_alarm_type}{}
 \label{RC_customize_ODB_idx_ODB_tree_Alarms}
\hypertarget{RC_customize_ODB_idx_ODB_tree_Alarms}{}
 \label{RC_customize_ODB_RC_alarm_classes}
\hypertarget{RC_customize_ODB_RC_alarm_classes}{}
 \label{RC_customize_ODB_RC_alarm_type}
\hypertarget{RC_customize_ODB_RC_alarm_type}{}
 

 \hypertarget{RC_customize_ODB_RC_ODB_Alarms_Tree}{}\paragraph{ODB /Alarms Tree}\label{RC_customize_ODB_RC_ODB_Alarms_Tree}
The {\bfseries  ODB /Alarms tree } contains user and system information related to alarms. When the ODB is created,
\begin{DoxyItemize}
\item two {\bfseries Classes} of alarm are created : {\bfseries Alarm} and {\bfseries Warning} 
\item two {\bfseries Alarms} are created: {\bfseries Demo} {\bfseries ODB} and {\bfseries Demo} {\bfseries Periodic} 
\item by default, the {\bfseries alarm system is NOT active}
\end{DoxyItemize}

Currently, the overall alarm is checked once every minute. Once the alarm has been triggered, the message associated with the alarm can be repeated at a different rate. The {\bfseries Alarms} structure is split into 2 sections:
\begin{DoxyItemize}
\item {\bfseries \char`\"{}Alarms\char`\"{}} which define the condition to be tested. The user can create as many Alarms as desired, but each must be one of the four defined \hyperlink{RC_customize_ODB_RC_alarm_types}{Alarm Types} .
\item {\bfseries \char`\"{}Classes\char`\"{}} which define the action to be taken when the alarm occurs. Two Classes (Alarm and Warning) are defined by default. The user can add more Classes as desired.
\end{DoxyItemize}

\label{RC_customize_ODB_RC_alarm_types}
\hypertarget{RC_customize_ODB_RC_alarm_types}{}
 The four available Alarm Types are shown in the following table. They are defined in \hyperlink{midas_8h}{midas.h}. \par
\begin{table}[h]\begin{TabularC}{4}
\hline
Alarm Type  &INT value &Explanation  

\\\cline{1-3}
Internal alarms  &AT\_\-INTERNAL &1 &Trigger on internal (program) alarm setting through the use of the {\itshape al\_\-...()\/} functions.  

\\\cline{1-4}
Program alarms  &AT\_\-PROGRAM &2 &Triggered on condition of the state of the defined task.  

\\\cline{1-4}
Evaluated  &AT\_\-EVALUATED &3 &Triggered by ODB value on given arithmetical condition.  

\\\cline{1-4}
Periodic alarms  &AT\_\-PERIODIC &4 &Triggered by timeout condition defined in the alarm setting.   \\\cline{1-4}
\end{TabularC}
\centering
\caption{Above: Defined Alarm Types. }
\end{table}


In order to make the system flexible, each alarm class may perform different actions when an alarm is given. For example, it may write a system message, write to the elog, stop the run or spawn a detached script listed in the ODB variable /Programs/Classes/Execute command. This feature is used when an \hyperlink{RC_customize_ODB_RC_alarms_email}{Alarm triggers Email or SMS alerts} .\hypertarget{RC_customize_ODB_RC_evaluated_alarm_condition}{}\subparagraph{Evaluated Alarm conditions}\label{RC_customize_ODB_RC_evaluated_alarm_condition}
The alarm {\bfseries condition} for evaluated alarms is entered into the ODB key /Alarms/Alarms/$<$alarm\_\-name$>$/Condition  where $<$alarm\_\-name$>$ is the name of the alarm. See \hyperlink{RC_customize_ODB_RC_condition}{condition key}.

The {\bfseries condition} may be simply a comparison between any ODB variable and a threshold parameter, e.g. 
\begin{DoxyCode}
 /Runinfo/Run number > 100
\end{DoxyCode}
 or it may be an evaluated condition. One can write conditions like 
\begin{DoxyCode}
  /Equipment/HV/Variables/Input[*] < 100
\end{DoxyCode}
 or 
\begin{DoxyCode}
  /Equipment/HV/Variables/Input[2-3] < 100
\end{DoxyCode}
 to check all values from an array or a certain range. If one array element fulfills the alarm condition, the alarm is triggerrd. In addition, bit-\/wise alarm conditions are possible, e.g. 
\begin{DoxyCode}
  /Equipment/Environment/Variables/Input[0] & 8
\end{DoxyCode}
 The alarm is triggered if bit \#2 is set in Input\mbox{[}0\mbox{]}.



\hypertarget{RC_customize_ODB_RC_explanation_of_alarms_tree}{}\subparagraph{Meaning of the keys in the /Alarms ODB tree}\label{RC_customize_ODB_RC_explanation_of_alarms_tree}
\begin{table}[h]\begin{TabularC}{6}
\hline
Keys in the ODB tree /Alarms   \\\cline{1-1}
ODB Key  &Explanation  

\\\cline{1-2}
Alarms  &\par
 &\par
 &\par
 &DIR

&\par
 

\\\cline{1-6}
\par
 &\label{RC_customize_ODB_RC_alarm_system_active}
\hypertarget{RC_customize_ODB_RC_alarm_system_active}{}
 Alarm system active &\par
 &\par
 &BOOL &If set to \char`\"{}y\char`\"{}the alarm system is active. Set to \char`\"{}n\char`\"{} to deactivate.  

\\\cline{1-6}
\par
 &Alarms &\par
 &\par
 &DIR &Sub-\/tree defining each individual alarm condition. 

\\\cline{1-6}
\par
 &\par
 &Demo odb &\par
 &DIR &Name of one of the defined alarms 

\\\cline{1-6}
\par
 &\par
 &\par
 &\label{RC_customize_ODB_RC_active}
\hypertarget{RC_customize_ODB_RC_active}{}
 Active &BOOL &If set to \char`\"{}y\char`\"{} , this particular alarm is active.  

\\\cline{1-6}
\par
 &\par
 &\par
 &Triggered &INT &If non-\/zero, alarm is triggered. Filled by System.  

\\\cline{1-6}
\par
 &\par
 &\par
 &Type &INT &One of the listed \hyperlink{RC_customize_ODB_RC_alarm_types}{Alarm Types}  

\\\cline{1-6}
\par
 &\par
 &\par
 &Check interval &INT &Frequency in seconds that alarm condition is checked  

\\\cline{1-6}
\par
 &\par
 &\par
 &Checked last &DWORD &Written by Alarm System  

\\\cline{1-6}
\par
 &\par
 &\par
 &Time triggered first &STRING &Written by Alarm System  

\\\cline{1-6}
\par
 &\par
 &\par
 &Time triggered last &STRING &Written by Alarm System  

\\\cline{1-6}
\par
 &\par
 &\par
 &\label{RC_customize_ODB_RC_condition}
\hypertarget{RC_customize_ODB_RC_condition}{}
 Condition &STRING &\hyperlink{RC_customize_ODB_RC_evaluated_alarm_condition}{Condition} on which alarm should trigger.  

\\\cline{1-6}
\par
 &\par
 &\par
 &Alarm class &STRING &Set to one of the existing Alarm classes, e.g. Alarm, Warning  

\\\cline{1-6}
\par
 &\par
 &\par
 &Alarm message &STRING &Message to be written when alarm triggers  

\\\cline{1-6}
\par
 &Classes &\par
 &\par
 &DIR &Sub-\/tree defining each individual action to be performed by a pre-\/defined and requested alarm. 

\\\cline{1-6}
\par
 &\par
 &Warning &\par
 &DIR &Name of one of the defined classes 

\\\cline{1-6}
\par
 &\par
 &\par
 &Write System Message &BOOL &If set to \char`\"{}y\char`\"{} a message will be sent to the System log when alarm is triggered.  

\\\cline{1-6}
\par
 &\par
 &\par
 &Write Elog Message &BOOL &If set to \char`\"{}y\char`\"{} a message will be written to the Elog when alarm is triggered  

\\\cline{1-6}
\par
 &\par
 &\par
 &System message interval &INT &Interval in seconds between successive system messages when alarm is triggered  

\\\cline{1-6}
\par
 &\par
 &\par
 &System message last &DWORD &Filled by System...  

\\\cline{1-6}
\par
 &\par
 &\par
 &Execute command &STRING &Command to be executed when alarm is triggered.  

\\\cline{1-6}
\par
 &\par
 &\par
 &Execute last &DWORD &\par
  

\\\cline{1-6}
\par
 &\par
 &\par
 &Stop run &BOOL &\par
  

\\\cline{1-6}
\par
 &\par
 &\par
 &Display BGColor &STRING &Background colour of \hyperlink{RC_mhttpd_Alarm_page_RC_mhttpd_alarm_banner}{alarm banner} (mhttpd only).  

\\\cline{1-6}
\par
 &\par
 &\par
 &Display FGColor &STRING &Foreground colour of \hyperlink{RC_mhttpd_Alarm_page_RC_mhttpd_alarm_banner}{alarm banner} (mhttpd only).   \\\cline{1-6}
\end{TabularC}
\centering
\caption{Above: Meaning of keys in the ODB /Alarms tree. }
\end{table}
\par


\par
\hypertarget{RC_customize_ODB_RC_alarms_tree_example}{}\subparagraph{Examples of an /Alarms tree}\label{RC_customize_ODB_RC_alarms_tree_example}
Part of the {\bfseries  /Alarms } tree is shown below using \hyperlink{RC_odbedit_utility}{odbedit} (see also \hyperlink{RC_mhttpd_Alarm_page}{mhttpd Alarm page}).


\begin{DoxyCode}
[local:pol:S]/>cd /alarms
[local:pol:S]/Alarms>ls
Alarm system active             y
Alarms
Classes
\end{DoxyCode}


Some of the types of alarm under the {\bfseries  /Alarms/Alarms } tree for an experiment are shown below: 
\begin{DoxyCode}
[local:pol:S]/Alarms>ls -r -lt
Key name                        Type    #Val  Size  Last Opn Mode Value
---------------------------------------------------------------------------
Alarms                          DIR
    Alarm system active         BOOL    1     4     4h   0   RWD  y
    Alarms                      DIR
        Demo ODB                DIR
            Active              BOOL    1     4     >99d 0   RWD  n
            Triggered           INT     1     4     >99d 0   RWD  0
            Type                INT     1     4     >99d 0   RWD  3
            Check interval      INT     1     4     >99d 0   RWD  60
            Checked last        DWORD   1     4     >99d 0   RWD  0
            Time triggered firstSTRING  1     32    >99d 0   RWD
            Time triggered last STRING  1     32    >99d 0   RWD
            Condition           STRING  1     256   >99d 0   RWD  /Runinfo/Run nu
      mber > 100
            Alarm Class         STRING  1     32    >99d 0   RWD  Alarm
            Alarm Message       STRING  1     80    >99d 0   RWD  Run number beca
      me too large
        Demo periodic           DIR
              Active              BOOL    1     4     >99d 0   RWD  n
            Triggered           INT     1     4     >99d 0   RWD  0
            Type                INT     1     4     >99d 0   RWD  4
            Check interval      INT     1     4     >99d 0   RWD  28800
            Checked last        DWORD   1     4     >99d 0   RWD  1058817867
            Time triggered firstSTRING  1     32    >99d 0   RWD
            Time triggered last STRING  1     32    >99d 0   RWD
            Condition           STRING  1     256   >99d 0   RWD
            Alarm Class         STRING  1     32    >99d 0   RWD  Warning
            Alarm Message       STRING  1     80    >99d 0   RWD  Please do your 
      shift checks
        fePOL                   DIR
            Active              BOOL    1     4     19s  0   RWD  y
            Triggered           INT     1     4     19s  0   RWD  205
            Type                INT     1     4     3s   0   RWD  2
            Check interval      INT     1     4     19s  0   RWD  60
            Checked last        DWORD   1     4     19s  0   RWD  1259196026
            Time triggered firstSTRING  1     32    19s  0   RWD  Wed Nov 25 12:5
      9:33 2009
            Time triggered last STRING  1     32    19s  0   RWD  Wed Nov 25 16:4
      0:26 2009
            Condition           STRING  1     256   3s   0   RWD  Program not run
      ning
            Alarm Class         STRING  1     32    19s  0   RWD  Caution
            Alarm Message       STRING  1     80    19s  0   RWD  Program fePOL i
      s not running
        thr2 trip               DIR
            Active              BOOL    1     4     3s   0   RWD  y
            Triggered           INT     1     4     3s   0   RWD  0
            Type                INT     1     4     3s   0   RWD  3
            Check interval      INT     1     4     3s   0   RWD  15
            Checked last        DWORD   1     4     3s   0   RWD  1259196042
            Time triggered firstSTRING  1     32    3s   0   RWD
            Time triggered last STRING  1     32    3s   0   RWD
            Condition           STRING  1     256   3s   0   RWD  /Equipment/Info
       ODB/Variables/last failed thr test = 2
            Alarm Class         STRING  1     32    3s   0   RWD  Threshold
            Alarm Message       STRING  1     80    3s   0   RWD  Laser threshold
       check failed
\end{DoxyCode}


In the above example,
\begin{DoxyItemize}
\item {\bfseries Demo odb} and {\bfseries Demo periodic} were created when the ODB was created.
\item The alarm {\bfseries Fepol} was added automatically when the user filled the \hyperlink{RC_customize_ODB_RC_programs_Alarm_class}{alarm class} field in the {\bfseries /Programs/fepol} sub-\/tree.
\item The other alarm {\bfseries thr2\_\-trip} was added by the user. \par

\end{DoxyItemize}

Four Classes of alarms (Alarm, Caution, Warning and Threshold) are defined under the /Alarms/Classes tree for this experiment. Alarm and Warning were created when the ODB was created. The user added two more classes, Caution and Threshold, by \hyperlink{RC_odbedit_examples_RC_odbedit_copy}{copying} and \hyperlink{RC_odbedit_examples_RC_odbedit_set}{editing} one of the existing classes. The Classes defined for the experiment are shown below:


\begin{DoxyCode}
   Classes                      DIR
        Alarm                   DIR
            Write system messageBOOL    1     4     27h  0   RWD  y
            Write Elog message  BOOL    1     4     27h  0   RWD  n
            System message interINT     1     4     27h  0   RWD  60
            System message last DWORD   1     4     27h  0   RWD  0
            Execute command     STRING  1     256   27h  0   RWD
            Execute interval    INT     1     4     27h  0   RWD  0
            Execute last        DWORD   1     4     27h  0   RWD  0
            Stop run            BOOL    1     4     27h  0   RWD  n
            Display BGColor     STRING  1     32    27h  0   RWD  red
            Display FGColor     STRING  1     32    27h  0   RWD  black
        Warning                 DIR
            Write system messageBOOL    1     4     >99d 0   RWD  y
            Write Elog message  BOOL    1     4     >99d 0   RWD  n
            System message interINT     1     4     >99d 0   RWD  60
            System message last DWORD   1     4     >99d 0   RWD  0
            Execute command     STRING  1     256   >99d 0   RWD
            Execute interval    INT     1     4     >99d 0   RWD  0
            Execute last        DWORD   1     4     >99d 0   RWD  0
            Stop run            BOOL    1     4     >99d 0   RWD  n
            Display BGColor     STRING  1     32    >99d 0   RWD  red
            Display FGColor     STRING  1     32    >99d 0   RWD  black
      Caution                 DIR
            Write system messageBOOL    1     4     19s  0   RWD  y
            Write Elog message  BOOL    1     4     19s  0   RWD  n
            System message interINT     1     4     19s  0   RWD  60
            System message last DWORD   1     4     19s  0   RWD  1259196026
            Execute command     STRING  1     256   19s  0   RWD
            Execute interval    INT     1     4     19s  0   RWD  0
            Execute last        DWORD   1     4     19s  0   RWD  0
            Stop run            BOOL    1     4     19s  0   RWD  y
            Display BGColor     STRING  1     32    19s  0   RWD  blue
            Display FGColor     STRING  1     32    19s  0   RWD  red
       Threshold               DIR
            Write system messageBOOL    1     4     >99d 0   RWD  n
            Write Elog message  BOOL    1     4     >99d 0   RWD  n
            System message interINT     1     4     >99d 0   RWD  60
            System message last DWORD   1     4     >99d 0   RWD  0
            Execute command     STRING  1     256   >99d 0   RWD
            Execute interval    INT     1     4     >99d 0   RWD  0
            Execute last        DWORD   1     4     >99d 0   RWD  0
            Stop run            BOOL    1     4     >99d 0   RWD  n
            Display BGColor     STRING  1     32    >99d 0   RWD  yellow
            Display FGColor     STRING  1     32    >99d 0   RWD  black
\end{DoxyCode}


\par


\par
\hypertarget{RC_customize_ODB_RC_alarms_email}{}\subsubsection{Alarm triggers Email or SMS alerts}\label{RC_customize_ODB_RC_alarms_email}
It is also possible to have the MIDAS alarm system send email or SMS alerts to cell phones when alarms are triggered. This can be configured by defining an ODB alarm on a critical ODB parameter, e.g. 
\begin{DoxyCode}
/Alarms/Alarms/Liquid Level
Active                   y
Triggered                0 (0x0)
Type                     3 (0x3)
Check interval          60 (0x3C)
Checked last    1227690148 (0x492D10A4)
Time triggered first    (empty)
Time triggered last     (empty)
Condition               /Equipment/Environment/Variables/Input[0] < 10
Alarm Class             Level Alarm
Alarm Message           Liquid Level is only %s
\end{DoxyCode}
 In this example, the alarm triggers an alarm of class \char`\"{}Level Alarm\char`\"{}. This alarm class is defined as follows: 
\begin{DoxyCode}
/Alarms/Classes/Level Alarm
Write system message    y
Write Elog message      n
System message interval 600 (0x258)
System message last     0 (0x0)
Execute command         /home/midas/level_alarm '%s'
Execute interval        1800 (0x708)
Execute last            0 (0x0)
Stop run                n
Display BGColor         red
Display FGColor         black
\end{DoxyCode}
 The key here is to call a script \char`\"{}level\_\-alarm\char`\"{}, which can send emails. Use something like: 
\begin{DoxyCode}
#/bin/csh
echo $1 | mail -s \"Level Alarm\" your.name@domain.edu
odbedit -c 'msg 2 level_alarm \"Alarm was sent to your.name@domain.edu\"'
\end{DoxyCode}
 The second command just generates a MIDAS system message for confirmation. Most cell phones (depends on the provider) have an email address. If you send an email there, it will be translated into a SMS message.

The script file above can of course be more complicated. A perl script could be used that parses an address list, so other interested parties can register by adding his/her email address to that list. The script may also collects some other slow control variables (like pressure, temperature) and combine them into the SMS message.

For very sensitive systems, having an alarm via SMS may not be sufficient, since the alarm system could be down (e.g. computer crash, network failure). In this case 'negative alarms' can be used. For example, every 30 minutes the system may send an SMS with the current parameter values. If the expected message is not received, it may indicate that something in the MIDAS system is wrong.\hypertarget{RC_customize_ODB_RC_ODB_Alarm_system_implementation}{}\subsubsection{Implementation of the MIDAS Alarm System}\label{RC_customize_ODB_RC_ODB_Alarm_system_implementation}
Alarms are checked inside \hyperlink{group__alfunctioncode_gaf31864a8bc5fe779057e81bde12167a9}{alarm.c::al\_\-check()}. This function is called by \hyperlink{group__cmfunctionc_ga115565c5a1d9591fcabf844c1dd624f8}{cm\_\-yield()} every 10 seconds and by rpc\_\-server\_\-thread(), also every 10 seconds. For remote MIDAS clients, their \hyperlink{group__alfunctioncode_gaf31864a8bc5fe779057e81bde12167a9}{al\_\-check()} issues an RPC\_\-AL\_\-CHECK RPC call into the MIDAS server utility \hyperlink{RC_customize_ODB_RC_mserver_utility}{mserver}, where rpc\_\-server\_\-dispatch() calls the local \hyperlink{group__alfunctioncode_gaf31864a8bc5fe779057e81bde12167a9}{al\_\-check()}. As result, all alarm checks run inside a process directly attached to the local MIDAS shared memory (inside a local client or inside an mserver process for a remote client). Each and every MIDAS client runs the alarm checks. To prevent race conditions between different MIDAS clients, access to \hyperlink{group__alfunctioncode_gaf31864a8bc5fe779057e81bde12167a9}{al\_\-check()} is serialized using the \hyperlink{structALARM}{ALARM} semaphore. Inside \hyperlink{group__alfunctioncode_gaf31864a8bc5fe779057e81bde12167a9}{al\_\-check()}, alarms are triggered using \hyperlink{group__alfunctioncode_gac024cd8160dc8b9418f05a63678f6c68}{al\_\-trigger\_\-alarm()}, which in turn calls al\_\-trigger\_\-class(). Inside al\_\-trigger\_\-class(), the alarm is recorded into an elog or into midas.log using cm\_\-msg(MTALK).

Special note should be made of the ODB setting \char`\"{}/Alarm/Classes/xxx/System
message interval\char`\"{}, which has a surprising effect -\/ after an alarm is recorded into system messages (using cm\_\-msg(MTALK)), no record is made of any subsequent alarms until the time interval set by this variable elapses. With default value of 60 seconds, after one alarm, no more alarms are recorded for 60 seconds. Also, because all the alarms are checked at the same time, only the first triggered alarm will be recorded.

As of \hyperlink{alarm_8c}{alarm.c} rev 4683,  \char`\"{}System message interval\char`\"{} is set to 0 ensures that every alarm is recorded into the MIDAS log file. (In previous revisions, this setting may still miss some alarms).

There are 3 types of alarms:

1) \char`\"{}program not running\char`\"{} alarms.

These alarms are enabled in ODB by setting /Programs/ppp/Alarm class. Each time \hyperlink{group__alfunctioncode_gaf31864a8bc5fe779057e81bde12167a9}{al\_\-check()} runs, every program listed in /Programs is tested using \char`\"{}cm\_\-exist()\char`\"{} and if the program is not running, the time of first failure is remembered in /Programs/ppp/First failed.

If the program has not been running for longer than the time set in ODB key /Programs/ppp/Check interval, an alarm is triggered (if enabled by /Programs/ppp/Alarm class and the program is restarted (if enabled by /Programs/ppp/Auto restart).

The \char`\"{}not running\char`\"{} condition is tested every 10 seconds (each time \hyperlink{group__alfunctioncode_gaf31864a8bc5fe779057e81bde12167a9}{al\_\-check()} is called), but the frequency of \char`\"{}program not running\char`\"{} alarms can be reduced by increasing the value of /Alarms/Alarms/ppp/Check interval (default value 60 seconds). This can be useful if System message interval is set to zero.

2) \char`\"{}evaluated\char`\"{} alarms

3) \char`\"{}periodic\char`\"{} alarms

There is nothing surprising in these alarms. Each alarm is checked with a time period set by /Alarm/xxx/Check interval. The value of an evaluated alarm is computed using al\_\-evaluate\_\-condition().

 \par


\label{index_end}
\hypertarget{index_end}{}
 \subsection{Event Notification (Hot-\/Link)}\label{RC_Hot_Link}
\par
 

\par
 \label{RC_Hot_Link_idx_hotlink}
\hypertarget{RC_Hot_Link_idx_hotlink}{}
 \label{RC_Hot_Link_idx_event_notification-see-hotlink}
\hypertarget{RC_Hot_Link_idx_event_notification-see-hotlink}{}
 \hypertarget{RC_Hot_Link_RC_Hot_Link_Intro}{}\subsubsection{Introduction}\label{RC_Hot_Link_RC_Hot_Link_Intro}
MIDAS implements event notification through {\bfseries  \char`\"{}hot-\/links\char`\"{} }. Once a hot-\/link is established to a key in the ODB, {\bfseries immediately} that key is accessed, a call-\/back routine associated with the hot-\/link is called to perform whatever action has been programmed. The MIDAS system uses hot-\/links to update keys in the ODB for communication between system clients (e.g. \hyperlink{F_History_logging_F_Frontend_History_Event}{history system}).

Users often use hot-\/links to immediately set some hardware to a new value. The new value may have been input into the ODB, either by the user or another client. Without a hot-\/link, the program setting the hardware values would have to continually poll the ODB to see if any values had changed; otherwise the new value would not be transmitted to the hardware until the next time the ODB set values were read and applied (for example, at the beginning of a run). \par




\hypertarget{RC_Hot_Link_RC_Example_Hot_Link}{}\subsubsection{How to set up a Hot-\/Link}\label{RC_Hot_Link_RC_Example_Hot_Link}
It is often desirable to modify hardware \hyperlink{structparameters}{parameters} (such as discriminator levels or trigger logic) connected to the frontend computer. Assuming that the required hardware is accessible from the frontend code, these \hyperlink{structparameters}{parameters} are easily controllable when a hot-\/link is established between the frontend and the ODB itself.

\begin{center} Hot-\/Link process  \end{center} 

First the \hyperlink{structparameters}{parameters} have to be defined in the ODB under the Settings tree for the given equipment. Let's assume we have two discriminator levels belonging to the trigger electronics, which should be controllable. The following commands define these levels in the ODB: 
\begin{DoxyCode}
[local]/>cd /Equipment/Trigger/
[local]Trigger>create key Settings
[local]Trigger>cd Settings
[local]Settings>create int level1
[local]Settings>create int level2
[local]Settings>ls
\end{DoxyCode}


\label{RC_Hot_Link_idx_experim-dot-h}
\hypertarget{RC_Hot_Link_idx_experim-dot-h}{}
 \label{RC_Hot_Link_RC_experim_dot_h}
\hypertarget{RC_Hot_Link_RC_experim_dot_h}{}
 The frontend can now map a C structure to these settings. In order to simplify this process, \hyperlink{RC_odbedit}{ODBEdit} can be requested to generate a header file containing this C structure. The odbedit command \hyperlink{RC_odbedit_examples_RC_odbedit_make}{make} generates in the current directory the header file {\bfseries \hyperlink{experim_8h}{experim.h}} . \par
 This file can be copied to the frontend directory (if necessary) and included in the frontend source code. It contains a section with a C structure of the trigger settings and an ASCII representation: 
\begin{DoxyCode}
typedef struct {
  INT       level1;
  INT       level2;
  TRIGGER_SETTINGS;

#define TRIGGER_SETTINGS_STR(_name) char *_name[] = {\
"[.]",\
"level1 = INT : 0",\
"level2 = INT : 0",\

"",\
NULL  
\end{DoxyCode}


This definition can be used to define a C structure containing the \hyperlink{structparameters}{parameters} in \hyperlink{frontend_8c}{frontend.c}: 
\begin{DoxyCode}
#include <experim.h>

TRIGGER_SETTINGS trigger_settings;
\end{DoxyCode}


A hot-\/link between the ODB values and the C structure is established in the \hyperlink{mfe_8c_a802849119d469feb2d1deee1be9593ac}{frontend\_\-init()} routine: 
\begin{DoxyCode}
INT frontend_init()
{HNDLE hDB, hkey;
TRIGGER_SETTINGS_STR(trigger_settings_str);

  cm_get_experiment_database(&hDB, NULL);

  db_create_record(hDB, 0,
    "/Equipment/Trigger/Settings",
    strcomb(trigger_settings_str));

  db_find_key(hDB, 0, "/Equipment/Trigger/Settings", &hkey);

  if (db_open_record(hDB, hkey,
      &trigger_settings,
      sizeof(trigger_settings), MODE_READ,
      trigger_update, NULL) != DB_SUCCESS)
    {
    cm_msg(MERROR, "frontend_init",
      "Cannot open Trigger Settings in ODB");
    return -1;
     
  return SUCCESS;
\end{DoxyCode}


The \hyperlink{group__odbfunctionc_ga59b971e77416b2b463e2e63f1b05342b}{db\_\-create\_\-record()} function re-\/creates the settings sub-\/tree in the ODB from the ASCII representation in case it has been corrupted or deleted. The \hyperlink{group__odbfunctionc_ga852bc9fa8ee4d0884b328aeb0b0cfd63}{db\_\-open\_\-record()} now establishes the hot-\/link between the settings in the ODB and the trigger\_\-settings structure. Each time the ODB settings are modified, the changes are written to the trigger\_\-settings structure and the callback routine trigger\_\-update() is executed afterwards. This routine has the task to set the hardware according to the settings in the trigger\_\-settings structure.

It may look like: 
\begin{DoxyCode}
void trigger_update(INT hDB, INT hkey)
{
  printf("New levels: %d %d",
    trigger_settings.level1,
    trigger_settings.level2);
\end{DoxyCode}


Of course the printf() function should be replaced by a function which accesses

the hardware properly. Modifying the trigger values with ODBEdit can test the whole scheme: 
\begin{DoxyCode}
[local]/>cd /Equipment/Trigger/Settings
[local]Settings>set level1 123
[local]Settings>set level2 456
\end{DoxyCode}
 Immediately after each modification the frontend should display the new values. The settings can be saved to a file and loaded back later: 
\begin{DoxyCode}
[local]/>cd /Equipment/Trigger/Settings
[local]Settings>save settings.odb
[local]Settings>set level1 789
[local]Settings>load settings.odb
\end{DoxyCode}
 The settings can also be modified from any application just by accessing the ODB. Following listing is a complete user application that modifies the trigger level: 
\begin{DoxyCode}
#include <midas.h>

main()
{
HNDLE hDB;
INT   level;
  cm_connect_experiment("", "Sample", "Test",
                        NULL);
  cm_get_experiment_database(&hDB, NULL);

  level = 321;
  db_set_value(hDB, 0,
    "/Equipment/Trigger/Settings/Level1",
    &level, sizeof(INT), 1, TID_INT);

  cm_disconnect_experiment();
\end{DoxyCode}
 The following figure summarizes the involved components:

To make sure a hot-\/link exists, one can use the \hyperlink{RC_odbedit}{odbedit} command \hyperlink{RC_odbedit_examples_RC_odbedit_sor}{sor -\/ show open records} : 
\begin{DoxyCode}
 [local]Settings>cd /
[local]/>sor
/Equipment/Trigger/Settings open 1 times by ...
\end{DoxyCode}


\par
 \label{index_end}
\hypertarget{index_end}{}


 \par
 